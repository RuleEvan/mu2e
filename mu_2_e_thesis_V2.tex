\documentclass{book}[12pt]
\usepackage{amsmath}
\usepackage{graphicx}
\usepackage{braket}
\usepackage[margin=1.0in]{geometry}
\usepackage[force]{feynmp-auto}
\usepackage{amssymb, amsthm}
\usepackage{tikz}
\usepackage{tikz-cd}
\usepackage{slashed}
\usepackage[numbers]{natbib}
\usepackage{subfig}
\usepackage{hyperref}
\newcommand{\prc}{Phys. Rev. C}
\newcommand{\prd}{Phys. Rev. D}
\usepackage[T1]{fontenc}
\usepackage[utf8]{inputenc}
\usepackage{notoccite}
\DeclareUnicodeCharacter{2032}{$^\prime$}
\DeclareUnicodeCharacter{2212}{-}
\newcommand{\pvec}[1]{\vec{#1}\mkern2mu\vphantom{#1}}
\title{Effective Theory of $\mu^-\rightarrow e^-$ Conversion in Nuclei}
\author{Evan Rule}

\begin{document}
\pagenumbering{gobble}
\begin{titlepage}
\begin{center}
\vspace*{1cm}
\textbf{Nuclear Effective Theory of $\mu\rightarrow e$ Conversion}\\
\vspace*{1cm}
by\\
\vspace*{0.5cm}
Evan Johnson Rule\\
\vspace*{1cm}
A dissertation submitted in partial satisfaction of the\\
\vspace*{0.25cm}
requirements for the degree of\\
\vspace*{0.25cm}
Doctor of Philosophy\\
\vspace*{0.25cm}
in\\
\vspace*{0.25cm}
Physics\\
\vspace*{0.25cm}
in the \\
\vspace*{0.25cm}
Graduate Division\\
\vspace*{0.25cm}
of the\\
\vspace*{0.25cm}
University of California, Berkeley\\
\vspace*{2cm}
Committee in charge:\\
\vspace*{1cm}
Professor Wick Haxton, Chair\\
\vspace*{0.25cm}
Professor Yury Kolomensky\\
\vspace*{0.25cm}
Professor Karl A. van Bibber\\
\vspace*{2cm}
Summer 2022
\end{center}
\end{titlepage}
\thispagestyle{plain}
\begin{center}
\textbf{Nuclear Effective Theory of $\mu\rightarrow e$ Conversion}

\vspace{4cm}
Copyright 2022

by

Evan Johnson Rule
\end{center}
\newpage
\pagenumbering{arabic}
\thispagestyle{plain}
\begin{center}
\textbf{Abstract}

\vspace{2cm}
Nuclear Effective Theory of $\mu\rightarrow e$ Conversion

\vspace{0.5cm}
by

\vspace{0.5cm}
Evan Johnson Rule

\vspace{0.5cm}
Doctor of Philosophy in Physics

\vspace{0.5cm}
University of California, Berkeley

\vspace{0.5cm}
Professor Wick Haxton, Chair
\end{center}
The coming decade promises exceptional experimental progress in searches for lepton flavor-violating $\mu\rightarrow e$ conversion due to efforts at Fermilab (Mu2E) and J-PARC (COMET). 
\frontmatter
\tableofcontents
\listoffigures
\listoftables
\mainmatter
\chapter{Introduction}
Throughout the modern history of particle physics, tests of lepton flavor violation have played a crucial role in advancing theoretical understanding. When it was first hypothesized that the weak interaction is mediated by charged bosons, only a single neutrino flavor was known to exist. Based on this assumption, Feinberg \cite{PhysRev.110.1482} in 1958 noted that the charged bosons of the weak force would induce $\mu^+\rightarrow e^+ + \gamma$ at the level of $10^{-4}$, a branching ratio already excluded by the upper limit of $2\times 10^{-5}$ on this process set in 1955 by the Columbia University Nevis cyclotron \cite{PhysRev.98.240}. The work of Feinberg immediately led to the two-neutrino hypothesis, where separate neutrinos couple to the muon and electron, respectively, thereby forbidding (in the absence of neutrino flavor violation) the $\mu^+\rightarrow e^+ + \gamma$ decay. When the two-neutrino hypothesis was confirmed at Brookhaven National Laboratory \cite{PhysRevLett.9.36} in 1965, it was the first indication of a conserved quantum number associated with lepton flavor, extending to both the charged and neutral leptons\footnote{This progress also contributed to the eventual recognition that the ``$\mu$ meson'', as it was known at the time of Feinberg's seminal work, is not associated with the strong nuclear force and is in fact a lepton.}. To this day, lepton flavor is enshrined in the Lagrangian of the standard model of particle physics as a conserved quantity. 

On the other hand, the modern philosophy of particle physics is that global symmetries that are not the result of an underlying gauge symmetry are ``accidental'' and ultimately violated, though perhaps to a highly suppressed degree. This is precisely the expectation for the accidental standard-model symmetries of baryon and lepton number conservation. Many extensions of the standard model, including supersymmetry and other Grand Unified Theories (GUTs), lead to the violation of baryon and lepton number at high energies, permitting, for example, a free proton to decay on an extremely long timescale. 

Through the observation of neutrino oscillations, it is already known that lepton flavor is not exactly conserved in nature, and some modification of the standard model is required to account for this physics. Given that neutrinos oscillate between flavor eigenstates as they propagate and that charged leptons interact with neutrinos through the electroweak force, one can immediately envision processes, for example those shown in Figure \ref{fig:neutrino_clfv}, through which charged lepton flavor is violated at one-loop level. The rate for such processes, however, is suppressed by $(m_{\nu}/m_W)^4$, leading to an overall branching ratio that is $\sim 40$ orders of magnitude below current experimental sensitivities. This implies that any observation of charged lepton flavor violation (CLFV) is an unambiguous sign of physics beyond the standard model (BSM).

\begin{figure}
\centering
\subfloat[]{
\begin{fmffile}{neutrino_clfv1}
\begin{fmfgraph*}(120,120)
\fmfleft{i1} \fmfright{o1}
\fmf{fermion}{i1,v1}
\fmf{fermion,label=$\nu_{\mu}$}{v1,v2}
\fmf{fermion,label=$\nu_e$}{v2,v3}
\fmf{fermion}{v3,o1}
\fmf{photon,left,tension=0,label=$W$}{v1,v3}
\fmflabel{$\mu$}{i1}
\fmflabel{$e$}{o1}
\fmfv{decoration.shape=cross}{v2}
\end{fmfgraph*}
\end{fmffile}
}
\hspace{5cm}
\subfloat[]{
\begin{fmffile}{neutrino_clfv2}
\begin{fmfgraph*}(120,120)
\fmfleft{i1} \fmfright{o1} \fmftop{t1,t2,t3,t4,t5}
\fmf{fermion,tension=100}{i1,v1}
\fmf{fermion,label=$\nu_{\mu}$,tension=100}{v1,v2}
\fmf{fermion,label=$\nu_e$,tension=100}{v2,v3}
\fmf{fermion,tension=100}{v3,o1}
\fmf{photon,left=0.4,label=$W$}{v1,v4}
\fmf{photon,left=0.4,label=$W$}{v4,v3}
\fmf{photon,label=$\gamma$,tension=1.4}{v4,t3}
\fmflabel{$\mu$}{i1}
\fmflabel{$e$}{o1}
\fmfv{decoration.shape=cross}{v2}
\end{fmfgraph*}
\end{fmffile}
}
\caption{Examples of CLFV induced by neutrino flavor oscillations. The CLFV rate which results from this source alone is unobservably small.}
\label{fig:neutrino_clfv}
\end{figure}
 

In fact, observable rates of CLFV are a consequence of many proposed BSM theories including, among others, extensions which generate neutrino masses \cite{PhysRevD.67.076006,Abada:2015oba}, account for the muon $g-2$ anomaly \cite{Lindner:2016bgg,Li:2021lnz}, exploit supersymmetry \cite{Carvalho:2002bq,PhysRevD.74.116002,Figueiredo:2013tea,Gomez:2017dhl} or other mechanisms \cite{Ciafaloni:1995ad,Benbrik:2010cf,Feldmann:2016hvo} for unification, and explain the observed matter/antimatter asymmetry \cite{PhysRevD.83.076007,Merlo:2018rin}. In addition to the $\mu\rightarrow e\gamma$ reaction originally considered by Feinberg, the CLFV processes of $\mu\rightarrow e$ conversion in nuclei and $\mu\rightarrow 3e$ are among the most sensitive tests of new physics beyond the standard model. In particular, such processes provide a more general constraint on supersymmetric grand unification than either proton decay or neutrino masses \citep{Barbieri_1994}.

In this work, our primary focus is the process of $\mu\rightarrow e$ conversion, in which a muon is captured into the Coulomb field of an atomic nucleus. 
Once bound, the muon quickly  de-excites to the $1s$ ground state of the nuclear Coulomb field. The timescale for transition to the ground state (within $10^{-9}$ to $10^{-12}$ seconds) is much faster than standard model muon decay processes ($\sim 10^{-5}$ seconds), and therefore one may describe the muon as residing in the $1s$ state. There are two primary standard model processes through which the muons then decay: decay in orbit
\begin{equation}
\mu^-\rightarrow e^- + \bar{\nu}_e+\nu_{\mu},
\end{equation}
and standard muon capture
\begin{equation}
\mu^- + A(Z,N)\rightarrow \nu_{\mu}+A(Z-1,N+1),
\end{equation}
where $Z$ is the nuclear charge and $N$ the number of neutrons. Both of these processes conserve total lepton number as well as lepton flavor; the former produces a continuous spectrum of electrons which constitute the primary experimental background in searches for CLFV $\mu\rightarrow e$ conversion. We are interested in the process
\begin{equation}
\mu^-+A(Z,N)\rightarrow e^- + A(Z,N),
\end{equation}
which conserves total lepton number but violates lepton flavor  by converting a muon into an electron with no neutrino byproducts. The relevant quantity that is constrained by experiments is the branching ratio
\begin{equation}
B(\mu^-+(A,Z)\rightarrow e^- + (A,Z))\equiv\frac{\Gamma(\mu^-+(A,Z)\rightarrow e^-+(A,Z))}{\Gamma(\mu^-+(A,Z)\rightarrow\nu_{\mu}+A(Z-1,N+1))},
\end{equation}
where the numerator is the rate for the CLFV conversion process, and the denominator is the rate of standard muon capture. The current best limit on this branching ratio is $B(\mu^-\rightarrow e^-)<7\times 10^{-13}$ (90\% confidence level), set by the SINDRUM-II collaboration \citep{Bertl2006} using Au as the target nucleus. The next generation of experiments, Mu2e at Fermilab \citep{Bernstein_2019} and the COherent Muon to Electron Transition (COMET) experiment \citep{10.3389/fphy.2018.00133} at Japan Proton Research Complex (J-PARC), are expected to improve this limit by four orders of magnitude. An upgrade of the Mu2e experiment has recently been proposed which would improve upon this sensitivity by a further order of magnitude \cite{Mu2e:2018osu}.

Significant advances in the sensitivity of searches for $\mu\rightarrow e\gamma$ and $\mu\rightarrow 3e$ are also soon expected with the advent, respectively, of the Mu to E Gamma (MEG) II and Mu3e experiments, both located at the Paul Scherrer Institute (PSI). The branching ratios for these two processes are measured relative to the dominant free-muon decay mode; that is
\begin{equation}
B(\mu\rightarrow e\gamma)\equiv\frac{\Gamma\left(\mu\rightarrow e+\gamma\right)}{\Gamma\left(\mu\rightarrow e+2\nu\right)},\;\;B(\mu\rightarrow 3e)\equiv\frac{\Gamma\left(\mu\rightarrow 3e\right)}{\Gamma\left(\mu\rightarrow e+2\nu\right)}.
\end{equation} 
The existing branching ratio limits for $\mu\rightarrow e$ conversion, $\mu\rightarrow e\gamma$ and $\mu\rightarrow 3e$ are shown in Table \ref{tab:existing_rates}. The corresponding limits expected at the next-generation experiments are shown in Table \ref{tab:future_rates}. 

 \renewcommand{\arraystretch}{1.5}
\begin{table}[h]
\centering
\begin{tabular}{lllll}
\hline
\hline
Process & Limit & Experiment/Lab & Year & Reference \\ 
\hline
$\mu^-+$Cu $\rightarrow e^-+$Cu &$<1.6 \times 10^{-8}$ & SREL & 1972 & \cite{PhysRevLett.28.1469}\\
$\mu^-+^{32}$S $\rightarrow e^-+^{32}$S &$<7 \times 10^{-11}$ & SIN & 1981 & \cite{BADERTSCHER1982406} \\
$\mu^-+$Pb $\rightarrow e^-+$Pb & $<4.6 \times 10^{-11}$ & SINDRUM II & 1996 & \cite{PhysRevLett.76.200}\\
$\mu^-+$Ti $\rightarrow e^-+$Ti &$<6.1 \times 10^{-13}$ & SINDRUM II & 1998 & \cite{wintz} \\
$\mu^-+$Au $\rightarrow e^-+$Au & $<7.0 \times 10^{-13}$ & SINDRUM II & 2006 & \cite{SINDRUMII:2006dvw} \\
$\mu^+\rightarrow e^+\gamma$ & $<4.2 \times 10^{-13}$ & MEG at PSI & 2016 & \cite{themegcollaboration2016search}\\
$\mu^+\rightarrow e^+e^-e^+$ & $<1.0 \times 10^{-12}$ & SINDRUM & 1988 & \cite{BELLGARDT19881}\\
\hline
\hline
\end{tabular}
\caption{Existing branching ratio limits on the CLFV processes $\mu \rightarrow e$, $\mu\rightarrow e\gamma$ and $\mu\rightarrow 3e$. All limits correspond to 90\% confidence level.}
\label{tab:existing_rates}
\end{table} 
 
 \begin{table}[h]
\centering
\begin{tabular}{lllll}
\hline
\hline
Process & Limit & Experiment/Lab & Reference \\ 
\hline
$\mu^-+^{27}$Al $\rightarrow e^-+^{27}$Al &$<8 \times 10^{-17}$ & Mu2e at Fermilab & \cite{Mu2e:2014fns}\\
$\mu^-+^{27}$Al $\rightarrow e^-+^{27}$Al &$<7 \times 10^{-15}$ & COMET (Phase I) at J-PARC & \cite{10.3389/fphy.2018.00133} \\
$\mu^-+^{27}$Al $\rightarrow e^-+^{27}$Al &$<7 \times 10^{-17}$ & COMET (Phase II) at J-PARC & \cite{10.3389/fphy.2018.00133} \\
$\mu^+\rightarrow e^+\gamma$ & $<6 \times 10^{-14}$ & MEG II at PSI & \cite{MEGII:2018kmf}\\
$\mu^+\rightarrow e^+e^-e^+$ & $<6 \times 10^{-15}$ & Mu3e at PSI (Phase I) & \cite{10.21468/SciPostPhysProc.5.020}\\
$\mu^+\rightarrow e^+e^-e^+$ & $<4 \times 10^{-16}$ & Mu3e at PSI (Phase II) & \cite{10.21468/SciPostPhysProc.5.020}\\
\hline
\hline
\end{tabular}
\caption{Expected limits on CLFV branching ratios at next-generation experiments. All limits correspond to 90\% confidence level.}
\label{tab:future_rates}
\end{table} 

Depending on the underlying CLFV physics, these three processes may be strongly interconnected or essentially independent. For example, an operator which mediates $\mu\rightarrow e\gamma$ can be embedded in a nucleus (the photon is virtual and exchanged with the nuclear charge) and thereby induce $\mu\rightarrow e$ conversion. On the other hand, mechanisms for $\mu\rightarrow e$ conversion exist which are entirely independent of the photonic CLFV process. 

Particular UV models of CLFV should provide unambiguous predictions for the branching ratios $B(\mu\rightarrow e)$, $B(\mu\rightarrow e\gamma)$, and $B(\mu\rightarrow e)$ in various target nuclei. Therefore if these quantities can be computed with well-understood uncertainties, then measurements of CLFV processes can be used to exclude specific CLFV models. 
Relating the low-energy $\mu\rightarrow e$ conversion process to the underlying UV theory of CLFV, however, is a formidable theoretical challenge due in part to the significant range of energy scales which the problem spans. The situation is illustrated in Figure \ref{fig:eft_tower}. The CLFV physics is associated with an energy scale $\Lambda_{CLFV}$ -- typically well above the electroweak scale -- at which the CLFV couplings (potentially to new degrees of freedom) are order one. In the spirit of effective theory, at energies below $\Lambda_{CLFV}$, we may integrate out the new degrees of freedom, giving rise to a set of effective operators in terms of standard-model degrees of freedom.  

Two major theoretical hurdles arise as one attempts to extract from the standard-model effective theory a prediction for the $\mu\rightarrow e$ conversion branching ratio. At energies $\gtrsim 1$ GeV, the strong force physics is described by quark and gluon degrees of freedom and a perturbative treatment of the theory of quantum chromodynamics (QCD). The conversion experiments, on the other hand, are performed at sufficiently low energies that the quarks hadronize into nucleons; although QCD is still the correct theory in this regime, it is strongly-coupled and essentially intractable. Instead, one must undertake a non-perturbative matching between quark and hadron degrees of freedom that requires input from either experiment or lattice QCD calculations. Fortunately, we are entering an era where precision lattice QCD calculations make it possible to quantify the errors associated with this matching procedure. 

The nucleon-level operators thus obtained must ultimately be embedded in a nucleus, where the form of the response is constrained by the angular momentum, isospin, parity and time-reversal symmetries of the nuclear ground state. This is the second major theoretical hurdle in connecting the UV theory to experiment due to the complex nuclear many-body physics required to evaluate the relevant response functions. In order to construct a many-body wave function for a nucleus like $^{27}$Al in the nuclear shell-model, one relies upon phenomenological interactions which have been tuned to reproduce low-energy nuclear observables such as charge radii and low-lying spectra. The errors associated with these treatments are not rigorously quantified and can certainly be significant. 

These difficulties can be circumvented if one restricts focus to the case of coherent $\mu\rightarrow e$ conversion, in which the only nuclear operator under consideration is the monopole charge operator. The term ``coherent'' refers to the fact that this operator couples to every nucleon in the nucleus, and therefore the amplitude is enhanced by roughly a factor of $A=N+Z$ relative to the incoherent process. Restricting to the coherent case results in a dramatic simplification of the nuclear physics: rather than relying on difficult and imprecise modeling of the nuclear many-body wave function, the coherent amplitude can be computed from an experimentally determined quantity, the scalar nucleon density. Thus, if one connects the coherent nucleon operator to its quark-level counterparts -- relying on lattice QCD to quantify the associated matching errors -- it is possible to complete the chain of effective theories from the scale of $\mu\rightarrow e$ conversion up to the high-energy realm of candidate UV CLFV theories, yielding a prediction for the conversion branching ratio with well-understood uncertainties. This construction was recently completed though next-to-leading order in the quark/hadron matching \cite{2018PhRvC..98a5208B,Cirigliano:2022ekw}, providing a valuable tool for discriminating among BSM theories.

On the other hand, this ``top-down'' approach is not general and can only be applied to exclude those particular UV models which yield a leading coherent response. While it may be helpful to exclude particular candidate UV theories of CLFV,  $\mu\rightarrow e$ conversion has the advantage that the nuclear target can be varied, potentially providing additional complimentary measurements. One would like a way to extract all of the information contained in the branching ratio $B(\mu\rightarrow e)$.  

If one begins from the most general effective theory of CLFV at the level of SM EFT and proceeds down to the nuclear scale then one must reproduce the most general nuclear-level theory. However, it is highly likely that in doing so one has horribly obscured the underlying CLFV physics, convoluting it with the nuclear physics. But it is clear that the CLFV physics does not depend on the choice of nuclear target. As we demonstrate in this work, if we begin by formulating an effective theory directly at the nuclear scale, then we can ensure a factorization between the CLFV physics -- which is independent of the choice of target -- and the nuclear physics.


We demonstrate how this can be achieved with a nuclear-level effective theory of the conversion process. In particular, we identify six CLFV response functions and two interference terms which represent the most general coupling of the leptons to the target nucleus.
 
While the kinematics are ideal for detecting the electron which results from a flavor-violating decay of a trapped muon, the relationship between the observed low-energy process and the underlying CLFV physics is heavily obscured by the intervening nuclear physics. As the nature of possible CLFV operators is almost entirely unconstrained, one should consider the problem in general. The existing literature on $\mu\rightarrow e$ conversion, however, is quite specific, with many studies restricted to the case of so-called ``coherent conversion'' in which the underlying operator carries zero angular momentum and couples equally to protons and neutrons. As we shall discuss in detail, such approaches can be useful for excluding particular UV models of CLFV. In this work, we detail the construction of a nuclear-level effective theory of $\mu\rightarrow e$ conversion, first reported in \cite{rule2021nucleonlevel}. 

The sensitivity of $\mu\rightarrow e$ conversion experiments is achieved in part due to the fact that if the nucleus remains in its ground state then the outgoing electron will have an energy at the endpoint of the spectrum of background electrons emitted in standard model $\mu\rightarrow e + 2\nu$ decays. Restricting to the case without nuclear excitation, which we refer to as \textit{elastic} $\mu\rightarrow e$ conversion, limits the operators that can contribute due to the approximate parity and time-reversal symmetries of the nuclear ground state. We formulate the most general response for elastic $\mu\rightarrow e$ conversion, showing that six response functions and two interference terms can be probed through an ensemble of measurements on different nuclear targets. We also show that there are underlying CLFV operators which are not probed by the elastic process. If CLFV arises only in these couplings, then $\mu\rightarrow e$ experiments will only be sensitive to excited state processes.

\begin{figure}
\centering
\includegraphics[scale=1.0]{eft_tower.png}
\caption{Sketch of the various energy scales that are relevant to $\mu\rightarrow e$ conversion and the effective theories that can be employed in each regime.}
\label{fig:eft_tower}
\end{figure}

If the process occurs without nuclear excitation - the nucleus remains in its ground state - one finds
\begin{equation}
E_e=m_{\mu}-E_{\mu}^\mathrm{bind}-\frac{\vec{q}^{\;2}}{2M_T}
\end{equation}
where $\vec{q}$ is the three-momentum transferred from the muon to the electron, $m_{\mu}$ and $M_T$ are, respectively, the muon and nuclear masses, and $E_{\mu}^\mathrm{bind}$ is the muon's binding energy, defined here as a positive quantity. Restricting to the elastic case maximizes the energy of the outgoing electron, allowing experimentalists to discriminate the CLFV process from the standard-model three-body decay $\mu^-\rightarrow e^-+\nu_{\mu}+\bar{\nu}_e$ as few background electrons are produced near the endpoint energy. 

For example, direction detection of weakly-interacting massive particle (WIMP) dark matter scattering off of nuclei and searches for neutrinoless double beta decay. 

\chapter{Treatment of the Leptonic Fields}
Compared to the nuclear physics, the leptonic physics of $\mu\rightarrow e$ conversion is relatively straightforward and can be computed to high accuracy. In particular, it is known that the muon occupies the $1s$ state of the nuclear Coulomb field. Nuclear charge distributions have been determined from electron scattering experiments and so one may solve (numerically) the Dirac equation in the field of the nuclear Coulomb potential. This procedure yields the wave function and determines the binding energy of the muon. Conservation of energy then dictates the energy of the outgoing electron, and again one may solve the Dirac equation for the electron in the Coulomb field of the nucleus. The downside of this approach is that one is now burdened with numerical solutions for the lepton wave functions which can make subsequent calculations very cumbersome.

In this chapter, we describe in detail how to obtain the Dirac solutions for the muon and electron. We demonstrate the difficulties that one encounters when attempting to use these solutions in a fully general setup and explain the special cases where the precise numerical solutions can be employed with minimal overhead. Finally, based on careful study of the Dirac solutions, we introduce approximate forms which dramatically simply the general formalism and illuminate the underlying physics. 
\section{Solutions of the Dirac Equation for a Spherically Symmetric Potential}
The Dirac equation for a particle of mass $m$ in a spherically symmetric potential $V(r)$ may be written as
\begin{equation}
E\psi=\left[-i\gamma_5\sigma_r\left(\partial_r+\frac{1}{r}-\frac{\gamma_0}{r}K\right)+V(r)+\bar{m}\gamma_0\right]\psi
\end{equation}
with
\begin{equation}
\gamma_5=\left(\begin{array}{cc}
0 & I_2\\
I_2 & 0
\end{array}\right),\hspace*{1cm}
\gamma_0=\left(\begin{array}{cc}
I_2 & 0\\
0 & -I_2
\end{array}\right)
\end{equation}
\begin{equation}
\sigma_r=\left(\begin{array}{cc}
\vec{\sigma}\cdot\hat{r} & 0\\
0 & \vec{\sigma}\cdot\hat{r}
\end{array}\right),\hspace*{1cm}
K=\left(\begin{array}{cc}
\vec{\sigma}\cdot\vec{L}+I_2 & 0\\
0 & -\left(\vec{\sigma}\cdot\vec{L}+I_2\right)
\end{array}\right).
\end{equation}
The mass $\bar{m}$ is the reduced mass of the lepton
\begin{equation}
\bar{m}=\frac{mM_T}{m+M_T},
\end{equation}
where $m$ is the lepton mass and $M_T$ is the nuclear target mass.

The solutions can then be expressed as eigenfunctions of the operators $J^2$, $J_z$ and $K$. Letting the corresponding eigenvalues be represented by $j$, $m$, and $\kappa$, respectively, the solution has the generic form
\begin{equation}
\psi_{m}^{\kappa}(\vec{r})=\left(\begin{array}{c}
\frac{G_\kappa(r)}{r}\;\Omega^\ell_{jm}(\hat{r})\\
i\frac{F_{\kappa}(r)}{r}\;\Omega^{\ell'}_{jm}(\hat{r})
\end{array}\right),
\end{equation}
where $\Omega^{\ell}_{jm}$ is a spinor spherical harmonic
\begin{equation}
\Omega^{\ell}_{jm}(\hat{r})=\sum_{m_{\ell}m_s}\braket{\ell\;m_{\ell}\;\frac{1}{2}\;m_s|j\;m}Y_{\ell m_{\ell}}(\hat{r})\;\xi_{m_s},
\end{equation}
and $\xi_{m_s}$ is a Pauli spinor. The spinor spherical harmonics satisfy
\begin{equation}
\begin{split}
L^2\Omega^{\ell}_{jm}&=\ell(\ell+1)\Omega^{\ell}_{jm}\\
J^2\Omega^{\ell}_{jm}&=j(j+1)\Omega^{\ell}_{jm}\\
J_z\Omega^{\ell}_{jm}&=m\Omega^{\ell}_{jm}\\
\left(\vec{\sigma}\cdot\vec{L}+I_2\right)\Omega^{\ell}_{jm}&=-\kappa\Omega^{\ell}_{jm}
\end{split}
\end{equation}
The solutions are indexed by $\kappa=...,-3,-2,-1,1,2,3,...$ where $j=|\kappa|-\frac{1}{2}$ and 
\begin{equation}
\kappa=\left\{\begin{array}{rl}
-(\ell+1) & \kappa < 0\\
\ell & \kappa > 0
\end{array}\right. .
\end{equation} 
In the case of $\mu\rightarrow e$ conversion, the muon is known to be in the $\kappa=-1$ state. In principle, the electron can be produced in any partial wave limited by the fact that it must couple with the $j=1/2$ muon to the nucleus. The radial wave functions are obtained by solving the coupled equations
\begin{equation}
\begin{split}
\frac{dG}{dr}&=-\frac{\kappa}{r}G+\left(E-V(r)+\bar{m}\right)F\\
\frac{dF}{dr}&=\frac{\kappa}{r}F-\left(E-V(r)-\bar{m}\right)G.
\end{split}
\end{equation} 
The Coulomb potential that we employ is based on a 2-parameter Fermi model of the proton density
\begin{equation}
\rho_p(r)=\frac{\rho_0}{1+e^{(r-c)/\beta}},
\end{equation}
where the parameters $c$ and $\beta$ are fit to electron scattering data, and the normalization constant $\rho_0$ is determined by our convention
\begin{equation}
\int dr\;r^2\rho_p(r) = Z.
\end{equation}
In fact, $\rho_0$ can be determined analytically in terms of the polylogarithm function Li$_n(z)$,
\begin{equation}
\rho_0=-\frac{Z}{2\beta^3\mathrm{Li}_3\left(-\exp\left[c/\beta\right]\right)},
\end{equation}
as can the associated Coulomb potential
\begin{equation}
V_C(r)=-\frac{\alpha Z}{r}\left\{1+\frac{\rho_0\beta^3}{Z}\left[\frac{r}{\beta}\mathrm{Li}_2\left(-e^{(c-r)/\beta}\right)+2\mathrm{Li}_3\left(-e^{(r-c)\beta}\right)\right]\right\}
\label{eq:VCoulomb}
\end{equation}
The values of $c$ and $\beta$ that we employ for the nuclei of interest are given in Table \ref{tab:input}, along with the implied value of the RMS charge radius $\sqrt{\braket{r^2}}$, which can also be expressed analytically for the 2-parameter Fermi model
\begin{equation}
\sqrt{\braket{r^2}}=\sqrt{12}\beta \left[\frac{\mathrm{Li}_5\left(-\exp[c/\beta]\right)}{\mathrm{Li}_3\left(-\exp[c/\beta]\right)}\right]^{1/2}.
\end{equation} 
The Dirac solutions obtained from these potentials for the nuclei $^{27}$Al and $^{48}$Ti are shown for the muon in Figure \ref{fig:muon} and for the electron in Figure \ref{fig:Alelectron} and Figure \ref{fig:Tielectron}, respectively. With this prescription, one may obtain Dirac solutions for the $\kappa=-1$ bound muon, determine the binding energy of this state and hence the energy of the outgoing electron and then solve for electron radial wave functions for any $\kappa$.
\section{The Coherent Case and Its Limitations}
The downside of utilizing the Dirac solutions directly in calculations of the $\mu\rightarrow e$ branching ratio lies in the fact that the lepton current must be integrated against the nuclear current which, given the numerical nature of the lepton solutions, precludes a simplified expression for all but the most trivial operators. In general, the effective interaction Hamiltonian consists of terms of the form
\begin{equation}
H_i= c_i \int d^3 x \;\bar{\psi}_e(\vec{x})\mathcal{O}_L\psi_{\mu}(\vec{x})\bar{\psi}_N(\vec{x})\mathcal{O}_N\psi_N(\vec{x})
\end{equation}
where $c_i$ is a low-energy constant and the operators $\mathcal{O}_L$, $\mathcal{O}_N$ may carry Lorentz indices, in which case they couple to an overall scalar. To some extent, one may factorize the leptonic and nuclear currents by introducing an auxiliary  coordinate $\vec{y}$ and inserting an intermediate delta function
\begin{equation}
\delta(\vec{x}-\vec{y})=\int \frac{d^3q}{(2\pi)^3}e^{i\vec{q}\cdot(\vec{x}-\vec{y})}
\end{equation}
to obtain
\begin{equation}
H_i=c_i\int \frac{d^3q}{(2\pi)^3}\;\int d^3x\;e^{i\vec{q}\cdot\vec{x}}\;\bar{\psi}_e(\vec{x})\mathcal{O}_L\psi_\mu(\vec{x})\int d^3y\;e^{-i\vec{q}\cdot\vec{y}}\;\bar{\psi}_N(\vec{y})\mathcal{O}_N\psi_N(\vec{y})
\end{equation}
In this form, we may perform separate multipole decompositions for the leptons and the nucleons. As a simple example, let us consider a scalar-scalar coupling of the leptons to the nucleons 
\begin{equation}
\mathcal{O}_L=1_L,\;\mathcal{O}_N=1_N.
\end{equation}
Performing the angular integral $d\Omega_q$ yields
\begin{equation}
H=\frac{2}{\pi}c\int_0^{\infty}dq\;q^2 \sum_{L=0}^{\infty}\sum_{M=-L}^L\int d^3x\;j_L(qx)Y^*_{LM}(\hat{x})\;\bar{\psi}_e(\vec{x})\psi_{\mu}(\vec{x})\int d^3y\;j_L(qy)Y_{LM}(\hat{y})\;\bar{\psi}_N(\vec{y})\psi_N(\vec{y})
\label{eq:H_full_multi}
\end{equation}
In practice, the summation over $L$ is truncated by the requirement that $L$ must couple to the total angular momentum of the nuclear ground state. Focusing on the leptonic multipoles, the muon is known to occupy the $\kappa=-1$ state, whereas the electron can be in any state which couples with the muon to total angular momentum $L$. Therefore we label the electron state by $\kappa$, $j$, and $m_e$ to write
\begin{equation}
\begin{split}
&\int d^3x\;j_L(qx)Y^*_{LM}(\hat{x})\bar{\psi}_e(\vec{x})\psi_{\mu}(\vec{x})\\
&=\int d^3x\;j_L(qx)Y^*_{LM}(\hat{x})\frac{1}{x^2}\left(G_\kappa^{(e)}(x)G_{-1}^{\mu}(x)\Omega^{\dag\ell}_{jm_e}(\hat{x})\Omega^0_{\frac{1}{2}m_{\mu}}(\hat{x})-F^{(e)}_\kappa(x)F_{-1}^{(\mu)}(x)\Omega^{\dag\ell'}_{jm_e}(\hat{x})\Omega^1_{\frac{1}{2}m_{\mu}}(\hat{x})\right)
\end{split}
\end{equation}
In principle, the angular integrals can be computed by recoupling the spinor spherical harmonics in terms of a single ordinary spherical harmonic
\begin{equation}
\begin{split}
\Omega^{\dag\ell_1}_{j_1m_1}(\hat{r})\Omega^{\ell_2}_{j_2m_2}(\hat{r})&=\sum_L (-1)^{j_1+m_1+j_2+L+\frac{1}{2}}\left\{\begin{array}{ccc}
\ell_1 & \ell_2 & L\\
j_2 & j_1 & \frac{1}{2}
\end{array}\right\}\\
&\times\sqrt{\frac{(2j_1+1)(2j_2+1)(2\ell_1+1)(2\ell_2+1)}{4\pi(2L+1)}}C^{L0}_{\ell_10\ell_20}C^{LM}_{j_1-m_1j_2m_2}Y_{LM}(\hat{r})
\end{split}
\end{equation}
We can already see that if one wants to retain generic $L$ multipoles while considering all relevant electron partial waves, the calculation quickly becomes quite cumbersome. On the other hand, the lowest multipole $L=0$ is relatively simple to compute. In this case, the only electron state that contributes is $\kappa=-1$, and we find
\begin{equation}
\int d^3x\;j_0(qx)Y^*_{00}(\hat{x})\;\bar{\psi}_e(\vec{x})\psi_{\mu}(\vec{x})=\frac{1}{2\sqrt{\pi}}\int_0^{\infty}dx\;j_0(qx)\;\left(G^{(e)}_{-1}(x)G^{(\mu)}_{-1}(x)-F_{-1}^{(e)}(x)F_{-1}^{(\mu)}(x)\right)\xi^{\dag}_{m_e}\xi_{m_{\mu}}
\end{equation}
The evaluation of the nuclear wave functions also simplifies dramatically for the $L=0$ multipole
\begin{equation}
\int d^3y\;j_0(qy)Y_{0,0}\;\bar{\psi}_N(\vec{y})\psi_N(\vec{y})=\frac{1}{2\sqrt{\pi}}\int_0^{\infty}dy\;y^2\;j_0(qy)\rho_N(y),
\end{equation}
where $\rho_N(r)=\rho_p(r)+\rho_n(r)$ is the isoscalar nuclear density. Conveniently, $\rho_p(r)$ and $\rho_n(r)$ have both been determined experimentally in a wide range of nuclei: the former is determined -- as noted above -- through measurements of elastic electron scattering off of nuclei whereas the latter can be determined, for example, through experiments on pionic atoms. Finally, we may perform the integral over $q$ to obtain a delta function $\delta(x-y)$, allowing us to write
\begin{equation}
H_{L=0}=\frac{c}{4\pi}\int_0^{\infty}dx\;\left(G_{-1}^{(e)}(x)G_{-1}^{(\mu)}(x)-F_{-1}^{(e)}(x)F_{-1}^{(\mu)}(x)\right)\rho_N(x)
\end{equation}
Thus, by restricting to the $L=0$ multipole, we have arrived at a very simple expression for the conversion amplitude, with all of the nuclear physics captured by a measured quantity $\rho_N(x)$. Is this simplification justified?

The underlying nuclear operator is the isoscalar nucleon density
\begin{equation}
\bar{\psi}_N(\vec{x})\psi_N(\vec{x})\rightarrow\hat{\rho}(\vec{x})=\sum_{i=1}^A\delta(\vec{x}-\vec{x}_i)
\end{equation} 
which admits a multipole decomposition
\begin{equation}
\begin{split}
M_{LM}(q)&=\int d^3x\;j_L(qx)Y_{LM}(\hat{x})\;\hat{\rho}(\vec{x})\\
&=\sum_{i=1}^Aj_L(qx_i)Y_{LM}(\hat{x}_i)
\end{split}
\end{equation}
The key observation is that when these operators are evaluated between nuclear ground state wave functions, the $L=0$ multipole is unique in that it sums coherently over every nucleon in the nucleus; operators with $L>0$ can sum only over certain subsets of nucleons. Therefore one expects a relative enhancement of $A$ in the amplitude for the coherent operator compared to incoherent operators. Thus it is justified to retain only the $L=0$ multipole in Eq. \ref{eq:H_full_multi}. It is crucial, however, that the only nuclear operator under consideration is the nucleon density $\hat{\rho}(\vec{x})$. This argument cannot be extended to other nuclear charges or currents which do not give rise to coherent multipole operators and therefore provide no justification for truncating the multipole expansion. Additionally, nuclear matrix elements of such operators cannot be evaluated in terms of the scalar nucleon density $\rho_N(r)$, as they are sensitive to the details of nuclear structure. 


For this reason, studies which utilize the Dirac solutions directly in calculations typically restrict to the so-called ``coherent'' conversion process in which the underlying nuclear operator is both an angular momentum and isospin scalar. The fact that the nuclear physics depends only on a measured quantity is a huge advantage of the restriction to the coherent process; it completely avoids the major theoretical hurdle of embedding the nucleon-level operators in the nucleus.

%Thus, in the coherent case the interaction between the leptons and the nucleus is distilled into the relatively simple overlap integrals
%\begin{equation}
%\begin{split}
%\tau^{(-1)}&=\frac{1}{m_{\mu}^{5/2}}\int dr\;\left(g_{-1}^{(e)}g_{-1}^{(\mu)}-f_{-1}^{(e)}f_{-1}^{(\mu)}\right)\rho_N\\
%\tau^{(+1)}&=\frac{i}{m_{\mu}^{5/2}}\int dr\;\left(f_{+1}^{(e)}g_{-1}^{(\mu)}+g_{+1}^{(e)}f_{-1}^{(\mu)}\right)\rho_N.
%\end{split}
%\end{equation}
%As no angular momentum is transferred and the muon is fixed in the $\kappa=-1$ state, only $\kappa=\pm 1$ states of the electron contribute. Likewise, only the monopole density of the nucleus is sampled. 


One can get away with neglecting higher multipole contributions only because of the coherent enhancement of the isoscalar monopole. The ground state of $^{27}$Al has total angular momentum $J_i=\frac{5}{2}$, and so can couple to operators up to $J=5$. If we were to consider such operators, then we would have to sum over all partial waves of the electron which can contribute. With the angular momentum of the muon fixed at $j=\frac{1}{2}$, this summation includes electron states up to $j=\frac{11}{2}$; that is $\kappa=\pm 1,\pm 2, ...,\pm 6$. While it is straightforward to obtain these solutions numerically, considerable angular momentum algebra . Moreover, one can no longer compute the nuclear physics directly from the scalar nucleon density: higher multipoles depend on the details of nuclear structure. The same is true if one attempts to include spin-dependent or velocity-dependent operators.

As the nature of CLFV physics is as yet entirely unknown -- depending on the particular BSM model, coherent conversion may not be the leading response -- one would like to consider the most general interaction between the leptons and the nuclear target. In order to add these nuclear operators while retaining a simple form for the theory, we introduce approximate forms for the muon and electron wave functions. We now show how we can replace the numerical lepton wave functions by, for the muon, the simplest possible wave function: a constant in the upper component, and for the electron, perhaps the second-simplest wave function: a free Dirac plane wave.
\section{Approximate Treatment of the Outgoing Electron}
\begin{table*}
\centering
 \begin{tabular}{|c|c|c|c|c|c|c|c|c|}
\hline
\hline
 ~Nucleus~ & ~$c$ (fm)~&  ~$\beta$ (fm)~ &~$\sqrt{\langle r^2 \rangle}$~(fm)~& ~$E_\mu^\mathrm{bind}$ (MeV)~  & ~~~$Z_\mathrm{eff}$~~~ &  R & $q$ (MeV)  & $q_\mathrm{eff}$ (MeV)  \\[0.4cm]
\hline
$^{12}_6$C & 2.215 & 0.491 & 2.505 & 0.1000 & 5.7030 & 0.8587 & 105.07 & 108.40 \\
$^{16}_8$O & 2.534 & 0.514 & 2.739 & 0.1775 & 7.4210 & 0.7982 &  105.11 & 109.16 \\
$^{19}_9$F & 2.580 & 0.567 & 2.904 & 0.2242 & 8.2298 & 0.7646 & 105.12 & 109.44 \\
$^{23}_{11}$Na & 2.760 & 0.543 & 2.940 & 0.3337 & 9.8547 & 0.7190 &  105.07 & 110.25 \\
$^{27}_{13}$Al & 3.070 & 0.519 & 3.062 & 0.4630 & 11.3086 & 0.6583 & 104.98 & 110.81 \\
$^{28}_{14}$Si & 3.140 & 0.537 & 3.146 & 0.5346 & 12.0009 & 0.6299 & 104.91 & 111.03 \\
$^{32}_{16}$S & 3.161 & 0.578 & 3.239 &  0.6924 &  13.1839 & 0.5595 & 104.78 &  111.56 \\
$^{40}_{20}$Ca & 3.621 & 0.563 & 3.499 & 1.0585 & 15.6916 & 0.4830 & 104.45 & 112.28 \\
$^{48}_{22}$Ti & 3.843 & 0.588 & 3.693 & 1.2615 & 16.6562 & 0.4340 &  104.28 & 112.43 \\
$^{56}_{26}$Fe & 4.111 & 0.558 & 3.800 & 1.7182 & 18.6028 & 0.3363 & 103.84 & 113.16 \\
$^{63}_{29}$Cu & 4.218 & 0.596 & 3.947 & 2.0885 & 19.8563 & 0.3210 & 103.48 & 113.50 \\
$^{184}_{74}$W & 6.51 & 0.535 & 5.42 & 9.0812 & 33.6287 & 0.0939 & 96.55 & 114.93\\[0.15cm]
 \hline
 \end{tabular}
  \caption{\label{tab:input}Input parameters and output quantities for the muon and electron Dirac solutions discussed in the text.}
\end{table*}
When the $\mu\rightarrow e$ conversion process occurs without nuclear excitation, then conservation of energy requires that the energy of the outgoing electron $E_e$ be given by
\begin{equation}
E_e=m_{\mu}-E^\mathrm{bind}_{\mu}-\frac{\vec{q}^{\;2}}{2M_T},
\end{equation}
where $\vec{q}$ is the three-momentum transferred from the nucleus to the electron, $m_{\mu}$ and $M_T$ are, respectively, the muon and nuclear masses, and $E_{\mu}^\mathrm{bind}$ is the muon's binding energy, defined here as a positive quantity. Working to first order in $m_{\mu}/M_T$ and ignoring smaller quantities in $1/M_T$, we find
\begin{equation}
\vec{q}^{\;2}=\frac{M_T}{m_{\mu}+M_T}\left[\left(m_{\mu}-E_{\mu}^\mathrm{bind}\right)^2-m_e^2\right].
\end{equation}
As shown in Table \ref{tab:input}, the muon binding energy is small compared to its rest mass -- even for very heavy nuclei. Therefore the outgoing electron receives nearly all of the muon's rest mass as kinetic energy and thus is ultra-relativistic. In this limit, the electron's wave function may be reasonably approximated as a free Dirac plane wave
\begin{equation}
\psi_e^{(r)}(\vec{x})=\sqrt{\frac{m_e}{E_e}}e^{i\vec{q}\cdot\vec{x}}u^{(r)}(q),
\end{equation}
where the basis spinor is defined in the convention of Bjorken \& Drell
\begin{equation}
u^{(r)}(q)=\sqrt{\frac{E+m}{2m}}\left(\begin{array}{c}
\xi^{(r)}\\
\frac{\vec{\sigma}\cdot\vec{q}}{E+m}\xi^{(r)}
\end{array}\right)
\end{equation}
The plane wave form implies a partial wave expansion from which we can make the identification
\begin{equation}
\begin{split}
G_{\kappa}(r)&=qrj_{\ell}(qr)\\
F_{\kappa}(r)&=\sqrt{\frac{E-m_e}{E+m_e}}\left\{\begin{array}{cc}
qrj_{\ell-1}(qr), & \kappa>0\\
-qrj_{\ell+1}(qr), &\kappa < 0,
\end{array}\right.
\end{split}
\end{equation}
where in the limit of an ultra-relativistic electron, we may take $\sqrt{(E-m_e)/(E+m_e)}\approx 1$. Although they capture the general qualitative behavior of the outgoing electron, the free Dirac solutions differ significantly from the exact numerical Coulomb solutions as shown in Figures \ref{fig:Alelectron}, \ref{fig:Tielectron}, and \ref{fig:Welectron}.

We can improve the agreement with the numerical solutions by employing the Effective Momentum Approximation (EMA), which retains the plane wave form but attempts to account for the Coulomb distortion of the electron wave function by the nuclear charge. The attractive Coulomb potential produces two physical effects on the electron wave function relative to the free plane wave: first, the wavelength is shortened; second, the probability near the origin (i.e. close to the nuclear charge) is increased. Both of these effects can be captured in a simple manner by replacing the Coulomb potential of Eq. \ref{eq:VCoulomb}, which is computed for a finite charge distribution and therefore rather complex, by a constant potential well whose depth is equated with the average of the nuclear Coulomb potential over the nuclear charge
\begin{equation}
\bar{V}_C\equiv \frac{\int dr \;r^2 \rho(r) V_C(r)}{\int dr \;r^2 \rho(r)}.
\end{equation}
Locally in this potential, the momentum of the electron is the effective momentum
\begin{equation}
\vec{q}_\mathrm{eff}^{\;2}=\frac{M_T}{M_T+m_{\mu}}\left[\left(m_{\mu}+E_{\mu}^\mathrm{bind}-\bar{V}_C\right)^2-m_e^2\right]
\end{equation}

\begin{figure}
\centering
\includegraphics[scale=0.47]{Fig2_Alelectron.pdf}
\caption{The Dirac Coulomb solutions $G(r)$ and $F(r)$ for the highly-relativistic outgoing electron produced in $\mu\rightarrow e$ conversion in $^{27}$Al (green line) are compared to the free solution (orange) and to the free solution evaluated with $q_\mathrm{eff}$ (blue dashed), for low partial waves. The nuclear charge distribution is shown by the shading (arbitrary normalization). The agreement between the Coulomb and free solutions evaluated with $q_\mathrm{eff}$ is quite good.}
\label{fig:Alelectron}
\end{figure}
\begin{figure}
\centering
\includegraphics[scale=0.47]{Fig3_Tielectron.pdf}
\caption{Same as Figure \ref{fig:Alelectron} but for $^{48}$Ti}.
\label{fig:Tielectron}
\end{figure}
\begin{figure}
\centering
\includegraphics[scale=0.47]{Fig4_Welectron.pdf}
\caption{Same as Figure \ref{fig:Alelectron} but for $^{184}$W}.
\label{fig:Welectron}
\end{figure}

As in the free plane wave, the partial waves can be identified as
\begin{equation}
\begin{split}
G_{\kappa}(r)&=q_\mathrm{eff}rj_{\ell}(q_\mathrm{eff}r)\\
F_{\kappa}(r)&=q_\mathrm{eff}r\left\{\begin{array}{cc}
qrj_{\ell-1}(q_\mathrm{eff}r), & \kappa>0\\
-qrj_{\ell+1}(q_\mathrm{eff}r), &\kappa < 0
\end{array}\right.
\end{split}
\label{eq:ema_bessel}
\end{equation}
We see that compared to the free Dirac plane wave, the effective momentum wave functions have their wavelength shifted from $q\rightarrow q_\mathrm{eff}$ and are rescaled by a factor $q_\mathrm{eff}/q$. For an attractive potential $\bar{V}_C < 0$ and therefore $q_\mathrm{eff}>q$. This implies that the effective momentum approximation shortens the wavelength of the plane wave and increases the amplitude. Figures \ref{fig:Alelectron}, \ref{fig:Tielectron}, and \ref{fig:Welectron} show the resulting effective momentum wave functions in various partial waves in the target nuclei $^{27}$Al, $^{48}$Ti, and $^{184}$W, respectively. Visually, we see how well the EMA wave functions match the numerical Coulomb solutions, particularly over the scale of the nuclear density. As the $\mu\rightarrow e$ transition amplitude generically depends on the integral of the CLFV lepton current over a specific nuclear transition density, it is most important that the EMA wave functions agree with the numerical reference solution in regions where the nuclear density is highest. Even in the very heavy nucleus $^{184}$W, the agreement between the EMA wave functions and the Coulomb solutions over the scale of the nucleus is quite good, especially compared to the free Dirac plane wave.

More quantitatively, we can assess the validity of the effective momentum approximation by computing the relative root-mean-square error 
\begin{equation}
\braket{\delta G^2}_\mathrm{RMS}\equiv \left(\frac{\int dr\;\rho(r)\left(G(r)-G_\mathrm{eff}(r)\right)^2}{\int dr\;\rho(r)G(r)^2}\right)^{1/2}
\label{eq:rms_error}
\end{equation}
where $G(r)$ is the numerical Coulomb solution and $G_\mathrm{eff}(r)$ is the effective momentum plane wave solution given by Eq. \ref{eq:ema_bessel}. An analogous definition can be made for $\braket{\delta F^2}_\mathrm{RMS}$. This quantity should provide a reliable estimate of the expected error in generic $\mu\rightarrow e$ transition matrix elements. The results for all of our nuclei of interest are shown in Table \ref{tab:ema_comp}. We see that for all of the light and medium-mass targets of primary interest, the relative errors incurred are at or below the level of 2\% in each partial wave. Even in our fiducial heavy nucleus, $^{184}$W, the effective momentum approximation continues to perform very well, with errors consistently below the 10\% level. Also shown in Table \ref{tab:ema_comp} is the relative RMS error of the uncorrected free Dirac plane wave solution at the physical momentum $q$ relative to the numerical Coulomb solution. We see that even in the lightest targets, the uncorrected plane wave suffers $\sim$ 5\% errors which quickly grow with $A$ to exceed 10\%. In all of the nuclei considered, the EMA reduces the RMS error by roughly a factor of 5-10 across the various partial waves. 

When considering the most general effective theory of $\mu\rightarrow e$ conversion, we must rely on many-body nuclear physics calculations which, even at the present state-of-the-art, do not have rigorous uncertainty quantification, especially for the medium-mass nulcei of primary interest. In our approach, we rely on nuclear shell model wave functions which are obtained from phenomenological interactions that have been tuned to reproduce nuclear charge radii and low-lying spectra. The errors which result when such interactions are used to compute generic nuclear response functions are not well understood. Certainly, errors at least at the level of 10\% would not be unexpected. For this reason, we are able to tolerate the few percent error in the electron wave functions introduced through the EMA. As we shall demonstrate, the dramatic simplifications which this approximation enables are very much worth the associated errors, especially when searches for $\mu\rightarrow e$ conversion are in the initial discovery phase.

\begin{table}
\centering
\begin{tabular}{|c|cccccccc|}
\hline
 & \multicolumn{8}{c|}{\% RMS Error}\\
 \hline
Target & $G_{-1}$ & $F_{-1}$ & $G_{+1}$ & $F_{+1}$ & $G_{-2}$ & $F_{-2}$ & $G_{+2}$ & $F_{+2}$ \\
\hline
$^{12}$C & 0.34\% & 0.44\% & 0.72\% & 0.24\% & 1.05\% & 1.78\% & 	1.32\% & 1.53\% \\
$^{12}$C & 2.22\% & 4.87\% & 5.29\% & 1.99\% & 3.68\% & 5.90\% & 6.32\% & 3.23\% \\
\hline
$^{16}$O & 0.47\% & 0.61\% & 0.95\% & 0.23\% & 0.90\% & 1.60\% & 1.16\% & 1.37\% \\
$^{16}$O & 2.86\% & 5.83\% & 6.24\% & 2.58\% & 4.56\% & 7.39\% & 7.80\% & 4.13\% \\
\hline
$^{19}$F & 0.54\% & 0.72\% & 1.07\% & 0.27\% & 0.78\% & 1.44\% & 1.02\% & 1.22\% \\
$^{19}$F & 3.19\% & 6.13\% & 6.53\% & 2.94\% & 4.89\% & 7.85\% & 8.25\% & 4.46\% \\
\hline
$^{23}$Na & 0.64\% & 0.91\% & 1.27\% & 0.34\% & 0.81\% & 1.48\% & 1.07\% & 1.25\% \\
$^{23}$Na & 3.79\% & 7.29\% & 7.69\% & 3.56\% & 5.85\% & 9.53\% & 9.93\% & 5.43\% \\
\hline
$^{27}$Al & 0.70\% & 1.05\% & 1.42\% & 0.41\% & 0.66\% & 1.27\% & 0.90\% & 1.08\% \\
$^{27}$Al & 4.35\% & 8.02\% & 8.41\% & 4.17\% & 6.60\% & 10.92\% & 11.31\% & 6.18\% \\
\hline
$^{28}$Si & 0.74\% & 1.13\% & 1.50\% & 0.46\% & 0.60\% & 1.17\% & 0.83\% & 0.98\% \\
$^{28}$Si & 4.72\% & 8.31\% & 8.69\% & 4.57\% & 6.91\% & 11.42\% & 11.80\% & 6.50\% \\
\hline
$^{32}$S & 0.84\% & 1.33\% & 1.70\% & 0.57\% & 0.61\% & 1.10\% & 0.83\% & 0.91\% \\
$^{32}$S & 5.52\% & 9.11\% & 9.47\% & 5.39\% & 7.63\% & 12.51\% & 12.88\% & 7.24\% \\
\hline
$^{40}$Ca & 0.95\% & 1.51\% & 1.87\% & 0.76\% & 0.60\% & 0.85\% & 0.84\% & 0.64\% \\
$^{40}$Ca & 7.24\% & 9.98\% & 10.33\% & 7.19\% & 8.69\% & 14.46\% & 14.82\% & 8.31\% \\
\hline
$^{48}$Ti & 1.04\% & 1.57\% & 1.92\% & 0.91\% & 0.76\% & 0.87\% &  1.02\% & 0.61\% \\
$^{48}$Ti & 8.47\% & 10.13\% & 10.46\% & 8.47\% & 8.99\% & 14.91\% &  15.27\% & 8.63\% \\
\hline
$^{56}$Fe & 1.23\% & 1.73\% & 2.07\% & 1.16\% & 0.93\% & 1.05\% & 1.30\% & 0.67\% \\
$^{56}$Fe & 10.54\% & 11.17\% & 11.49\% & 10.57\% & 10.08\% & 16.90\% & 17.25\% & 9.74\% \\
\hline
$^{63}$Cu & 1.46\% & 1.89\% & 2.22\% & 1.42\% & 1.16\% & 1.30\% & 1.60\% & 0.85\% \\
$^{63}$Cu & 12.35\% & 11.95\% & 12.24\% & 12.39\% & 10.85\% & 17.87\% & 18.21\% & 10.53\% \\
\hline
$^{184}$W & 6.64\% & 4.39\% & 4.38\% & 6.67\% & 2.60\% & 3.43\% & 3.76\% & 2.40\% \\
$^{184}$W & 47.79\% & 28.17\% & 28.13\% & 47.91\% & 25.77\% & 26.60\% & 26.86\% & 25.76\% \\
\hline
\end{tabular}
\caption{Relative root-mean-square error $\braket{\delta G^2}_\mathrm{RMS}$, $\braket{\delta F^2}_\mathrm{RMS}$ between the numerical Coulomb solutions $G$, $F$ and the free Dirac plane wave solutions $G_q$, $F_q$. For each target, the top line corresponds to the effective momentum plane wave solutions with momentum $q_\mathrm{eff}$ whereas the bottom line corresponds to the plane wave solutions with the physical momentum $q$.}
\label{tab:ema_comp}
\end{table}




\section{Approximate Treatment of the Bound Muon}
\begin{figure}
\centering
\includegraphics[scale=0.48]{Fig1_Muon.pdf}
\caption{Lower panels: muon $\kappa=-1$ bound state Dirac solutions $G(r)/r$ (orange line) and $F(r)/r$ (green) are shown for $^{27}$Al (left) and $^{48}$Ti (right), along with the Schroedinger solutions (blue dashed). These solutions are computed for extended nuclear charges, using the parameterizations of Table \ref{tab:input}, and are properly normalized. Upper panels: the $G(r)$ (orange), $F(r)$ (green), and Schroedinger (red) solutions; also shown are the volume-weighted charge distributions $r^2\rho(r)j_0(qr)$ and $r^2\rho(r)j_2(qr)$ (shaded). The overall normalization (but not the relative normalization) of the two densities is arbitrary. As the muon wave functions varies slowly over the nuclear, it is appropriate to use an average value: the black dotted line is the value obtained by averaging over $r^2\rho(r)j_0(qr)$.}
\label{fig:muon}
\end{figure}

%\begin{table}
%\centering
%\begin{tabular}{|c|c|c|c|c|c|c|c|c|}
%\hline
% & \multicolumn{8}{c|}{\% Variation}\\
% \hline
%Target & $\braket{g_{-1}}$ & $\braket{f_{-1}}$ & $\braket{g_{+1}}$ & $\braket{f_{+1}}$ & $\braket{g_{-2}}$ & $\braket{f_{-2}}$ & $\braket{g_{+2}}$ & $\braket{f_{+2}}$ \\
%\hline
%$^{12}$C & -0.48\% & -0.92\% & -1.38\% & -0.007\% & 0.28\% & 0.29\% & -0.17\% & 0.77\% \\
%$^{16}$O & -0.51\% & -1.10\% & -1.55\% & -0.05\% & 0.15\% & 0.11\% & -0.34\% & 0.63\% \\
%$^{19}$F & -0.51\% & -1.17\% & -1.62\% & -0.05\% & 0.003\% & -0.03\% & -0.48\% & 0.48\% \\
%$^{23}$Na & -0.56\% & -1.34\% & -1.79\% & -0.10\% & 0.03\% & -0.04\% & -0.49\% & 0.51\% \\
%$^{27}$Al & -0.51\% & -1.45\% & 1.90\% & -0.06\% & -0.12\% & -0.27\% & -0.72\% & 0.35\% \\
%$^{28}$Si & -0.48\% & -1.50\% & -1.95\% & -0.03\% & -0.23\% & -0.40\% & -0.84\% & 0.24\% \\
%$^{32}$S & -0.46\% & -1.63\% & -2.08\% & -0.02\% & -0.36\% & -0.55\% & -0.99\% & 0.11\% \\
%$^{40}$Ca & -0.16\% & -1.68\% & -2.12\% & 0.29\% & -0.73\% & -1.06\% & -1.49\% & -0.27\% \\
%$^{48}$Ti & 0.15\% & -1.61\% & -2.05\% & 0.59\% & -0.96\% & -1.36\% &  -1.79\% & -0.51\% \\
%$^{56}$Fe & 0.49\% & -1.64\% & -2.08\% & 0.93\% & -1.17\% & -1.68\% & -2.11\% & -0.71\% \\
%$^{63}$Cu & 0.85\% & -1.63\% & -2.07\% & 1.29\% & -1.37\% & -1.96\% & -2.39\% & -0.92\% \\
%$^{184}$W & 30.18\% & 2.58\% & 2.14\% & 30.63\% & -0.11\% & -2.95\% & -3.37\% & 0.30\% \\
%\hline
%\end{tabular}
%\caption{Relative error in the averages $\braket{g_{\kappa}}\equiv\int dr\;r^2\rho(r)g_{\kappa}(r)$ and $\braket{f_{\kappa}}$ incurred by replacing the numerical Coulomb solution with the corresponding effective momentum approximation.}
%\label{tab:ema_comp}
%\end{table}





\begin{figure}
\centering
\subfloat[]{
\includegraphics[scale=0.53]{al27_rho_nsm_vs_2pf.png}
}
\subfloat[]{
\includegraphics[scale=0.52]{ti48_rho_nsm_vs_2pf.png}
}
\caption{Comparison of the proton density $r^2\rho_p(r)$ obtained from the nuclear shell model (blue curve) and the 2-parameter Fermi function fit to elastic electron scattering data (orange curve) in the target nuclei $^{27}$Al (left) and $^{48}$Ti (right). The Fermi function parameters $c$ and $\beta$ are given in Table \ref{tab:input}. For both nuclei, the nuclear oscillator parameter $b$ has been tuned to reproduce the measured RMS charge radius. The particular shell model interactions employed were Brown-Wildenthal and KB3P for $^{27}$Al and $^{48}$Ti, respectively.}
\end{figure}

Muons which are trapped by the target quickly de-excite to the $1s$ $(\kappa=-1)$ orbital of the nuclear Coulomb field. As discussed above, it is possible to obtain a highly-accurate wave function for the bound muon by numerical solution of the Dirac equation. Despite the complexity of these solutions, the physics of the captured muon can be understood rather simply. 

First, for light- to medium-mass nuclei, the muon is non-relativistic. Figure \ref{fig:muon} shows the upper and lower components of the muon's Dirac wave function, as well as the corresponding Schrodinger solution in the nuclei $^{27}$Al and $^{48}$Ti. For both targets, the Schrodinger solution is an excellent approximation to the upper component of the Dirac solution. Furthermore, we can appraise the relative importance of the muon's lower component by evaluating the ratio $F(r)/G(r)$. As the behavior of the muon wave functions is most important over the extent of the nucleus, a natural radius at which to measure this ratio is the radius $R_N$ at which the nuclear density $r^2\rho(r)$ peaks. Table \ref{tab:f_g_ratio} shows the resulting ratios $|F(R_N)/G(R_N)|$ evaluated in several nuclei of interest ranging from the very light $^{12}$C to the very heavy $^{184}$W. We see that even in the nucleus $^{27}$Al, the lower component amounts to roughly a $3\%$ correction to the $\mu\rightarrow e$ conversion amplitude. For sufficiently heavy nuclei, relativistic effects exceed the 10\% level. For now, we shall neglect entirely the lower component of the muon wave function. In Section \ref{sec:muon_lower}, we demonstrate that the muon's lower component can be generated by the action of the muon velocity operator on the upper component of the wave function. 


\begin{table}
\centering
\begin{tabular}{|c|c|c|c|c|c|c|c|c|}
\hline
Target & $\big|\frac{F(R_0)}{G(R_0)}\big|$ & $\Delta\braket{M_0}$ & Target & $\big|\frac{F(R_0)}{G(R_0)}\big|$ & $\Delta\braket{M_0}$ & Target & $\big|\frac{F(R_0)}{G(R_0)}\big|$ & $\Delta\braket{M_0}$ \\
\hline
$^{12}$C & 0.013 & 0.007 & $^{27}$Al & 0.030 & 0.019 & $^{48}$Ti & 0.049 & 0.039 \\
$^{16}$O & 0.018 & 0.010 & $^{28}$Si & 0.032 & 0.021 & $^{56}$Fe & 0.057 & 0.048 \\
$^{19}$F & 0.020 & 0.012 & $^{32}$S & 0.036 & 0.024 & $^{63}$Cu & 0.062 & 0.056\\
$^{23}$Na & 0.025 & 0.015 & $^{40}$Ca & 0.045 & 0.033 & $^{184}$W & 0.13 & 0.26\\
\hline
\end{tabular}
\caption{Ratio of the lower $F(r)$ to upper $G(r)$ components of the muon's Dirac wave function evaluated at the radius $R_0$ where the nuclear density  $r^2\rho(r)$ peaks.}
\label{tab:f_g_ratio}
\end{table}

Having dispensed with the lower component of the muon's Dirac wave function, we now aim to replace the upper component by an approximate form. Focusing on $^{27}$Al, we note that the muonic Bohr radius $a_\mu^0\approx 19.7$ fm is large compared to either the nuclear radius $r_N^\mathrm{rms}=\sqrt{\braket{r^2}}\approx 3.1$ fm or the scale over which the outgoing electron wave function varies, which is given by the first zero of the Bessel function $j_0(qr)$, $r_e\sim\frac{\pi}{q}\sim\frac{\pi}{m_{\mu}}\sim 5.9$ fm. Adopting the EMA plane wave form for the electron, then the muon wave function will be folded with ground state matrix elements of local nuclear operators as
\begin{equation}
\int d^3r\;e^{-i\vec{q}_\mathrm{eff}\cdot\vec{r}}\phi^{\mu}_{1s}(\vec{r})\braket{g.s.|\sum_{i=1}^A\mathcal{O}(i)\delta(\vec{r}-\vec{r}_i)|g.s.}
\end{equation}
It is instructive to consider the case of a point-like nucleus. In this limit, two effects emerge: first, the Coulomb potential of the nucleus is that of a point-like charge, and the resulting wave function of the muon has a known analytic form. Second, as the nuclear density is concentrated entirely at the origin, the only relevant part of the muon wave function is the value at $\vec{r}=0$.  Explicitly
\begin{equation}
\begin{split}
&\int d^3r\;e^{-i\vec{q}_\mathrm{eff}\cdot\vec{r}}\phi^{\mu}_{1s}(\vec{r})\braket{g.s.|\sum_{i=1}^A\mathcal{O}(i)\delta(\vec{r}-\vec{r}_i)|g.s.}\rightarrow \phi^{Z}_{1s}(\vec{0})\braket{g.s.|\sum_{i=1}^A\mathcal{O}(i)|g.s.},
\end{split}
\end{equation}
where it is known from the point-like Schrodinger solution that
\begin{equation}
\phi_{1s}^Z(\vec{0})=\frac{1}{\sqrt{\pi}}\left[\frac{Z\alpha\mu c}{\hbar}\right]^{3/2}.
\end{equation}
Returning to the case of an extended nuclear charge distribution, as the upper component of the muon wave function varies slowly over the extent of the nuclear density, we may replace the complicated numerical solution by a constant value
\begin{equation}
\int d^3r\;e^{-i\vec{q}_\mathrm{eff}\cdot\vec{r}}\phi^{\mu}_{1s}(\vec{r})\braket{g.s.|\sum_{i=1}^A\mathcal{O}(i)\delta(\vec{r}-\vec{r}_i)|g.s.}=\phi_{1s}^\mathrm{avg}\int d^3r\;e^{-i\vec{q}_\mathrm{eff}\cdot\vec{r}}\braket{g.s.|\sum_{i=1}^A\mathcal{O}(i)\delta(\vec{r}-\vec{r}_i)|g.s.}
\end{equation}
Of course, there are many inequivalent but physically reasonable ways to define the average value $\phi_{1s}^\mathrm{avg}$. As the isoscalar monopole operator 
\begin{equation}
M_{00;0}(q)=\sum_{i=1}^A j_0(qr_i)Y_{00}(\hat{r}_i)
\end{equation}
is expected to dominate the elastic $\mu\rightarrow e$ transition amplitude, we define the average muon value as 
\begin{equation}
\begin{split}
\phi_{1s}^\mathrm{avg}&\equiv \frac{\braket{g.s.|M_{00;0}(q_\mathrm{eff})\phi^{\mu}_{1s}(\vec{r}_i)|g.s.}}{\braket{g.s.|M_{00;0}(q_\mathrm{eff})|g.s.}}\\
&=\frac{1}{\sqrt{4\pi}}\frac{\int dr\;r\rho(r)j_0(q_\mathrm{eff}r)G(r)}{\int dr\;r^2\rho(r)j_0(q_\mathrm{eff}r)},
\end{split}
\end{equation}
so that the $M_{00;0}$ transition amplitude is exactly reproduced. In analogy with the point-like limit, we can parameterize the average muon value in terms of the point-like Schrodinger solution with an effective nuclear charge $Z_\mathrm{eff}$
\begin{equation}
\phi_{1s}^\mathrm{avg}=\phi^{Z_\mathrm{eff}}_{1s}(\vec{0})=\frac{1}{\sqrt{\pi}}\left[\frac{Z_\mathrm{eff}\alpha\mu c}{\hbar}\right]^{3/2}.
\end{equation}
An alternative parameterization that is common in the muon capture literature (see e.g. \cite{1975mpwi.conf..114W}) is to write the average wave function in terms of a reduction factor $R<1$ defined by
\begin{equation}
|\phi^\mathrm{avg}_{1s}(\vec{0})|^2\equiv R |\phi^Z_{1s}(\vec{0})|^2,
\end{equation}
from which it follows that $R$ is given by the cube of the ratio of effective charge to the physical charge:
\begin{equation}
R=\left(\frac{Z_\mathrm{eff}}{Z}\right)^3.
\end{equation}
The values of $R$ and $Z_\mathrm{eff}$ for our nuclear targets of interest are shown in Table \ref{tab:input}. It is clear from these results that the finite nuclear size is crucial to the physics of $\mu\rightarrow e$ conversion; even for the relatively light nucleus $^{27}$Al (with a charge radius $\sqrt{\braket{r^2}}=3.062$ fm) the coherent conversion rate is reduced by $\approx 65$\% compared to the point-like nucleus result.

As discussed above, we have chosen to define the average muon value in order to exactly reproduce the matrix element of the isoscalar monopole operator. Therefore, errors will only be incurred when this constant value is applied in the calculation of other transition amplitudes. For example, consider the isovector monopole operator 
\begin{equation}
M_{00;1}(q)=\sum_{i=1}^A j_0(qr_i)Y_{00}(\hat{r}_i)\tau_3(i),
\end{equation}
which depends on the isovector nuclear density $\rho(p)-\rho(n)$. In $^{27}$Al, the isovector density is approximately that of a single $1d_{5/2}$ proton hole in a $^{28}$Si core. We see from Table \ref{tab:average1} that one incurs an error of $\approx 2$\% in the $\mu\rightarrow e$ decay rate by using the value of $R$ ($Z_\mathrm{eff}$) defined for the operator $M_{00;0}$ to compute matrix elements of $M_{00;1}$.

By a similar token, the ground state of $^{27}$Al has a total angular momentum $J=5/2$, and therefore the total nuclear charge operator contains not just the monopole but also $J=2$ and $J=4$ contributions as well. Table \ref{tab:average1} demonstrates that the inclusion of these higher spin operators only affects the computed value of $R$ (and hence the $\mu\rightarrow e$ rate) by $\approx 0.01$\%. We also consider the error incurred in the calculation of the spin-dependent operators $\Sigma'$ and $\Sigma''$, which correspond, respectively, to the transverse and longitudinal components of the nuclear spin current. In $^{27}$Al, the typical size of these errors is $\lesssim 3$\%.



\begin{table}
\centering
 \begin{tabular}{|c|c|c|c|c|c|c|}
 \hline
 \rule{0cm}{0.4cm}
 &  \multicolumn{5}{c}{~~~~~~~$R$ and \% variation}  &\\[0.15cm]
\hline
\rule{0cm}{0.5cm}
 target & $W_{M_0}^{00}$ & $W_M^{00}$ & $W_{M_0}^{pp}$ & $W_M^{pp}$ & $W_{M_0}^{11}$ & $W_M^{11}$ \\[0.15cm]
 \hline
 \rule{0cm}{0.4cm}
 $^{27}$Al & 0.6566 & 0.6565 & 0.6570 & 0.6569  & 0.6434 & 0.6417  \\
  & 0.00\% & -0.01\% & 0.07\% & 0.06\% &  -2.01\% & -2.26\% \\
 \rule{0cm}{0.4cm}
  $^{63}$Cu & 0.3204 & 0.3204 & 0.3200 & 0.3199 & 0.3287 & 0.3281  \\
  & 0.00\% & -0.01\% & -0.15\% & -0.16\%   & 2.58\% & 2.38\% \\ [.15cm]
 \hline
\rule{0cm}{0.5cm}
 target & $W_{\Sigma^\prime_1}^{00}$ & $W_{\Sigma^\prime}^{00}$ & $W_{\Sigma^\prime_1}^{pp}$ & $W_{\Sigma^\prime}^{pp}$ &  $W_{\Sigma^\prime_1}^{11}$ & $W_{\Sigma^\prime}^{11}$ \\[0.15cm]
 \hline
 \rule{0cm}{0.4cm}
 $^{27}$Al & 0.6475 & 0.6463 & 0.6513 & 0.6495 & 0.6561 & 0.6533 \\
  & -1.38\% & -1.56\% & -0.81\% & -1.08\% &  -0.07\% & -0.49\% \\
 \rule{0cm}{0.4cm}
  $^{63}$Cu & 0.3689 & 0.3141 & 0.3673 & 0.3085 &  0.3655 &  0.3026 \\
  & 15.11\% & -1.99\% & 14.62\% & -3.72\% & 14.05\% &  -5.58\% \\ [.15cm]
  \hline
\rule{0cm}{0.5cm}
 target & $W_{\Sigma^{\prime \prime}_1}^{00}$ & $W_{\Sigma^{\prime \prime}}^{00}$ & $W_{\Sigma^{\prime \prime}_1}^{pp}$ & $W_{\Sigma^{\prime \prime}}^{pp}$ & $W_{\Sigma^{\prime \prime}_1}^{11}$ & $W_{\Sigma^{\prime \prime}}^{11}$ \\[0.15cm]
 \hline
 \rule{0cm}{0.4cm}
 $^{27}$Al & 0.6345 & 0.6339 & 0.6338 & 0.6332 &  0.6331 &  0.6323 \\
  & -3.36\% & -3.45\% & -3.46\% & -3.56\% & -3.58\% & -3.69\% \\
 \rule{0cm}{0.4cm}
  $^{63}$Cu & 0.3275 & 0.2951 & 0.3228 & 0.2921 & 0.3182 & 0.2891  \\
  & 2.20\% & -7.92\% & 0.74\% & -8.85\% & -0.71\% & -9.77\% \\ [.15cm]
  \hline
 \end{tabular}
  \caption{ \label{tab:average1} The exact transition probability containing the muon wave function $G(r)/r$ has been computed for the odd-proton targets $^{27}$Al and $^{63}$Cu, for the charge $M$,
  transverse spin $\Sigma^\prime$, and longitudinal spin $\Sigma^{\prime \prime}$ operators, and for isoscalar (00), isovector (11), and proton (pp) couplings.
  The probability is then re-expressed with an effective constant Coulomb density proportional to the point-nucleus Schr\"{o}dinger density, $R |\phi_{1s}^Z(\vec{0})|^2$, where $R$ is a parameter adjusted to 
  reproduce the exact result.  For cases in which the operator appear with a subscript 0, e.g., $M_0$, this exercise was performed
  by retaining only the leading multipole ($J_0$=0 for $M$ and $J_0$=1 for $\Sigma^\prime$ and $\Sigma^{\prime \prime}$).  Otherwise, is was done using the full
  structure function.  The table shows the weak dependence of $R$ on isospin, choice of operator, and structure function treatment (full, or only leading multipole).  Finally, we take $R$
  from the isoscalar monopole charge operator $M_0$ as a standard value, and express the other values in terms of their percentage deviation from this value (second rows).   The approximate constancy
  of $R$ can be exploited to write the NRET $\mu \rightarrow e$ conversion rate in a particularly simple form.}
\end{table}


\begin{table}
\centering
 \begin{tabular}{|c|c|c|c|c|c|c|}
 \hline
 \rule{0cm}{0.4cm}
 &  \multicolumn{5}{c}{~~~~~~~${\langle f \rangle/ \langle g \rangle}$ and \% variation}  &\\[0.15cm]
\hline
\rule{0cm}{0.4cm}
 target & $W_{M_0}^{00}$ & $W_M^{00}$ & $W_{M_0}^{pp}$ & $W_M^{pp}$ & $W_{M_0}^{11}$ & $W_M^{11}$ \\[0.15cm]
 \hline
 \rule{0cm}{0.4cm}
 $^{27}$Al &-0.0265 &-0.0265 &-0.0264 &-0.0264 &-0.0288 & -0.0290 \\
  & 0.00\% & 0.02\% &- 0.28\% & -0.24\% & 8.65\% & 9.59\% \\
 \rule{0cm}{0.4cm}
  $^{63}$Cu & -0.0481 & -0.0481 & -0.0482 & -0.0482 & -0.0453 & -0.0455 \\
  & 0.00\% & 0.01\% & 0.35\% & 0.35\% & -5.70\% & -5.39\% \\ [.15cm]
 \hline
\rule{0cm}{0.5cm}
 target & $W_{\Sigma^\prime_1}^{00}$ & $W_{\Sigma^\prime}^{00}$ & $W_{\Sigma^\prime_1}^{pp}$ & $W_{\Sigma^\prime}^{pp}$ & $W_{\Sigma^\prime_1}^{11}$ & $W_{\Sigma^\prime}^{11}$ \\[0.15cm]
 \hline
 \rule{0cm}{0.4cm}
 $^{27}$Al & -0.0286 & -0.0287 & -0.0276 & -0.0279 & -0.0265 & -0.0269  \\
  & 7.81\% & 8.44\% & 4.31\% & 5.33\% & -0.14\% & 1.54\% \\
 \rule{0cm}{0.4cm}
  $^{63}$Cu & -0.0325 & -0.0447 & -0.0333 & -0.0465 & -0.0342 & -0.0485 \\
  & -32.32\% & -6.94\% & -30.74\% & -3.25\% & -28.89\% & 0.82\% \\ [.15cm]
  \hline
\rule{0cm}{0.5cm}
 target & $W_{\Sigma^{\prime \prime}_1}^{00}$ & $W_{\Sigma^{\prime \prime}}^{00}$ & $W_{\Sigma^{\prime \prime}_1}^{pp}$ & $W_{\Sigma^{\prime \prime}}^{pp}$ & $W_{\Sigma^{\prime \prime}_1}^{11}$ & $W_{\Sigma^{\prime \prime}}^{11}$ \\[0.15cm]
 \hline
 \rule{0cm}{0.4cm}
 $^{27}$Al & -0.0304 & -0.0305 & -0.0302 & -0.0303 &  -0.0300 &  -0.301 \\
  & 14.62\% & 14.94\% & 13.95\% & 14.35\% & 13.24\% & 13.72\% \\
 \rule{0cm}{0.4cm}
  $^{63}$Cu & -0.0418 & -0.0503 & -0.0437 & -0.0518 & -0.0457 &  -0.0532 \\
  & -13.14\% & 4.63\% & -9.07\% & 7.65\% &  -4.95\% & 10.68\% \\ [.15cm]
  \hline
 \end{tabular}
  \caption{ \label{tab:average2} As in Table \ref{tab:average1}, except that the structure-function-averaged quantity calculated is the ratio of the muon lower to upper components $\langle f \rangle/\langle g \rangle$  See text..}
\end{table}
\section{Comparison to Other Works}
Not long after Feinberg's original paper on $\mu\rightarrow e\gamma$, Weinberg \& Feinberg considered the analogous process in a muonic atom wherein the photon is virtual and exchanged with the nuclear charge. This was the first calculation of $\mu\rightarrow e$ conversion in nuclei. In their formulation, the outgoing electron is treated as a free Dirac plane wave. In analogy with earlier work on the process of standard muon capture \cite{RevModPhys.21.133}, the bound muon is taken to be a constant parameterized by the effective charge $Z_\mathrm{eff}$. The definition of $Z_\mathrm{eff}$ employed first by Wheeler and later by Primakoff in the study of standard muon capture
\begin{equation}
Z_\mathrm{eff}^4\equiv \frac{1}{4\pi}Z^4 \int dr\;r^2\rho_p(r)|\phi_{1s}^{\mu}(r)|^2
\end{equation}
was adopted by Weinberg \& Feinberg and 

Two important distinctions should be noted. First, $Z_\mathrm{eff}'$ is computed by averaging the muon \textit{probability} over the nuclear density whereas $Z_\mathrm{eff}$ is computed by averaging the muon amplitude $G(r)$ over the isoscalar monopole transition density $j_0(q_\mathrm{eff}r)\rho(r)$. The later quantity has the advantage that it exactly reproduces the leading coherent operator contributing to elastic $\mu\rightarrow e$ conversion. On the other hand, $Z'_\mathrm{eff}$ is more appropriate for the inclusive process of standard muon capture in which no single transition is expected to dominate. In particular, following Primakoff, the rate for standard muon capture obeys
\begin{equation}
\begin{split}
\Gamma_\mathrm{capture}&\sim\sum_fF(E_f)|\braket{i|\Omega\phi^{\mu}_{1s}(r)|f}|^2\\
&\sim F(\bar{E}_f)\sum_f|\braket{i|\Omega\phi_{1s}^{\mu}(r)|f}|^2\\
&\sim F(\bar{E}_f)\braket{i|\Omega^{\dag}\Omega|\phi_{1s}^{\mu}(r)|^2|i}\\
&\sim \frac{\int d^3r\;\rho(r)|\phi_{1s}^{\mu}(r)|^2}{\int d^3r\;\rho(r)}F(\bar{E}_f)\braket{i|\Omega^{\dag}\Omega|i},
\end{split}
\end{equation}
where $\Omega$ is a nuclear operator, and after replacing the final-state phase space factor $F(E_f)$ by an average value the sum over final states has been completed by the closure approximation. Therefore, in the case of muon capture, the appropriate average value of the muon wave function is in fact the ground state average of the muon probability.

The second distinction is that $\braket{|\psi_{\mu}|^2}_{\rho}$ is proportional to $Z_\mathrm{eff}^{'4}$ whereas clearly the Schrodinger solution implies that $|\phi_{1s}^{Z_\mathrm{eff}}(\vec{0})|^3\sim Z_{\mathrm{eff}}^3$. Where does the extra power of the effective charge come from? The answer lies in the origin of this quantity in the standard muon capture literature wherein the process is inclusive 

\begin{table}
\centering
\begin{tabular}{llllcc}
\hline
\hline
Author(s) & Year [Ref] & Nuclei & Operators & $\psi_e$ & $\psi_{\mu}$\\
\hline
\hline
Weinberg \& Feinberg$^{\dag}$& 1959 \cite{PhysRevLett.3.111} & Many & $M_{0;p}$ & Plane wave & $\braket{|\psi_{\mu}|^2}_\rho$ \\
Marciano \& Sanda & 1977 \cite{PhysRevLett.38.1512} & Many &$M_{0;\tau}$ & Plane wave & $\braket{|\psi_{\mu}|^2}_\rho$ \\
Shanker & 1979 \cite{PhysRevD.20.1608} & Many & $M_{0;\tau}$ & Dirac & Dirac \\
 Kosmas \& Vergados$^{\dag}$ & 1990 \cite{KOSMAS1990641} & Many & $M_{0;\tau}$ & Plane wave & $\braket{|\psi_{\mu}|^2}_\rho$ \\
Chiang et al.$^{\dag}$ & 1993 \cite{CHIANG1993526} & Many & $M_{0;\tau}$ & Plane wave & $\braket{|\psi_{\mu}|^2}_\rho$ \\
Kosmas et al.$^{\dag}$ & 1993 \cite{KOSMAS1994637} & $^{48}$Ti & $M_{0;\tau}$ & Plane wave & $\braket{|\psi_{\mu}|^2}_\rho$ \\
 Czarnecki, Marciano, \& Melnikov & 1998 \cite{czarnecki1998} & $^{27}$Al, $^{48}$Ti, $^{208}$Pb & $M_{0;\tau}$ & Dirac & Dirac \\
Siiskonen, Suhonen, \& Kosmas$^{\dag}$ & 2000 \cite{PhysRevC.62.035502} & $^{27}$Al, $^{48}$Ti & $M_{J;\tau}\;\Sigma'_{J;\tau}\;\Sigma''_{J;\tau}$ & Plane wave & $\braket{|\psi_{\mu}|^2}_\rho$ \\
Kosmas$^{\dag}$ & 2001 \cite{KOSMAS2001443} & $^{48}$Ti, $^{208}$Pb & $M_{J;\tau}\;\Sigma'_{J;\tau}\;\Sigma''_{J;\tau}$ & Plane wave & $\braket{|\psi_{\mu}|^2}_\rho$ \\
Kitano, Koike, \& Okada & 2002 \cite{PhysRevD.66.096002} & Many & $M_{0;\tau}$ & Dirac & Dirac \\
Kosmas & 2003 \cite{kosmas2003} & Many & $M_{0;\tau}$ & Plane wave & Dirac \\
Cirigliano et al. & 2009 \cite{cirigliano2009} & Many & $M_{0;\tau}$ & Dirac & Dirac \\
Crivellin et al. & 2017 \cite{crivellin2017} & $^{27}$Al, $^{197}$Au & $M_{0;\tau}$ & Dirac & Dirac \\
Bartolotta \& Ramsey-Musolf & 2018 \cite{2018PhRvC..98a5208B} & $^{27}$Al & $M_{0;\tau}$ & Dirac & Dirac \\
Cirigliano, Davidson, \& Kuno & 2018 \cite{cirigliano2017} & $^{27}$Al & $M_{0;\tau}\;\Sigma'_{J;\tau}\;\Sigma''_{J;\tau}$ & Plane wave & $|\phi_{1s}^{Z}(\vec{0})|^2$ \\
Davidson, Kuno, \& Saporta & 2018 \cite{davidson2018} & $^{27}$Al, Ti & $M_{0;\tau}\;\Sigma'_{J;\tau}\;\Sigma''_{J;\tau}$ & Plane wave & $|\phi_{1s}^{Z}(\vec{0})|^2$\\
Civitarese \& Tarutina$^{\dag}$ & 2019 \cite{PhysRevC.99.065504} & $^{208}$Pb & $M_{0;\tau}$ & Plane wave & $|G(R_N)|^2$ \\
 Heeck, Szafron, \& Uesaka & 2022 \cite{2022arXiv220300702H} & Many & $M_{0;\tau}$ & Dirac & Dirac \\
Cirigliano et al. & 2022 \cite{Cirigliano:2022ekw} & $^{27}$Al,$^{48}$Ti,$^{197}$Au,$^{208}$Pb & $M_{0;\tau}$& Dirac & Dirac \\
\hline
\hline
\end{tabular}
\caption{An incomplete survey of elastic $\mu\rightarrow e$ conversion studies including the nuclear targets considered, the nuclear multipole operators evaluated, and the form of the lepton wave functions employed. $\mathcal{O}_{J;\tau}$ means that both isospin structures and all allowed $J$ were included. For the Dirac electron, all of the references surveyed restrict attention to the lowest partial waves $\kappa=\pm 1$. Besides the Dirac solution, the remaining forms of the muon wave function are all constant approximations: $\braket{|\psi_{\mu}|^2}_{\rho}$ is obtained by averaging the probability of the Dirac solution over the nuclear density, $|\phi^Z_\mathrm{1s}(\vec{0})|^2$ is the probability of the point-like Schrodinger solution evaluated at the origin, $|G(R_N)|^2$ is the upper component of the muon's Dirac wave function evaluated at the nuclear radius. Superscript $\dag$ indicates that the reference considers the inelastic process as well, although the information in the table reflects only the treatment of the elastic process.}
\end{table}
%\section{Full Muon Expressions}
%\begin{equation}
%\begin{split}
%\bar{\psi}_e(\vec{x})\psi_{\mu}(\vec{x})&=\frac{1}{\sqrt{2}}\frac{q_\mathrm{eff}}{q}\sum_{L=0}^{\infty}(-i)^L\sqrt{2L+1}Y_{L,0}(\hat{x})\bigg\{j_L(q_\mathrm{eff}x)G^{(\mu)}_{-1}(x)\\
%&+\left(\frac{L+1}{2L+1}j_{L+1}(q_\mathrm{eff}x)-\frac{L}{2L+1}j_{L-1}(q_\mathrm{eff}x)\right)F_{-1}^{(\mu)}(x)\bigg\}
%\end{split}
%\end{equation}
\section{The Nuclear Diffraction Minimum}
Before proceeding to the construction of the effective theory, it is worth discussing the nuclear diffraction minimum in some detail. Generally, one would expect the nuclear form factor
\begin{equation}
F_0(q_\mathrm{eff})=\int dr \;r^2j_0(q_\mathrm{eff}r)\rho(r)
\end{equation} 
to grow proportionally with $A$. This is certainly true at $q=0$ as
\begin{equation}
F_0(0) = A.
\end{equation} 
However, as is apparent for $W^{184}$ in Figure \ref{fig:Welectron}, for heavier nuclei the first zero of the Bessel function $j_0$ leads to a cancellation and a reduction in the strength of the $F_0$ form factor. This is the first nuclear diffraction minimum. In these cases, the nuclear density extends far enough to overlap significantly with the first zero of the $L=0$ Bessel function, resulting in a cancellation between positive and negative contributions and increased sensitivity to any errors in the electron wave function. In higher partial waves, the first zero of the Bessel function is further out and so the cancellation is not as significant (if it occurs at all). In $^{184}$W, this minimum occurs at a momentum of $q=133.52$ MeV; the effective momentum $q_\mathrm{eff}=117.48$ is sufficiently close to the minimum to significantly depress the coherent contribution. On the other hand, the first diffraction minimum in $^{27}$Al occurs at $q=270.93$ MeV, quite far from the effective momentum $q_\mathrm{eff}=110.81$ MeV. 
\begin{table}
\centering
\begin{tabular}{|c|c|c|c|c|c|c|}
\hline
 & $^{12}$C & $^{16}$O & $^{19}$F & $^{23}$Na & $^{27}$Al & $^{28}$Si \\
 \hline
 $F_0(q_\mathrm{eff})$ & 8.748 & 10.880 & 12.312 & 14.582 & 16.281 & 16.380 \\
$F_0(q_\mathrm{eff})/A$ & 0.729 & 0.680 & 0.648 & 0.634 & 0.603 & 0.585\\
\hline
  & $^{32}S$ & $^{40}$Ca & $^{48}$Ti & $^{56}$Fe & $^{63}$Cu & $^{184}$W \\
\hline
 $F_0(q_\mathrm{eff})$ & 17.952 & 20.04 & 22.032 & 23.968 & 25.011 & 15.088\\
$F_0(q_\mathrm{eff})/A$  & 0.561 & 0.501 & 0.459 & 0.428 & 0.397 & 0.082\\
  \hline
\end{tabular}
\caption{The coherent form factor $F_0(q_\mathrm{eff})$ is computed for each target nucleus and compared to $F_0(0)=A$ to see the effect of the nuclear diffraction minimum.}
\end{table}
\chapter{Nucleon-level Effective Operators}
\label{chap:nucleon_level_eft}
In the previous chapter, we demonstrated that the essential physical behavior of the bound muon and the outgoing electron can be captured by approximate functions which dramatically simplify the description of the leptons in $\mu\rightarrow e$ conversion. In this chapter, we discuss the nucleon degrees of freedom. After performing a standard non-relativistic reduction of the nucleon currents, we demonstrate that when these results are combined with the approximate lepton wave functions, a simplified effective theory emerges which is constructed from four basic Hermitian vector operators acting between Pauli spinors. Beyond first order in $v_N/c$, ambiguities arise in the non-relativistic reduction of the nucleon currents \cite{Serot:1978vj}.
\section{Power-counting and Important Parameters}
The construction of an effective theory involves the enumeration of all operators which respect the chosen symmetries. The complete effective theory basis is generally an infinite set of operators, each with an unknown coupling coefficient. In order for the effective theory to have predictive power, we must reduce this basis to a finite number of operators by imposing a power-counting scheme, allowing us to identify and retain all operators through a given order in the power-counting. This truncation should translate to a small but systematically improvable (by progressing to higher-order in the power-counting) error in the calculation of observables.

The relevant dimensionless parameters include the magnitude of the three-momentum transfer in units of the nucleon mass $q_\mathrm{eff}/m_N\approx 1/10$, the multipole parameter $y\gtrsim \frac{1}{4}$. Three velocities. The average nucleon velocity $\vec{v}_N$, the muon velocity $\vec{v}_{\mu}$, and the nuclear recoil velocity $\vec{v}_T$. Treating $^{27}$Al as a single unpaired $1d_{5/2}$ proton, $\sqrt{\vec{v}_N^{\;2}}\approx 0.21$.


\section{Single-nucleon Effective Operator Basis}
The available scalar Hermitians operators are those that can be constructed from the lepton and nucleon identity operators $1_L$ and $1_N$ and from the four dimensionless three-vectors
\begin{equation}
i\hat{q}=\frac{i\vec{q}}{|\vec{q}|},\;\vec{v}_N,\;\vec{\sigma}_L,\;\vec{\sigma}_N.
\end{equation}
Here $\hat{q}$ is the unit vector along the three-momentum transfer to the leptons (or alternatively the velocity of the outgoing ultra-relativistic electron). We identify a total of 16 operators:
\begin{equation}
\begin{split}
\mathcal{O}_1&=1_L\;1_N\\
\mathcal{O}_2'&=1_L\;i\hat{q}\cdot\vec{v}_N\\
\mathcal{O}_3&=1_L\;i\hat{q}\cdot\left[\vec{v}_N\times\vec{\sigma}_N\right]\\
\mathcal{O}_4&=\vec{\sigma}_L\cdot\vec{\sigma}_N\\
\mathcal{O}_5&=\vec{\sigma}_L\cdot\left(i\hat{q}\times\vec{v}_N\right)\\
\mathcal{O}_6&=i\hat{q}\cdot\vec{\sigma}_L\;i\hat{q}\cdot\vec{\sigma}_N\\
\mathcal{O}_7&=1_L\;\vec{v}_N\cdot\vec{\sigma}_N\\
\mathcal{O}_8&=\vec{\sigma}_L\cdot\vec{v}_N\\
\mathcal{O}_9&=\vec{\sigma}_L\cdot\left(i\hat{q}\times\vec{\sigma}_N\right)\\
\mathcal{O}_{10}&=1_L\;i\hat{q}\cdot\vec{\sigma}_N\\
\mathcal{O}_{11}&=i\hat{q}\cdot\vec{\sigma}_L\;1_N\\
\mathcal{O}_{12}&=\vec{\sigma}_L\cdot\left[\vec{v}\times\vec{\sigma}_N\right]\\
\mathcal{O}'_{13}&=\vec{\sigma}_L\cdot\left(i\hat{q}\times\left[\vec{v}_N\times\vec{\sigma}_N\right]\right)\\
\mathcal{O}_{14}&=i\hat{q}\cdot\vec{\sigma}_L\;\vec{v}_N\cdot\vec{\sigma}_N\\
\mathcal{O}_{15}&=i\hat{q}\cdot\vec{\sigma}_L\;i\hat{q}\cdot\left[\vec{v}_N\times\vec{\sigma}_N\right]\\
\mathcal{O}_{16}'&=i\hat{q}\cdot\vec{\sigma}_L\;i\hat{q}\cdot\vec{v}_N
\end{split}
\label{eq:basis_NRET}
\end{equation}
Twelve of the sixteen operators arise from scalar and vector exchanges whereas $\mathcal{O}_3,\mathcal{O}_{12},\mathcal{O}_{13}'$ and $\mathcal{O}_{15}$ arise from higher-spin mediators. The sixteen Pauli-reduced operators can also be obtained from a Lorentz-invariant current reduction.

The operators in Eq. \ref{eq:basis_NRET} form the basis of our effective theory; that is, we may write our effective Lagrangian density as
\begin{equation}
\mathcal{L}_\mathrm{eff}=\sum_{\alpha=n,p}\sum_{i=1}^{16}c_i^{\alpha}\mathcal{O}_i^{\alpha},
\end{equation}
where the $c_i$ are the unknown low-energy constants (LECs) of the effective theory which must be determined from experiment or by matching to a predictive UV theory. We have allowed for each operator to couple distinctly to protons and neutrons. Equivalently, we can transform to the isospin basis to write
\begin{equation}
\mathcal{L}_\mathrm{eff}=\sum_{\tau=0,1}\sum_{i=1}^{16}c^{\tau}_1\mathcal{O}_it^{\tau},
\label{eq:L_NRET_iso}
\end{equation}
where $c^0_i=\frac{1}{2}\left(c_i^p+c_i^n\right)$ and $c_i^1=\frac{1}{2}\left(c_i^p-c_i^n\right)$, and the isospin matrices are $t^0=1$ and $t^1=\tau_3$. Therefore the ET contains a total of 32 unknown parameters associated with 16 space/spin operators each of which can have distinct couplings to protons and neutrons. If we exclude operators that are not associated with spin-0 or spin-1 mediators, 12 space/spin operators and 24 couplings remain. Our aim is to determine the specific constraints that a program of elastic $\mu\rightarrow e$ conversion measurements can place on these LECs.

As defined in Eq. \ref{eq:L_NRET_iso}, the low-energy constants carry dimensions of $1/(\mathrm{mass})^2$. Because experimental results are conventionally expressed as a ratio with respect to the standard-model muon decay rate, it is convenient to introduce a set of dimensionless LECs $\tilde{c}_i$ defined with respect to the weak scale
\begin{equation}
c_i\equiv \tilde{c}_i/v^2=\sqrt{2}G_F\tilde{c}_i,
\end{equation}
where $v=246.2$ GeV is the Higgs vacuum expectation value and $G_F=1.166\times 10^{-5}$GeV$^{-2}$ is the Fermi constant. Alternatively, given an experimental measurement (or limit) on an LEC, one can define an energy characteristic of the CLFV,
\begin{equation}
\Lambda_i^{\tau}\equiv\frac{1}{\sqrt{|c_i^{\tau}|}}=\frac{v}{\sqrt{|\tilde{c}_i^{\tau}|}}.
\end{equation}
This is the energy scale one would associate with the beyond-standard-model CLFV physics. One may characterize the sensitivity of a particular experimental search in terms of the reach in energy scale $\Lambda^\tau_i$.

Having specified the effective Lagrangian, it is relatively straightforward to obtain an expression for the $\mu\rightarrow e$ decay rate. In order to arrive at an simple expression which factorizes the nuclear physics from the CLFV physics, we proceed by performing a multipole expansion of the nuclear charges and currents. 
\section{One-Body Nuclear Charges and Currents}
The nucleon-level effective operators of Eq. \ref{eq:basis_NRET} contain two single-nucleon charges: scalar $1_N$ and axial $\vec{v}_N\cdot\vec{\sigma}_N$, and three single-nucleon currents: spin $\vec{\sigma}_N$, velocity $\vec{v}_N$ and spin-velocity $\vec{\sigma}_N\times\vec{v}_N$. These single-nucleon operators can be embedded in a many-nucleon system by defining the total nuclear charge $J_0(\vec{x})$ and axial charge $J_0^A(\vec{x})$ operators as
\begin{equation}
\begin{split}
J_0(\vec{x})&=\sum_{i=1}^A \delta(\vec{x}-\vec{x}_i)\\
J_0^A(\vec{x})&=\sum_{i=1}^A \frac{1}{2m_N}\left[-\frac{1}{i}\overleftarrow{\nabla}_i\cdot\vec{\sigma}(i)\delta(\vec{x}-\vec{x}_i)+\delta(\vec{x}-\vec{x}_i)\vec{\sigma}(i)\cdot\frac{1}{i}\overrightarrow{\nabla}_i\right],
\end{split}
\end{equation}
and the total nuclear velocity current $\vec{J}_c(\vec{x})$, spin current $\vec{J}_A(\vec{x})$ and spin-velocity current $\vec{J}_M(\vec{x})$ operators as
\begin{equation}
\begin{split}
\vec{J}_c(\vec{x})&=\sum_{i=1}^A \frac{1}{2m_N}\left[-\frac{1}{i}\overleftarrow{\nabla}_i\delta(\vec{x}-\vec{x}_i)+\delta(\vec{x}-\vec{x}_i)\frac{1}{i}\overrightarrow{\nabla}_i\right]\\
\vec{J}_A(\vec{x})&=\sum_{i=1}^A\vec{\sigma}(i)\delta(\vec{x}-\vec{x}_i)\\
\vec{J}_M(\vec{x})&=\sum_{i=1}^A \frac{1}{2m_N}\left[\overleftarrow{\nabla}_i\times\vec{\sigma}(i)\delta(\vec{x}-\vec{x}_i)+\delta(\vec{x}-\vec{x}_i)\vec{\sigma}(i)\times\overrightarrow{\nabla}_i\right].
\end{split}
\end{equation}
Defining the leptonic charges and currents
\begin{equation}
\begin{split}
&l_0^{\tau}=c_1^{\tau}1_L+c_{11}^{\tau}i\hat{q}\cdot\vec{\sigma}_L\\
&l_0^{A\;\tau}=c_7^{\tau}1_L+c_{14}^{\tau}i\hat{q}\cdot\vec{\sigma}_L\\
&\vec{l}_5^{\tau}=c_4^{\tau}\vec{\sigma}_L+c_6^{\tau}i\hat{q}\cdot\vec{\sigma}_Li\hat{q}-c_9^{\tau}i\hat{q}\times\vec{\sigma}_L+c_{10}^{\tau}i\hat{q}1_L\\
&\vec{l}_M^{\tau}=c_2^{\tau}i\hat{q}1_L-c_5^{\tau}i\hat{q}\times\vec{\sigma}_L+c_8^{\tau}\vec{\sigma}_L+c_{16}^{\tau}i\hat{q}\cdot\vec{\sigma}_Li\hat{q}\\
&\vec{l}_E^{\tau}=-c_3^{\tau}\hat{q}1_L+c_{12}^{\tau}i\vec{\sigma}_L+c_{13}^{\tau}\hat{q}\times\vec{\sigma}_L-ic_{15}^{\tau}\hat{q}\cdot\vec{\sigma}_L\hat{q}.
\end{split}
\end{equation}
we may write our effective Hamiltonian density as
\begin{equation}
\begin{split}
\mathcal{H}_\mathrm{eff}(\vec{x})&=\sqrt{\frac{E_e}{2m_e}}|\phi_{1s}^{Z_\mathrm{eff}}(\vec{0})|\frac{q_\mathrm{eff}}{q}e^{-i\vec{q}_\mathrm{eff}\cdot\vec{x}}\sum_{\tau=0,1}\left[l_0^{\tau}\sum_{i=1}^A\delta(\vec{x}-\vec{x}_i)]\right.\\
&\left.+l_0^{A\;\tau}\sum_{i=1}^A\frac{1}{2m_N}\left(-\frac{1}{i}\overleftarrow{\nabla}_i\cdot\vec{\sigma}(i)\delta(\vec{x}-\vec{x}_i)+\delta(\vec{x}-\vec{x}_i)\vec{\sigma}(i)\cdot\frac{1}{i}\overrightarrow{\nabla}\right)\right.\\
&+\vec{l}_5^{\tau}\cdot\sum_{i=1}^A\vec{\sigma}(i)\delta(\vec{x}-\vec{x}_i)+\vec{l}_M^{\tau}\cdot\sum_{i=1}^A\frac{1}{2m_N}\left(-\frac{1}{i}\overleftarrow{\nabla}_i\delta(\vec{x}-\vec{x}_i)+\delta(\vec{x}-\vec{x}_i)\frac{1}{i}\overrightarrow{\nabla}_i\right)\\
&\left.+\vec{l}_E^{\tau}\cdot\sum_{i=1}^A\frac{1}{2m_N}\left(\overleftarrow{\nabla}_i\times\vec{\sigma}(i)\delta(\vec{x}-\vec{x}_i)+\delta(\vec{x}-\vec{x}_i)\vec{\sigma}(i)\times\overrightarrow{\nabla}_i\right)\right]_{int} t^{\tau}(i)
\end{split}
\label{eq:H_NRET}
\end{equation}
Here the subscript $int$ denotes that technically the $A$ single-nucleon velocities appearing in the expression should be replaced by the $A-1$ relative (or Jacobi) velocities. There are techniques for addressing this issue connected with, for example, shell model techniques that employ single-nucleon coordinates but nevertheless create wave functions where the nuclear center-of-mass is in a definite state, thereby removing the extra degrees of freedom. More commonly, though, this issue is ignored.

\section{Multipole Decomposition}
The effective Hamiltonian now has the form of a plane wave multiplying either a nuclear charge as $e^{i\vec{q}\cdot\vec{x}}J_0(\vec{x})$ or current as $e^{i\vec{q}\cdot\vec{x}}\vec{J}(\vec{x})$. In both cases, we may expand the exponential plane wave factor into partial waves and thereby perform a multipole decomposition of the corresponding nuclear charge/current. Orienting our coordinate system along the direction of three-momentum-transfer $(\hat{e}_x,\hat{e}_y,\hat{e}_z=\hat{q})$, then the plane wave factor may be expanded in partial waves as
\begin{equation}
e^{i\vec{q}\cdot\vec{x}}=\sum_{J=0}^{\infty}\sqrt{4\pi(2J+1)}i^Jj_J(qx)Y_{J0}(\hat{x}).
\end{equation}
It immediately follows that any local charge density $J_0(\vec{x})$ can be decomposed into multipole operators with good angular momentum quantum numbers $J,M$ as
\begin{equation}
\mathcal{M}_{JM}(q)=\int d^3x\left[j_J(qx)Y_{JM}(\hat{x})\right]J_0(\vec{x})
\end{equation}
When the plane wave factor multiplies a vector current, then we may expand the vector current in the spherical basis as
\begin{equation}
\vec{J}=\sum_{\lambda}J_{\lambda}e^{\dag}_\lambda
\end{equation}
and then use the fact that
\begin{equation}
\hat{e}_\lambda Y_{lm}(\hat{x})=\sum_{JM}\braket{l\;m\;1\;\lambda|J\;M}\vec{Y}_{J\;l\;M}(\hat{x})
\end{equation}
to write
\begin{equation}
\begin{split}
\hat{e}_{\lambda}e^{i\vec{q}\cdot\vec{x}}&=-\frac{i}{q}\sum_{J=0}^{\infty}\sqrt{4\pi(2J+1)}i^J\vec{\nabla}\left(j_J(qx)Y_{J0}(\hat{x})\right),\;\mathrm{for}\;\lambda=0\\
&=-\sum_{J\geq 1}^{\infty}\sqrt{2\pi(2J+1)}i^J\left[\lambda j_J(qx)\vec{Y}_{JJ\lambda}(\hat{x})+\frac{1}{q}\vec{\nabla}\times\left(j_J(qx)\vec{Y}_{JJ\lambda}(\hat{x})\right)\right],\;\mathrm{for}\;\lambda=\pm 1
\end{split}
\end{equation}
We see that three unique tensor structures have arisen, reflecting the fact that each vector current can be decomposed into longitudinal, transverse-magnetic, and transverse-electric components. The corresponding multipole operators are
\begin{equation}
\begin{split}
\mathcal{L}_{JM}(q)&=\frac{i}{q}\int d^3x \left[\vec{\nabla}\left(j_J(qx)Y_{JM}(\hat{x})\right)\right]\cdot\vec{J}(\vec{x})\\
\mathcal{T}^{\mathrm{mag}}_{JM}(q)&=\int d^3x\left[j_J(qx)\vec{Y}_{J\;J\;M}(\hat{x})\right]\cdot\vec{J}(\vec{x})\\
\mathcal{T}^{\mathrm{el}}_{JM}(q)&=\frac{1}{q}\int d^3x\left[\vec{\nabla}\times\left(j_J(qx)Y_{JM}(\hat{x})\right)\right]\cdot\vec{J}(\vec{x})
\end{split}
\end{equation}
The multipole projections defined in this section are valid for any local nuclear charges $J_0(\vec{x})$ and currents $\vec{J}(\vec{x})$. In the next section, we will construct these projections for the two nuclear charges and three nuclear currents which arise in the effective theory of elastic $\mu\rightarrow e$ conversion.
%The master equation is due to Walecka
%\begin{equation}
%\begin{split}
%\frac{1}{2J_i+1}\sum_{M_i,M_f}|\braket{f|H_I|i}|^2&=\frac{4\pi}{2J_i+1}\left\{\sum_{J\geq 1}^{\infty}\left[\frac{1}{2}(\vec{l}\cdot\vec{l}^*-l_3l_3^*)\left(|\braket{J_f||\mathcal{T}^{\mathrm{mag}}_J||J_i}|^2+|\braket{J_f||\mathcal{T}^{\mathrm{el}}_J||J_i}|^2\right)\right.\right.\\
%&-i\left.\left(\vec{l}\times\vec{l}^*\right)_3\mathrm{Re}\left(\braket{J_f||\mathcal{T}^{\mathrm{mag}}_J||J_i}\braket{J_f||\mathcal{T}^{\mathrm{el}}_J||J_i}^*\right)\right]\\
%&+\left.\sum_{J=0}^{\infty}\left[l_3l_3^*|\braket{J_f||\mathcal{L}_J||J_i}|^2+l_0l_0^*|\braket{J_f||\mathcal{M}_J||J_i}|^2-2\mathrm{Re}\left(l_3l_0^*\braket{J_f||\mathcal{L}_j||J_i}\braket{J_f||\mathcal{M}_J||J_i}^*\right)\right]\right\},
%\end{split}
%\label{eq:master_eq}
%\end{equation}
%where the multipole operators are defined as different projections of the nuclear current,

%The nuclear current here is completely general and we will consider specific models of the nuclear current. The decay rate is given by Fermi's golden rule
%\begin{equation}
%\omega_{fi}=2\pi\frac{V4\pi |\vec{q}|^2}{(2\pi)^3}\frac{1}{2}\sum_{\substack{\mathrm{lepton}\\\mathrm{spins}}}\frac{1}{2J_i+1}\sum_{M_i,M_f}|\braket{f|H_I|i}|^2,
%\end{equation}
%where $J_i$ is the angular momentum of the initial nuclear wavefunction. 

\section{Projections of One-Body Nuclear Currents and Charges}
\label{sec:single_nucleon_operators}
The two nuclear charges and three nuclear currents which arise in the effective theory should admit a total of eleven distinct multipole response operators corresponding to one projection of each charge and three independent projections of each current. Indeed, we find the following projections of $J_0$, $J_0^A$, $\vec{J}_c$, $\vec{J}_A$, and $\vec{J}_M$:
\begin{equation}
\begin{split}
\mathcal{M}_{JM}(J_0)&=\sum_{i=1}^AM_{JM}(q\vec{x}_i)\\
\mathcal{M}_{JM}(J^5_0)&=-i\frac{q}{m_N}\sum_{i=1}^A \left[\Omega_{JM}(q\vec{x}_i)+\frac{1}{2}\Sigma''_{JM}(q\vec{x}_i)\right]\\
\mathcal{L}_{JM}(\vec{J}_A)&=i\sum_{i=1}^A\Sigma''_{J,M}(q\vec{x}_i),\\
\mathcal{T}^{\mathrm{el}}_{JM}(\vec{J}_A)&=i\sum_{i=1}^A\Sigma'_{JM}(q\vec{x}_i),\\
\mathcal{T}^{\mathrm{mag}}_{JM}(\vec{J}_A)&=\sum_{i=1}^A\Sigma_{JM}(q\vec{x}_i)\\
\mathcal{L}_{JM}(\vec{J}_M)&=\frac{q}{m_N}\sum_{i=1}^A \left[\Delta''_{JM}(q\vec{x}_i)-\frac{1}{2}M_{JM}(q\vec{x}_i)\right]\\
\mathcal{T}^{\mathrm{el}}_{JM}(\vec{J}_M)&=\frac{q}{m_N}\sum_{i=1}^A\Delta'_{JM}(q\vec{x}_i)\\
\mathcal{T}^{\mathrm{mag}}_{JM}(\vec{J}_M)&=-i\frac{q}{m_N}\sum_{i=1}^A\Delta_{JM}(q\vec{x}_i)\\
\mathcal{L}_{JM}(\vec{J}_E)&=\frac{q}{m_N}\sum_{i=1}^A\Phi^{''}_{JM}(q\vec{x}_i)\\
\mathcal{T}^{\mathrm{el}}_{JM}(\vec{J}_E)&=\frac{q}{m_N}\sum_{i=1}^A\left[\Phi'_{JM}(q\vec{x}_i)+\frac{1}{2}\Sigma_{JM}(q\vec{x}_i)\right]\\
\mathcal{T}^{\mathrm{mag}}_{JM}(\vec{J}_E)&=-i\frac{q}{m_N}\sum_{i=1}^A\left[\Phi_{JM}(q\vec{x}_i)-\frac{1}{2}\Sigma'_{JM}(q\vec{x}_i)\right]
\end{split}
\end{equation}
where the single-nucleon multipole operators which arise are those familiar from the study of semi-leptonic weak interactions \cite{DONNELLY1979103,SEROT1979408}. Adding a label to denote isospin, the single-nucleon response functions are
\begin{equation}
\begin{split}
M_{JM;\tau}(q)&\equiv \sum_{i=1}^AM_{JM}(q\vec{x}_i)\;t^{\tau}(i)\\
\Omega_{JM;\tau}(q)&\equiv \sum_{i=1}^AM_{JM}(q\vec{x}_i)\vec{\sigma}(i)\cdot\frac{1}{q}\vec{\nabla}_i\;t^{\tau}(i)\\
\Delta_{JM;\tau}(q)&\equiv \sum_{i=1}^A \vec{M}_{JJM}(q\vec{x}_i)\cdot\frac{1}{q}\vec{\nabla}_i\;t^{\tau}(i)\\
\Delta'_{JM;\tau}(q)&\equiv -i\sum_{i=1}^A \left\{\frac{1}{q}\vec{\nabla}_i\times\vec{M}_{JJM}(q\vec{x}_i)\right\}\cdot\frac{1}{q}\vec{\nabla}_i\;t^{\tau}(i)\\
\Delta''_{JM;\tau}(q)&\equiv \sum_{i=1}^A\left(\frac{1}{q}\vec{\nabla}_iM_{JM}(q\vec{x}_i)\right)\cdot\frac{1}{q}\vec{\nabla}_i\;t^{\tau}(i)\\
\Sigma_{JM;\tau}(q)&\equiv \sum_{i=1}^A\vec{M}_{JJM}(q\vec{x}_i)\cdot\vec{\sigma}(i)\;t^{\tau}(i)\\
\Sigma'_{JM;\tau}(q)&\equiv -i\sum_{i=1}^A\left\{\frac{1}{q}\vec{\nabla}_i\times\vec{M}_{JJM}(q\vec{x}_i)\right\}\cdot\vec{\sigma}(i)\;t^{\tau}(i)\\
\Sigma''_{JM;\tau}(q)&\equiv \sum_{i=1}^A\left\{\frac{1}{q}\vec{\nabla}_iM_{JM}(q\vec{x}_i)\right\}\cdot\vec{\sigma}(i)\;t^{\tau}(i)\\
\Phi_{JM;\tau}(q)&\equiv i\sum_{i=1}^A\vec{M}_{JJM}(q\vec{x}_i)\cdot\left(\vec{\sigma}(i)\times\frac{1}{q}\vec{\nabla}_i\right)\;t^{\tau}(i)\\
\Phi'_{JM;\tau}(q)&\equiv\sum_{i=1}^A\left(\frac{1}{q}\vec{\nabla}_i\times\vec{M}_{JJM}(q\vec{x}_i)\right)\cdot\left(\vec{\sigma}(i)\times\frac{1}{q}\vec{\nabla}_i\right)\;t^{\tau}(i)\\
\Phi''_{JM;\tau}(q)&\equiv i\sum_{i=1}^A\left(\frac{1}{q}\vec{\nabla}_iM_{JM}(q\vec{x}_i)\right)\cdot\left(\vec{\sigma}(i)\times\frac{1}{q}\vec{\nabla}_i\right)\;t^{\tau}(i)
\label{eq:single_nucleon_responses}
\end{split}
\end{equation}
where 
\begin{equation}
\begin{split}
M_{JM}(q\vec{x})&=j_J(qx)Y_{JM}(\hat{x})\\
\vec{M}_{JLM}(q\vec{x})&=j_L(qx)\vec{Y}_{JLM}(\hat{x})
\end{split}
\end{equation}
and $\vec{Y}_{JLM}$ is a vector spherical harmonic (see Appendix \ref{sec:vector_spherical_harmonics}). 

In addition to carrying angular momentum $(J,M)$ each multipole operator has a well-defined transformation under parity $\vec{x}\rightarrow -\vec{x}$: the operators $M$, $\Delta'$, $\Delta''$, $\Sigma$, $\Phi'$ and $\Phi''$ are \textit{normal parity} operators which transform with a phase $(-1)^J$ under parity whereas $\Omega$, $\Delta$, $\Sigma'$, $\Sigma''$ and $\Phi$ are \textit{abnormal parity} operators which transform with a phase $(-1)^{J+1}$ under parity. 

Another attractive feature of this formulation is that matrix elements of the above multipole operators evaluated between harmonic oscillator states can be expressed analytically in terms of the dimensionless quantity $y=(qb/2)^2$, where $b$ is the parameter which sets the length scale of the harmonic oscillator states. In particular, letting $T_J(q\vec{r})$ represent any of the 11 single-particle operators, we have
\begin{equation}
\braket{n'\left(\ell'\;1/2\right)j'||T_J(q\vec{r})||n\left(\ell\;1/2\right)j}=\frac{1}{\sqrt{4\pi}}y^{(J-K)/2}e^{-y}p(y),
\label{eq:multipole_polynomial}
\end{equation}
where $K=2$ for the normal parity operators $M$, $\Delta'$, $\Delta''$, $\Sigma$, $\Phi'$, and $\Phi''$, and where $K=1$ for the abnormal parity operators $\Omega$, $\Delta$, $\Sigma'$, $\Sigma''$, and $\Phi$. The function $p(y)$ is a finite-degree polynomial in $y$. See Appendix \ref{app:single_nucleon_response} for a detailed derivation of Eq. \ref{eq:multipole_polynomial} and other properties of the single-nucleon response functions. For the choices of phase conventions in our definitions, all of the matrix elements are real. In order to discuss time-reversal symmetry, we define the following transformed operators
\begin{equation}
\begin{split}
\tilde{\Omega}_{JM}(q)&\equiv\Omega_{JM}(q)+\frac{1}{2}\Sigma''_{JM}(q)\\
\tilde{\Delta}''_{JM}(q)&\equiv\Delta''_{JM}(q)-\frac{1}{2}M_{JM}(q)\\
\tilde{\Phi}_{JM}(q)&\equiv \Phi_{JM}(q)-\frac{1}{2}\Sigma'_{JM}(q)\\
\tilde{\Phi}'_{JM}(q)&\equiv \Phi'_{JM}(q)+\frac{1}{2}\Sigma_{JM}(q)
\end{split}
\end{equation}
which have well-defined transformations under the exchange of initial and final single-particle states
\begin{equation}
\braket{n(l\frac{1}{2})j||T_J(q\vec{x})||n'(l'(\frac{1}{2})j'}=(-1)^{\lambda}\braket{n'(l'\frac{1}{2})j'||T_J(q\vec{x})||n(l\frac{1}{2})j},
\end{equation}
with $\lambda=j'-j$ for the operators $M$, $\Delta$, $\Sigma'$, $\Sigma''$, $\tilde{\Phi}'$ and $\Phi''$ and $\lambda=j'+j$ for the operators $\tilde{\Omega}$, $\Delta'$, $\tilde{\Delta}''$, $\Sigma$ and $\tilde{\Phi}$. 

The parity and time-reversal properties of the eleven single-nucleon response functions are summarized in Table \ref{tab:multipole_symmetries}. We see that the five operators for which $\lambda=j'+j$ under exchange of initial and final states always have opposite $P$ and $T$ transformations for a given $J$. For example, the operator $\tilde{\Omega}_{00}$ is even under parity but odd under time-reversal. The restriction of the nucleus to remain in the ground state throughout the $\mu\rightarrow e$ conversion process now has profound consequences for the form of the nuclear response; although the operator $\tilde{\Omega}_{00}$ conserves parity, it violates time-reversal symmetry and therefore cannot contribute to the elastic conversion process. The same is true for $\Delta'$, $\tilde{\Delta}''$, $\Sigma$ and $\tilde{\Phi}$; the fact that these operators always have opposite $P$ and $T$ symmetries excludes them from the elastic process. Thus the nuclear response is limited to the six allowed response functions $M$, $\Delta$, $\Sigma'$, $\Sigma''$, $\tilde{\Phi}'$ and $\Phi''$. We will now derive an expression for the $\mu\rightarrow e$ conversion rate in terms of these six response functions.


\begin{table}
\centering
\begin{tabular}{|c|c|c|c|c|c|}
\hline
Projection & Charge/Current & Operator & Even J & Odd J & LECs Probed\\
\hline
Charge & $1_N$ & $M_{JM}$ & E-E & O-O & $c_1,c_{11}$\\
Charge & $\vec{v}_N\cdot\vec{\sigma}_N$ & $\tilde{\Omega}_{JM}$ & O-E & E-O & $c_7,c_{14}$\\
Longitudinal & $\vec{\sigma}_N$ & $\Sigma''_{JM}$ & O-O & E-E & $c_4,c_6,c_{10}$\\
Transverse magnetic & " & $\Sigma_{JM}$ & E-O & O-E & $c_4,c_9$\\
Transverse electric & " & $\Sigma'_{JM}$ & O-O & E-E & $c_4,c_9$\\
Longitudinal & $\vec{v}_N$ & $\tilde{\Delta}''_{JM}$ & E-O & O-E & $c_2,c_8,c_{16}$\\
Transverse magnetic & " & $\Delta_{JM}$ & O-O & E-E & $c_5,c_8$\\
Transverse electric & " & $\Delta'_{JM}$ & E-O & O-E & $c_5,c_8$\\
Longitudinal & $\vec{v}_N\times\vec{\sigma}_N$ & $\Phi''_{JM}$ & E-E & O-O & $c_3,c_{12},c_{15}$ \\
Transverse magnetic & " & $\tilde{\Phi}_{JM}$ & O-E & E-O & $c_{12},c_{13}$\\
Transverse electric & " & $\tilde{\Phi}'_{JM}$ & E-E & O-O & $c_{12},c_{13}$ \\
\hline
\end{tabular}
\caption{Parity and time transformation properties of the eleven single-nucleon response functions. Based on these results, elastic nuclear matrix elements can involve only $M_{JM}$, $\Delta_{JM}$, $\Sigma'_{JM}$, $\Sigma^{''}_{JM}$ and $\Phi''_{JM}$.}
\label{tab:multipole_symmetries}
\end{table}
Equation \ref{eq:master_eq} gives the unpolarized transition probability in terms of multipole projections of the nuclear charge and current. Let us now consider specific one-body nuclear charge and current operators. These currents can all be derived by taking the non-relativistic limit of various nucleon spinor currents (see Appendix). 

\section{Elastic $\mu\rightarrow \lowercase{e}$ Decay Rate}
Knowing the multipole decomposition of the relevant nuclear charges and currents, as well as the parity and time-reversal transformation properties of the resulting single nucleon operators, we now proceed to compute the $\mu\rightarrow e$ decay rate. Beginning from Eq. \ref{eq:H_NRET} and letting $j_i=j_f=j_N$ be the total nuclear angular momentum, $m_i,m_f$ the initial and final magnetic quantum numbers of the nuclear state, and $s_i,s_f$ the initial and final magnetic quantum numbers of the leptons, we compute the decay amplitude 
\begin{equation}
\mathcal{M}=\braket{\frac{1}{2}s_f;j_Nm_f|\;\int d^3x\;\mathcal{H}(\vec{x})\;|\frac{1}{2}s_i;j_Nm_i}
\end{equation}
by performing the multipole decomposition of the nuclear charges and currents. The result is
\begin{equation}
\begin{split}
\mathcal{M}&=\sqrt{\frac{E_e}{2m_e}}|\phi_{1s}^{Z_\mathrm{eff}}(\vec{0})|\frac{q_\mathrm{eff}}{q}\sum_{\tau=0,1}\bra{\frac{1}{2}m_{s_f};j_Nm_f}\left[\sum_{J=0,2,...}^{\infty}\sqrt{4\pi(2J+1)}(-i)^J\left[l_0^{\tau}M_{J0;\tau}(q_\mathrm{eff})+\frac{q_\mathrm{eff}}{m_N}l_{E0}^{\tau}\Phi''_{J0;\tau}(q_\mathrm{eff})\right]\right.\\
&+\sum_{J=1,3,...}^{\infty}\sqrt{2\pi(2J+1)}(-i)^J\sum_{\lambda =\pm 1}(-1)^\lambda\left[il_{5\lambda}^{\tau}\Sigma'_{J-\lambda;\tau}(q_\mathrm{eff})-i\frac{q_\mathrm{eff}}{m_N}l_{M\lambda}^{\tau}\lambda\Delta_{J-\lambda;\tau}(q_\mathrm{eff})\right]\\
&+\sum_{J=2,4,...}^{\infty}\sqrt{2\pi(2J+1)}(-i)^J\sum_{\lambda=\pm 1}(-1)^\lambda \left[\frac{q_\mathrm{eff}}{m_N}l_{E\lambda}^{\tau}\tilde{\Phi}'_{J-\lambda;\tau}(q_\mathrm{eff})\right]\\
&+\sum_{J=1,3,...}^{\infty}\sqrt{4\pi(2J+1)}(-i)^J\left[il_{50}^{\tau}\Sigma''_{J0;\tau}(q_\mathrm{eff})\right]\Bigg]\ket{\frac{1}{2}m_{s_i};j_Nm_i},
\end{split}
\end{equation}
where we note that the leptonic current $l_0^{A\;\tau}$ does not appear because the nuclear axial charge operator is excluded from the elastic conversion process by $P$ and $T$ symmetry. Next we write the amplitude in terms of reduced matrix elements of the nuclear multipole operators using the Wigner-Eckart theorem
\begin{equation}
\braket{j_Nm_f|T_{JM}|j_Nm_i}=(-1)^{j_N-m_f}\left(\begin{array}{ccc}
j_N & J & j_N\\
-m_f & M & m_i
\end{array}\right)\braket{j_N||T_J||j_N}
\end{equation}
which allows us to compute the nuclear spin-averaged amplitude squared
\begin{equation}
\frac{1}{2j_N+1}\sum_{m_f,m_i}|\mathcal{M}|^2
\end{equation}
using the completeness relation for the 3-$j$ symbols
\begin{equation}
\frac{1}{2j_N+1}\sum_{m_f,m_i}\left(\begin{array}{ccc}
j_N & J & j_N\\
-m_f & M & m_i
\end{array}\right)\left(\begin{array}{ccc}
j_N & J' & j_N\\
-m_f & M' & m_i
\end{array}\right)=\delta_{JJ'}\delta_{MM'}\frac{1}{2J+1}\frac{1}{2j_N+1}.
\end{equation}
The resulting expression is
% Then just as in the case of DM-Nucleus scattering
%\begin{equation}
%\begin{split}
%&\frac{1}{2J_i+1}\sum_{M_i,M_f}|\braket{f|H_I|i}|^2=\frac{4\pi}{2J_i+1}\left[\sum_{J=1,3,...}^{\infty}|\braket{J_i||\;\vec{l}_5\cdot\hat{q}\;\Sigma''_J(q)\;||J_i}|^2\right.\\
%&+\sum_{J=0,2,...}^{\infty}\left\{|\braket{J_i||\;l_0\;M_J(q)\;||J_i}|^2+|\braket{J_i||\;\vec{l}_E\cdot\hat{q}\;\frac{q}{m_N}\Phi''_J(q)\;||J_i}|^2\right.\\
%&+\left.2\mathrm{Re}\left[\braket{J_i||\;\vec{l}_E\cdot\hat{q}\;\frac{q}{m_N}\Phi''_J(q)\;||J_i}\braket{J_i||\;l_0\;M_J(q)\;||J_i}^*\right]\right\}\\
%&+\frac{q^2}{2m_N^2}\sum_{J=2,4,...}^{\infty}\left(\braket{J_i||\;\vec{l}_E\;\tilde{\Phi}'_J(q)\;||J_i}\cdot\braket{J_i||\;\vec{l}_E\;\tilde{\Phi}'_J(q)\;||J_i}^*-|\braket{J_i||\;\vec{l}_E\cdot\hat{q}\;\tilde{\Phi}'_J(q)\;||J_i}|^2\right)\\
%&+\sum_{J=1,3,...}^{\infty}\left\{\frac{q^2}{2m_N^2}\left(\braket{J_i||\;\vec{l}_M\;\Delta_J(q)\;||J_i}\cdot\braket{J_i||\vec{l}_M\;\Delta_J(q)\;||J_i}^*-|\braket{J_i||\;\vec{l}_M\cdot\hat{q}\;\Delta_J(q)\;||J_i}|^2\right)\right.\\
%+\frac{1}{2}\left(\braket{J_i||\;\vec{l}_5\;\Sigma'_J(q)\;||J_i}\cdot\braket{J_i||\;\vec{l}_5\;\Sigma'_J(q)\;||J_i}^*-|\braket{J_i||\;\vec{l}_A\cdot\hat{q}\;\Sigma'_J(q)\;||J_i}|^2\right)\\
%&+\left.\left. 2\mathrm{Re}\left[i\hat{q}\cdot\braket{J_i||\;\vec{l}_M\;\frac{q}{m_N}\Delta_J(q)\;||J_i}\times\braket{J_i||\;\vec{l}_5\;\Sigma'_J(q)\;||J_i}^*\right]\right\}\right]
%\end{split}
%\label{eq:master_elastic}
%\end{equation}
\begin{equation}
\begin{split}
&\frac{1}{2j_N+1}\sum_{m_f,m_i}|\mathcal{M}|^2=\frac{E_e}{2m_e}|\phi_{1s}^{Z_\mathrm{eff}}(\vec{0})|^2\frac{q^2_\mathrm{eff}}{q^2}\frac{4\pi}{2j_N+1}\sum_{\tau=0,1}\sum_{\tau'=0,1}\\
&\Bigg\{\sum_{J=0,2,...}^{\infty}\Bigg(\braket{l_0^{\tau}}\braket{l_0^{\tau'}}^*\braket{j_N||M_{J,\tau}(q_\mathrm{eff})||j_N}\braket{j_N||M_{J,\tau'}(q_\mathrm{eff})||j_N}\\
&+\frac{\vec{q}_\mathrm{eff}}{m_N}\cdot\braket{\vec{l}_E^{\tau}}\frac{\vec{q}_\mathrm{eff}}{m_N}\cdot\braket{\vec{l}_E^{\tau'}}^*\braket{j_N||\Phi''_{J,\tau}(q_\mathrm{eff})||j_N}\braket{j_N||\Phi''_{J,\tau'}(q_\mathrm{eff})||j_N}\\
&+\frac{2\vec{q}_\mathrm{eff}}{m_N}\cdot\mathrm{Re}\left[\braket{\vec{l}_E^{\tau}}\braket{l_0^{\tau'}}^*\right]\braket{j_N||\Phi''_{J,\tau}(q_\mathrm{eff})||j_N}\braket{j_N||M_{J,\tau'}(q_\mathrm{eff})||j_N}\Bigg)\\
&+\sum_{J=2,4,...}^{\infty}\frac{1}{2}\left(\frac{q^2_\mathrm{eff}}{m_N^2}\braket{\vec{l}_E^{\tau}}\cdot\braket{\vec{l}_E^{\tau'}}^*-\frac{\vec{q}_\mathrm{eff}}{m_N}\cdot\braket{\vec{l}_E^{\tau}}\frac{\vec{q}_\mathrm{eff}}{m_N}\cdot\braket{\vec{l}_E^{\tau'}}^*\right)\braket{j_N||\tilde{\Phi}'_{J,\tau}(q_\mathrm{eff})||j_N}\braket{j_N||\tilde{\Phi}'_{J,\tau'}(q_\mathrm{eff})||j_N}\\
&+\sum_{J=1,3,...}^{\infty}\Bigg(\hat{q}\cdot\braket{\vec{l}_5^{\tau}}\hat{q}\cdot\braket{\vec{l}_5^{\tau'}}^*\braket{j_N||\Sigma''_{J,\tau}(q_\mathrm{eff})||j_i}\braket{j_f||\Sigma''_{J,\tau'}(q_\mathrm{eff})||j_N}\\
&+\frac{1}{2}\left(\braket{\vec{l}_5^{\tau}}\cdot\braket{\vec{l}_5^{\tau'}}^*-\hat{q}\cdot\braket{\vec{l}_5^{\tau}}\hat{q}\cdot\braket{\vec{l}_5^{\tau'}}^*\right)\braket{j_N||\Sigma'_{J,\tau}(q_\mathrm{eff})||j_N}\braket{j_N||\Sigma'_{J,\tau'}(q_\mathrm{eff})||j_N}\\
&+\frac{1}{2}\left(\frac{q^2_\mathrm{eff}}{m_N^2}\braket{\vec{l}_M^{\tau}}\cdot\braket{\vec{l}_M^{\tau'}}^*-\frac{\vec{q}_\mathrm{eff}}{m_N}\cdot\braket{\vec{l}_M^{\tau}}\frac{\vec{q}_\mathrm{eff}}{m_N}\cdot\braket{\vec{l}_M^{\tau'}}^*\right)\braket{j_N||\Delta_{J,\tau}(q_\mathrm{eff})||j_N}\braket{j_N||\Delta_{J,\tau'}(q_\mathrm{eff})||j_N}\\
&+\frac{\vec{q}_\mathrm{eff}}{m_N}\cdot\mathrm{Re}\left[i\braket{\vec{l}_M^{\tau}}\times\braket{\vec{l}_5^{\tau'}}^*\right]\braket{j_N||\Delta_{J,\tau}(q_\mathrm{eff})||j_N}\braket{j_N||\Sigma'_{J,\tau'}(q_\mathrm{eff})||j_N}\Bigg)\Bigg\},
\end{split}
\end{equation}
where we have introduced the shorthand $\braket{l}\equiv\braket{\frac{1}{2}s_f|l|\frac{1}{2}s_i}$ for the leptonic matrix elements. The sum over lepton spins can now be computed in a straightforward way by performing the traces over the leptonic currents. For example
\begin{equation}
\begin{split}
\frac{1}{2}\sum_{s_f,s_i}\braket{l_0^{\tau}}\braket{l_0^{\tau'}}^*&=\frac{1}{2}\sum_{s_f,s_i}\xi^{\dag}_{s_f}\Big(c_1^{\tau}1_L+c_{11}^{\tau}i\hat{q}\cdot\vec{\sigma}_L\Big)\xi_{s_i}\xi^{\dag}_{s_i}\Big(c_1^{\tau'*}1_L-c_{11}^{\tau'*}i\hat{q}\cdot\vec{\sigma}_L\Big)\xi_{s_f}\\
&=\frac{1}{2}\mathrm{Tr}\left[\Big(c_1^{\tau}1_L+c_{11}^{\tau}i\hat{q}\cdot\vec{\sigma}_L\Big)\Big(c_1^{\tau'*}1_L-c_{11}^{\tau'*}i\hat{q}\cdot\vec{\sigma}_L\Big)\right]\\
&=c_1^{\tau}c_1^{\tau'*}+c_{11}^{\tau}c_{11}^{\tau'*},
\end{split}
\end{equation} 
where the final result depends only on the LECs of the CLFV operators. Performing the spin summation for the remaining leptonic currents yields 
\begin{equation}
\begin{split}
&\frac{1}{2}\frac{1}{2J_N+1}\sum_{\mathrm{spins}}|\mathcal{M}|^2=\frac{E_e}{2m_e}|\phi_{1s}^{Z_\mathrm{eff}}(\vec{0})|^2\frac{q_\mathrm{eff}^2}{q^2}\frac{4\pi}{2J_N+1}\sum_{\tau=0,1}\sum_{\tau'=0,1}\\
&\left\{\sum_{J=0,2,...}^{\infty}\left[R^{\tau\tau'}_M\braket{J_N||\;M_{J;\tau}(q_\mathrm{eff})\;||J_N}\braket{J_N||\;M_{J;\tau'}(q_\mathrm{eff})||J_N}\right.\right.\\
&+\frac{q^{2}_\mathrm{eff}}{m_N^2}R_{\Phi''}^{\tau\tau'}\braket{J_N||\Phi^{''}_{J;\tau}(q_\mathrm{eff})||J_N}\braket{J_N||\Phi^{''}_{J;\tau'}(q_\mathrm{eff})||J_N}\\
&\left.-\frac{2q_\mathrm{eff}}{m_N}R_{\Phi''M}^{\tau\tau'}\braket{J_N||\Phi''_{J;\tau}(q_\mathrm{eff})||J_N}\braket{J_N||M_{J;
\tau'}(q_\mathrm{eff})||J_N}\right]\\
&+\sum_{J=2,4,...}^{\infty}\left[\frac{q_\mathrm{eff}^{2}}{m_N^2}R_{\tilde{\Phi'}}^{\tau\tau'}\braket{J_N||\tilde{\Phi}'_{J;\tau}(q_\mathrm{eff})||J_N}\braket{J_N||\tilde{\Phi}'_{J;\tau'}(q_\mathrm{eff})||J_N}\right]\\
&+\sum_{J=1,3,...}^{\infty}\left[R_{\Sigma^{''}}^{\tau\tau'}\braket{J_N||\Sigma^{''}_{J;\tau}(q_\mathrm{eff})||J_N}\braket{J_N||\Sigma^{''}_{J;\tau'}(q_\mathrm{eff})||J_N}\right.\\
&+R_{\Sigma'}^{\tau\tau'}\braket{J_N||\Sigma'_{J;\tau}(q_\mathrm{eff})||J_N}\braket{J_N||\Sigma'_{J;\tau'}(q_\mathrm{eff})||J_N}\\
&+\frac{q_\mathrm{eff}^{2}}{m_N^2}R_{\Delta}^{\tau\tau'}\braket{J_N||\Delta_{J;\tau}(q_\mathrm{eff})||J_N}\braket{J_N||\Delta_{J;\tau'}(q_\mathrm{eff})||J_N}\\
&\left.\left.-\frac{2q_\mathrm{eff}}{m_N}R_{\Delta\Sigma'}^{\tau\tau'}\braket{J_N||\Delta_{J;\tau}(q_\mathrm{eff})||J_N}\braket{J_N||\Sigma'_{J;\tau'}(q_\mathrm{eff})||J_N}\right]\right\},
\end{split}
\end{equation}
where we have defined the following linear combinations of the LECs:
\begin{equation}
\begin{split}
R_M^{\tau\tau'}&\equiv c_1^{\tau}c_1^{\tau'*}+c_{11}^{\tau}c_{11}^{\tau'*}\\
R_{\Phi^{''}}^{\tau\tau'}&\equiv c_3^{\tau}c_3^{\tau'*}+\left(c^{\tau}_{12}-c_{15}^{\tau}\right)\left(c_{12}^{\tau'*}-c_{15}^{\tau'*}\right)\\
R^{\tau\tau'}_{\Phi^{''}M}&\equiv \mathrm{Re}\left[c_3^{\tau}c_1^{\tau'*}-(c_{12}^{\tau}-c_{15}^{\tau})c_{11}^{\tau'*}\right]\\
R_{\tilde{\Phi}'}^{\tau\tau'}&\equiv c_{12}^{\tau}c_{12}^{\tau'*}+c_{13}^{\tau}c_{13}^{\tau'*}\\
R_{\Sigma^{''}}^{\tau\tau'}&\equiv (c_4^{\tau}-c_6^{\tau})(c_4^{\tau'*}-c_6^{\tau'*})+c_{10}^{\tau}c_{10}^{\tau'*}\\
R_{\Sigma'}^{\tau\tau'}&\equiv c_4^{\tau}c_4^{\tau'*}+c_9^{\tau}c_9^{\tau'*}\\
R_{\Delta}^{\tau\tau'}&\equiv c_5^{\tau}c_5^{\tau'*}+c_8^{\tau}c_8^{\tau'*}\\
R_{\Delta\Sigma'}^{\tau\tau'}&\equiv \mathrm{Re}\left[c_5^{\tau}c_4^{\tau'*}+c_8^{\tau}c_9^{\tau'*}\right].
\end{split}
\end{equation}
The $R$ coefficients defined above are in terms of the dimensionful LECs $c_i^{\tau}$. If we consider the analogous dimensionless leptonic tensors $\tilde{R}$ defined in terms of $\tilde{c}_i^{\tau}$, then the elastic CLFV decay rate can be expressed as 
\begin{equation}
\begin{split}
\Gamma\left(\mu\rightarrow e\right)=\frac{G_F^2}{\pi}\frac{q_\mathrm{eff}^2}{1+\frac{q}{M_T}}|\phi_{1s}^{Z_\mathrm{eff}}(\vec{0})|^2\sum_{\tau=0,1}&\sum_{\tau'=0,1}\Bigg\{\left[\tilde{R}_M^{\tau\tau'}W_M^{\tau\tau'}(q_\mathrm{eff})+\tilde{R}_{\Sigma''}^{\tau\tau'}W_{\Sigma''}^{\tau\tau'}(q_\mathrm{eff})+\tilde{R}_{\Sigma'}^{\tau\tau'}W_{\Sigma'}^{\tau\tau'}(q_\mathrm{eff})\right]\\
&+\frac{q_\mathrm{eff}^2}{m_N^2}\left[\tilde{R}_{\Phi''}^{\tau\tau'}W_{\Phi''}^{\tau\tau'}(q_\mathrm{eff})+\tilde{R}_{\tilde{\Phi}'}^{\tau\tau'}W_{\tilde{\Phi}'}^{\tau\tau'}(q_\mathrm{eff})+\tilde{R}_{\Delta}^{\tau\tau'}W_{\Delta}^{\tau\tau'}(q_\mathrm{eff})\right]\\
&-\frac{2q_\mathrm{eff}}{m_N}\left[\tilde{R}_{\Phi'' M}^{\tau\tau'}W_{\Phi'' M}^{\tau\tau'}(q_\mathrm{eff})+\tilde{R}_{\Delta\Sigma'}^{\tau\tau'}W_{\Delta\Sigma'}^{\tau\tau'}(q_\mathrm{eff})\right]\Bigg\}
\end{split}
\label{eq:mu2e_rate}
\end{equation}
where the factor $(1+q/M_T)^{-1}$ accounts for the distortion of phase space by nuclear recoil, and where we have defined the nuclear response functions
\begin{equation}
\begin{split}
W_O^{\tau\tau'}(q_\mathrm{eff})&\equiv \frac{4\pi}{2j_N+1}\sum_{J=0,2,...}^{\infty}\braket{j_N||O_{J;\tau}(q_\mathrm{eff})||j_N}\braket{j_N||O_{J;\tau'}(q_\mathrm{eff})||j_N}\;\mathrm{for}\;O=M,\Phi''\\
W_O^{\tau\tau'}(q_\mathrm{eff})&\equiv \frac{4\pi}{2j_N+1} \sum_{J=1,3,...}^{\infty}\braket{j_N||O_{J;\tau}(q_\mathrm{eff})||j_N}\braket{j_N||O_{J;\tau'}(q_\mathrm{eff})||j_N}\;\mathrm{for}\;O=\Sigma',\Sigma'',\Delta\\
W_{\tilde{\Phi'}}^{\tau\tau'}(q_\mathrm{eff})&\equiv \frac{4\pi}{2j_N+1} \sum_{J=2,4,...}^{\infty}\braket{j_N||\tilde{\Phi}'_{J;\tau}(q_\mathrm{eff})||j_N}\braket{j_N||\tilde{\Phi}'_{J;\tau'}(q_\mathrm{eff})||j_N}\\
W_{\Phi''M}^{\tau\tau'}(q_\mathrm{eff})&\equiv \frac{4\pi}{2j_N+1} \sum_{J=0,2,...}^{\infty}\braket{j_N||\Phi''_{J;\tau}(q_\mathrm{eff})||j_N}\braket{j_N||M_{J;\tau'}(q_\mathrm{eff})||j_N}\\
W_{\Delta\Sigma'}^{\tau\tau'}(q_\mathrm{eff})&\equiv \frac{4\pi}{2j_N+1} \sum_{J=1,3,...}^{\infty}\braket{j_N||\Delta_{J;\tau}(q_\mathrm{eff})||j_N}\braket{j_N||\Sigma'_{J;\tau'}(q_\mathrm{eff})||j_N}\\
\end{split}
\label{eq:w_response}
\end{equation}
Both the nuclear response functions $W$ and the leptonic response functions $\tilde{R}$ are dimensionless. If all dimensionful pre-factors in Eq. \ref{eq:mu2e_rate} are evaluated in GeV units, rates in 1/sec will be obtained by dividing by $\hbar$.

As expected from consideration of parity and time-reversal symmetry, the $\mu\rightarrow e$ conversion rate can be expressed in terms of the six allowed single-nucleon response functions as well as two interference terms. Each nuclear response function $W$ is multiplied by a corresponding leptonic response function $R$. This is precisely the factorization between nuclear physics and CLFV physics which we hoped to achieve in formulating an effective theory of $\mu\rightarrow e$ conversion at the nuclear scale. As one varies the choice of nuclear target, the nuclear response functions $W$ vary depending on the details of nuclear structure; the parameters of the lepton approximations $q_\mathrm{eff}$ and $Z_\mathrm{eff}$ also change depending on the target. Crucially, the low-energy constants of the single-nucleon CLFV operators, the $c_i^{\tau}$, should not depend on the nuclear target, and therefore the leptonic response functions $R$ will inherent this target-independence. Thus by performing an ensemble of measurements of $\mu\rightarrow e$ conversion in a range of nuclear targets, an experimentalist can, in principle, use Eq. \ref{eq:mu2e_rate} to constrain and/or determine the values of the $R$ coefficients. As we have formulated  the most general nuclear-scale effective theory, such a determination would extract the maximum amount of information that can be obtained about CLFV operators in observations of elastic $\mu\rightarrow e$ conversion. Crucially, one cannot determine the values of individual operators coefficients $c_i^{\tau}$, only the particular linear combinations specified by the leptonic response functions $R$. In Section \ref{sec:form_factors} we discuss situations in which the target-independence of the CLFV LECs is (weakly) violated.  

In contrast to the nucleon-level effective theory, we recognize the nuclear-scale effective theory whose operators are the six allowed nuclear responses with LECs given by the $R$ leptonic response functions. In a sense, the nuclear-scale effective theory is the most natural effective theory for studying elastic $\mu\rightarrow e$ conversion as it interfaces directly with the experiments.

In order to utilize our formalism to place constraints on the CLFV parameters, one must be able to evaluate the nuclear response functions for the chosen nuclear target. We will now demonstrate how this can be done within the paradigm of the nuclear shell model.
\section{Shell Model Evaluation of Nuclear Responses}
Calculation of the nuclear response functions defined in Eq. \ref{eq:w_response} requires the evaluation of the matrix elements
\begin{equation}
\braket{j_N||T_{J;\tau}(q_\mathrm{eff})||j_N},
\end{equation}
where $\ket{j_N}$ is a wave function for the nuclear ground state and $T_{J;\tau}$ is any of the basic single-nucleon operators in Eq. \ref{eq:single_nucleon_responses}. Here the isospin of the nuclear states has been suppressed. We will demonstrate how to compute these matrix elements using the nuclear shell-model. Restoring isospin explicitly, we may write the desired matrix element in terms of one which has been doubly-reduced in total angular momentum $J$ and total isospin $\tau$
\begin{equation}
\braket{j_N;TM_T||T_{J;\tau}(q_\mathrm{eff})||j_N;TM_T}=(-1)^{T-M_T}\left(\begin{array}{ccc}
T & \tau & T\\
-M_T & 0 & M_T
\end{array}\right)\braket{j_N;T|||T_{J;\tau}(q_\mathrm{eff})|||j_N;T}
\end{equation}
Each nuclear operator is a sum of single-nucleon operators
\begin{equation}
T_{J;\tau}(q)=\sum_{i=1}^AT_{J;\tau}(q\vec{x}_i),
\end{equation}
and therefore the doubly-reduced total nuclear matrix element can be expressed in terms of the doubly-reduced one-body density matrix $\rho_{J;\tau}(ab)$  (see Appendix \ref{app:density}) as
\begin{equation}
\braket{j_N;T|||T_{J;\tau}(q_\mathrm{eff})|||j_N;T}=\sum_{a,b}\rho_{J;\tau}(ab)\;\braket{n_a\left(\ell_a\;1/2\right)j_a;1/2|||T_{J;\tau}(q_\mathrm{eff})|||n_b\left(\ell_b\;1/2\right)j_b;1/2},
\end{equation}
where the sum extends over all single-particle \textit{orbits}, labeled by harmonic oscillator quantum numbers $n$, $\ell$ and $j$ but not $j_Z$. As $T_{J;\tau}(q_\mathrm{eff})=T_J(q_\mathrm{eff})t^\tau$, the single-particle isospin matrix elements are readily evaluated
\begin{equation}
\braket{1/2||t^{\tau}||1/2}=\left\{\begin{array}{cc}
\sqrt{2}, & \tau=0\\
\sqrt{6}, & \tau=1.
\end{array}\right.
\end{equation}
From Section \ref{sec:single_nucleon_operators} (and Appendix \ref{app:single_nucleon_response}), we know how to evaluate the matrix elements of $T_J$ between single-particle harmonic oscillator states in terms of the dimensionless quantity $y=(q_\mathrm{eff}b/2)^2$. All that remains is to obtain the density matrix
%\begin{equation}
%\rho_{J,\tau}(ab)=\frac{1}{\sqrt{(2J+1)(2\tau+1)}}\braket{j_N;T|||c^{\dag}_a\tilde{c}_b|||j_N;T},
%\end{equation}

For example, we model the nucleus $^{27}$Al using the $2s$-$1d$ valence space above an inert $^{16}$O core; that is, we assume that the 8 lowest-energy protons and neutrons are fixed in their orbitals whereas the remaining 5 protons and 6 neutrons can occupy any of the 12 states (for each species) in the $1d_{5/2}$-$2s_{1/2}$-$1d_{3/2}$ valence space, while respecting Pauli exclusion. The core, although trivial to model, is crucial to include as it contributes significantly to the coherent response. In total, 80,115 basis states are required to describe the state of the 11 valence nucleons. The ground-state configuration is determined by diagonalizing an effective Hamiltonian in the valence space. Table \ref{tab:nsm_params} reports the model space employed and the effective interactions available for the nuclear targets considered in this work. In addition to the shell structure, the harmonic oscillator model space is only fully specified once the oscillator length scale $b$ has been chosen. To this end, one may employ the well-known empirical formula
\begin{equation}
b=\sqrt{\frac{41.467}{45\bar{A}^{-1/3}-25\bar{A}^{-2/3}}}\;\mathrm{fm},
\label{eq:b_empirical}
\end{equation}
where $\bar{A}$ is the isotope-averaged nucleon number. In many cases, the effective shell model interactions have been tuned to reproduce nuclear charge radii as well as low-lying nuclear spectra. Alternatively to the empirical formula, one may also determine a value of $b$ which, in combination with the nuclear density obtained from the shell-model, accurately reproduces the known nuclear charge radius. These independent determinations of the oscillator parameter $b$ are compared in Table \ref{tab:nsm_params}. The agreement is generally quite good.

As the size of the shell-model basis grows combinatorially in the number of states and particles -- the basis size for $^{56}$Fe is 501,113,392 -- it quickly becomes infeasible to diagonalize the Hamiltonian directly. Fortunately, one can obtain a converged result for the ground-state nuclear wave function using the iterative Lanczos algorithm. In particular, we employ the massively-parallel Lanczos-algorithm code BIGSTICK \cite{2013CoPhC.184.2761J,2018arXiv180108432J} in order to obtain the ground-state of the chosen effective interaction within the model space. BIGSTICK can also directly output the one-body density matrices required in the evaluation of the nuclear response functions.
\begin{table}
\centering
\begin{tabular}{|c|c|c|c|c|c|}
\hline
Target & Isotopes & SM Space & Interactions & $b$ (fm) & $y=(q_\mathrm{eff}b/2)^2$\\
\hline
C & 12,13 & $1p$ & \cite{COHEN19651} & 1.67/1.70 & 0.21/0.22\\
O & 16,18 & $2s$-$1d$ & \cite{PhysRevC.74.034315,bw} & 1.73/1.83 & 0.23/0.26\\
  &       & $4\hbar\omega$ & \cite{PhysRevLett.65.1325} & 1.73/1.80 & 0.23/0.25\\
F & 19    & $2s$-$1d$ & \cite{PhysRevC.74.034315,bw} & 1.76/1.88 & 0.24/0.27\\
Na & 23 & $2s$-$1d$ & '' & 1.80/1.83 & 0.25/0.26\\
Al & 27 & $2s$-$1d$ & '' & 1.84/1.85 & 0.27/0.27\\
Si & 28-30 & $2s$-$1d$ & '' & 1.85/1.89 & 0.27/0.28\\
S & 32-34 & $2s$-$1d$ & '' & 1.88/1.91 & 0.28/0.29\\
Ca & 40,42,44 & $2p$-$1f$ & \cite{kb3g,gxpf1,kbp} & 1.94/2.02 & 0.30/0.33\\
Ti & 46-50 & $2p$-$1f$ & '' & 1.99/2.09 & 0.32/0.35\\
Fe & 54,56-58 & $2p$-$1f$ & '' & 2.03/2.08 & 0.34/0.36\\
Cu & 63,65 & $1f_\frac{5}{2}$-$2p$-$1g_\frac{9}{2}$ & \cite{jun45,jj44b,gcn2850} & 2.07/2.12 & 0.35/0.37\\
\hline
\end{tabular}
\caption{Nuclear shell-model spaces employed and available effective interactions. The first entry for the oscillator parameter $b$ is calculated from the empirical formula Eq. \ref{eq:b_empirical}. The second entry is the value of $b$ which reproduces the measured nuclear charge radius. These values of $b$ are then used in calculation of the corresponding value of $y$.}
\label{tab:nsm_params}
\end{table}
\section{Nuclear Multipole Power Counting}
Frequently when one performs a multipole decomposition, the expansion converges rapidly enough that the desired accuracy can be achieved by retaining only the first few leading multipoles. In this section, we demonstrate that this is explicitly \textit{not} the case in elastic $\mu\rightarrow e$ conversion. As demonstrated by Eq. \ref{eq:multipole_polynomial}, the parameter $y=\left(bq_\mathrm{eff}/2\right)^2$ governs the convergence of the nuclear multipole expansion. For $\mu\rightarrow e$ conversion in the nuclei of interest, the effective three-momentum transfer is of order the muon mass $q_\mathrm{eff}\sim m_{\mu}$ and the nuclear oscillator parameter $b\sim 2$ fm, and therefore $y\approx 0.3$. The exact value of $y$ for each nuclear target may be found in Table \ref{tab:nsm_params}. 

Not only is the dimensionless parameter $y$ large enough to warrant the inclusion of terms beyond the leading multipole, but it turns out that any truncation scheme in $y$ is an uncontrolled approximation. Eq. \ref{eq:multipole_polynomial} describes the dependence on $y$ of single-particle matrix elements. As discussed in the previous section, the total nuclear matrix elements can be decomposed into a sum of single-particle matrix elements multiplied by the corresponding one-body density matrix factor. The details of nuclear structure which are encoded in the density matrix determine the overall $y$-dependence of the nuclear response functions $W(y)$ and may significantly enhance or suppress contributions from higher-$J$ multipoles. One frequently exploited example of this is the coherent enhancement of the isoscalar monopole operator $M_{0;0}$. Let us specialize for the moment to $^{27}$Al. At the level of single-particle matrix elements, one naively expects the quadrupole operator $M_{2;0}$ to be suppressed by $y\approx 0.27$ relative to the monopole. Due to coherent enhancement of the monopole, however, the total nuclear matrix element of $M_{2;0}$ is suppressed by roughly an additional order of magnitude relative to the monopole $\braket{j_N||M_{2;0}(q_\mathrm{eff})||j_N}/\braket{j_N||M_{0;0}(q_\mathrm{eff})||j_N}\approx 0.027$. Thus in the case of coherent conversion, one is justified in retaining only the leading multipole; the resulting error in the $\mu\rightarrow e$ rate is less than $0.1\%$.

The isoscalar charge operator is very much a special case. Let us repeat the previous exercise with the corresponding isovector operators. Now there is no coherent enhancement, and the quadrupole operator $M_{2;1}$ is suppressed by roughly the expected factor of $y$: $\braket{j_N||M_{2;1}(q_\mathrm{eff})||j_N}/\braket{j_N||M_{0;1}(q_\mathrm{eff})||j_N}\approx -0.23$. Truncating the multipole expansion at leading order then introduces a 4.8\% error in the conversion rate. A point of caution should be made in regard to calculations of coherent $\mu\rightarrow e$ conversion in the literature. In many works specialized to coherent conversion \cite{PhysRevD.20.1608,Czarnecki_1998,PhysRevD.66.096002,cirigliano2009,crivellin2017,2018PhRvC..98a5208B,2022arXiv220300702H,Cirigliano:2022ekw}, the isovector operator is retained even though the multipole expansion is truncated at leading order. The neglect of higher multipoles should be justified when the isoscalar component is significant enough to provide a coherent enhancement, for example if the coupling is purely to either protons or neutrons, $(1\pm\tau_3)/2$. However, in cases where the nuclear ground state carries sufficient angular momentum $j_N\geq 1$, one should not trust the result of these calculations when the coupling becomes nearly isovector, the coherent enhancement diminishes, and higher multipoles become significant.

The impact of higher order multipoles is also significant for spin-dependent nuclear operators. Among the 11 targets which we focus on in this work, there are three - Na, Al, and Cu - with only odd isotopes, an unpaired proton, and a ground-state angular momentum $j_N\geq \frac{3}{2}$ so that more than one multipole operator contributes to the total response function. Consider the spin-dependent interaction $\mathcal{O}_4=\vec{\sigma}_L\cdot\vec{\sigma}_N$ which generates the $\Sigma'_J$ and $\Sigma''_J$ response functions comprised of odd $J$ multipoles, and suppose that the isospin coupling is $(1+\tau_3)/2$ so that the operator couples only to the unpaired proton. Evaluating the $\mu\rightarrow e$ decay rate first retaining only the leading $J=1$ multipoles and then with all contributing multipoles, we find rate increases of 22.4\% in Na, 4.7\% in Al, and 65.4\% in Cu. In Al, the truncation of the multipole expansion has produced only a modest error in the decay rate whereas in Cu the error is nearly $o(1)$.

%Table \ref{tab:higher_multi_al27} shows the relative size of the nuclear matrix elements of the six multipole operators contributing to elastic $\mu\rightarrow e$ conversion in $^{27}$Al. In the case of the isoscalar charge multipoles, the $M_2$ and $M_4$ operators are highly suppressed relative to the leading coherent $M_0$ operator. This is a direct result of the coherence: the $M_0$ operator receives contributions from all $A$ nucleons whereas the $M_2$ and $M_4$ cannot couple to the core nucleons. Therefore, if one restricts to the case where the dominant response is the isoscalar nuclear charge, then it is well-justified to retain only the isoscalar $M_0$ operator. This is an essential step in treatments of coherent $\mu\rightarrow e$ conversion which employ exact numerical Dirac wave functions for the leptons; the $M_0$ multipole is relatively simple to compute whereas higher order mutipoles would require considerable angular momentum algebra. To our knowledge, these corrections have never been computed exactly.

%\begin{table}
%\centering
%\begin{tabular}{|c|cccccc|}
%\hline
% &~~~0~~~&~~~1~~~&~~~2~~~&~~~3~~~&~~~4~~~&~~~5~~~\\
% \hline
% $|\braket{J_i||M_{J;0}(q_\mathrm{eff})||J_i}|^2/|\braket{J_i||M_{0;0}(q_\mathrm{eff})||J_i}|^2$ & 1 & - & $7.2\times 10^{-4}$ & - & $3.4\times 10^{-7}$ & - \\
%  $|\braket{J_i||M_{J;1}(q_\mathrm{eff})||J_i}|^2/|\braket{J_i||M_{0;1}(q_\mathrm{eff})||J_i}|^2$ & 1 & - & 0.046 & - & $7.5\times 10^{-5}$ & - \\
%   $|\braket{J_i||\Delta_{J;0}(q_\mathrm{eff})||J_i}|^2/|\braket{J_i||\Delta_{1;0}(q_\mathrm{eff})||J_i}|^2$ & - & 1 & - & 0.0024 & - & 0 \\
%  $|\braket{J_i||\Delta_{J;1}(q_\mathrm{eff})||J_i}|^2/|\braket{J_i||\Delta_{1;1}(q_\mathrm{eff})||J_i}|^2$ & - & 1 & - & 0.0045 & - & 0 \\
%   $|\braket{J_i||\Sigma'_{J;0}(q_\mathrm{eff})||J_i}|^2/|\braket{J_i||\Sigma'_{1;0}(q_\mathrm{eff})||J_i}|^2$ & - & 1 & - & 0.039 & - & 0.0033 \\
%  $|\braket{J_i||\Sigma'_{J;1}(q_\mathrm{eff})||J_i}|^2/|\braket{J_i||\Sigma'_{1;1}(q_\mathrm{eff})||J_i}|^2$ & - & 1 & - & 0.078 & - & 0.0063 \\
%     $|\braket{J_i||\Sigma''_{J;0}(q_\mathrm{eff})||J_i}|^2/|\braket{J_i||\Sigma''_{1;0}(q_\mathrm{eff})||J_i}|^2$ & - & 1 & - & 0.030 & - & 0.0024 \\
%  $|\braket{J_i||\Sigma''_{J;1}(q_\mathrm{eff})||J_i}|^2/|\braket{J_i||\Sigma''_{1;1}(q_\mathrm{eff})||J_i}|^2$ & - & 1 & - & 0.038 & - & 0.0029 \\
%       $|\braket{J_i||\tilde{\Phi}'_{J;0}(q_\mathrm{eff})||J_i}|^2/|\braket{J_i||\tilde{\Phi}'_{2;0}(q_\mathrm{eff})||J_i}|^2$ & - & - & 1 & - & 260 & - \\
%  $|\braket{J_i||\tilde{\Phi}'_{J;1}(q_\mathrm{eff})||J_i}|^2/|\braket{J_i||\tilde{\Phi}'_{2;1}(q_\mathrm{eff})||J_i}|^2$ & - & - & 1 & - & 0.020 & \\
%         $|\braket{J_i||\Phi''_{J;0}(q_\mathrm{eff})||J_i}|^2/|\braket{J_i||\Phi''_{0;0}(q_\mathrm{eff})||J_i}|^2$ & 1 & - & $2.2\times 10^{-4}$ & - & $6.4\times 10^{-5}$ & \\
%  $|\braket{J_i||\Phi''_{J;1}(q_\mathrm{eff})||J_i}|^2/|\braket{J_i||\Phi''_{0;1}(q_\mathrm{eff})||J_i}|^2$ & 1 & - & 0.99 & - & 0.025 & \\
% \hline
%\end{tabular}
%\caption{Relative size of higher multipole contributions to elastic $\mu\rightarrow e$ conversion computed for $^{27}$Al.}
%\label{tab:higher_multi_al27}
%\end{table}
\section{The Muon Lower Component and Muon Velocity Operator}
\label{sec:muon_lower}
For $\kappa=-1$, the radial Dirac equation implies that
\begin{equation}
\frac{d}{dr}\left(\frac{G(r)}{r}\right)=\left(2\bar{m}-E^\mathrm{bind}_{\mu}-V(r)\right)\frac{F(r)}{r}.
\end{equation}
Just as we did for the electron, we may replace the Coulomb potential by an average value $\bar{V}_C$ and write
\begin{equation}
\frac{d}{dr}\left(\frac{G(r)}{r}\right)=2m^*F(r),
\end{equation}
where
\begin{equation}
m^*\equiv \bar{m}-\left(E^\mathrm{bind}_{\mu}+\bar{V}_C\right)/2
\end{equation}
is an effective muon mass which varies from the physical reduced muon mass $\bar{m}$ by $\approx 2.6$\% in $^{27}$Al. Using the fact that 
\begin{equation}
\Omega^1_{\frac{1}{2}m}(\hat{r})=-\vec{\sigma}\cdot\hat{r}\;Y_{00}(\hat{r})\xi_m
\end{equation}
we can express both components of the muon wave function in terms of the radial function $G(r)$ as
\begin{equation}
\begin{split}
\psi^{\mu}_{\kappa=-1}(\vec{r})&=\left(\begin{array}{c}
\xi_m\\
\frac{\vec{\sigma}\cdot\vec{p}_{\mu}}{2m^*}\xi_m
\end{array}\right)\frac{G(r)}{r}Y_{00}(\hat{r}),
\end{split}
\label{eq:muon_f_eff}
\end{equation}
where $\vec{p}_{\mu}$ is the muon momentum operator. Figure \ref{fig:muon_f_eff} compares the numerically obtained Coulomb solution $F(r)$ to the effective lower component wave function in Eq. \ref{eq:muon_f_eff} in four target nuclei. From the lightest nucleus, $^{12}$C, to the heaviest, $^{184}$W, the effective form of the lower component reproduces very well the reference solution; even in $^{184}$W, the relative root-mean-square error -- defined in analogy with Eq. \ref{eq:rms_error} -- is $\approx 1$\%. In the next-generation target $^{27}$Al, the RMS error is $\approx 0.5$\%. 

\begin{figure}
\centering
\includegraphics[scale=0.67]{muon_f_eff_combined.png}
\caption{Comparison of the exact result (green line) for the absolute value of the muon wave function lower component $|F(r)/r|$ and the effective wave function (blue dashed line) given by Eq. \ref{eq:muon_f_eff} in four target nuclei: $^{12}$C, $^{27}$Al, $^{63}$Cu, and $^{184}$W. The grey shaded region shows the extent of the nuclear density $r^2\rho(r)$ (arbitrary normalization). Note that the scale of the plot is different across the four nuclei. In all cases, the effective wave function provides a good approximation of the muon's lower component.} 
\label{fig:muon_f_eff}
\end{figure}
\begin{equation}
\begin{split}
e^{-i\vec{q}_\mathrm{eff}\cdot\vec{x}}\;\hat{q}\cdot\vec{v}_{\mu}\left(g(x)\right)&=-\frac{i}{m_{\mu}}e^{-i\vec{q}_{\mathrm{eff}}\cdot\vec{x}}\;\hat{q}\cdot\hat{x}\;\frac{dg}{dx}\\
&=\frac{1}{m_{\mu}q_\mathrm{eff}}\frac{dg}{dx}\;\hat{x}\cdot\vec{\nabla}\left(e^{-i\vec{q}_\mathrm{eff}\cdot\vec{x}}\right)\\
&=\frac{1}{m_{\mu}q_\mathrm{eff}}\sum_{J=0}^{\infty}\sqrt{4\pi(2J+1)}(-i)^J\;\frac{d}{dx}\left(j_J(q_\mathrm{eff}x)\right)\frac{dg}{dx}\;Y_{J0}(\hat{x})
\end{split}
\end{equation}
The new operators are
\begin{equation}
\begin{split}
\mathcal{O}_2^{f'}&=i\hat{q}\cdot\frac{\vec{v}_{\mu}}{2}\;1_N\\
\mathcal{O}_3^f&=i\hat{q}\cdot\left[\frac{\vec{v}_{\mu}}{2}\times\vec{\sigma}_L\right]\;1_N\\
\mathcal{O}_5^f&=\left(i\hat{q}\times\frac{\vec{v}_{\mu}}{2}\right)\cdot\vec{\sigma}_N\\
\mathcal{O}_7^f&=\frac{\vec{v}_{\mu}}{2}\cdot\vec{\sigma}_L\;1_N\\
\mathcal{O}_8^f&=\frac{\vec{v}_{\mu}}{2}\cdot\vec{\sigma}_N\\
\mathcal{O}_{12}^f&=\left[\frac{\vec{v}_{\mu}}{2}\times\vec{\sigma}_L\right]\cdot\vec{\sigma}_N\\
\mathcal{O}_{13}^{f'}&=\left(i\hat{q}\times\left[\frac{\vec{v}_{\mu}}{2}\times\vec{\sigma}_L\right]\right)\cdot\vec{\sigma}_N\\
\mathcal{O}_{14}^f&=\frac{\vec{v}_{\mu}}{2}\cdot\vec{\sigma}_L\;i\hat{q}\cdot\vec{\sigma}_N\\
\mathcal{O}_{15}^f&=i\hat{q}\cdot\left[\frac{\vec{v}_{\mu}}{2}\times\vec{\sigma}_L\right]\;i\hat{q}\cdot\vec{\sigma}_N\\
\mathcal{O}_{16}^{f'}&=i\hat{q}\cdot\frac{\vec{v}_{\mu}}{2}\;i\hat{q}\cdot\vec{\sigma}_N 
\end{split}
\end{equation}
\begin{equation}
\begin{split}
l_0^{\tau}&\rightarrow l_0^{\tau}+b_2^{\tau}i\hat{q}\cdot\frac{\vec{v}_{\mu}}{2}+b_3^{\tau}i\hat{q}\cdot\left[\frac{\vec{v}_{\mu}}{2}\times\vec{\sigma}_L\right]+b_7^{\tau}\frac{\vec{v}_{\mu}}{2}\cdot\vec{\sigma}_L\\
&\equiv l_0^{\tau}+l_{0f}^{\tau}\\
\vec{l}_5^{\tau}&\rightarrow \vec{l}_5^{\tau}+b_5^{\tau}i\hat{q}\times\frac{\vec{v}_{\mu}}{2}+b_8^{\tau}\frac{\vec{v}_{\mu}}{2}+b_{12}^{\tau}\frac{\vec{v}_{\mu}}{2}\times\vec{\sigma}_L+b_{13}^{\tau}i\hat{q}\times\left[\frac{\vec{v}_{\mu}}{2}\times\vec{\sigma}_L\right]\\&+b_{14}^{\tau}\frac{\vec{v}_{\mu}}{2}\cdot\vec{\sigma}_L\;i\hat{q}
+b_{15}^{\tau}i\hat{q}\cdot\left[\frac{\vec{v}_{\mu}}{2}\times\vec{\sigma}_L\right]i\hat{q}+b_{16}^{\tau}i\hat{q}\cdot\frac{\vec{v}_{\mu}}{2}\;i\hat{q}\\
&\equiv \vec{l}_5^{\tau}+\vec{l}_{5f}^{\tau}
\end{split}
\end{equation}
We find six new nuclear operators associated with the muon lower component
\begin{equation}
\begin{split}
M^{(1)}_{JM;\tau}(q)&\equiv \sum_{i=1}^A\sqrt{J(J+1)}\frac{1}{qx_i}j_J(qx_i)Y_{JM}(\hat{x}_i)\;t^{\tau}(i)\\
M^{(2)}_{JM;\tau}(q)&\equiv \sum_{i=1}^A\frac{d}{dqx_i}j_J(qx_i)Y_{JM}(\hat{x}_i)\;t^{\tau}(i)\\
\Sigma^{'(0)}_{JM;\tau}(q)&\equiv \sum_{i=1}^Aj_J(qx_i)\left[\sqrt{\frac{J}{2J+1}}\vec{Y}_{JJ+1M}(\hat{x}_i)+\sqrt{\frac{J+1}{2J+1}}\vec{Y}_{JJ-1M}(\hat{x}_i)\right]\cdot\vec{\sigma}_N(i)\;t^{\tau}(i)\\
\Sigma^{'(2)}_{JM;\tau}(q)&\equiv\sum_{i=1}^A\left[-\sqrt{\frac{J}{2J+1}}\frac{d}{dqx_i}j_{J+1}(qx_i)\vec{Y}_{JJ+1M}(\hat{x}_i)+\sqrt{\frac{J+1}{2J+1}}\frac{d}{dqx_i}j_{J-1}(qx_i)\vec{Y}_{JJ-1M}(\hat{x}_i)\right]\cdot\vec{\sigma}_N(i)\;t^{\tau}(i)\\
\Sigma^{''(0)}_{JM;\tau}(q)&\equiv\sum_{i=1}^Aj_J(qx_i)\left[-\sqrt{\frac{J+1}{2J+1}}\vec{Y}_{JJ+1M}(\hat{x}_i)+\sqrt{\frac{J}{2J+1}}\vec{Y}_{JJ-1M}(\hat{x}_i)\right]\cdot\vec{\sigma}_N(i)\;t^{\tau}(i)\\
\Sigma^{''(2)}_{JM;\tau}(q)&\equiv\sum_{i=1}^A\left[\sqrt{\frac{J+1}{2J+1}}\frac{d}{dqx_i}j_{J+1}(qx_i)\vec{Y}_{JJ+1M}(\hat{x}_i)+\sqrt{\frac{J}{2J+1}}\frac{d}{dqx_i}j_{J-1}(qx_i)\vec{Y}_{JJ-1M}(\hat{x}_i)\right]\cdot\vec{\sigma}_N(i)\;t^{\tau}(i)
\end{split}
\end{equation}
\section{Form Factors and LEC-independence}
\label{sec:form_factors}
The momentum-transfer-squared $q^2$ is a completely-invariant scalar quantity and so for every operator $\mathcal{O}$ which is allowed by symmetries, then so is $q^{2n}\mathcal{O}$ for every $n\geq 0$. Therefore we can group all such operators using a form factor
\begin{equation}
c_0\mathcal{O}+c_2 q^2\mathcal{O}+c_4q^4\mathcal{O}+...=F_{\mathcal{O}}\left(\frac{q^2}{\Lambda^2}\right)\mathcal{O}
\end{equation}
Massless mediators can be added by including $1/q^2$ terms. The form factor associated with one-pion exchange is the pion propagator
\begin{equation}
F_{\pi}(
\end{equation}
The leading order $\pi NN$ interaction is given by
\begin{equation}
\begin{split}
\mathcal{L}&=-\frac{\mathring{g}_A}{\mathring{f}_{\pi}}\bar{\mathcal{N}}_V S^{\mu}_V\vec{\tau}\cdot\partial_{\mu}\vec{\pi}\mathcal{N}_V\\
&=-\frac{i}{2}\frac{\mathring{g}_A}{\mathring{f}_{\pi}}q_i\pi_a\bar{\mathcal{N}}_V\left(\begin{array}{cc}
\sigma^i & 0 \\
0 & \sigma^i
\end{array}\right)\tau^a\mathcal{N}_V
\end{split}
\end{equation}
In the single-pion exchange of diagram \ref{fig:mm_decay_diagrams}, the exchanged pion must be neutral $\pi^0$ as the leptons conserve charge. The leptons can couple to the pion through non-derivative 
\begin{equation}
\bar{\psi}_e\psi_{\mu}\pi^0,\;\bar{\psi}_e\gamma^5\psi_{\mu}\pi^0,
\end{equation}
or derivative couplings
\begin{equation}
\bar{\psi}_e\gamma^{\mu}\psi_{\mu}\partial_{\mu}\pi^0,\;\bar{\psi}_e\gamma^{\mu}\gamma^5\psi_{\mu}\partial_{\mu}\pi^0
\end{equation}
Thus the form of the nuclear current is fixed, in fact it must be the axial current $\vec{J}_A(\vec{x})$.

\section{Connection to Relativistic Amplitudes}
The relativistic $\mu$-to-$e$ nucleon-level interactions are constructed from the available leptonic scalar
\begin{equation}
\bar{\chi}_e\chi_{\mu},\;\bar{\chi}_ei\gamma^5\chi_{\mu}
\end{equation}
and four-vector
\begin{equation}
\bar{\chi}_e\gamma^{\mu},\;\bar{\chi}_ei\sigma^{\mu\nu}\frac{q_{\nu}}{m_L}\chi_{\mu},\;\bar{\chi}_e\gamma^{\mu}\gamma^5\chi_{\mu},\;\bar{\chi}_e\sigma^{\mu\nu}\frac{q_{\nu}}{m_L}\gamma^5\chi_{\mu}
\end{equation}
amplitudes, contracting these with their nucleon counterparts to form all possible bi-linear scalars. Note that any propagator effects can be absorbed into the LECs because of the fixed momentum transfer $q$ flowing through the propagator. The nonrelativistic nuclear reduction neglects terms in $\vec{v}_N^2\sim\frac{1}{100}$ and $\frac{q^2}{m_N^2}\sim\frac{1}{100}$.

In this reduction four of our effective Pauli operators, $\mathcal{O}_3$, $\mathcal{O}_{12}$, $\mathcal{O}_{13}'$, and $\mathcal{O}_{15}$, play no role. These are the interactions associated with the nuclear spin-velocity three-current $\vec{v}_N\times\vec{\sigma}_N$, which cannot be generated by scalar or vector couplings to the nuclear, but will arise for more general mediators. For example, the pseudotensor interaction
\begin{equation}
\bar{\chi}_ei\sigma^{\mu\nu}\gamma^5\chi_{\mu}\bar{N}i\sigma_{\mu\nu}\gamma^5N
\end{equation}
reduces to the Pauli form
\begin{equation}
\begin{split}
&\frac{q}{m_N}1_L1_N+2i\hat{q}\cdot\left[\vec{v}_N\times\vec{\sigma}_N\right]-2\vec{\sigma}_L\cdot\vec{\sigma}_N-2\vec{\sigma}_L\cdot\left(\hat{q}\times\left[\vec{v}_N\times\vec{\sigma}_N\right]\right)\\
&=\frac{q}{m_N}\mathcal{O}_1+2\mathcal{O}_3-2\mathcal{O}_4+2i\mathcal{O}_{13}'
\end{split}
\end{equation}
generating $\mathcal{O}_3$ and $\mathcal{O}_{13}'$. The four operators associated with $\vec{v}_N\times\vec{\sigma}_N$ should be retained to ensure the EFT formulation is general.
\begin{table}
\centering
{\renewcommand{\arraystretch}{1.5}
\begin{tabular}{|c|c|c|c|c|}
\hline
\hline
$j$ & $\mathcal{L}^j_\mathrm{int}$ & Pauli Reduction & $\sum_ic_i\mathcal{O}_i$ & P/T\\
\hline
1 & $\bar{\chi}_e\chi_{\mu} \bar{N}N$ & $1_L1_N$ & $\mathcal{O}_1$ & E/E\\
2 & $\bar{\chi}_e\chi_{\mu} \bar{N}i\gamma^5N$ & $1_L\left(i\frac{\vec{q}}{2m_N}\cdot\vec{\sigma}_N\right)$ & $\frac{q}{2m_N}\mathcal{O}_{10}$ & O/O\\
3 & $\bar{\chi}_ei\gamma^5\chi_{\mu} \bar{N}N$ & $\left(-i\hat{q}\cdot\vec{\sigma}_L\right)1_N$ & $-\mathcal{O}_{11}$ & O/O\\
4 & $\bar{\chi}_ei\gamma^5\chi_{\mu} \bar{N}i\gamma^5N$ & $\left(-i\hat{q}\cdot\vec{\sigma}_L\right)\left(i\frac{\vec{q}}{2m_N}\cdot\vec{\sigma}_N\right)$ & $-\frac{q}{2m_N}\mathcal{O}_6$ & E/E\\
5 & $\bar{\chi}_e\gamma^{\mu}\chi_{\mu}\bar{N}\gamma_{\mu}N$ & $1_L1_N$ & $\mathcal{O}_1$ & E/E\\
 & & $-\left(\hat{q}1_L-i\hat{q}\times\vec{\sigma}_L\right)\cdot\left(\vec{v}_N+i\frac{\vec{q}}{2m_N}\times\vec{\sigma}_N\right)$ & $+i\mathcal{O}_2'-\mathcal{O}_5-\frac{q}{2m_N}\left(\mathcal{O}_4+\mathcal{O}_6\right)$ & \\
6 & $\bar{\chi}_e\gamma^{\mu}\chi_{\mu}\bar{N}i\sigma_{\mu\alpha}\frac{q^{\alpha}}{m_N}N$ & $-\left(\hat{q}1_L-i\hat{q}\times\vec{\sigma}_L\right)\cdot\left(-i\frac{\vec{q}}{m_N}\times\vec{\sigma}_N\right)$ & $\frac{q}{m_N}\left(\mathcal{O}_4+\mathcal{O}_6\right)$ & E/E\\
7 & $\bar{\chi}_e\gamma^{\mu}\chi_{\mu}\bar{N}\gamma_{\mu}\gamma^5N$ & $1_L\left(\vec{v}_N\cdot\vec{\sigma}_N\right)-\left(\hat{q}1_L-i\hat{q}\times\vec{\sigma}_L\right)\cdot\vec{\sigma}_N$ & $\mathcal{O}_7+i\mathcal{O}_{10}-\mathcal{O}_9$ & O/E\\
8 & $\bar{\chi}_e\gamma^{\mu}\chi_{\mu}\bar{N}\sigma_{\mu\alpha}\frac{q^{\alpha}}{m_N}\gamma^5N$ & $1_L\left(-i\frac{\vec{q}}{m_N}\cdot\vec{\sigma}_N\right)$ & $-\frac{q}{m_N}\mathcal{O}_{10}$ & O/O\\
9 & $\bar{\chi}_ei\sigma^{\mu\nu}\frac{q_{\nu}}{m_L}\chi_{\mu}\bar{N}\gamma_{\mu}N$ & $-\frac{q}{m_L}1_L1_N$ & $-\frac{q}{m_L}\mathcal{O}_{1}$ & E/E\\
 & & $-\left(-i\frac{\vec{q}}{m_N}\times\vec{\sigma}_L\right)\cdot\left(\vec{v}_N+i\frac{\vec{q}}{2m_N}\times\vec{\sigma}_N\right)$ & $-\frac{q}{m_L}\left(\mathcal{O}_5+\frac{q}{2m_N}(\mathcal{O}_4+\mathcal{O}_6)\right)$ & \\
10 & $\bar{\chi}_ei\sigma^{\mu\nu}\frac{q_{\nu}}{m_L}\chi_{\mu}\bar{N}i\sigma_{\mu\alpha}\frac{q^{\alpha}}{m_N}N$ & $-\left(-i\frac{\vec{q}}{m_L}\times\vec{\sigma}_L\right)\cdot\left(-i\frac{\vec{q}}{m_N}\times\vec{\sigma}_N\right)$ & $\frac{q}{m_L}\frac{q}{m_N}\left(\mathcal{O}_4+\mathcal{O}_6\right)$ & E/E\\
11 & $\bar{\chi}_ei\sigma^{\mu\nu}\frac{q_{\nu}}{m_L}\chi_{\mu}\bar{N}\gamma_{\mu}\gamma^5N$ & $\left(-\frac{q}{m_L}1_L\right)\vec{v}_N\cdot\vec{\sigma}_N-\left(-i\frac{\vec{q}}{m_L}\times\vec{\sigma}_L\right)\cdot\vec{\sigma}_N$ & $-\frac{q}{m_L}\left(\mathcal{O}_7+\mathcal{O}_9\right)$ & O/E\\
12 & $\bar{\chi}_ei\sigma^{\mu\nu}\frac{q_{\nu}}{m_L}\chi_{\mu}\bar{N}\sigma_{\mu\alpha}\frac{q^{\alpha}}{m_N}\gamma^5N$ & $\left(-\frac{q}{m_L}1_L\right)\left(-i\frac{\vec{q}}{m_N}\cdot\vec{\sigma}_N\right)$ & $\frac{q}{m_L}\frac{q}{m_N}\mathcal{O}_{10}$ & O/O\\
13 & $\bar{\chi}_e\gamma^{\mu}\gamma^5\chi_{\mu}\bar{N}\gamma_{\mu}N$ & $(\hat{q}\cdot\vec{\sigma}_L)1_N-\vec{\sigma}_L\cdot\left(\vec{v}_N+i\frac{\vec{q}}{2m_N}\times\vec{\sigma}_N\right)$ & $-i\mathcal{O}_{11}-\mathcal{O}_8-\frac{q}{2m_N}\mathcal{O}_9$ & O/E\\
14 & $\bar{\chi}_e\gamma^{\mu}\gamma^5\chi_{\mu}\bar{N}i\sigma_{\mu\alpha}\frac{q^{\alpha}}{m_N}N$ & $-\vec{\sigma}_L\cdot\left(-i\frac{\vec{q}}{m_N}\times\vec{\sigma}_N\right)$ & $\frac{q}{m_N}\mathcal{O}_9$ & O/E\\
15 & $\bar{\chi}_e\gamma^{\mu}\gamma^5\chi_{\mu}\bar{N}\gamma_{\mu}\gamma^5N$ & $(\hat{q}\cdot\vec{\sigma}_L)(\vec{v}_N\cdot\vec{\sigma}_N)-\vec{\sigma}_L\cdot\vec{\sigma}_N$ & $-i\mathcal{O}_{14}-\mathcal{O}_4$ & E/E\\
16 & $\bar{\chi}_e\gamma^{\mu}\gamma^5\chi_{\mu}\bar{N}\sigma_{\mu\alpha}\frac{q^{\alpha}}{m_N}\gamma^5 N$ & $(\hat{q}\cdot\vec{\sigma}_L)\left(-i\frac{\vec{q}}{m_N}\cdot\vec{\sigma}_N\right)$ & $i\frac{q}{m_N}\mathcal{O}_6$ & E/O\\
17 & $\bar{\chi}_e\sigma^{\mu\nu}\frac{q_{\nu}}{m_L}\gamma^5\chi_{\mu}\bar{N}\gamma_{\mu}N$ & $\left(-i\frac{\vec{q}}{m_L}\cdot\vec{\sigma}_L\right)1_N$ & $-\frac{q}{m_L}\mathcal{O}_{11}$ & O/O\\
 & & $-i\frac{q}{m_L}\left(\vec{\sigma}_L-\hat{q}\hat{q}\cdot\vec{\sigma}_L\right)\cdot\left(\vec{v}_N+i\frac{\vec{q}}{2m_N}\times\vec{\sigma}_N\right)$ & $-\frac{q}{m_L}\left(i\mathcal{O}_8+i\frac{q}{2m_N}\mathcal{O}_9+i\mathcal{O}_{16}'\right)$ & \\
 18 & $\bar{\chi}_e\sigma^{\mu\nu}\frac{q_{\nu}}{m_L}\gamma^5\chi_{\mu}\bar{N}i\sigma_{\mu\alpha}\frac{q^{\alpha}}{m_N}N$ & $-i\frac{q}{m_L}\left(\vec{\sigma}_L-\hat{q}\hat{q}\cdot\vec{\sigma}_L\right)\cdot\left(-i\frac{\vec{q}}{m_N}\times\vec{\sigma}_N\right)$ & $i\frac{q}{m_L}\frac{q}{m_N}\mathcal{O}_9$ & O/O\\
19 & $\bar{\chi}_e\sigma^{\mu\nu}\frac{q_{\nu}}{m_L}\gamma^5\chi_{\mu}\bar{N}\gamma_{\mu}\gamma^5 N$ & $\left(-i\frac{\vec{q}}{m_L}\cdot\vec{\sigma}_L\right)\left(\vec{v}_N\cdot\vec{\sigma}_N\right)$ & $-\frac{q}{m_L}\mathcal{O}_{14}$ & E/O\\
 & & $-i\frac{q}{m_L}\left(\vec{\sigma}_L-\hat{q}\hat{q}\cdot\vec{\sigma}_L\right)\cdot\vec{\sigma}_N$ & $-\frac{q}{m_L}\left(i\mathcal{O}_4+i\mathcal{O}_6\right)$ & \\
20 & $\bar{\chi}_e\sigma^{\mu\nu}\frac{q_{\nu}}{m_L}\gamma^5\chi_{\mu}\bar{N}\sigma_{\mu\alpha}\frac{q^{\alpha}}{m_N}\gamma^5N$ & $\left(-i\frac{\vec{q}}{m_L}\cdot\vec{\sigma}_L\right)\left(-i\frac{\vec{q}}{m_N}\cdot\vec{\sigma}_N\right)$ & $\frac{q}{m_L}\frac{q}{m_N}\mathcal{O}_6$ & E/E\\
\hline
\hline
\end{tabular}}
\caption{Relativistic $\mu$-to-$e$ conversion amplitudes $\mathcal{L}_\mathrm{int}^j$, the corresponding linear combinations of the $\mathcal{O}_i$ resulting from the Pauli reduction, and the transformation properties of the interactions (even E or odd O) under parity and time reversal. Bjorken and Drell spinor and gamma matrix conventions are used. An overall factor of $\sqrt{\frac{E_e}{2m_e}}|R_{1s}^\mathrm{Z_{eff}}(0)|$ has been removed from the third column.}
\label{tab:operator_list}
\end{table}
\begin{table}
\centering
{\renewcommand{\arraystretch}{1.5}
\begin{tabular}{|c|c|c|c|c|}
\hline
\hline
$j$ & $\mathcal{L}^j_\mathrm{int}$ & Pauli Reduction & $\sum_i c_i\mathcal{O}_i$ & P/T\\
\hline
21 & $\bar{\chi}_e\sigma^{\mu\nu}\chi_{\mu}\bar{N}\sigma_{\mu\nu}N$ & $-\frac{q}{m_N}1_L1_N$ & $-\frac{q}{m_N}\mathcal{O}_1$ & E/E\\
 & & $-2i\hat{q}\cdot\left[\vec{v}_N\times\vec{\sigma}_N\right]+2\vec{\sigma}_L\cdot\vec{\sigma}_N+2\vec{\sigma}_L\cdot\left(\hat{q}\times\left[\vec{v}_N\times\vec{\sigma}_N\right]\right)$ & $-2\mathcal{O}_3+2\mathcal{O}_4-2i\mathcal{O}_{13}'$ &\\
 22 & $\bar{\chi}_ei\sigma^{\mu\nu}\gamma^5\chi_{\mu}\bar{N}\sigma_{\mu\nu}N$ & $2i\hat{q}\cdot\vec{\sigma}_N-2\vec{\sigma}_L\cdot\left(i\hat{q}\times\vec{\sigma}_N\right)$ & $2\left(\mathcal{O}_{10}-i\mathcal{O}_9\right)$ & O/O \\
  & & $+i\frac{\vec{q}}{m_N}\cdot\vec{\sigma}_L-2\vec{\sigma}_L\cdot\left(\vec{v}_N\times\vec{\sigma}_N\right)$ & $+\frac{q}{m_N}\mathcal{O}_{11}-2\mathcal{O}_{12}$ &\\
 \hline
 \hline
\end{tabular}}
\end{table}
\begin{table}
\centering
{\renewcommand{\arraystretch}{1.5}
\begin{tabular}{|c|c|c|c|}
\hline
\hline
$j$ & $\mathcal{L}^j_\mathrm{int}$ & Pauli Reduction & $\sum_ic_i\mathcal{O}_i$\\
\hline
1 & $\bar{\chi}_e\chi_{\mu} \bar{N}N$ & $-\frac{1}{2}\hat{q}\cdot\vec{v}_{\mu}1_N-\frac{i}{2}\hat{q}\cdot\left[\vec{v}_{\mu}\times\vec{\sigma}_L\right]1_N$ & $i\mathcal{O}^{f'}_{2}-\mathcal{O}^f_3$ \\
3 & $\bar{\chi}_ei\gamma^5\chi_{\mu} \bar{N}N$ & $\frac{i}{2}\vec{v}_\mu\cdot\vec{\sigma}_L1_N$ & $i\mathcal{O}^f_{7}$ \\
5 & $\bar{\chi}_e\gamma^{\mu}\chi_{\mu}\bar{N}\gamma_{\mu}N$ & $\frac{1}{2}\hat{q}\cdot\vec{v}_{\mu}1_N+\frac{i}{2}\hat{q}\cdot\left[\vec{v}_{\mu}\times\vec{\sigma}_L\right]1_N$ & $-i\mathcal{O}^{f'}_2+\mathcal{O}^f_3$ \\
7 & $\bar{\chi}_e\gamma^{\mu}\chi_{\mu}\bar{N}\gamma_{\mu}\gamma^5N$  & $-\frac{1}{2}\vec{v}_{\mu}\cdot\vec{\sigma}_N-\frac{i}{2}\left[\vec{v}_{\mu}\times\vec{\sigma}_L\right]\cdot\vec{\sigma}_N$ & $-\mathcal{O}^f_8-i\mathcal{O}^f_{12}$ \\
9 & $\bar{\chi}_ei\sigma^{\mu\nu}\frac{q_{\nu}}{m_L}\chi_{\mu}\bar{N}\gamma_{\mu}N$ & $\frac{q}{2m_L}\left(\hat{q}\cdot\vec{v}_{\mu}+i\hat{q}\cdot\left[\vec{v}_{\mu}\times\vec{\sigma}_L\right]\right)1_N$ & $\frac{q}{m_L}\left(-i\mathcal{O}^{f'}_2+\mathcal{O}^f_3\right)$ \\
11 & $\bar{\chi}_ei\sigma^{\mu\nu}\frac{q_{\nu}}{m_L}\chi_{\mu}\bar{N}\gamma_{\mu}\gamma^5N$ & $\frac{q}{2m_L}\left(\vec{v}_{\mu}\cdot\vec{\sigma}_N+i\left[\vec{v}_{\mu}\times\vec{\sigma}_L\right]\cdot\vec{\sigma}_N\right.$ & $\frac{q}{m_L}\left(\mathcal{O}_8^f+i\mathcal{O}_{12}^f\right.$ \\
  & & $\left.-i\vec{q}\cdot\left[\vec{v}_{\mu}\times\vec{\sigma}_L\right]\vec{q}\cdot\vec{\sigma}_N-\hat{q}\cdot\vec{v}_{\mu}\hat{q}\cdot\vec{\sigma}_N\right)$ & $\left.+i\mathcal{O}_{15}^f+\mathcal{O}_{16}^{f'}\right)$ \\
13 & $\bar{\chi}_e\gamma^{\mu}\gamma^5\chi_{\mu}\bar{N}\gamma_{\mu}N$ & $\frac{1}{2}\vec{v}_{\mu}\cdot\vec{\sigma}_L1_N$ & $\mathcal{O}^f_7$ \\
15 & $\bar{\chi}_e\gamma^{\mu}\gamma^5\chi_{\mu}\bar{N}\gamma_{\mu}\gamma^5N$ & $\frac{i}{2}\left[\hat{q}\times\vec{v}_{\mu}\right]\cdot\vec{\sigma}_N-\frac{1}{2}\left(\hat{q}\times\left[v_{\mu}\times\vec{\sigma}_L\right]\right)\cdot\vec{\sigma}_N$ & $\mathcal{O}_5^f+i\mathcal{O}_{13}^{f'}$ \\
& & $-\frac{1}{2}\vec{v}_{\mu}\cdot\vec{\sigma}_L\;\hat{q}\cdot\vec{\sigma}_N$ & $+i\mathcal{O}_{14}^f$\\
17 & $\bar{\chi}_e\sigma^{\mu\nu}\frac{q_{\nu}}{m_L}\gamma^5\chi_{\mu}\bar{N}\gamma_{\mu}N$ & $\frac{iq}{2m_L}\vec{v}_{\mu}\cdot\vec{\sigma}_L1_N$ & $\frac{iq}{m_L}\mathcal{O}^f_7$ \\
19 & $\bar{\chi}_e\sigma^{\mu\nu}\frac{q_{\nu}}{m_L}\gamma^5\chi_{\mu}\bar{N}\gamma_{\mu}\gamma^5 N$ & $\frac{q}{2m_L}\left(\left[\hat{q}\times\vec{v}_{\mu}\right]\cdot\vec{\sigma}_N+\left(i\hat{q}\times\left[\vec{v}_{\mu}\times\vec{\sigma}_L\right]\right)\cdot\vec{\sigma}_N\right)$ & $\frac{q}{m_L}\left(-i\mathcal{O}^f_5+\mathcal{O}^f_{13}\right)$ \\
\hline
\hline
\end{tabular}}
\caption{Relativistic $\mu$-to-$e$ conversion amplitudes $\mathcal{L}_\mathrm{int}^j$, the corresponding linear combinations of the $\mathcal{O}_i$ resulting from the Pauli reduction, and the transformation properties of the interactions (even E or odd O) under parity and time reversal. Bjorken and Drell spinor and gamma matrix conventions are used. An overall factor of $\sqrt{\frac{E_e}{2m_e}}|R_{1s}^\mathrm{Z_{eff}}(0)|$ has been removed from the third column.}
\label{tab:operator_list}
\end{table}
\chapter{Nuclear Response Function Properties}
The factorized form of the conversion rate implies a clear path to extracting all of the information about CLFV operators which can be probed in elastic $\mu\rightarrow e$ conversion. Indeed, the nuclear response functions $W$ can be interpreted as the ``nuclear dials'' available to an experimentalist, dials which can be tuned through nuclear target selection in order to vary the $W$ functions and thereby probe different linear combinations of the CLFV response functions. To aid in this endeavor, one would like to understand the general properties of the six (twelve, accounting for isospin) allowed nuclear response functions and their dependence on macroscopic nuclear quantities.

Broadly, we may distinguish three of the allowed nuclear responses, $M$, $\Sigma'$ and $\Sigma''$, as velocity-\textit{independent} and the remaining three, $\Delta$, $\tilde{\Phi}$' and $\Phi''$, as velocity-\textit{dependent}. The velocities in question - consistent with our definition of the effective interaction in terms of intrinsic nuclear coordinates - are the relative (or Jacobi) velocities of the nucleons. As these quantities are purely internal to the nuclear system, the velocity-dependent operators must vanish in the limit of a point-like nucleus. One way of approaching the point-like limit is to consider that the nucleus is probed by a very long wavelength operator, i.e. $q_\mathrm{eff}\rightarrow 0$. We note that in the expression for the $\mu\rightarrow e$ decay rate, each of the velocity-dependent operators is accompanied by a factor $q_\mathrm{eff}^2/m_N^2$, reflecting the fact that these responses vanish in the point-like limit. 

On the other hand, the velocity-independent operators survive in the point-like limit; indeed, the leading multipoles have rather simple forms at $q_\mathrm{eff}=0$:
\begin{equation}
\begin{split}
M_{00}(0)&=\frac{1}{\sqrt{4\pi}}\sum_{i=1}^A 1(i)\\
\Sigma'_{1M}(0)&=\frac{1}{\sqrt{6\pi}}\sum_{i=1}^A\sigma_{1M}(i)\\
\Sigma''_{1M}(0)&=\frac{1}{\sqrt{12\pi}}\sum_{i=1}^A\sigma_{1M}(i)
\end{split}
\end{equation}
These are the total charge and total spin operators for the nucleus, macroscopic quantities which do not depend on any internal structure of the nucleus. It is interesting to note that the transverse-electric $\Sigma'_1$ and longitudinal $\Sigma''_1$ projections of the spin current are proportional in the long-wavelength limit, though of course they are distinct at finite $q_\mathrm{eff}$. If a target with total ground-state angular momentum $j_N=0$ is selected, then only the coherent operator $M_{00}$ will contribute; the spin-dependent operators require a nuclear state with $j_N\geq \frac{1}{2}$. 


Although the prefactor $q_\mathrm{eff}^2/m_N^2$ causes their response functions to vanish in the limit of zero momentum-transfer, the velocity-dependent operators can be expanded for long wavelengths as
\begin{equation}
\begin{split}
\Delta_{1M}(0)&=-\frac{1}{\sqrt{24\pi}}\sum_{i=1}^A\ell_{1M}(i)\\
\tilde{\Phi}'_{2M}(0)&=-\frac{1}{\sqrt{20\pi}}\sum_{i=1}^A\left[\vec{x}(i)\otimes\left(\vec{\sigma}(i)\times\frac{1}{i}\vec{\nabla}(i)\right)_1\right]_{2M}\\
\Phi''_{JM}(0)&=\left\{\begin{array}{cc}
-\frac{1}{6\sqrt{\pi}}\sum_{i=1}^A\vec{\sigma}(i)\cdot\vec{\ell}(i), & J=0\\
-\frac{1}{\sqrt{30\pi}}\sum_{i=1}^A\left[\vec{x}(i)\otimes\left(\vec{\sigma}(i)\times\frac{1}{i}\vec{\nabla}\right)_1\right]_{2M}, & J=2
\end{array}\right.,
\end{split}
\end{equation}
where $\vec{\ell}$ is the orbital angular momentum operator, $\otimes$ denotes a spherical tensor product whereas $\times$ denotes the conventional cross product. The lowest multipole of the transverse-electric projection of the spin-velocity current is $J=2$ and therefore $\tilde{\Phi}'$ can only contribute if $j_N\geq 1$.

The momentum-dependence of the nuclear response functions $W^{\tau\tau}_\mathcal{O}(q_\mathrm{eff})$ for $^{27}$Al is shown in Figure \ref{fig:Al_responses}. In Figure \ref{fig:Cu_responses}, we show the corresponding results for Cu - a case analogous to Al, as the only stable isotopes, $^{63}$Cu and $^{65}$Cu, have an unpaired nucleon. The nuclear responses for Al were computed by diagonalizing the USDA interaction in the $2s$-$1d$ shell-model space whereas for Cu we employed the GCN2850 interaction in the $1f_{5/2}$-$2p$-$1g_{9/2}$ valence space. The ground state of $^{27}$Al has total angular momentum $j_N=\frac{5}{2}$, and both isotopes of Cu have ground-state angular momentum $j_N=\frac{3}{2}$. The angular momentum selection rules therefore allow all six response functions to contribute. Similarly, the nuclei $^{27}$Al, $^{63}$Cu, and $^{65}$Cu carry total isospin $T=\frac{1}{2}$, $T=\frac{5}{2}$, and $T=\frac{7}{2}$, respectively, thereby permitting a coupling to isovector operators.

The isoscalar nuclear response is dominated, unsurprisingly, by the charge operator, particularly the coherent contribution of $M_{00}$. 

The spin-orbit operator satisfies
\begin{equation}
\begin{split}
\vec{\sigma}\cdot\vec{\ell}&=\vec{j}^2-\vec{\ell}^2-\vec{s}^2\\
&=j(j+1)-\ell(\ell+1)-\frac{3}{4}
\end{split}
\end{equation}
There are $2(\ell+1)$ states where $j=\ell+\frac{1}{2}$ for which the spin-orbit term evaluates to $\ell$ and $2\ell$ states where $j=\ell-\frac{1}{2}$ and the spin-orbit term evaluates to $-(\ell+1)$. In nuclei, the strong nuclear spin-orbit force breaks the degeneracy between the $j=\ell\pm\frac{1}{2}$ subshells and moves the spin-aligned subshell $j=\ell+\frac{1}{2}$ to lower energy. As this subshell is progressively filled with nucleons, the spin-orbit response of $\Phi''_{0}$ sums coherently. Once the spin-anti-aligned $j=\ell-\frac{1}{2}$ subshell starts to be filled, the spin-orbit contribution cancels against that from the spin-aligned subshell. When both subshells are completely occupied, the total spin-orbit contribution is identically zero.

$^{27}$Al is an ideal nucleus for exploiting the semi-coherence of $\Phi''_{0}$: in the extreme single-particle picture, all five states of the spin-aligned $1d_{5/2}$ subshell are occupied while the anti-aligned $1d_{3/2}$ subshell is entirely vacant. Even after accounting for the additional momentum suppression factor  - not included in the figures - $q_\mathrm{eff}^2/m_N^2W_{\Phi''}^{00}(q_\mathrm{eff})\approx 0.15$ is larger than either of the spin responses $W_{\Sigma'}^{00}(q_\mathrm{eff})\approx 0.09$, $W_{\Sigma''}^{00}(q_\mathrm{eff})\approx 0.11$. 

Cu also shows a strong semi-coherent response from the operator $\Phi''_0$. Protons completely fill the $1f_{7/2}$ subshell, leaving the anti-aligned counterpart vacant. In $^{63}$Cu ($^{65}$Cu), neutrons completely fill the $1f_{7/2}$ and $2p_{3/2}$ subshells with two (four) neutrons occupying the anti-aligned $1f_{5/2}$ subshell, partially negating the coherence of the spin-orbit operator.  Nonetheless, we find that the semi-coherence significantly enhances the response: $q_\mathrm{eff}^2/m_N^2W_{\Phi''}^{00}(q_\mathrm{eff})\approx 0.78$ is larger than either of the spin responses $W_{\Sigma'}^{00}(q_\mathrm{eff})\approx 0.12$, $W_{\Sigma''}^{00}(q_\mathrm{eff})\approx 0.15$. 

Consequently, in the targets Al and Cu, we identify a hierarchy of isoscalar response functions
\begin{equation}
W_M^{00}\gg \bigg\{W^{00}_{\Sigma'},W_{\Sigma''}^{00},\frac{q_\mathrm{eff}^2}{m_N^2}W^{00}_{\Phi''}\bigg\}\gg\bigg\{\frac{q_\mathrm{eff}^2}{m_N^2}W^{00}_{\Delta},\frac{q^2_\mathrm{eff}}{m_N^2}W_{\tilde{\Phi}'}^{00}\bigg\}
\end{equation}
As $\Phi''$ is associated with the spin-velocity current $\vec{v}_N\times\vec{\sigma}_N$ that arises for tensor and other more exotic interactions, targets like Al and Cu have a special sensitivity to velocity-dependent couplings associated with such interactions. 
\begin{figure}
\centering
\includegraphics[scale=0.6]{Fig4_Alformfactor.pdf}
\caption{The nuclear response functions $W^{\tau\tau}_\mathcal{O}(q_\mathrm{eff})$ for the six operators contributing to elastic $\mu\rightarrow e$ conversion in $^{27}$Al. The left (right) panel gives the results for the isoscalar (isovector) coupling. The response functions are needed at the three-momentum transfer $q_\mathrm{eff}$ indicated by the dashed line. The results in blue correspond to charge and spin couplings, while those in orange correspond to the velocity-dependent operators where the response functions are accompanied by the additional factor $q^2_\mathrm{eff}/m_N^2\sim 0.014$.}
\label{fig:Al_responses}
\end{figure}
\begin{figure}
\centering
\includegraphics[scale=0.6]{Fig5_Cuformfactor.pdf}
\caption{As in Figure \ref{fig:Al_responses} but for Cu.}
\label{fig:Cu_responses}
\end{figure}

Figure \ref{fig:v_ind_responses} shows the strength of the various velocity-independent response functions across eleven different nuclear targets, ranging in mass from the lightest, carbon, to the heaviest, copper. One can plainly see how the coherent response increases with nucleon number $A$, though it is also dampened by the nuclear diffraction minimum for heavier nuclei. The isovector charge response function also generally exhibits the expected behavior and is roughly proportional to the isospin asymmetry $|Z-N|$. Examining the spin-dependent responses, we see that the targets which are composed primary of odd-nucleon isotopes (F, Na, Al, and Cu) provide the strongest response. The minor impurities of $^{13}$C, $^{27}$Si, $^{33}$S, $^{47}$Ti/$^{49}$Ti, and $^{57}$Fe are not sufficient to produce much of a spin-dependent response in those targets. It is interesting to note how strong the $\Sigma'$ and $\Sigma''$ responses are in $^{19}$F.

Turning to the velocity-dependent response functions in Figure \ref{fig:v_dep_responses}, the trends are a bit tougher to discern. It is clear that $\tilde{\Phi}'$, the transverse-electric projection of the spin-velocity current, is consistently the weakest nuclear response, owing to the fact that its leading multipole is $J=2$. Similarly, the transverse-magnetic projection of convective current, $\Delta$, is also generally suppressed, though it is worth noting that $^{27}$Al shows the strongest $\Delta$ response in both isospin channels of all nuclei considered. Finally, as discussed above, the longitudinal projection of the spin-velocity current, $\Phi''$, sums coherently over spin-aligned $j=\ell+\frac{1}{2}$ and $j=\ell-\frac{1}{2}$ subshells, but vanishes when both subshells are fully occupied. As such, this contribution vanishes entirely for the doubly-magic nuclei $^{16}$O and $^{40}$Ca (though the response for the natural targets O and Ca is nonzero due to valence neutrons in the non-magic isotopes). The spin-orbit response is strongest when the lower-energy spin-aligned subshells are filled and the anti-aligned subshells are vacant. In Ti especially, we see how the response is primarily driven by neutrons filling the $1f_{7/2}$ shell, leading to a strong isovector response.

Comparing across all twelve allowed response functions, it appears that $^{27}$Al is a rather good choice of initial nuclear target for the next-generation experiments.
\begin{figure}
\centering
\includegraphics[scale=0.62]{v_independent_combined.pdf}
\caption{Velocity-independent response functions computed for the eleven nuclear targets of interest.}
\label{fig:v_ind_responses}
\end{figure}
\begin{figure}
\centering
\includegraphics[scale=0.62]{v_dependent_combined.pdf}
\caption{Velocity-dependent response functions computed for the eleven nuclear targets of interest and multiplied by the corresponding momentum suppression factor $q_\mathrm{eff}^2/m_N^2$.}
\label{fig:v_dep_responses}
\end{figure}
For example, if a $\mu\rightarrow e$ conversion signal were observed with an Al target but not with a Ca target, it would be good evidence that the underlying CLFV operators are nuclear spin-dependent.
\chapter{LEC Analysis}
Having demonstrated how the nuclear response functions can be computed for a specified target using nuclear many-body techniques, the only unknowns which remain in our expression for the decay rate $\Gamma(\mu\rightarrow e)$ are the CLFV low-energy constants $\tilde{c}_i^{\tau}$. Of course, these quantities can be constrained by experimental measurements of the branching ratio 
\begin{equation}
B\left(\mu\rightarrow e\right)=\frac{\Gamma\left(\mu\rightarrow e\right)}{\Gamma\left(\mu\rightarrow \nu_{\mu}\right)},
\end{equation}
where $\Gamma(\mu\rightarrow \nu_{\mu})$ is the rate of standard muon capture. 

Over the next five years, new experiments employing high-intensity pulsed muon beams should lead to substantial improvement on $\mu\rightarrow e$ conversion limits. The COMET experiment at J-PARC is expected to reach a branching ratio sensitivity $B(\mu\rightarrow e)<7\times 10^{-15}$ (90\% CL) in Phase-I and, ultimately, $B(\mu\rightarrow e)\lesssim\times 10^{-17}$ in Phase-II. The Mu2e experiment at Fermilab is expected to reach a branching rato sensitivity of $7\times 10^{-17}$ (90\% CL), and a proposed followup experiment Mu2e II, which will take advantage of future beam upgrades at Fermilab, could improve this limit by another order of magnitude to $7\times 10^{-18}$ (90\% CL). Both COMET and Mu2e will employ Al targets. A second J-PARC experiment has been proposed by the DeeMe collaboration and aims to achieve a $\mu\rightarrow e$ branching ratio limit of $1\times 10^{-13}$ for a graphite target. A followup experiment, hoping to achieve a branching ratio sensitivity $2\times 10{-14}$ with a silicon carbide target has also been discussed.

With the exception of the SIN experiment on $^{32}$S, all past and planned experiments have employed natural targets; therefore, we carry out shell-model calculations for each isotope with natural abundance $>0.2\%$ and compute the total $\mu\rightarrow e$ conversion rate as a sum over the isotopes weighted by their abundance.

In order to estimate the impact of these proposed new experiments on CLFV bounds, w
\begin{equation}
\Gamma\left(\mu\rightarrow\nu_{\mu}\right)=\left\{\begin{array}{lr}
0.0378 & \mathrm{C}\\
0.703 & \mathrm{Al}\\
0.865 & \mathrm{Si}\\
1.351 & \mathrm{S}\\
2.592 & \mathrm{Ti}\\
5.673 & \mathrm{Cu}
\end{array}\right\}\times 10^6\mathrm{/s}
\end{equation}
which were obtained by computing the weighted averages of the measurements compiled in \cite{PhysRevC.35.2212}.   

Table \ref{tab:LEC_limits} shows the maximum magnitude of each LEC that is consistent with the given branching ratio limit. More physically, assuming that the LEC under consideration is natural at the scale of CLFV physics, one may convert the LEC limit to an approximate scale probed by the given operator
\begin{equation}
\Lambda_i^{\tau}=\frac{v}{\sqrt{y_i^{\tau}}}
\end{equation}
We see that the operators which generate the coherent response, $\tilde{c}_1^0$ and $\tilde{c}_{11}^0$, probe the highest scale of new physics, up to $10^4$ TeV in $^{27}$Al at a branching ratio sensitivity $\sim 10^{-17}$. By this same measure, the existing limit in Ti provides a constraint up to $\sim$ 900 TeV; the next-generation experiments will improve the reach in energy scale by more than an order of magnitude. The isovector operators $\tilde{c}_1^1$ and $\tilde{c}_{11}^1$, which do not benefit from the coherent enhancement, are still capable of probing up to $\sim 2,000$ TeV. The Mu2e/COMET design sensitivity is so impressive that even the weakest probe, $\tilde{c}_{13}^{\tau}$, which generates only the highly-suppressed $\tilde{\Phi}'$ response function, is capable of probing up to $\sim$200 TeV in the isoscalar case and $\sim$500 TeV in the isovector.

\begin{table}
\label{tab:LEC_limits}
\caption{Limits on the CLFV LECs imposed by $\mu \rightarrow e$ conversion branching ratios. 
Given the specified branching ratio limits, the allowed values of the dimensionless LECs are 
given by $|\tilde{c}_i^\tau| \lesssim y_i^\tau$ where the $y_i^\tau$ are the column entries.  The $^\dagger$s indicate  
branching ratios achievable in planned experiments (see text).   E-$x \, \equiv \,10^{-x}.$}
\begin{tabular}{|c|c|c|c|c|c|c|}
\hline
 & & & & & &  \\[-7.5pt]
Target  &~~~~~ Al~~~~~&~~~~~C~~~~~&~~~~SiC~~~~&~~~$^{32}$S~~~~~&~~~~~Ti~~~~~&~~~~~Cu~~~~~ \\[1.6pt]
\hline 
 & & & & & &  \\[-7.5pt]
$\begin{array}{l} \mathrm{Branching} \\  \mathrm{Ratio} \end{array}$ & $10^{-17 \, \dagger}$ & $10^{-13 \, \dagger}$ & $2 \times 10^{-14 \, \dagger}$ & $7 \times 10^{-11}$ \cite{BADERTSCHER1982406} &  $6.1 \times 10^{-13}$ \cite{wintz} & $1.6 \times 10^{-8}$ \cite{PhysRevLett.28.1469} \\
 & & & & & &  \\[-7.5pt]
\hline
 & & & & & &  \\[-5.5pt]
$\tilde{c}_1^0,\tilde{c}_{11}^0$ &~~3.994E-10~~ & ~~5.103E-8~~ & ~~1.767E-8 ~~& ~~1.030E-6~~ & ~~7.380E-8 ~~& ~~1.207E-5~~ \\
$\tilde{c}_1^1,\tilde{c}_{11}^1$ &1.238E-8 & 6.294E-6 & 1.398E-6 & - & 1.316E-6 & 1.861E-4 \\
$\tilde{c}_3^0,\tilde{c}_{15}^0$ &1.608E-8 & 3.960E-6 &7.331E-7 & 4.542E-5 & 3.801E-6 & 3.487E-4 \\
$\tilde{c}_3^1,\tilde{c}_{15}^1$ &1.860E-7 & 1.322E-4 & 4.034E-5 & - & 7.344E-6 & 2.129E-3 \\
$\tilde{c}_4^0$ & 1.418E-8 & 9.366E-6 & 4.109E-6 & - & 1.504E-5 & 5.905E-4 \\
$\tilde{c}_4^1$ & 1.713E-8 & 1.055E-5 & 4.886E-6 & - & 1.718E-5 & 6.148E-4 \\
$\tilde{c}_5^0,\tilde{c}_8^0$ &7.774E-8 & 9.606E-5 & 7.128E-5 & - & 5.802E-5 & 9.021E-3 \\
$\tilde{c}_5^1,\tilde{c}_8^1$ &1.164E-7 & 1.558E-4 & 7.311E-5 & - & 6.521E-5 & 2.691E-2 \\
$\tilde{c}_6^0,\tilde{c}_{10}^0$ & 1.954E-8 & 1.082E-5 & 5.463E-6 & - & 1.794E-5 & 8.733E-4 \\
$\tilde{c}_6^1,\tilde{c}_{10}^1$ & 2.151E-8 & 1.201E-5 & 6.130E-6 & - & 1.999E-5 & 8.738E-4 \\
$\tilde{c}_9^0$ & 2.061E-8 & 1.870E-5 & 6.237E-6 & - & 2.758E-5 & 8.015E-4 \\
$\tilde{c}_9^1$ & 2.833E-8 & 2.213E-5 & 8.089E-6 & - & 3.360E-5 & 8.651E-4 \\
$\tilde{c}_{12}^0$ & 1.608E-8 & 3.960E-6 & 7.331E-7 & 4.542E-5 & 3.797E-6 & 3.487E-4 \\
$\tilde{c}_{12}^1$ & 1.388E-7 & 1.322E-4 & 4.034E-5 & - & 7.342E-6 & 2.096E-3 \\
$\tilde{c}_{13}^0$ & 1.787E-6 & - & - & - & 8.422E-5 & 5.277E-2 \\
$\tilde{c}_{13}^1$ & 2.085E-7 & - & - & - & 3.718E-4 & 1.179E-2 \\[2.6pt]
\hline
\end{tabular}
\end{table} 

\begin{table}
\caption{Approximate scale $\Lambda_i^{\tau}$ probed by the effective CLFV operators at a given $\mu \rightarrow e$ conversion branching ratio limit.  The $^\dagger$s indicate  
branching ratios achievable in planned experiments (see text).   E-$x \, \equiv \,10^{-x}.$}
\label{tab:LECscale}
\begin{tabular}{|c|c|c|c|c|c|c|}
\hline
 & & & & & &  \\[-7.5pt]
Target  &~~~~~ Al~~~~~&~~~~~C~~~~~&~~~~SiC~~~~&~~~$^{32}$S~~~~~&~~~~~Ti~~~~~&~~~~~Cu~~~~~ \\[1.6pt]
\hline 
 & & & & & &  \\[-7.5pt]
$\begin{array}{l} \mathrm{Branching} \\  \mathrm{Ratio} \end{array}$ & $10^{-17 \, \dagger}$ & $10^{-13 \, \dagger}$ & $2 \times 10^{-14 \, \dagger}$ & $7 \times 10^{-11}$ \cite{BADERTSCHER1982406} &  $6.1 \times 10^{-13}$ \cite{wintz} & $1.6 \times 10^{-8}$ \cite{PhysRevLett.28.1469} \\
 & & & & & &  \\[-7.5pt]
\hline
 & & & & & &  \\[-5.5pt]
$\tilde{c}_1^0,\tilde{c}_{11}^0$ &  10,000 TeV & 1,000 TeV & 2,000 TeV & 200 TeV & 900 TeV & 70 TeV\\
$\tilde{c}_1^1,\tilde{c}_{11}^1$ & 2,000 TeV & 100 TeV & 200 TeV & - & 200 TeV & 20 TeV\\
$\tilde{c}_3^0,\tilde{c}_{15}^0$ & 2,000 TeV & 100 TeV & 300 TeV & 40 TeV & 100 TeV & 10 TeV\\
$\tilde{c}_3^1,\tilde{c}_{15}^1$ & 600 TeV & 20 TeV & 40 TeV & - & 90 TeV & 5 TeV\\
$\tilde{c}_4^0$ & 2,000 TeV & 80 TeV & 100 TeV & - & 60 TeV & 10 TeV\\
$\tilde{c}_4^1$ & 2,000 TeV & 80 TeV & 100 TeV & - & 60 TeV & 10 TeV\\
$\tilde{c}_5^0,\tilde{c}_8^0$ & 900 TeV & 30 TeV & 30 TeV & - & 30 TeV & 3 TeV\\
$\tilde{c}_5^1,\tilde{c}_8^1$ & 700 TeV & 20 TeV & 30 TeV & - & 30 TeV & 2 TeV\\
$\tilde{c}_6^0,\tilde{c}_{10}^0$ & 2,000 TeV & 70 TeV & 100 TeV & - & 60 TeV & 8 TeV\\
$\tilde{c}_6^1,\tilde{c}_{10}^1$ & 2,000 TeV & 70 TeV & 100 TeV & - & 60 TeV & 8 TeV\\
$\tilde{c}_9^0$ & 2,000 TeV & 60 TeV & 100 TeV & - & 50 TeV & 9 TeV\\
$\tilde{c}_9^1$ & 1,000 TeV & 50 TeV & 90 TeV & - & 40 TeV & 8 TeV\\
$\tilde{c}_{12}^0$ & 2,000 TeV & 100 TeV & 300 TeV & 40 TeV & 100 TeV & 10 TeV\\
$\tilde{c}_{12}^1$ & 700 TeV & 20 TeV & 40 TeV & - & 90 TeV & 5 TeV\\
$\tilde{c}_{13}^0$ & 200 TeV & - & - & - & 30 TeV & 1 TeV\\
$\tilde{c}_{13}^1$ & 500 TeV & - & - & - & 10 TeV & 2 TeV\\[2.6pt]
\hline
\end{tabular}
\end{table} 
\chapter{Relationship to $\mu\rightarrow \lowercase{e}\gamma$ and $\mu\rightarrow 3\lowercase{e}$}
$\mu\rightarrow e$ conversion in nuclei is just one probe of possible CLFV physics. First, it is worth mentioning that there is significant theoretical and experimental interest in CLFV processes involving the tau lepton, in addition to the muon and electron. While we have so far centered our discussion around dedicated CLFV experiments (like Mu2e and COMET), many important constraints have been obtained from colliders, especially the Large Hadron Collider (LHC). In particular, searches at A Toroidal LHC Apparatus (ATLAS), Compact Muon Solenoid (CMS) and Large Hadron Collider beauty (LHCb), have constrained the branching ratios of CLFV processes including the $Z$-boson decays $Z\rightarrow e\tau$ and $Z\rightarrow \mu\tau$ \cite{ATLAS:2020zlz,ATLAS:2021bdj}, the Higgs boson decays $h\rightarrow e\mu$, $h\rightarrow \mu\tau$ and $h\rightarrow e\tau$ \cite{ATLAS:2016joj,CMS:2017con,CMS:2021rsq}, the $B$ meson decays $B^+\rightarrow K^+\mu^-\tau^+$ \cite{LHCb:2020khb}, $B^+\rightarrow K^+\mu^{\pm}e^{\mp}$ \cite{LHCb:2019bix}, $B_s^{0}\rightarrow \tau^{\pm}\mu^{\mp}$ and $B^0\rightarrow \tau^{\pm}\mu^{\mp}$ \cite{LHCb:2019ujz}, and the purely leptonic decay $\tau\rightarrow 3\mu$ \cite{LHCb:2014kws,ATLAS:2016jts,CMS:2020kwy}. In the next few years, the Belle II experiment \cite{Belle-II:2018jsg} at the High Energy Accelerator Research Organization (KEK) in Japan is expected to produce competitive limits on a range of CLFV processes including $\tau\rightarrow e\gamma$, $\tau\rightarrow e\ell^+\ell^-$, and $B^0\rightarrow K^{*0}\ell^+\ell^-$, and the future Electron-Ion Collider (EIC) \cite{AbdulKhalek:2022erw} at Brookhaven National Laboratory (BNL) is expected to further strengthen limits on CLFV $\tau$ decays \cite{Cirigliano:2021img}. The existing limits on $\tau\rightarrow e$ transitions are typically much weaker than those on $\mu\rightarrow e$ decays, with $BR\left(\tau^{\pm}\rightarrow e^{\pm}X\right)\lesssim 10^{-8}$ \cite{Zyla:2020zbs} for $X=\{\gamma,\pi\pi,...\}$. On the other hand, specific UV models, such as the SUSY seesaw model \cite{Ellis:1999uq}, may lead to an enhanced rate for CLFV processes involving the tau lepton.

Many of the above-mentioned processes are interrelated. For example, if the CLFV vertex that mediates $h\rightarrow e\mu$ exists, then it will induce $\mu\rightarrow e$ conversion through the exchange of a virtual Higgs and $\mu\rightarrow e\gamma$ through one- and two-loop diagrams. Thus, an ensemble of measurements from colliders and dedicated decay experiments will play a complementary role in distinguishing among candidate UV theories of CLFV. In this chapter, limiting our discussion to CLFV $\mu$ decays, we explore the relationship between $\mu\rightarrow e$ conversion, $\mu\rightarrow e\gamma$ and $\mu\rightarrow 3e$. We discuss the limits which can be placed on operator coefficients if Mu2e/COMET, MEG-II, and Mu3e achieve their design sensitivity and comment on the wide range of possible detection scenarios at this next-generation of experiments.


\begin{figure}
\subfloat[]{
\begin{fmffile}{photon_on_shell}
\begin{fmfgraph*}(120,120)
\fmfleft{i1,i2} \fmfright{o1,o2}\fmfbottom{b1}
\fmf{fermion}{i2,v1}
\fmf{fermion}{v1,o2}
\fmfv{d.sh=circle,d.f=shaded}{v1}
\fmflabel{$\mu$}{i2}
\fmflabel{$e$}{o2}
\fmf{photon,label=$\gamma$}{v1,b1}
\end{fmfgraph*}
\end{fmffile}
}
\hfill
\subfloat[]{
\begin{fmffile}{photon_exchange}
\begin{fmfgraph*}(120,120)
\fmfleft{i1,i2} \fmfright{o1,o2}
\fmf{dbl_plain}{i1,v1}
\fmf{dbl_plain}{v1,o1}
\fmf{photon,label=$\gamma$}{v1,v2}
\fmfv{d.sh=circle,d.f=shaded}{v2}
\fmf{fermion}{i2,v2}
\fmf{fermion}{v2,o2}
\fmflabel{$N$}{o1}
\fmflabel{$N$}{i1}
\fmflabel{$\mu$}{i2}
\fmflabel{$e$}{o2}
\end{fmfgraph*}
\end{fmffile}
}
\caption{(a) On-shell $\mu\rightarrow e\gamma$ decay mediated by CLFV vertex. (b) Conversion process induced by the $\mu\rightarrow e\gamma$ vertex. The photon is virtual and exchanged with the nuclear charge.}
\label{fig:mu_e_gamma_diagrams}
\end{figure}

The most general CLFV electromagnetic vertex coupling a photon to the leptons is 
\begin{equation}
\begin{split}
\Gamma^{\mu}_{\mu\rightarrow e}&=\frac{1}{\Lambda^2}\left(q^2\gamma^{\mu}-q^{\mu}\slashed{q}\right)\left[\tilde{f}_R(q^2)+i\tilde{f}_A(q^2)\gamma_5\right]\\
&+i\frac{m_{\mu}}{\Lambda^2}\sigma^{\mu\nu}q_{\nu}\left[\tilde{f}_M(q^2)+i\tilde{f}_E(q^2)\gamma_5\right],
\end{split}
\label{eq:em_clfv}
\end{equation}
where $\Lambda$ is the scale of CLFV physics, and the subscripts $R$, $A$, $M$, and $E$ denote the induced (dimensionless) CLFV charge radius, anapole, magnetic dipole, and electric dipole form factors, respectively.  

In the case of on-shell photon production, shown in Figure \ref{fig:mu_e_gamma_diagrams} (a), the four-momentum-transfer satisfies $q^2=0$, and the charge radius and anapole form factors do not contribute. The resulting $\mu\rightarrow e\gamma$ decay rate has the simple form
\begin{equation}
\Gamma\left(\mu\rightarrow e\gamma\right)=\frac{1}{2}\alpha\frac{m_{\mu}^5}{\Lambda^4}\left(|\tilde{f}_M(0)|^2+|\tilde{f}_E(0)|^2\right)
\end{equation}
To arrive at the branching ratio, we normalize by the rate of standard model $\mu\rightarrow e+2\nu$ decay; the result is exceedingly simple:
\begin{equation}
B(\mu\rightarrow e\gamma)=192\alpha\pi^3\left(\frac{v}{\Lambda}\right)^4\left(|\tilde{f}_{M}(0)|^2+|\tilde{f}_E(0)|^2\right).
\end{equation}
In comparison to $\mu\rightarrow e$ conversion, the on-shell $\mu^+\rightarrow e^+\gamma$ process -- the focus of the MEG and MEG-II experiments -- is a relatively clean probe of CLFV physics, as it does not depend on any nuclear physics. The drawback is that one must detect both the positron and the photon, with backgrounds primarily coming from radiative muon decay $\mu^+\rightarrow e^+\nu\bar{\nu}\gamma$ and from accidental coincidence of a positron from Michel decay $\mu^+\rightarrow e^+\nu\bar{\nu}$ with a photon originating from either radiative muon decay, bremsstrahlung, or $e^+e^-\rightarrow \gamma\gamma$ annihilation. Consequently, the expected branching ratio sensitivity at the next-generation Mu to E Gamma (MEG) II experiment, $B(\mu\rightarrow e\gamma)< 6\times 10^{-14}$ (90\% CL), is roughly three orders of magnitude less sensitive than the corresponding limits on $\mu\rightarrow e$ conversion expected at Mu2e and COMET.

If $\mu\rightarrow e$ proceeds through exchange of a virtual photon, as in Figure \ref{fig:mu_e_gamma_diagrams} (b), then all four CLFV electromagnetic form factors in Equation \ref{eq:em_clfv} contribute. The photon couples directly to the nuclear Coulomb charge, and therefore the $\mu\rightarrow e$ conversion amplitude is
\begin{equation}
\frac{4\pi\alpha}{q^2}\bar{\chi}_e\Gamma^{\mu}_{\mu\rightarrow e}(q^2)\chi_{\mu}\;\bar{N}\gamma_{\mu}\left(\frac{1+\tau_3}{2}\right)N.
\end{equation}
Retaining the dominant coherent contribution and approximating the fixed four-momentum of the photon propagator as $q^2\sim -m_{\mu}^2$, we find the following values for the nucleon-level effective theory LECs:
\begin{equation}
\begin{split}
\tilde{c}_1^0&=\tilde{c}_1^1=2\pi\alpha\frac{v^2}{\Lambda^2}\left[\tilde{f}_R(-m_{\mu}^2)+\tilde{f}_M(-m_{\mu}^2)\right]\\
\tilde{c}_{11}^0&=\tilde{c}_{11}^1=2\pi\alpha\frac{v^2}{\Lambda^2}\left[\tilde{f}_A(-m_{\mu}^2)-\tilde{f}_E(-m_{\mu}^2)\right].
\end{split}
\end{equation}
If we assume that the momentum-dependence of the form factors is mild $\tilde{f}(-m_{\mu}^2)\approx \tilde{f}(0)$, then the rates for $\mu\rightarrow e$ conversion and $\mu\rightarrow e\gamma$ will be correlated through their dependence on the dipole form factors $\tilde{f}_M$ and $\tilde{f}_E$. We can explore this relationship further by setting $\tilde{f}_A=\tilde{f}_E=0$ and assuming that $|\tilde{f}_R|+|\tilde{f}_M|=1$. This latter condition is the requirement that at least one of the non-zero form factors is approximately natural at the scale of CLFV physics $\Lambda$. Figure \ref{fig:mu2e_meg_mu3e} shows how the excluded regions of parameter space in $\Lambda$ and the ratio $|\tilde{f}_R/\tilde{f}_M$ from the experiment. As expected, we see that the on-shell process provides no constraint in the limit $\tilde{f}_R\gg \tilde{f}_M$. For the conversion process, two cases must be distinguished. If $\tilde{f}_R$ and $\tilde{f}_M$ have the same sign, then the total rate is independent of the ratio $|\tilde{f}_R/\tilde{f}_M|$, as we have assumed $|\tilde{f}_R|+|\tilde{f}_M|=1$. If, however, the charge radius and magnetic dipole form factors are opposite sign, then they interfere destructively, leading to a total cancellation when $|\tilde{f}_R/\tilde{f}_M|=1$ (the cancellation is not actually exact; only the leading coherent contribution cancels, and the typically suppressed spin and velocity-dependent operators become the leading response. We can infer from Table \ref{tab:LECscale} that the limits on $\Lambda$ are at least a factor of five weaker in this case). In this cancellation scenario, it is possible for a $\mu\rightarrow e\gamma$ signal to be observed at MEG-II without a corresponding $\mu\rightarrow e$ conversion singal in Mu2e/COMET Phase-II. Otherwise, one generically expects the conversion experiments to be a better probe of electromagnetic CLFV couplings than the on-shell photon experiments, despite the fact that the rate for virtual photon $\mu\rightarrow e$ conversion is suppressed by a factor of $\alpha$ relative to the on-shell process; the conversion experiments aim to achieve roughly six thousand times better sensitivity.

\begin{figure}
\centering
\includegraphics[scale=0.83]{Fig6_muegamma_b.pdf}
\caption{Exclusion curves for the CLFV elecromagnetic coupling considered in Eq. \ref{eq:em_clfv} for the case $\tilde{f}_A=\tilde{f}_E=0$ and $|\tilde{f}_R|+|\tilde{f}_M|=1$. The dashed (solid) black curve shows the (expected) limit for on-shell $\mu\rightarrow e\gamma$ conversion obtained from MEG (MEG-II). The branching ratio limits are $B(\mu\rightarrow e\gamma)<4.2\times 10^{-13}$ for the MEG experiment and $6\times 10^{-14}$ for MEG-II. The dashed (solid) green curve shows the (expected) limit for $\mu\rightarrow 3e$ conversion obtained from Mu3e Phase-I (Phase-II). The branching ratio limit is $B(\mu\rightarrow 3e)<2.0\times 10^{-15}$ $(10^{-16})$ for Phase-I (II)  of Mu3e. The orange curve corresponds to the existing limit $B(\mu\rightarrow 3e)<1.0\times 10^{-12}$ obtained by SINDRUM. The solid red (blue) curve corresponds to a $\mu\rightarrow e$ branching ratio limit $B(\mu\rightarrow e)<10^{-17}$ $(7\times 10^{-15})$ for the case where $\tilde{f}_R$ and $\tilde{f}_M$ contribute with the same sign. The dashed red (blue) curve is analogous to the solid red (blue) curve for the case where $\tilde{f}_R$ and $\tilde{f}_M$ are opposite sign. In this case, the charge radius and magnetic dipole contributions to the coherent conversion on nuclei cancel when $|\tilde{f}_R/\tilde{f}_M|=1$.}
\label{fig:mu2e_meg_mu3e}
\end{figure}

Let us consider the following effective Lagrangian which mediates $\mu\rightarrow 3e$
\begin{equation}
\mathcal{L}=\frac{1}{\Lambda^2}\left(C_{\mu eee}^L\bar{e}^c_L\mu_L\bar{e}_Le^c_L+C^R_{\mu eee}\bar{e}^c_R\mu_R\bar{e}_Re^c_R\right),
\label{eq:L_mu3e}
\end{equation}
where $C^L_{\mu eee}$ and $C^R_{\mu eee}$ are Wilson coefficients. The branching ratio is then given by the simple expression
\begin{equation}
B(\mu\rightarrow 3e)=\frac{1}{2}\left(\frac{v}{\Lambda}\right)^4\left(|C^L_{\mu eee}|^2+|C^R_{\mu eee}|^2\right).
\end{equation}

In general, there are dimension-six operators which mediate $\mu\rightarrow 3e$ beyond those included in Eq. \ref{eq:L_mu3e}. We have restricted to those operators unique in that at one-loop order (see Figure \ref{fig:mu3e_diagrms} (a)) they generate an effective $\mu\rightarrow e\gamma$ vertex which is enhanced by a large logarithm; the resulting electromagnetic couplings are \cite{Raidal:1997hq,Cirigliano:2004mv,PhysRevLett.93.231802}
\begin{equation}
\begin{split}
f_R&=-\frac{m_{\mu}^2}{\Lambda^2}\frac{1}{(4\pi)^2}\frac{2}{3}\left(C^L_{\mu eee}+C^R_{\mu eee}\right)\mathrm{ln}\frac{-q^2}{\Lambda^2}\\
f_A&=-i\frac{m_{\mu}^2}{\Lambda^2}\frac{1}{(4\pi)^2}\frac{2}{3}\left(C^L_{\mu eee}-C^R_{\mu eee}\right)\mathrm{ln}\frac{-q^2}{\Lambda^2},
\end{split}
\end{equation}
where we have retained only the large logarithm contributions.

\begin{figure}
\centering
\subfloat[]{
\begin{fmffile}{mu3e_loop}
\begin{fmfgraph*}(130,130)
\fmfcmd{
    path quadrant, q[], otimes;
    quadrant = (0, 0) -- (0.5, 0) & quartercircle & (0, 0.5) -- (0, 0);
    for i=1 upto 4: q[i] = quadrant rotated (45 + 90*i); endfor
    otimes = q[1] & q[2] & q[3] & q[4] -- cycle;
}
\fmfwizard
\fmfleft{i1} \fmfright{o1}\fmfbottom{b1}
\fmf{fermion}{i1,v1}
\fmf{fermion}{v1,o1}
\fmf{fermion,right=1.0,tension=0.3,label=$e^-$}{v1,v2}
\fmf{fermion,right=1.0,tension=0.3,label=$e^+$}{v2,v1}
\fmfv{d.sh=otimes,d.f=empty}{v1}
\fmf{photon,label=$\gamma$}{v2,b1}
\fmflabel{$\mu$}{i1}
\fmflabel{$e$}{o1}
\end{fmfgraph*}
\end{fmffile}
}
\hspace{3cm}
\subfloat[]{
\begin{fmffile}{mu3e_dipole}
\begin{fmfgraph*}(130,130)
\fmfcmd{
    path quadrant, q[], otimes;
    quadrant = (0, 0) -- (0.5, 0) & quartercircle & (0, 0.5) -- (0, 0);
    for i=1 upto 4: q[i] = quadrant rotated (45 + 90*i); endfor
    otimes = q[1] & q[2] & q[3] & q[4] -- cycle;
}
\fmfwizard
\fmfleft{i1,i2,i3,i4} \fmfright{o1,o2,o3,o4}
\fmf{fermion}{i3,v1}
\fmf{fermion}{v1,o3}
\fmf{fermion,tension=0.8}{o2,v2}
\fmf{fermion,tension=1.2}{v2,o1}
\fmfv{d.sh=circle,d.f=shaded}{v1}
\fmf{photon,tension=1.5,label=$\gamma$}{v1,v2}
\fmflabel{$\mu$}{i3}
\fmflabel{$e^-$}{o1}
\fmflabel{$e^+$}{o2}
\fmflabel{$e$}{o3}
\end{fmfgraph*}
\end{fmffile}
}
\caption{(a) One-loop diagram in which an underlying $\mu\rightarrow 3e$ vertex (Eq. \ref{eq:L_mu3e}) generates $\mu\rightarrow e\gamma$. This contribution is only nonzero for off-shell photons. (b) Tree-level diagram in which $\mu\rightarrow e$ induces $\mu\rightarrow 3e$ via pair production.}
\label{fig:mu3e_diagrams}
\end{figure}

Assuming that these are the only contributions to the charge radius and anapole form factors, these relations can be inverted to write the $\mu\rightarrow 3e$ branching ratio in terms of the induced couplings
\begin{equation}
B(\mu\rightarrow 3e)=\left(\frac{v}{\Lambda}\right)^4\left(\frac{12\pi^2}{\mathrm{ln}\left(m_{\mu}^2/\Lambda^2\right)}\right)^{1/2}\left(|\tilde{f}_R(-m_{\mu}^2)|^2+|\tilde{f}_A(-m_{\mu}^2)|^2\right)
\end{equation}
This expression describes the $\mu\rightarrow 3e$ process when the CLFV is dominated by the charge radius (or anapole) coupling but vanishes in the limit where the magnetic (or electric) dipole moment coupling becomes large. In the latter case, there must be a significant $\mu\rightarrow e\gamma$ vertex, and the $\mu\rightarrow 3e$ process then proceeds through a virtual photon decaying to an electron/positron pair, as in Figure \ref{fig:mu3e_diagrams} (b). In this limit, the branching ratios satisfy a simple relation \cite{PhysRevD.53.2442}
\begin{equation}
B(\mu\rightarrow 3e)\approx \frac{\alpha}{3\pi}\left(\mathrm{ln}\frac{m_{\mu}^2}{m_e^2}-2\right)B(\mu\rightarrow e\gamma)
\end{equation}

We may now interpret the $\mu\rightarrow 3e$ exclusion curves in Figure \ref{fig:mu2e_meg_mu3e}: when $|\tilde{f}_R|\gg |\tilde{f}_M$, $\mu\rightarrow 3e$ proceeds through the four-Fermion interaction of Eq. \ref{eq:L_mu3e} whereas $\mu\rightarrow e$ is induced by the large-logarithm-enhanced loop diagram. Despite the logarithmic enhancement, the $\alpha$-suppression of the virtual photon process is significant enough that -- assuming final phase sensitivities -- Mu3e will provide a stronger constraint on the charge-radius-dominated region of parameter space. In contrast, when $|\tilde{f}_R|\ll |\tilde{f}_M$, then both $\mu\rightarrow e$ and $\mu\rightarrow 3e$ are suppressed by $\alpha$, and thus the advantage of Mu2e and COMET in sensitivity over Mu3e translates into a stronger constraint on dipole-dominated couplings. 

We reiterate that the scenario discussed here is not general. Although Eq. \ref{eq:em_clfv} is the most general CLFV electromagnetic coupling describing the $\mu\rightarrow e\gamma$ process, additional sources of $\mu\rightarrow 3e$ and $\mu\rightarrow e$ will alter the interplay between these three processes. Nonetheless, it is worth taking a moment to survey the landscape of Figure \ref{fig:mu2e_meg_mu3e} to get a scope of the rich variety of experimental outcomes which are possible in the next generation. If we consider the charge-radius-dominated scenario $|\tilde{f}_R/\tilde{f}_M|\gg 1$, then it is extremely unlikely that we will detect the on-shell process at MEG-II. Depending on the scale of CLFV physics, it may be possible to detect CLFV at both Mu2e/COMET and Mu3e or at only Mu3e (or of course, a null result at all experiments). At the other extreme, the magnetic dipole dominated scenario $|\tilde{f}_R/\tilde{f}_M|\gg 1$, depending on the scale $\Lambda$ we may observe (1) detection at MEG-II, Mu2e/COMET and Mu3e (2) detection at Mu2e/COMET and Mu3e but not MEG-II (3) detection only at Mu2e/COMET (4) no detection. Finally, if the source of CLFV is such that the charge radius and magnetic dipole couplings are roughly equal $|\tilde{f}_R/\tilde{f}_M|\approx 1$, then we must also consider the possible cancellation in the $\mu\rightarrow e$ conversion rate. Without cancellation, we could observe (1) CLFV signal at MEG-II, Mu2e/COMET and Mu3e (2) CLFV signal at Mu2e/COMET and Mu3e but not MEG-II (3) CLFV signal only at Mu3e (4) no CLFV signal. If the cancellation is significant, then we don't expect any signal from the conversion experiments Mu2e and COMET. One particularly interesting possibility is that we may observe a CLFV signal at MEG-II and Mu3e but not Mu2e/COMET. Such a result could be interpreted as good evidence for the cancellation scenario, though of course a more general analysis would be warranted.
\chapter{Inelastic $\mu\rightarrow \lowercase{e}$ Conversion}
We have learned from our effective theory of $\mu\rightarrow e$ conversion that focusing attention on the elastic process where the nucleus remains in the ground state restricts the possible responses from eleven independent multipole operators to six. 
\begin{figure}
\includegraphics[scale=1.0]{czarnecki_backgrounds.png}
\caption{The differential decay rate for standard model $\mu\rightarrow e+2\nu$ decays, taken from [cite czarnecki].}
\end{figure}
\section{Kinematics}
Given the form of the electron background from standard weak muon decays, the favorable energy window for experiments is typically ~3 MeV. Therefore we will restrict our consideration of inelastic processes to nuclear transitions with $\Delta E^\mathrm{nuc}=E^\mathrm{nuc}_f-E^\mathrm{nuc}_i \lesssim 3$ MeV. This energy transfer to the nucleus will be subtracted from the energy of the outgoing electron. As the energy transfer is still small compared to the energy of the outgoing electron, the electron will still be ultra-relativistic. We will employ the same effective momentum approximation but now the effective momentum is computed as
\begin{equation}
q^2_\mathrm{eff}=\frac{M_T}{m_{\mu}+M_T}\left[\left(m_{\mu}+E_{\mu}^\mathrm{bind}-\Delta E^\mathrm{nuc}-\bar{V}_c\right)^2-m_e^2\right]
\end{equation}
in order to account for the energy transferred to the nucleus. Otherwise the kinematics are unchanged.
\section{Current Conservation and Siegert's Theorem}
The continuity equation in momentum space takes the form
\begin{equation}
\vec{q}\cdot\vec{J}(\vec{q})=-i\dot{\rho}(\vec{q})=\left[H_0,\rho(\vec{q})\right],
\end{equation}
where $\rho$ is the charge and $H_0$ is the nuclear Hamiltonian.
\section{Expression for the Branching Ratio}
\begin{equation}
\begin{split}
\mathcal{M}_\mathrm{nucleus/EFT}&=\sum_{\tau=0,1}\bra{\frac{1}{2}m_{s_f};j_fm_f}\left[\sum_{J=0}^{\infty}\sqrt{4\pi(2J+1)}(-i)^J\left[l_0^{\tau}M_{J0;\tau}(q_\mathrm{eff})-il_0^{A\;\tau}\frac{q}{m_N}\tilde{\Omega}_{J0;\tau}(q_\mathrm{eff})\right]\right.\\
&+\sum_{J=1}^{\infty}\sqrt{2\pi(2J+1)}(-i)^J\sum_{\lambda =\pm 1}(-1)^\lambda\Big[l_{5\lambda}^{\tau}\left(\lambda\Sigma_{J-\lambda;\tau}(q_\mathrm{eff})+i\Sigma'_{J-\lambda;\tau}(q_\mathrm{eff})\right)\\
&-i\frac{q_\mathrm{eff}}{m_N}l_{M\lambda}^{\tau}\left(\lambda\Delta_{J-\lambda;\tau}(q_\mathrm{eff})+i\Delta'_{J-\lambda;\tau}(q_\mathrm{eff})\right)-i\frac{q_\mathrm{eff}}{m_N}l^{\tau}_{E\lambda}\left(\lambda\tilde{\Phi}_{J-\lambda;\tau}(q_\mathrm{eff})+i\tilde{\Phi}'_{J-\lambda;\tau}(q_\mathrm{eff})\right)\Big]\\
&+\sum_{J=0}^{\infty}\sqrt{4\pi(2J+1)}(-i)^J\left[il_{50}^{\tau}\Sigma''_{J0;\tau}(q_\mathrm{eff})+\frac{q_\mathrm{eff}}{m_N}l_{M0}^{\tau}\tilde{\Delta}''_{J0;\tau}(q_\mathrm{eff})+\frac{q}{m_N}l_{E0}^{\tau}\Phi''_{J0;\tau}(q_\mathrm{eff})\right]\Bigg]\ket{\frac{1}{2}m_{s_i};j_im_i}
\end{split}
\end{equation}
Let us assume that the nuclear transition conserves parity. Then the $\mu$-to-$e$ transition probability may be written
\begin{equation}
\begin{split}
&\frac{1}{2j_i+1}\sum_{m_f,m_i}|\braket{\frac{1}{2}m_{s_f};j_fm_f|\mathcal{M}^\mathrm{inelastic}_\mathrm{nucleus/EFT}|\frac{1}{2}m_{s_i};j_im_i}|^2=\frac{E_e}{2m_e}|R_{1s}^\mathrm{Z_{eff}}(0)|^2\frac{4\pi}{2j_i+1}\sum_{\tau=0,1}\sum_{\tau'=0,1}\\
&\Bigg\{\sum_{J=0,2,...}^{\infty}\Bigg(\braket{l_0^{\tau}}\braket{l_0^{\tau'}}^*\braket{j_f||M_{J,\tau}(q_\mathrm{eff})||j_i}\braket{j_f||M_{J,\tau'}(q_\mathrm{eff})||j_i}\\
&+\frac{\vec{q}_\mathrm{eff}}{m_N}\cdot\braket{\vec{l}_E^{\tau}}\frac{\vec{q}_\mathrm{eff}}{m_N}\cdot\braket{\vec{l}_E^{\tau'}}^*\braket{j_f||\Phi''_{J,\tau}(q_\mathrm{eff})||j_i}\braket{j_f||\Phi''_{J,\tau'}(q_\mathrm{eff})||j_i}\\
&+\frac{\vec{q}_\mathrm{eff}}{m_N}\cdot\braket{\vec{l}_M^{\tau}}\frac{\vec{q}_\mathrm{eff}}{m_N}\cdot\braket{\vec{l}_M^{\tau'}}^*\braket{j_f||\tilde{\Delta}''_{J,\tau}(q_\mathrm{eff})||j_i}\braket{j_f||\tilde{\Delta}''_{J,\tau'}(q_\mathrm{eff})||j_i}\\
&+\frac{2\vec{q}_\mathrm{eff}}{m_N}\cdot\mathrm{Re}\left[\braket{\vec{l}_E^{\tau}}\braket{l_0^{\tau'}}^*\right]\braket{j_f||\Phi''_{J,\tau}(q_\mathrm{eff})||j_i}\braket{j_f||M_{J,\tau'}(q_\mathrm{eff})||j_i}\\
&+\frac{2\vec{q}_\mathrm{eff}}{m_N}\cdot\mathrm{Re}\left[\braket{\vec{l}_M^{\tau}}\braket{l_0^{\tau'}}^*\right]\braket{j_f||\tilde{\Delta}''_{J,\tau}(q_\mathrm{eff})||j_i}\braket{j_f||M_{J,\tau'}(q_\mathrm{eff})||j_i}\\
&+2\mathrm{Re}\left[\frac{\vec{q}_\mathrm{eff}}{m_N}\cdot\braket{\vec{l}_E^{\tau}}\frac{\vec{q}_\mathrm{eff}}{m_N}\cdot\braket{\vec{l}_M^{\tau'}}^*\right]\braket{j_f||\Phi''_{J,\tau}(q_\mathrm{eff})||j_i}\braket{j_f||\tilde{\Delta}''_{J,\tau'}(q_\mathrm{eff})||j_i}\Bigg)\\
&+\sum_{J=2,4,...}^{\infty}\Bigg(\frac{1}{2}\left(\frac{q^2_\mathrm{eff}}{m_N^2}\braket{\vec{l}_E^{\tau}}\cdot\braket{\vec{l}_E^{\tau'}}^*-\frac{\vec{q}_\mathrm{eff}}{m_N}\cdot\braket{\vec{l}_E^{\tau}}\frac{\vec{q}_\mathrm{eff}}{m_N}\cdot\braket{\vec{l}_E^{\tau'}}^*\right)\braket{j_f||\tilde{\Phi}'_{J,\tau}(q_\mathrm{eff})||j_i}\braket{j_f||\tilde{\Phi}'_{J,\tau'}(q_\mathrm{eff})||j_i}\\
&+\frac{1}{2}\left(\frac{q_{\mathrm{eff}}^2}{m_N^2}\braket{\vec{l}_M^{\tau}}\cdot\braket{\vec{l}_M^{\tau'}}^*-\frac{\vec{q}_\mathrm{eff}}{m_N}\cdot\braket{\vec{l}_M^{\tau}}\frac{\vec{q}_\mathrm{eff}}{m_N}\cdot\braket{\vec{l}_M^{\tau'}}^*\right)\braket{j_f||\Delta'_{J,\tau}(q_\mathrm{eff})||j_i}\braket{j_f||\Delta'_{J,\tau'}(q_\mathrm{eff})||j_i}\\
&+\frac{1}{2}\left(\braket{\vec{l}_5^{\tau}}\cdot\braket{\vec{l}_5^{\tau'}}^*-\hat{q}\cdot\braket{\vec{l}_5^{\tau}}\hat{q}\cdot\braket{\vec{l}_5^{\tau'}}^*\right)\braket{j_f||\Sigma_{J,\tau}(q_\mathrm{eff})||j_i}\braket{j_f||\Sigma_{J,\tau'}(q_\mathrm{eff})||j_i}\\
&-\frac{\vec{q}_\mathrm{eff}}{m_N}\cdot\mathrm{Re}\left[i\braket{\vec{l}_M^{\tau}}\times\braket{\vec{l}_5^{\tau'}}^*\right]\braket{j_f||\Delta'_{J,\tau}(q_\mathrm{eff})||j_i}\braket{j_f||\Sigma_{J,\tau'}(q_\mathrm{eff})||j_i}\\
&-\frac{\vec{q}_\mathrm{eff}}{m_N}\cdot\mathrm{Re}\left[i\braket{\vec{l}_5^{\tau}}\times\braket{\vec{l}_E^{\tau'}}^*\right]\braket{j_f||\Sigma_{J,\tau}(q_\mathrm{eff})||j_i}\braket{j_f||\tilde{\Phi}'_{J,\tau'}(q_\mathrm{eff})||j_i}\\
&+\mathrm{Re}\left[\frac{q_\mathrm{eff}^2}{m_N^2}\braket{\vec{l}_M^{\tau}}\cdot\braket{\vec{l}_E^{\tau'}}^*-\frac{\vec{q}_\mathrm{eff}}{m_N}\cdot\braket{\vec{l}_M^{\tau}}\frac{\vec{q}_\mathrm{eff}}{m_N}\cdot\braket{\vec{l}_E^{\tau'}}^*\right]\braket{j_f||\Delta'_{J,\tau}(q_\mathrm{eff})||j_i}\braket{j_f||\tilde{\Phi}'_{J,\tau'}(q_\mathrm{eff})||j_i}\Bigg)\\
&+\sum_{J=1,3,...}^{\infty}\Bigg(\frac{q_\mathrm{eff}^2}{m_N^2}\braket{l_0^{A\;\tau}}\braket{l_0^{A\;\tau'}}^*\braket{j_f||\tilde{\Omega}_{J,\tau}(q_\mathrm{eff})||j_i}\braket{j_f||\tilde{\Omega}_{J,\tau'}(q_\mathrm{eff})||j_i}\\
&+\hat{q}\cdot\braket{\vec{l}_5^{\tau}}\hat{q}\cdot\braket{\vec{l}_5^{\tau'}}^*\braket{j_f||\Sigma''_{J,\tau}(q_\mathrm{eff})||j_i}\braket{j_f||\Sigma''_{J,\tau'}(q_\mathrm{eff})||j_i}\\
&+\frac{1}{2}\left(\braket{\vec{l}_5^{\tau}}\cdot\braket{\vec{l}_5^{\tau'}}^*-\hat{q}\cdot\braket{\vec{l}_5^{\tau}}\hat{q}\cdot\braket{\vec{l}_5^{\tau'}}^*\right)\braket{j_f||\Sigma'_{J,\tau}(q_\mathrm{eff})||j_i}\braket{j_f||\Sigma'_{J,\tau'}(q_\mathrm{eff})||j_i}\\
&+\frac{1}{2}\left(\frac{q^2_\mathrm{eff}}{m_N^2}\braket{\vec{l}_M^{\tau}}\cdot\braket{\vec{l}_M^{\tau'}}^*-\frac{\vec{q}_\mathrm{eff}}{m_N}\cdot\braket{\vec{l}_M^{\tau}}\frac{\vec{q}_\mathrm{eff}}{m_N}\cdot\braket{\vec{l}_M^{\tau'}}^*\right)\braket{j_f||\Delta_{J,\tau}(q_\mathrm{eff})||j_i}\braket{j_f||\Delta_{J,\tau'}(q_\mathrm{eff})||j_i}\\
&+\frac{1}{2}\left(\frac{q_\mathrm{eff}^2}{m_N^2}\braket{\vec{l}_E^{\tau}}\cdot\braket{\vec{l}_E^{\tau'}}^*-\frac{\vec{q}_\mathrm{eff}}{m_N}\cdot\braket{\vec{l}_E^{\tau}}\frac{\vec{q}_\mathrm{eff}}{m_N}\cdot\braket{\vec{l}_E^{\tau'}}^*\right)\braket{j_f||\tilde{\Phi}_{J,\tau}(q_\mathrm{eff})||j_i}\braket{j_f||\tilde{\Phi}_{J,\tau'}(q_\mathrm{eff})||j_i}\\
&+\frac{\vec{q}_\mathrm{eff}}{m_N}\cdot\mathrm{Re}\left[i\braket{\vec{l}_M^{\tau}}\times\braket{\vec{l}_5^{\tau'}}^*\right]\braket{j_f||\Delta_{J,\tau}(q_\mathrm{eff})||j_i}\braket{j_f||\Sigma'_{J,\tau'}(q_\mathrm{eff})||j_i}\\
&+\frac{\vec{q}_\mathrm{eff}}{m_N}\cdot\mathrm{Re}\left[i\braket{\vec{l}_5^{\tau}}\times\braket{\vec{l}_E^{\tau'}}^*\right]\braket{j_f||\Sigma'_{J,\tau}(q_\mathrm{eff})||j_i}\braket{j_f||\tilde{\Phi}_{J,\tau'}(q_\mathrm{eff})||j_i}\\
&-\frac{2\vec{q}_\mathrm{eff}}{m_N}\cdot\mathrm{Re}\left[\braket{\vec{l}_5^{\tau}}\braket{l_0^{A\;\tau'}}^*\right]\braket{j_f||\Sigma''_{J,\tau}(q_\mathrm{eff})||j_i}\braket{j_f||\tilde{\Omega}_{J,\tau'}(q_\mathrm{eff})||j_i}\\
&+\mathrm{Re}\left[\frac{q_\mathrm{eff}^2}{m_N^2}\braket{\vec{l}_M^{\tau}}\cdot\braket{\vec{l}_E^{\tau'}}^*-\frac{\vec{q}_\mathrm{eff}}{m_N}\cdot\braket{\vec{l}_M^{\tau}}\frac{\vec{q}_\mathrm{eff}}{m_N}\cdot\braket{\vec{l}_E^{\tau'}}^*\right]\braket{j_f||\Delta_{J,\tau}(q_\mathrm{eff})||j_i}\braket{j_f||\tilde{\Phi}_{J,\tau'}(q_\mathrm{eff})||j_i}\Bigg)\Bigg\},
\end{split}
\end{equation}
We see that all eleven nuclear response functions, and various interference terms, contribute to this process. Some of the leptonic factors are not unique. This can be put in the form
\begin{equation}
\begin{split}
&\frac{1}{2j_i+1}\sum_{m_f,m_i}|\braket{\frac{1}{2}m_{s_f};j_fm_f|\mathcal{M}^\mathrm{inelastic,PC}_\mathrm{nucleus/EFT}|\frac{1}{2}m_{s_i};j_im_i}|^2=\frac{E_e}{2m_e}|R_{1s}^\mathrm{Z_{eff}}(0)|^2\frac{4\pi}{2j_i+1}\sum_{\tau=0,1}\sum_{\tau'=0,1}\\
&\Bigg\{\sum_{J=0,2,...}^{\infty}\Bigg(R_M^{\tau\tau'}\braket{j_f||M_{J,\tau}(q_\mathrm{eff})||j_i}\braket{j_f||M_{J,\tau'}(q_\mathrm{eff})||j_i}+\frac{q^2_\mathrm{eff}}{m_N^2}R_{\Phi''}^{\tau\tau'}\braket{j_f||\Phi''_{J,\tau}(q_\mathrm{eff})||j_i}\braket{j_f||\Phi''_{J,\tau'}(q_\mathrm{eff})||j_i}\\
&+\frac{q^2_\mathrm{eff}}{m_N^2}R_{\tilde{\Delta}''}^{\tau\tau'}\braket{j_f||\tilde{\Delta}''_{J,\tau}(q_\mathrm{eff})||j_i}\braket{j_f||\tilde{\Delta}''_{J,\tau'}(q_\mathrm{eff})||j_i}+\frac{2q_\mathrm{eff}}{m_N}R_{\Phi''M}^{\tau\tau'}\braket{j_f||\Phi''_{J,\tau}(q_\mathrm{eff})||j_i}\braket{j_f||M_{J,\tau'}(q_\mathrm{eff})||j_i}\\
&+\frac{2q_\mathrm{eff}}{m_N}R_{\tilde{\Delta}''M}^{\tau\tau'}\braket{j_f||\tilde{\Delta}''_{J,\tau}(q_\mathrm{eff})||j_i}\braket{j_f||M_{J,\tau'}(q_\mathrm{eff})||j_i}+2\frac{q_\mathrm{eff}^2}{m_N^2}R_{\Phi''\tilde{\Delta}''}^{\tau\tau'}\braket{j_f||\Phi''_{J,\tau}(q_\mathrm{eff})||j_i}\braket{j_f||\tilde{\Delta}''_{J,\tau'}(q_\mathrm{eff})||j_i}\Bigg)\\
&+\sum_{J=2,4,...}^{\infty}\Bigg(\frac{q^2_\mathrm{eff}}{m_N^2}R_{\tilde{\Phi}'}^{\tau\tau'}\braket{j_f||\tilde{\Phi}'_{J,\tau}(q_\mathrm{eff})||j_i}\braket{j_f||\tilde{\Phi}'_{J,\tau'}(q_\mathrm{eff})||j_i}+\frac{q_{\mathrm{eff}}^2}{m_N^2}R_{\Delta'}^{\tau\tau'}\braket{j_f||\Delta'_{J,\tau}(q_\mathrm{eff})||j_i}\braket{j_f||\Delta'_{J,\tau'}(q_\mathrm{eff})||j_i}\\
&+R_{\Sigma}^{\tau\tau'}\braket{j_f||\Sigma_{J,\tau}(q_\mathrm{eff})||j_i}\braket{j_f||\Sigma_{J,\tau'}(q_\mathrm{eff})||j_i}-\frac{q_\mathrm{eff}}{m_N}R_{\Delta'\Sigma}^{\tau\tau'}\braket{j_f||\Delta'_{J,\tau}(q_\mathrm{eff})||j_i}\braket{j_f||\Sigma_{J,\tau'}(q_\mathrm{eff})||j_i}\\
&-\frac{q_\mathrm{eff}}{m_N}R_{\Sigma\tilde{\Phi}'}^{\tau\tau'}\braket{j_f||\Sigma_{J,\tau}(q_\mathrm{eff})||j_i}\braket{j_f||\tilde{\Phi}'_{J,\tau'}(q_\mathrm{eff})||j_i}+\frac{q_\mathrm{eff}^2}{m_N^2}R_{\Delta'\tilde{\Phi}'}^{\tau\tau'}\braket{j_f||\Delta'_{J,\tau}(q_\mathrm{eff})||j_i}\braket{j_f||\tilde{\Phi}'_{J,\tau'}(q_\mathrm{eff})||j_i}\Bigg)\\
&+\sum_{J=1,3,...}^{\infty}\Bigg(\frac{q_\mathrm{eff}^2}{m_N^2}R_{\tilde{\Omega}}^{\tau\tau'}\braket{j_f||\tilde{\Omega}_{J,\tau}(q_\mathrm{eff})||j_i}\braket{j_f||\tilde{\Omega}_{J,\tau'}(q_\mathrm{eff})||j_i}+R_{\Sigma''}^{\tau\tau'}\braket{j_f||\Sigma''_{J,\tau}(q_\mathrm{eff})||j_i}\braket{j_f||\Sigma''_{J,\tau'}(q_\mathrm{eff})||j_i}\\
&+R_{\Sigma'}^{\tau\tau'}\braket{j_f||\Sigma'_{J,\tau}(q_\mathrm{eff})||j_i}\braket{j_f||\Sigma'_{J,\tau'}(q_\mathrm{eff})||j_i}+\frac{q^2_\mathrm{eff}}{m_N^2}R_{\Delta}^{\tau\tau'}\braket{j_f||\Delta_{J,\tau}(q_\mathrm{eff})||j_i}\braket{j_f||\Delta_{J,\tau'}(q_\mathrm{eff})||j_i}\\
&+\frac{q_\mathrm{eff}^2}{m_N^2}R_{\tilde{\Phi}}^{\tau\tau'}\braket{j_f||\tilde{\Phi}_{J,\tau}(q_\mathrm{eff})||j_i}\braket{j_f||\tilde{\Phi}_{J,\tau'}(q_\mathrm{eff})||j_i}+\frac{q_\mathrm{eff}}{m_N}R_{\Delta\Sigma'}^{\tau\tau'}\braket{j_f||\Delta_{J,\tau}(q_\mathrm{eff})||j_i}\braket{j_f||\Sigma'_{J,\tau'}(q_\mathrm{eff})||j_i}\\
&+\frac{q_\mathrm{eff}}{m_N}R_{\Sigma'\tilde{\Phi}}^{\tau\tau'}\braket{j_f||\Sigma'_{J,\tau}(q_\mathrm{eff})||j_i}\braket{j_f||\tilde{\Phi}_{J,\tau'}(q_\mathrm{eff})||j_i}-\frac{2q_\mathrm{eff}}{m_N}R_{\Sigma''\tilde{\Omega}}^{\tau\tau'}\braket{j_f||\Sigma''_{J,\tau}(q_\mathrm{eff})||j_i}\braket{j_f||\tilde{\Omega}_{J,\tau'}(q_\mathrm{eff})||j_i}\\&+\frac{q_\mathrm{eff}^2}{m_N^2}R_{\Delta\tilde{\Phi}}^{\tau\tau'}\braket{j_f||\Delta_{J,\tau}(q_\mathrm{eff})||j_i}\braket{j_f||\tilde{\Phi}_{J,\tau'}(q_\mathrm{eff})||j_i}\Bigg)\Bigg\},
\end{split}
\end{equation}
where the R's are given by
\begin{equation}
\begin{split}
R_M^{\tau\tau'}&=c_1^{\tau}c_1^{\tau'*}+c_{11}^{\tau}c_{11}^{\tau'*}\\
R_{\Phi''}^{\tau\tau'}&=c_3^{\tau}c_3^{\tau'*}+(c_{12}^{\tau}-c_{15}^{\tau})(c_{12}^{\tau'*}-c_{15}^{\tau'*})\\
R_{\tilde{\Delta}''}^{\tau\tau'}&=c_2^{\tau}c_2^{\tau'*}+(c_8^{\tau}-c_{16}^{\tau})(c_8^{\tau'*}-c_{16}^{\tau'*})\\
R_{\Phi''M}^{\tau\tau'*}&=\mathrm{Re}\left[-c_3^{\tau}c_1^{\tau'*}+(c_{12}^{\tau}-c_{15}^{\tau})c_{11}^{\tau'*}\right]\\
R_{\tilde{\Delta}''M}^{\tau\tau'}&=\mathrm{Re}\left[i\left(c_2^{\tau}c_1^{\tau'*}+i(c_8^{\tau}-c_{16}^{\tau})c_{11}^{\tau'*}\right)\right]\\
R_{\Phi''\tilde{\Delta}''}^{\tau\tau'}&=\mathrm{Re}\left[i\left(c_3^{\tau}c_2^{\tau'*}+i(c_{12}^{\tau}-c_{15}^{\tau})(c_8^{\tau'*}-c_{16}^{\tau'*})\right)\right]\\
R_{\Sigma''}^{\tau\tau'}&=(c_4^{\tau}-c_6^{\tau})(c_4^{\tau'*}-c_6^{\tau'*})+c_{10}^{\tau}c_{10}^{\tau'*}\\
R_{\tilde{\Omega}}^{\tau,\tau'}&=c_7^{\tau}c_7^{\tau'*}+c_{14}^{\tau}c_{14}^{\tau'*}\\
R_{\Sigma''\tilde{\Omega}}^{\tau\tau'}&=\mathrm{Re}\left[i\left((c_6^{\tau}-c_4^{\tau})c_{14}^{\tau'*}+c_{10}^{\tau}c_7^{\tau'*}\right)\right]\\
R_{\tilde{\Phi}}^{\tau\tau'}=R_{\tilde{\Phi}'}^{\tau\tau'}&=c_{12}^{\tau}c_{12}^{\tau'*}+c_{13}^{\tau}c_{13}^{\tau'*}\\
R_{\Sigma}^{\tau\tau'}=R_{\Sigma'}^{\tau\tau'}&=c_4^{\tau}c_4^{\tau'*}+c_9^{\tau}c_9^{\tau'*}\\
R_{\Delta}^{\tau\tau'}=R_{\Delta'}^{\tau\tau'}&=c_5^{\tau}c_5^{\tau'*}+c_8^{\tau}c_8^{\tau'*}\\
R_{\Delta'\Sigma}^{\tau\tau'}=R_{\Delta\Sigma'}^{\tau\tau'}&=\mathrm{Re}\left[c_5^{\tau}c_4^{\tau'*}+c_8^{\tau}c_9^{\tau'*}\right]\\
R_{\Delta\tilde{\Phi}}^{\tau\tau'}=R_{\Delta'\tilde{\Phi}'}^{\tau\tau'}&=\mathrm{Re}\left[i\left(c_5^{\tau}c_{13}^{\tau'*}+c_8^{\tau}c_{12}^{\tau'*}\right)\right]\\
R_{\Sigma'\tilde{\Phi}}^{\tau\tau'}=R_{\Sigma\tilde{\Phi}'}^{\tau\tau'}&=\mathrm{Re}\left[i\left(c_4^{\tau}c_{13}^{\tau'*}+c_9^{\tau}c_{12}^{\tau'*}\right)\right]\\
\end{split}
\end{equation}
Therefore we identify 15 independent nuclear responses allowing one to probe 15 independent linear combinations of the underlying LECs, compared to the 8 independent linear combinations probed by the elastic process. Moreover, all 16 (32 including isospin) of the LECs are probed by inelastic transitions whereas only 12 (24) are probed by the elastic process.

If the nucleus transitions to a state of opposite parity, then the nuclear response functions must change to opposite parity. The response functions $R$ are unchanged but the overall expression is
\section{Parity-violating Nuclear Transitions}
In the case where the nucleus transitions to a state of opposite parity, then the response functions exchange (even $J$ $\leftrightarrow$ odd $J$).

\chapter{Scalar-mediated Coherent $\mu\rightarrow \lowercase{e}$ Conversion}
At present, there exist a wealth of possible extensions of the standard model which are not in tension with existing limits on CLFV processes. Although any positive observation of CLFV is an unambiguous signal of BSM physics, an isolated measurement is not sufficient to distinguish between candidate UV theories. In this thesis, we have introduced the necessary formalism at the nuclear scale to extract all of the available information about CLFV operators from elastic $\mu\rightarrow e$ conversion experiments. Satisfied that we have constructed the most general effective theory at the single-nucleon level, our next aim is to connect constraints obtained on the low-energy constants of the nuclear effective theory to the coefficients in higher-scale effective theories through a successive matching procedure, eventually making contact with UV theories of CLFV. Short of the developing the general matching procedure, in this chapter we will focus our attention on the case of coherent $\mu\rightarrow e$ conversion mediated by scalar quark operators. We map these operators to the nucleon-level effective theory using chiral perturbation theory to match the quark and nucleon degrees of freedom. We will also demonstrate the advantages of this limited, top-down approach, wherein the resulting $\mu\rightarrow e$ conversion branching ratio can be computed with quantified uncertainty. The analysis and discussion will mirror that found in \cite{Cirigliano:2022ekw}.
%\begin{figure}
%\centering
%\subfloat[]{
%\begin{fmffile}{bsm_diagram}
%\begin{fmfgraph*}(120,120)
%\fmfleft{i1} \fmfright{o1}\fmfbottom{b1}
%\fmf{fermion}{i1,v1}
%\fmf{fermion}{v1,o1}
%\fmf{dashes,label=$X$}{v1,b1}
%\fmflabel{$\mu^-$}{i1}
%\fmflabel{$e^-$}{o1}
%\end{fmfgraph*}
%\end{fmffile}
%}
%\hfill
%\subfloat[]{
%\begin{fmffile}{light_quark_eft}
%\begin{fmfgraph*}(120,120)
%\fmfleft{i1,i2} \fmfright{o1,o2}
%\fmf{fermion}{i2,v1}
%\fmf{fermion}{v1,o2}
%\fmf{fermion}{i1,v1}
%\fmf{fermion}{v1,o1}
%\fmfblob{.15w}{v1}
%\fmflabel{$Q$}{i1}
%\fmflabel{$Q$}{o1}
%\fmflabel{$\mu^-$}{i2}
%\fmflabel{$e^-$}{o2}
%\end{fmfgraph*}
%\end{fmffile}
%}
%\hfill
%\subfloat[]{
%\begin{fmffile}{hbchpt}
%\begin{fmfgraph*}(120,120)
%\fmfleft{i1,i2} \fmfright{o1,o2}
%\fmf{fermion}{i2,v1}
%\fmf{fermion}{v1,o2}
%\fmf{dbl_plain}{i1,v1}
%\fmfv{decoration.shape=square,decoration.filled=shaded}{v1}
%\fmf{dbl_plain}{v1,o1}
%\fmflabel{$N$}{i1}
%\fmflabel{$N$}{o1}
%\fmflabel{$\mu^-$}{i2}
%\fmflabel{$e^-$}{o2}
%\end{fmfgraph*}
%\end{fmffile}
%}
%\caption{Example CLFV diagrams in theories at different scales: (a) Above the scale of new physics. (b) Below the scale of new physics, SM-EFT or LEFT. (c) Below the hadronization scale.}
%\label{fig:hbchpt}
%\end{figure}

\section{Quark-level Effective Theories: SM-EFT \& LEFT}
Augmenting the standard model (SM) with higher-dimensional SU$(3)_c\times$SU$(2)_L\times$U$(1)_Y$ invariant operators constructed from SM fields, one obtains the so-called standard model effective field theory (SM-EFT) \cite{Brivio:2017vri}. At dimension five, there is only one operator that obeys the SM gauge symmetries, the famous Weinberg operator \cite{PhysRevLett.43.1566}, which violates lepton number $L$ by two units and therefore permits processes such as neutrinoless double beta decay. The Weinberg operator is not relevant to the $L$ conserving CLFV processes which we consider; one the other hand, many different CLFV operators arise in SM-EFT at dimension six.

Below the scale of electroweak symmetry breaking, the SU$(2)_L\times$U$(1)_Y$ symmetry is broken to the electromagnetic gauge symmetry U$(1)_\mathrm{EM}$. The gauge symmetry which governs the effective theory is now SU$(3)_c\times$U$(1)_\mathrm{EM}$, and the consistent set of higher-dimensional operators is known as low-energy effective field theory (LEFT). The complete set of LEFT operators up to and including dimension six is known \cite{Jenkins:2017jig}, as is their anomalous dimension \cite{Jenkins:2017dyc}. At the electroweak scale, the operators of LEFT can be matched to the operators of SM-EFT. 

Thus, one may follow the general procedure of starting with a particular UV theory of CLFV at some energy scale $\Lambda$ above the electroweak scale. Below the scale of new physics, any heavy particles associated with CLFV can be integrated out, inducing higher dimensional SM-EFT operators. The SM-EFT operators can be renormalized down to the scale where the Higgs vacuum expectation value breaks electroweak symmetry; at this point, the SM-EFT operators are matched onto the operators of LEFT. This program of renormalization and matching was recently carried out explicitly for the case of CLFV operators mediating $\tau$ decays \cite{Cirigliano:2021img}. As the procedure is relatively straightforward and the physics is well-understood, we shall begin our discussion below the electroweak scale. Short of considering the most general basis of CLFV LEFT operators, we will restrict our attention to those operators which arise when CLFV is mediated by a heavy scalar particle (including, as we consider in detail in Section \ref{sec:CLFV_higgs}, the SM Higgs boson). We take the effective Lagrangian below the electroweak scale to be given by
\begin{equation}
\begin{split}
\mathcal{L}_\mathrm{eff}&=-\frac{1}{\Lambda^2}\sum_{\alpha=L,R}\bigg[C_{D\alpha}~m_{\mu}~\bar{e}\sigma^{\lambda\nu}P_{\alpha}\mu~F_{\lambda\nu}+\sum_{q=u,d,s,c,b,t}C^{(q)}_{S\alpha}~G_Fm_{\mu}m_q~\bar{q}q~\bar{e}P_{\alpha}\mu\\
&+C_{G\alpha}~G_Fm_{\mu}\alpha_s~G^a_{\lambda\nu}G^{a\;\lambda\nu}~\bar{e}P_{\alpha}\mu+\mathrm{h.c.}\bigg],
\label{eq:quark_L_eff}
\end{split}
\end{equation}
where $\Lambda$ is a mass scale associated with the CLFV physics, $P_{L/R}=\left(1\mp \gamma_5\right)/2$ are the chiral projection operators, $F_{\mu\nu}$ and $G^a_{\mu\nu}$ are, respectively, the photon and gluon field strength tensors, and the Wilson coefficients $C_{D\alpha}$, $C_{S\alpha}^{(q)}$ and $C_{G\alpha}$ are dimensionless. The effective theory of scalar-mediated coherent conversion is therefore specified by 16 dimensionless Wilson coefficients. 

The CLFV operators themselves carry the following mass dimension: the photon dipole is dimension five, the scalar quark coupling is dimension six, and the gluonic operator is dimension seven. We have included the factors of $m_q$ and $\alpha_s$ in the quark and gluon operators, respectively, to ensure that the Wilson coefficients $C_{S\alpha}^{(q)}$ and $C_{G\alpha}$ do not run under QCD renormalization. When the heavy quarks are integrated out at the GeV scale, their scalar quark couplings generate contributions to the gluonic Wilson coefficients
\begin{equation}
C_{G\alpha}\rightarrow C_{G\alpha}-\frac{1}{12\pi}\sum_{q=c,b,t}C^{(q)}_{S\alpha}.
\end{equation}
Otherwise the effective Lagrangian remains unchanged, with the summation of $q$ in Eq. \ref{eq:quark_L_eff} now restricted to $q=(u,d,s)$. This is the reason for retaining the dimension-seven gluon operator in our treatment: it is induced by the dimension-six heavy quark operators.

The photon dipole term induces the $\mu\rightarrow e$ conversion process through the exchange of a virtual photon with the target nucleus as in Figure \ref{fig:mu_e_gamma_diagrams} (b). The photon dipole operator can also mediate the $\mu\rightarrow e\gamma$ process -- as in Figure \ref{fig:mu_e_gamma_diagrams} (a) -- in which an on-shell photon is produced. The branching ratio for the on-shell process has the simple form 
\begin{equation}
B(\mu\rightarrow e\gamma)=96\pi^2\left(\frac{v}{\Lambda}\right)^4\left(|C_{DR}|^2+|C_{DL}|^2\right).
\end{equation} 
\section{Connecting Quarks to Nucleons with Chiral Effective Theory}
\label{sec:quarks_2_nucleons}
To make contact with $\mu\rightarrow e$ conversion experiments, we must continue to run down in energy past the QCD scale where the quarks confine into hadrons. Beyond this point, QCD becomes strongly coupled, and a perturbative treatment in terms of quark degrees of freedom becomes intractable. In order to proceed, we must match the quark theory to one in which the degrees of freedom are protons, neutrons, and light mesons. Retaining only the up and down quark, in the limit that $m_u,m_d\rightarrow 0$, the QCD Lagrangian exhibits a chiral symmetry SU$(2)_L\times$SU$(2)_R$ which allows for separate isospin rotations of right-handed and left-handed up and down quark doublets. Ultimately, the chiral symmetry is spontaneously broken to the vector subgroup SU$(2)_L\times$SU$(2)_R\rightarrow SU(2)_V$ (the chiral symmetry is also explicitly broken by the non-zero quark masses, leading to massive pions as the pseudo-Goldstone bosons). Nonetheless, we may construct effective hadronic operators which transform under the (broken) chiral symmetry. The resulting effective field theory is known as SU$(2)$ Chiral Perturbation Theory (ChPT).

In the case of $\mu\rightarrow e$ conversion, the momentum transfer is roughly equal to the strange quark mass $|\vec{q}|\approx m_s$. Technically, one should retain the strange quark and consider the  chiral symmetry SU$(3)_L\times$SU$(3)_R$ which rotates among the three lighest quarks; however, as the external nucleon states do not carry net strangeness and the momentum transfer is not significantly greater than the strange quark mass, one expects that the effect of the strange quark on CLFV processes will be suppressed relative to the contributions from up and down quarks. In that case, we may work in the SU$(2)$ chiral theory with the strange quark treated as an additional singlet under the chiral symmetry.

The one- and two-nucleon operators arising in SU$(2)$ ChPT through next-to-leading order (NLO) in chiral power-counting have been computed previously \cite{Crivellin:2014cta,2018PhRvC..98a5208B,Crivellin:2013ipa,Korber:2017ery}. The relevant diagrams are shown in Figure \ref{fig:mm_decay_diagrams}. 


\begin{figure}
\subfloat[]{
\begin{fmffile}{contact}
\begin{fmfgraph*}(120,120)
\fmfleft{i1,i2} \fmfright{o1,o2}
\fmf{dbl_plain}{i1,v1}
\fmf{dbl_plain}{v1,o1}
\fmf{fermion}{i2,v1}
\fmf{fermion}{v1,o2}
\fmflabel{$N$}{o1}
\fmflabel{$N$}{i1}
\fmflabel{$\mu$}{i2}
\fmflabel{$e$}{o2}
\end{fmfgraph*}
\end{fmffile}
}
\hfill
\subfloat[]{
\begin{fmffile}{two_pion_loop}
\begin{fmfgraph*}(120,120)
\fmfleft{i1,i2} \fmfright{o1,o2}
\fmf{dbl_plain}{i1,v1}
\fmf{dbl_plain}{v1,v2}
\fmf{dbl_plain}{v2,o1}
\fmf{dashes,label=$\pi$,label.side=left}{v1,v3}
\fmf{dashes,label=$\pi$}{v3,v2}
\fmf{fermion}{i2,v3}
\fmf{fermion}{v3,o2}
\fmflabel{$N$}{o1}
\fmflabel{$N$}{i1}
\fmflabel{$\mu$}{i2}
\fmflabel{$e$}{o2}
\end{fmfgraph*}
\end{fmffile}
}
\hfill
\subfloat[]{
\begin{fmffile}{two_nucleon_two_pion}
\begin{fmfgraph*}(120,120)
\fmfleft{i1,i2,i3} \fmfright{o1,o2,o3}
\fmf{dbl_plain}{i1,v1}
\fmf{dbl_plain}{v1,o1}
\fmf{dbl_plain}{i3,v3}
\fmf{dbl_plain}{v3,o3}
\fmf{dashes,label=$\pi$}{v1,v2}
\fmf{dashes,label=$\pi$}{v2,v3}
\fmf{fermion}{i2,v2}
\fmf{fermion}{v2,o2}
\fmflabel{$N$}{o1}
\fmflabel{$N$}{i1}
\fmflabel{$N$}{i3}
\fmflabel{$N$}{o3}
\fmflabel{$\mu$}{i2}
\fmflabel{$e$}{o2}
\end{fmfgraph*}
\end{fmffile}
}
\caption{Diagrams in HB-ChPT contributing to scalar-mediated $\mu\rightarrow e$ conversion through next-to-leading order: (a) LO $NNe\mu$ contact interaction. (b) NLO two-pion one-loop diagram with $\pi\pi e\mu$ vertex. (e) NLO two-nucleon two-pion-exchange diagram with $\pi\pi e\mu$ vertex.}
\label{fig:mm_decay_diagrams}
\end{figure}

\begin{equation}
\begin{split}
\braket{N(\mathbf{k}')|\left(C^{(u)}_{S\alpha}m_u\bar{u}u+C^{(d)}_{S\alpha}\bar{d}d\right)|N(\mathbf{k})}&\rightarrow \bar{N}'J^{(1)}_{ud,\alpha}(\mathbf{q})N\\
\braket{N(\mathbf{k}_1')N(\mathbf{k}_2')|\left(C^{(u)}_{S\alpha}m_u\bar{u}u+C^{(d)}_{S\alpha}m_d\bar{d}d\right)|N(\mathbf{k}_1)N(\mathbf{k}_2)}&\rightarrow \bar{N}_1'\bar{N}_2'J^{(2)}_{ud,\alpha}(\mathbf{q}_1,\mathbf{q}_2)N_1N_2\\
\braket{N(\mathbf{k}')|C_{S\alpha}^{(s)}m_s\bar{s}s|N(\mathbf{k})}&\rightarrow \bar{N}'J^{(1)}_{s,\alpha}(\mathbf{q})N\\
\braket{N(\mathbf{k}')|C_{G\alpha}\alpha_sG^{a}_{\lambda\nu}G^{a\;\lambda\nu}|N(\mathbf{k})}&\rightarrow \bar{N}'J^{(1)}_{G,\alpha}(\mathbf{q})N\\
\braket{N(\mathbf{k}_1')N(\mathbf{k}_2')|C_{G\alpha}\alpha_sG^{a}_{\lambda\nu}G^{a\;\lambda\nu}|N(\mathbf{k}_1)N(\mathbf{k}_2)}&\rightarrow \bar{N}_1'\bar{N}_2'J^{(2)}_{G,\alpha}(\mathbf{q}_1,\mathbf{q}_2)N_1N_2
\end{split}
\end{equation}

As $\mu\rightarrow e$ conversion proceeds through the interaction of the bound muon with nucleons, we must take hadronic matrix elements of the currents in our quark-level theory. 

\begin{subequations}
\begin{alignat}{4}
J^{(1)}_{ud,\alpha}(q)&=\left[\sigma_{\pi N}-\frac{3m_{\pi}^3 g_A^2}{64\pi f_{\pi}^2}F(q^2/m_{\pi}^2)\right]C^{(0)}_{S\alpha}-\frac{\delta m_N}{4}\tau_3 C_{S\alpha}^{(1)}\\
J^{(2)}_{ud,\alpha}\left(\vec{q}_1,\vec{q}_2\right)&=-\frac{g_A^2m_{\pi}^2}{4f_{\pi}^2}\frac{\vec{\sigma}_1\cdot\vec{q}_1\;\vec{\sigma}_2\cdot\vec{q}_2}{\left(q_1^2+m_{\pi}^2\right)\left(q_2^2+m_{\pi}^2\right)}\;\vec{\tau}_1\cdot\vec{\tau}_2\;C^{(0)}_{S\alpha}\\
J^{(1)}_{s,\alpha}(q)&=\left(\sigma_s-\dot{\sigma}_sq^2\right)C^{(s)}_{S\alpha}\\
J^{(1)}_{G,\alpha}(q)&=-\frac{8\pi}{9}C_{G\alpha}\left(m_N-\left[\sigma_{\pi N}-\frac{3m_{\pi}^3g_A^2}{64\pi f_{\pi}^2}F(q^2/m_{\pi}^2)\right]+\frac{\delta m_N}{2}\tau_3-\left(\sigma_s-\dot{\sigma}_sq^2\right)\right)\\
J^{(2)}_{G,\alpha}(\vec{q}_1,\vec{q}_2)&=-\frac{8\pi}{9}C_{G\alpha}\frac{g_A^2m_{\pi}^2}{4f_{\pi}^2}\frac{\vec{\sigma}_1\cdot\vec{q}_1\;\vec{\sigma}_2\cdot\vec{q}_2}{\left(q_1^2+m_{\pi}^2\right)\left(q_2^2+m_{\pi}^2\right)}\;\vec{\tau}_1\cdot\vec{\tau}_2
\end{alignat}
\end{subequations}
The isoscalar and isovector combinations of scalar Wilson coefficients are given by
\begin{equation}
\begin{split}
C^{(0)}_{S\alpha}&=\frac{1-\epsilon}{2}C_{S\alpha}^{(u)}+\frac{1+\epsilon}{2}C_{S\alpha}^{(d)}\\
C^{(1)}_{S\alpha}&=\left(1-\frac{1}{\epsilon}\right)C^{(u)}_{S\alpha}+\left(1+\frac{1}{\epsilon}\right)C_{S\alpha}^{(d)}
\end{split}
\end{equation}
\begin{equation}
\begin{split}
\sigma_{\pi N}&=\frac{1}{2}\braket{N|(m_u+m_d)(\bar{u}u+\bar{d}d)|N}\\
\sigma_s&=\braket{N|m_s\bar{s}s|N}\\
\epsilon&=\frac{m_d-m_u}{m_d+m_u}\\
\delta m_N&=\left(m_n-m_p\right)_\mathrm{strong}
\end{split}
\end{equation}
The single-nucleon scalar form factor $F(x)$, corresponding to the one-loop diagram in Figure \ref{fig:mm_decay_diagrams}, is given by
\begin{equation}
\begin{split}
F(x)&=\frac{2+x}{\sqrt{x}}\mathrm{arccot}\left(\frac{2}{\sqrt{x}}\right)-1\\
&\approx \frac{5}{12}x-\frac{7}{240}x^2+\frac{9}{2240}x^3+...
\end{split}
\end{equation}
Although the typical momentum transfer\footnote{As we will retain the full Dirac solution for the leptons, the three-momentum transfer is not a well-defined quantity, see Section \ref{sec:q_defined}.} is not small compared to the pion mass $(q/m_{\pi})^2\approx (m_{\mu}/m_{\pi})^2\approx 0.6$, the form factor dependence on $x$ is quite mild (higher order terms in the Taylor series are naturally suppressed). The first order term reproduces the full expression to $2\%$ accuracy for $x\leq 1$ and to $5\%$ accuracy for $1<x\leq 2$. Therefore we are justified in retaining only the linear term.

\begin{equation}
\begin{split}
\mathcal{A}^{(1)}&=-\frac{G_Fm_{\mu}}{\Lambda^2}\sum_{\alpha=L,R}\bar{N}'J^{(1)}_{\alpha}N\braket{\bar{e}P_{\alpha}\mu}\\
\mathcal{A}^{(2)}&=-\frac{G_Fm_{\mu}}{\Lambda^2}\sum_{\alpha=L,R}\bar{N}_1'\bar{N}_2'J_{\alpha}^{(2)}N_1N_2\braket{\bar{e}P_{\alpha}\mu}
\end{split}
\end{equation}
The nucleon currents can be separated in to isoscalar  and isovector components as
\begin{equation}
\begin{split}
J_{\alpha}^{(T=0)}(q^2)&=\left[\sigma_{\pi N}-\frac{3m_{\pi}^3g_A^2}{64\pi f_{\pi}^2}\left(F\left(q^2/m_{\pi}^2\right)+\frac{k_F}{m_{\pi}}f^{SI}_\mathrm{eff}\right)\right]\left(C^{(0)}_{S\alpha}+\frac{8\pi}{9}C_{G\alpha}\right)\\
&+\left(\sigma_s-\dot{\sigma}_sq^2\right)\left(C^{(s)}_{S\alpha}+\frac{8\pi}{9}C_{G\alpha}\right)-\frac{8\pi}{9}C_{G\alpha}m_N\\
J_{\alpha}^{(T=1)}(q^2)&=-\frac{\delta m_N}{4}\left(C_{S\alpha}^{(1)}+\frac{16\pi}{9}C_{G\alpha}\right)
\end{split}
\end{equation}
\begin{table}
\centering
\begin{tabular}{lcc}
\hline
\hline
Quantity & Accepted Value & Reference\\
\hline
$\sigma_{\pi N}$ & $59.1\pm 3.5$ MeV & \cite{hoferichter-2015} \\
$\sigma_s$ & $41\pm 9$ MeV & \cite{aoki2021flag} \\
$\dot{\sigma}_s$ & $0.3\pm 0.2$ GeV & \cite{hoferichter-2012}\\
$\epsilon$ & $0.365\pm 0.23$ & \cite{aoki2021flag} \\
$\delta m_N$ & $2.32\pm 0.17$ MeV & \cite{brantley2016strong} \\
$f_{\pi}$ & $92.07\pm 0.99$ MeV & \cite{aoki2021flag} \\
\hline
\hline
\end{tabular}
\caption{Physical parameters, their uncertainties and references.}
\label{tab:hadronic_params}
\end{table}
The two-nucleon currents $J^{(2)}_{ud,\alpha}$ and $J^{(2)}_{G\alpha}$ present a significant theoretical challenge. First, the nuclear effective theory that we have developed was based entirely on single-nucleon currents. According to the chiral theory, however, there exist two-nucleon currents which, based on naive dimensional analysis, can contribute at the 10\% level. Do these two-nucleon currents spoil our conclusion that $\mu\rightarrow e$ conversion is governed by six response functions (plus two interference terms)? 

The two-nucleon contribution is also important for ``top-down'' approaches where, for a prescribed BSM model of CLFV, one aims to compute the $\mu\rightarrow e$ branching ratio with quantified uncertainties. Clearly, in order to provide a clean prediction of e.g. $B(\mu\rightarrow e)/B(\mu\rightarrow e\gamma)$, we must be able to compute the strength of the two-nucleon term and understand the associated uncertainty. The difficultly with $J^{(2)}_{ud,\alpha}$ and $J_{G,\alpha}^{(2)}$ arises because -- unlike the single-nucleon currents $J^{(1)}_{ud,\alpha}$, $J^{(1)}_{s,\alpha}$ and $J^{(1)}_{G,\alpha}$ -- total nuclear matrix elements of the two-nucleon currents cannot be obtained with only the scalar nucleon density $\rho_N(r)$ but require knowledge of the two-nucleon correlation function. 

The two-nucleon correlation function can be obtained, for example, from a nuclear shell model calculation, as we describe in detail in \ref{sec:two_nucleon}. Alternatively, one may replace the two-nucleon operator by an effective one-body operator by averaging over a degenerate Fermi gas model of the target nucleus. More precisely, an effective one-body operator is obtained by performing a mean-field-like sum over direct and exchange terms
\begin{equation}
\braket{\alpha|\mathcal{O}^{(1)}|\beta}\equiv \sum_{\gamma}\braket{\alpha\gamma|\mathcal{O}^{(2)}|\beta\gamma}-\braket{\alpha\gamma|\mathcal{O}^{(2)}|\gamma\beta},
\end{equation}
where $\gamma$ sums over occupied core states. In the degenerate Fermi gas model, each core state is a direct product of space, spin, and isospin components
\begin{equation}
\ket{\alpha}=\ket{\mathbf{p}(\alpha)}\otimes \ket{\frac{1}{2}m_s(\alpha)}\otimes \ket{\frac{1}{2}m_t(\alpha)}.
\end{equation}
States of momentum $p$ are occupied up to the nuclear Fermi momentum $k_F$. The details of the averaging procedure when $\mathcal{O}^{(2)}\sim J^{(2)}_{ud,\alpha}$ are relegated to Appendix \ref{app:fga}. The net effect of performing the Fermi gas average is a shift in the single-nucleon coupling parameter
\begin{equation}
\sigma_{\pi N}\rightarrow \sigma_{\pi N}-\frac{3 g_A^2m_{\pi}^2k_F}{64\pi f_{\pi}^2}f^{SI}_\mathrm{eff},
\end{equation}
where $f_\mathrm{eff}^{SI}$ is a constant that encodes the strength of the effective one-body operator. 

Table \ref{tab:two_body} reports the value of $f_\mathrm{eff}^{SI}$ for several nuclei of interest obtained by the Fermi gas average and from a shell model evaluation of the two-nucleon charge operator. This overestimation has been observed previously in two-nucleon operators which give rise to nuclear anapole moments \cite{2002PhRvC..65d5502H}. In that study, which considered a variety of two-body nuclear currents -- none of them the operator of present concern -- and focused on the relatively heavy nuclei $^{133}$Cs and $^{205}$Tl, it was found that the Fermi gas average result was typically 2-3 times larger than the corresponding shell model result. We observe this same overestimation in the case at hand; the ratio of the FGA to NSM result ranges from 2.0 for $^{12}$C to 2.83 for $^{63}$Cu. It is worth noting that the value of the effective one-body coupling implied by the NSM calculation $f^{SI}_\mathrm{eff,NSM}$ is nearly constant across the range of nuclei considered in Table \ref{tab:two_body}. The NSM calculation itself is not completely rigorous; the two-nucleon correlation function is obtained in a soft Hilbert space which lacks the high-momentum modes required to resolve the very strong repulsion of two nucleons at short range. For a detailed discussion of this issue see Chapter \ref{chap:two_body}.

The merits of the Fermi gas average are two-fold. First, it yields an estimate of the strength of the two-nucleon operator without the need for a computationally intensive nuclear shell model calculation. Second, it provides a justification for basing our nuclear effective theory of $\mu\rightarrow e$ conversion on single-nucleon currents: any two-nucleon currents that might be relevant can be averaged to an effective one-body operator. Therefore we can interpret the single-nucleon $c_i$s of our theory as already containing information about two-nucleon currents as effective one-body operators. The general form of our effective theory -- along with the conclusion that six response functions are probed in elastic $\mu\rightarrow e$ conversion -- is unchanged as long as the effect of two-nucleon currents can be reliably captured by effective single-nucleon operators. At such time that $\mu\rightarrow e$ conversion is observed experimentally and detailed interpretation of the $c_i$s is needed, one may disentangle the various one- and two-nucleon contributions through detailed matching to HBChPT.


\begin{table}
\centering
\begin{tabular}{lccccccc}
\hline
\hline
Isotope & $B_{\mu}$ (MeV) & $Z_\mathrm{eff}$ & $k_F$ (MeV) & $E_e$ (MeV) & $f_\mathrm{eff,FGA}^{SI}$ & $f_\mathrm{eff,NSM}^{SI}$ & $f_\mathrm{eff,FGA}^{SD}$\\
\hline
$^{12}_6$C & 0.101 & 5.766 & \textbf{221.0} & 105.067 & $0.36^{+0.03}_{-0.18}$ & 0.18 & $0.38^{+0.04}_{-0.19}$\\
$^{16}_8$O & 0.179 & 7.539 & 225.7 & 105.111 & $0.38^{+0.03}_{-0.19}$ & 0.18 & $0.40^{+0.04}_{-0.20}$\\
$^{19}_9$F & 0.225 & 8.379 & 229.2 & 105.123 & $0.40^{+0.03}_{-0.20}$ & $(0.18,0.18,0.18)$ & $0.41^{+0.04}_{-0.21}$\\
$^{23}_{11}$Na & 0.335 & 10.060 & 233.8 & 105.065 & $0.41^{+0.03}_{-0.21}$ & $(0.18,0.18,0.18)$ & $0.42^{+0.04}_{0.21}$ \\
$^{27}_{13}$Al & 0.465 & 11.645 & 238.0 & 104.976 & $0.43^{+0.03}_{-0.22}$ & $(0.18, 0.18, 0.18)$ & $0.43^{+0.03}_{-0.22}$\\
$^{28}_{14}$Si & 0.537 & 12.408 & 239.0 & 104.912 & $0.43^{+0.03}_{-0.22}$ & $(0.18, 0.18, 0.18)$ & $0.44^{+0.03}_{-0.22}$ \\
$^{32}_{16}$S & 0.695 & 13.833 & 243.0 & 104.780 & $0.45^{+0.03}_{-0.23}$ & $(0.18,0.18,0.18)$ & $0.45^{+0.03}_{-0.23}$ \\
$^{40}_{20}$Ca & 1.063 & 16.487 & \textbf{251.0} & 104.450 & $0.48^{+0.03}_{-0.24}$ & 0.19 & $0.47^{+0.03}_{-0.24}$\\
$^{48}_{22}$Ti & 1.268 & 17.613 & 255.0 & 104.269 & $0.49^{+0.03}_{-0.25}$ & $(0.18,0.18,0.18)$ & $0.48^{+0.03}_{-0.24}$\\
$^{56}_{26}$Fe & 1.727 & 19.887 & 259.0 & 103.828 & $0.51^{+0.03}_{-0.26}$ & $(0.18,0.18, )$ & $0.48^{+0.03}_{-0.24}$ \\
$^{63}_{29}$Cu & 2.097 & 21.322 & 259.0 & 103.471 & $0.51^{+0.03}_{-0.26}$ & $(0.18, 0.18, 0.18)$ & $0.48^{+0.03}_{-0.24}$\\
\hline\hline
\end{tabular}
\caption{Input parameters and outputs for the one-body-averaging of the two-nucleon coherent operator. $k_F$ is the Fermi momentum, obtained by linear interpolation between the values measured in \cite{PhysRevLett.26.445}, $f^{SI}_\mathrm{eff,FGA}$ is the value of the effective spin-independent coupling obtained from the Fermi gas average, $f^{SI}_\mathrm{eff,NSM}$ is the value implied by the nuclear shell model evaluation of the two-nucleon operator. Shell model wave functions for $^{12}$C, $^{16}$O, and $^{40}$Ca are taken to be an inert core. The nuclei $^{19}$F, $^{23}$Na, $^{27}$Al, and $^{28}$Si are modeled in the $1d_{5/2}-2s_{1/2}-1d_{3/2}$ valence space above an inert $^{16}$O core. The nuclei $^{48}$Ti and $^{56}$Fe are modeled in the $1f_{7/2}-2p_{3/2}-2p_{1/2}-1f_{5/2}$ valence space above an inert $^{40}$Ca core. Finally $^{63}$Cu is modeled in the $2p_{3/2}-2p_{1/2}-1f_{5/2}-1g_{9/2}$ valence space above an inert $^{56}$Ni core. The interactions employed in the $sd$ valence space are (BW \cite{bw}, USDA \cite{PhysRevC.74.034315}, USDB \cite{PhysRevC.74.034315}), in the $fp$ valence space (GXPF1 \cite{gxpf1}, KB3G \cite{kb3g}, $\mathrm{KB}'$ \cite{kbp}), and in the $pfg$ valence space (JUN45 \cite{jun45}, GCN2850 \cite{gcn2850}, jj44b \cite{jj44b}).}
\label{tab:two_body}
\end{table}
Writing the standard muon capture rate $\Gamma_\mathrm{capt}=\kappa_\mathrm{capt}m_{\mu}^5/v^4$, the branching ratio for scalar-mediated $\mu\rightarrow e$ conversion can be expressed simply as 
\begin{equation}
B(\mu\rightarrow e)=\left(\frac{v}{\Lambda}\right)^4\left(\left|\tau^{(+1)}\right|^2+\left|\tau^{(-1)}\right|^2\right),
\end{equation}
where $\tau^{(\pm 1)}$ are the dimensionless overlap integrals
\begin{equation}
\begin{split}
\tau^{(-1)}&=\left(C_{DL}+C_{DR}\right)\tau^{(-1)}_D+\tau_S^{(-1)},\\
\tau^{(+1)}&=\left(C_{DL}-C_{DR}\right)\tau_D^{(+1)}-\tau_D^{(+1)}
\end{split}
\end{equation}
The overlap integrals for the dipole operator can be expressed in terms of the electric field produced by the nuclear charge
\begin{equation}
\begin{split}
\tau_D^{(-1)}&=-\frac{1}{m_{\mu}^{3/2}}\int dr\;r^2\;E(r)\;\left(g_{-1}^{(e)}f_{-1}^{(\mu)}+f_{-1}^{(e)}g_{-1}^{(\mu)}\right)\\
\tau_D^{(-1)}&=-\frac{i}{m_{\mu}^{3/2}}\int dr\;r^2\;E(r)\;\left(g_{+1}^{(e)}g_{-1}^{(\mu)}-f_{+1}^{(e)}f_{-1}^{(\mu)}\right)
\end{split}
\end{equation}
\begin{equation}
\begin{split}
\tau_S^{(-1)}&=\frac{1}{2}G_Fm_{\mu}^2\sum_{N=p,n}\left[\left(C^{\rho}_{NL}+C^{\rho}_{NR}\right)\tau^{(-1)}_{\rho_N}+\left(C^f_{NL}+C^f_{NR}\right)\tau^{(-1)}_{f_N}\right],\\
\tau_S^{(+1)}&=\frac{1}{2}G_Fm_{\mu}^2\sum_{N=p,n}\left[\left(C_{NL}^{\rho}-C_{NR}^{\rho}\right)\tau_{\rho_N}^{(+1)}+\left(C_{NL}^f-C_{NR}^f\right)\tau_{f_N}^{(+1)}\right],
\end{split}
\end{equation}
\begin{equation}
\begin{split}
m_{\mu}C_{N\alpha}^{\rho}&=\left(\sigma_{\pi N}-\frac{3g_A^2m_{\pi}^2}{64\pi f_{\pi}^2}k_Ff^\mathrm{SI}_\mathrm{eff}\right)\left(C_{S\alpha}^{(0)}+\frac{8\pi}{9}C_{G\alpha}\right)\mp\frac{\delta m_N}{4}\left(C_{S\alpha}^{(1)}+\frac{16\pi}{9}C_{G\alpha}\right)\\
&+\sigma_s\left(C_{S\alpha}^{(s)}+\frac{8\pi}{9}C_{G\alpha}\right)-\frac{8\pi}{9}C_{G\alpha}m_N,\\
m_{\mu}C_{N\alpha}^f&=-\frac{3g_A^2m_{\pi}^3}{64\pi f_{\pi}^2}\frac{5}{12}\left(C_{S\alpha}^{(0)}+\frac{8\pi}{9}C_{G\alpha}\right)-\dot{\sigma}_sm_{\pi}^2\left(C_{S\alpha}^{(s)}+\frac{8\pi}{9}C_{G\alpha}\right).
\end{split}
\end{equation}
The overlap integrals are
\begin{equation}
\begin{split}
\tau^{(-1)}_{\rho_N(f_N)}&=\frac{1}{m_{\mu}^{5/2}}\int dr\;r^2\left(g^{(e)}_{-1}g^{(\mu)}_{-1}-f_{-1}^{(e)}f_{-1}^{(\mu)}\right)\rho_N(f_N)\\
\tau^{(+1)}_{\rho_Nf_N}&=\frac{i}{m_{\mu}^{5/2}}\int dr\;r^2\left(f_{+1}^{(e)}g_{-1}^{(\mu)}+g_{+1}^{(e)}f_{-1}^{(\mu)}\right)\rho_N(f_N),
\end{split}
\end{equation}
where
\begin{equation}
f_N(r)=-\frac{1}{m_{\pi}^2}\left(\frac{\partial^2}{\partial r^2}+\frac{2}{r}\frac{\partial}{\partial r}\right)\rho_N(r)
\end{equation}

Note that the Fermi density parameters $\beta$ and $c$ employed in this work differ from those in \cite{Cirigliano:2022ekw}, which introduces a small discrepancy in the resulting values of the overlap integrals.
\section{Higgs-Mediated CLFV}
\label{sec:CLFV_higgs}
As an explicit example of the top-down formalism that we have developed, let us consider the Higgs-mediated CLFV model with effective Lagrangian
\begin{equation}
\mathcal{L}=-Y_{e\mu}\bar{e}P_R\mu h-Y_{\mu e}\bar{\mu}P_Reh+\mathrm{h.c.},
\label{eq:L_higgs}
\end{equation}
specified by two unknown Yukawa-like coupling parameters $Y_{e\mu}$, $Y_{\mu e}$. The tree-level Higgs exchange shown in Figure \ref{fig:CLFV_higgs_diagrams} (a) induces the scalar quark couplings
\begin{equation}
\begin{split}
\frac{1}{\Lambda^2}G_Fm_{\mu}vC^{(q)}_{SR}&=-\frac{1}{m_h^2}Y_{e\mu},\\
\frac{1}{\Lambda^2}G_Fm_{\mu}vC_{SL}^{(q)}&=-\frac{1}{m_h^2}Y_{\mu e},
\end{split}
\end{equation}
where $m_h=125$ GeV is the Higgs boson mass. Thus, in the Higgs-mediated model, the scalar quark Wilson coefficients are independent of quark flavor. 

The Yukawa-like CLFV interaction generates a coupling of the leptons to on-shell photons at one- and two-loop level. The one-loop diagrams are shown in Figure \ref{fig:CLFV_higgs_diagrams}. 
\begin{figure}
\centering
\subfloat[]{
\begin{fmffile}{clfv_higgs_tree_level}
\begin{fmfgraph*}(120,120)
\fmfstraight
\fmftop{t1,t2,t3} \fmfbottom{b1,b2,b3}
\fmf{fermion}{t1,t2}
\fmf{fermion}{t2,t3}
\fmf{fermion}{b1,b2}
\fmf{fermion}{b2,b3}
\fmf{dashes,label=$h$}{t2,b2}
\fmflabel{$q$}{b1}
\fmflabel{$q$}{b3}
\fmflabel{$\mu$}{t1}
\fmflabel{$e$}{t3}
\fmfdot{t2}
\end{fmfgraph*}
\end{fmffile}
}
\hfill
\subfloat[]{
\begin{fmffile}{higgs_clfv_one_loop_a}
\begin{fmfgraph*}(120,120)
\fmfleft{i1} \fmfright{o1} \fmfbottom{b1,b2,b3,b4,b5}
\fmf{fermion,tension=100}{i1,v1}
\fmf{fermion,label=$\mu$,tension=100,label.side=left}{v1,v2}
\fmf{fermion,label=$\mu$,tension=100,label.side=left}{v2,v3}
\fmf{fermion,tension=100}{v3,o1}
\fmf{dashes,left=1.0,label=$h$}{v1,v3}
\fmf{photon,label=$\gamma$,tension=1.4}{v2,b3}
\fmflabel{$\mu$}{i1}
\fmflabel{$e$}{o1}
\fmfdot{v3}
\end{fmfgraph*}
\end{fmffile}
}
\hfill
\subfloat[]{
\begin{fmffile}{higgs_clfv_one_loop_b}
\begin{fmfgraph*}(120,120)
\fmfleft{i1} \fmfright{o1} \fmfbottom{b1,b2,b3,b4,b5}
\fmf{fermion,tension=100}{i1,v1}
\fmf{fermion,label=$e$,tension=100,label.side=left}{v1,v2}
\fmf{fermion,label=$e$,tension=100,label.side=left}{v2,v3}
\fmf{fermion,tension=100}{v3,o1}
\fmf{dashes,left=1.0,label=$h$}{v1,v3}
\fmf{photon,label=$\gamma$,tension=1.4}{v2,b3}
\fmflabel{$\mu$}{i1}
\fmflabel{$e$}{o1}
\fmfdot{v1}
\end{fmfgraph*}
\end{fmffile}
}
\caption{CLFV Higgs Diagrams. (a) Tree-level Higgs exchange mediating $\mu\rightarrow e$ conversion (b) One-loop diagram inducing $\mu\rightarrow e\gamma$ with muon Yukawa coupling. (c) One-loop diagram inducing $\mu\rightarrow e\gamma$ with electron Yukawa coupling. The dotted vertex is the CLFV Higgs coupling.}
\label{fig:CLFV_higgs_diagrams}
\end{figure}
Given the hierarchy of the lepton Yukawas, $Y_{\mu\mu}\gg Y_{ee}$, diagram (b) dominates, and the one-loop contributions to the CLFV dipole couplings are given approximately by \cite{Harnik:2012pb}
\begin{equation}
\begin{split}
\frac{1}{\Lambda^2}C_{DR}^{1\mathrm{loop}}&\approx-\frac{e}{32\pi^2}\frac{1}{m_h^2}Y_{\mu\mu}\left[\log\frac{m_h^2}{m_{\mu}^2}-\frac{4}{3}\right]Y_{e\mu}\approx -5.27\times 10^{-6}\frac{1}{m_h^2}Y_{e\mu},\\
\frac{1}{\Lambda^2}C_{DL}^{1\mathrm{loop}}&\approx-\frac{e}{32\pi^2}\frac{1}{m_h^2}Y_{\mu\mu}\left[\log\frac{m_h^2}{m_{\mu}^2}-\frac{4}{3}\right]Y_{\mu e}\approx -5.27\times 10^{-6}\frac{1}{m_h^2}Y_{\mu e}.
\end{split}
\end{equation}
The leading one-loop diagram is suppressed by the muon's Yukawa coupling to the Higgs $Y_{\mu\mu}$. At two-loop order, this suppression can be avoided by the Bar-Zee diagrams shown in Figure \ref{fig:CLFV_higgs_diagrams_2_loops}, whose flavor-conserving analogs contribute to fermion electric dipole moments \cite{PhysRevLett.65.21,Abe:2013qla}. We do not reproduce the full result for the two-loop amplitudes here -- complete gauge-invariant expressions can be found in \cite{Cirigliano:2021img}. Numerically, the two-loop contributions to the dipole Wilson coefficients are roughly three orders of magnitude larger than the one-loop contributions
\begin{equation}
\begin{split}
\frac{1}{\Lambda^2}C_{DR}^{2\mathrm{loop}}&\approx -4.67\times 10^{-3}\frac{1}{m_h^2}Y_{e\mu},\\
\frac{1}{\Lambda^2}C_{DL}^{2\mathrm{loop}}&\approx -4.67\times 10^{-3}\frac{1}{m_h^2}Y_{\mu e},
\end{split}
\end{equation}
 and therefore dominate the response. The branching ratio can thus be written explicitly in terms of the unknown CLFV Yukawas as
\begin{equation}
B\left(\mu\rightarrow e\gamma\right)\approx 0.312\left(|Y_{e\mu}|^2+|Y_{\mu e}|^2\right),
\end{equation}
from which one can immediately exclude regions of parameter space using the existing limit from MEG, $B(\mu\rightarrow e\gamma)<4.2\times 10^{-13}$, and the expected limit from MEG II, $B(\mu\rightarrow e\gamma)<6\times 10^{-14}$. The resulting exclusion plot is shown in Figure \ref{fig:higgs_clfv_exclusion}.

As usual, the induced dipole operator can mediate $\mu\rightarrow e$ conversion through exchange of a virtual photon with the nuclear charge. In fact, the contribution to the conversion rate in $^{27}$Al from virtual photon exchange is roughly 4 times larger than the contribution from tree-level Higgs exchange -- the quark Yukawa in the latter diagram introduces a suppression factor $\approx m_N/v$ even for the heaviest quarks. 
\begin{figure}
\centering
\subfloat[]{
\begin{fmffile}{clfv_higgs_two_loop_a}
\begin{fmfgraph*}(120,120)
\fmfstraight
\fmftop{t1,t2,t3,t4,t5,t6}
\fmfbottom{b1,b2,b3,b4,b5}
\fmf{fermion,tension=100}{t1,t2}
\fmf{plain,tension=100}{t2,t3}
\fmf{fermion,label=$e$,tension=100,label.side=left}{t3,t4}
\fmf{plain,tension=100}{t4,t5}
\fmf{fermion,tension=100}{t5,t6}
\fmfpoly{phantom}{v2,v1,v3}
\fmf{fermion,left=0.7,tension=0.1}{v1,v2}
\fmf{fermion,left=0.5,tension=0.1,label=$t$}{v2,v3}
\fmf{fermion,left=0.5,tension=0.1,label=$t$}{v3,v1}
\fmf{photon,tension=4.0,label=$\gamma$}{v3,b3}
\fmf{dashes,tension=2.0,label=$h$,label.side=left}{v1,t2}
\fmf{photon,tension=2.0,label=$\gamma,, Z$}{v2,t5}
\fmflabel{$\mu$}{t1}
\fmflabel{$e$}{t6}
\fmfdot{t2}
\end{fmfgraph*}
\end{fmffile}
}
\hspace{2cm}
\subfloat[]{
\begin{fmffile}{clfv_higgs_two_loop_b}
\begin{fmfgraph*}(120,120)
\fmfstraight
\fmftop{t1,t2,t3,t4,t5,t6}
\fmfbottom{b1,b2,b3,b4,b5}
\fmf{fermion,tension=100}{t1,t2}
\fmf{plain,tension=100}{t2,t3}
\fmf{fermion,label=$e$,tension=100,label.side=left}{t3,t4}
\fmf{plain,tension=100}{t4,t5}
\fmf{fermion,tension=100}{t5,t6}
\fmfpoly{phantom}{v2,v1,v3}
\fmf{photon,left=0.7,tension=0.1}{v1,v2}
\fmf{photon,left=0.5,tension=0.1,label=$W$}{v2,v3}
\fmf{photon,left=0.5,tension=0.1,label=$W$}{v3,v1}
\fmf{photon,tension=4.0,label=$\gamma$}{v3,b3}
\fmf{dashes,tension=2.0,label=$h$,label.side=left}{v1,t2}
\fmf{photon,tension=2.0,label=$\gamma,, Z$}{v2,t5}
\fmflabel{$\mu$}{t1}
\fmflabel{$e$}{t6}
\fmfdot{t2}
\end{fmfgraph*}
\end{fmffile}
}\\[10mm]
\subfloat[]{
\begin{fmffile}{clfv_higgs_two_loop_c}
\begin{fmfgraph*}(120,120)
\fmfstraight
\fmftop{t1,t2,t3,t4,t5,t6}
\fmfbottom{b1,b2,b3,b4,b5}
\fmf{fermion,tension=100}{t1,t2}
\fmf{plain,tension=100}{t2,t3}
\fmf{fermion,label=$e$,tension=100,label.side=left}{t3,t4}
\fmf{plain,tension=100}{t4,t5}
\fmf{fermion,tension=100}{t5,t6}
\fmfpoly{phantom}{v2,v1,v3}
\fmf{photon,left=0.7,tension=0.1,label=$W$}{v1,v2}
\fmf{photon,left=0.5,tension=0.1}{v2,v3}
\fmf{photon,left=0.5,tension=0.1,label=$W$}{v3,v1}
\fmf{phantom,tension=4.0}{v3,b3}
\fmf{photon,label=$\gamma$}{b5,v2}
\fmf{dashes,tension=2.0,label=$h$,label.side=left}{v1,t2}
\fmf{photon,tension=2.0,label=$\gamma,, Z$,label.side=right}{v2,t5}
\fmflabel{$\mu$}{t1}
\fmflabel{$e$}{t6}
\fmfdot{t2}
\end{fmfgraph*}
\end{fmffile}
}
\hspace{2cm}
\subfloat[]{
\begin{fmffile}{clfv_higgs_two_loop_d}
\begin{fmfgraph*}(120,120)
\fmfstraight
\fmfleft{i1}
\fmfright{o1}
\fmfbottom{b1,b2,b3,b4,b5}
\fmftop{t1,t2,t3}
\fmf{fermion,tension=100}{i1,v1}
\fmf{fermion,tension=100,label=$e$,label.side=left}{v1,v2}
\fmf{fermion,tension=100,label=$e$,label.side=left}{v2,v3}
\fmf{fermion,tension=100,label=$e$,label.side=left}{v3,v4}
\fmf{fermion,tension=100}{v4,o1}
\fmf{photon,left=1/2,tension=1/3}{v2,v5}
\fmf{photon,left=1/2,tension=1/3,label=$Z$}{v5,v6}
\fmf{photon,left=1/2,tension=1/3}{v6,v4}
\fmf{dashes,left,tension=0.1,label=$h$}{v1,v5}
\fmf{photon,label=$\gamma$}{v3,b5}
\fmf{phantom,tension=0.45}{v5,t2}
\fmf{phantom,tension=0.2}{v6,t3}
\fmflabel{$\mu$}{i1}
\fmflabel{$e$}{o1}
\fmfdot{v1}
\end{fmfgraph*}
\end{fmffile}
}
\caption{Two-loop diagrams comprising the dominant contribution to $\mu\rightarrow e\gamma$ in Higgs-mediated CLFV.}
\label{fig:CLFV_higgs_diagrams_2_loops}
\end{figure}


We now demonstrate the primary advantage of the top-down approach by computing the branching ratio for $\mu\rightarrow e$ conversion with a quantified uncertainty 
\begin{equation}
\frac{B\left(\mu+\mathrm{Al}\rightarrow e+\mathrm{Al}\right)}{B\left(\mu\rightarrow e\gamma\right)}=\left(8.7\pm 0.3\right)\times 10^{-3},
\label{eq:BR_ratio_mu_e_gamma}
\end{equation}
where we have divided by the branching ratio for the on-shell photon process to remove any dependence on the unknown CLFV parameters. Therefore, if these quantities were to be measured at Mu2e/COMET and MEG-II, respectively, and their ratio fell significantly outside of the $1\sigma$ uncertainty window, then it would be good evidence to disfavor the Higgs-mediated CLFV model specified by Eq. \ref{eq:L_higgs}. The same is true when comparing the conversion rate in different nuclei. For instance, our top-down formalism implies that the following ratio must hold in the Higgs-mediated CLFV model:
\begin{equation}
\frac{B\left(\mu+\mathrm{Ti}\rightarrow e+\mathrm{Ti}\right)}{B\left(\mu+\mathrm{Al}\rightarrow e+\mathrm{Al}\right)}=1.5\pm 0.1
\label{eq:BR_ratio_Ti_Al}
\end{equation}
\begin{figure}
\centering
\includegraphics[scale=1.2]{higgs_clfv_exclusion.png}
\caption{Exclusion curves for the Higgs-mediated CLFV model considered in Eq. \ref{eq:L_higgs}. The dashed (solid) black curve shows the (expected) limit for on-shell $\mu\rightarrow e\gamma$ conversion obtained from MEG (MEG II). The branching ratio limits are $B(\mu\rightarrow e\gamma)<4.2\times 10^{-13}$ for the MEG experiment and $6\times 10^{-14}$ for MEG II. The red (blue) curve corresponds to a $\mu\rightarrow e$ branching ratio limit $B(\mu\rightarrow e)<10^{-17}$ $(7\times 10^{-15})$ in $^{27}$Al.}
\label{fig:higgs_clfv_exclusion}
\end{figure}
The primary source of error in the ratio in Eq. \ref{eq:BR_ratio_mu_e_gamma} is the uncertainty in the hadronic parameters whereas in Eq. \ref{eq:BR_ratio_Ti_Al} the primary source of error is the uncertainty in the neutron overlap integrals. When computing the ratio of $\mu\rightarrow e$ conversion rates in two different targets, the hadronic uncertainties are common to both nuclei and therefore have a negligible impact on the overall error.
\section{Matching to Nuclear Effective Theory}
The one-loop pion form factor $F(q^2/m_{\pi}^2)$ and the momentum-dependent correction to the strange quark sigma term $\dot{\sigma}_sq^2$ will be evaluated at the effective momentum transfer $q_\mathrm{eff}$.

The low-energy coefficient $c_1$ corresponds to the scalar leptonic charge and therefore depends on the \textit{sum} of left- and right-handed Wilson coefficients
\begin{equation}
\begin{split}
&\tilde{c}_1^0=\frac{1}{2\sqrt{2}}\frac{m_{\mu}}{\Lambda^2}\bigg\{\left[\sigma_{\pi N}-\frac{3g_A^2m_{\pi}^2}{64\pi f_{\pi}^2}\Big(k_Ff_\mathrm{eff}^{SI}+m_{\pi}F\left(q_\mathrm{eff}^2/m_{\pi}^2\right)\Big)\right]\left[C_{SR}^{(0)}+C_{SL}^{(0)}+\frac{8\pi}{9}(C_{GR}+C_{GL})\right]\\
&+\left(\sigma_s-\dot{\sigma}_sq_\mathrm{eff}^2\right)\left[C_{SR}^{(s)}+C_{SL}^{(s)}+\frac{8\pi}{9}\left(C_{GR}+C_{GL}\right)\right]+\frac{8\pi}{9}m_N\left(C_{GR}+C_{GL}\right)\bigg\}+\sqrt{\alpha\pi}v^2\left(C_{DR}+C_{DL}\right),
\end{split}
\end{equation}
whereas $c_{11}$ corresponds to the lepton axial charge and depends on the \textit{difference} of left- and right-handed Wilson coefficients
\begin{equation}
\begin{split}
&\tilde{c}_{11}^0=\frac{1}{2\sqrt{2}}\frac{m_{\mu}}{\Lambda^2}\bigg\{\left[\sigma_{\pi N}-\frac{3g_A^2m_{\pi}^2}{64\pi f_{\pi}^2}\Big(k_Ff_\mathrm{eff}^{SI}+m_{\pi}F\left(q_\mathrm{eff}^2/m_{\pi}^2\right)\Big)\right]\left[C_{SR}^{(0)}-C_{SL}^{(0)}+\frac{8\pi}{9}(C_{GR}-C_{GL})\right]\\
&+\left(\sigma_s-\dot{\sigma}_sq_\mathrm{eff}^2\right)\left[C_{SR}^{(s)}-C_{SL}^{(s)}+\frac{8\pi}{9}\left(C_{GR}-C_{GL}\right)\right]+\frac{8\pi}{9}m_N\left(C_{GR}-C_{GL}\right)\bigg\}+\sqrt{\alpha\pi}v^2\left(C_{DR}-C_{DL}\right).
\end{split}
\end{equation}
The same is true for the isovector operators
\begin{equation}
\begin{split}
\tilde{c}_1^1&=-\frac{1}{2\sqrt{2}}\frac{m_{\mu}}{\Lambda^2}\frac{\delta m_N}{4}\left[C_{SR}^{(1)}+C_{SL}^{(1)}+\frac{16\pi}{9}\left(C_{GR}+C_{GL}\right)\right]+\sqrt{\alpha\pi}v^2\left(C_{DR}+C_{DL}\right)\\
\tilde{c}_{11}^1&=-\frac{1}{2\sqrt{2}}\frac{m_{\mu}}{\Lambda^2}\frac{\delta m_N}{4}\left[C_{SR}^{(1)}-C_{SL}^{(1)}+\frac{16\pi}{9}\left(C_{GR}-C_{GL}\right)\right]+\sqrt{\alpha\pi}v^2\left(C_{DR}-C_{DL}\right)
\end{split}
\end{equation}
\chapter{Two-Nucleon Contribution to $\mu\rightarrow e$ Conversion}
In the previous chapter, we found that at next-to-leading order in the chiral power counting, there is a two-nucleon diagram that contributes to scalar-mediated $\mu\rightarrow e$ conversion. In this section, we will first discuss in detail how the two-nucleon operator can be evaluated in the nuclear shell model. Then we study the behavior of the two-nucleon operator as a function of the three-momentum transfer and relate this matrix element to the effective one-body operator obtained through the Fermi gas average procedure.
\section{Fourier Transform of Two-Nucleon Operator}
\label{sec:ft_two}
The two-nucleon current which contributes to $\mu\rightarrow e$ conversion has the form
\begin{equation}
J^{(2)}_{ud,\alpha}(\vec{q}_1,\vec{q}_2)=-\frac{g_A^2m_{\pi}^2}{4f_{\pi}^2}\rho\left(\vec{k}_1,\pvec{k}_1',\vec{k}_2,\pvec{k}_2'\right)\;\vec{\tau}(1)\cdot\vec{\tau}(2)\;C^{(0)}_{S\alpha},
\label{eq:two_body_current}
\end{equation}
where we have defined the two-nucleon density
\begin{equation}
\rho\left(\vec{k}_1,\pvec{k}_1',\vec{k}_2,\pvec{k}_2'\right)\equiv\frac{\vec{q}_1\cdot\vec{\sigma}(1)\;\vec{q}_2\cdot\vec{\sigma}(2)}{(q_1^2+m_{\pi}^2)(q_2^2+m_{\pi}^2)}
\end{equation}
The three-momentum transfers are defined to be
\begin{equation}
\vec{q}_1\equiv \vec{p}_1-\pvec{p}_1',\;\;\;\vec{q}_2\equiv \vec{p}_2-\pvec{p}_2',
\end{equation}
and we use the notation $q_i=|\vec{q}_i|$ to denote the length of the three-vectors. Then to transform to coordinate space and enforce momentum conservation, we have
\begin{equation}
\begin{split}
&\rho(\vec{x}_1,\pvec{x}_1',\vec{x}_2,\pvec{x}_2',\vec{q})\\
&=\frac{1}{(2\pi)^{12}}\int d^3p_1\;d^3p_1'\;d^3p_2\;d^3p'_2\;\tilde{J}_2(\vec{q}_1,\vec{q}_2,\vec{q})(2\pi)^3\delta(\vec{p}_1+\vec{p}_2-\pvec{p}'_1-\pvec{p}'_2-\vec{q})e^{i\left(\pvec{p}'_1\cdot\pvec{x}_1'+\pvec{p}'_2\cdot\pvec{x}'_2-\vec{p}_1\cdot\vec{x}_1-\vec{p}_2\cdot\vec{x}_2\right)}
\end{split}
\end{equation}
We transform our integration variables to
\begin{equation}
\begin{split}
\vec{q}_1&=\vec{p}_1-\pvec{p}'_1,\;\vec{P}_1=\frac{1}{2}\left(\vec{p}_1+\pvec{p}'_1\right)\\
\vec{q}_2&=\vec{p}_2-\pvec{p}'_2,\;\vec{P}_2=\frac{1}{2}\left(\vec{p}_2+\pvec{p}'_2\right),
\end{split}
\end{equation}
so that
\begin{equation}
\begin{split}
&\rho(\vec{x}_1,\pvec{x}_1',\vec{x}_2,\pvec{x}'_2,\vec{q})\\
&=\frac{1}{(2\pi)^9}\int d^3P_1\;d^3P_2\;d^3q_1\;d^3q_2\;\rho(\vec{q}_1,\vec{q}_2,\vec{q})\delta(\vec{q}_1+\vec{q}_2-\vec{q})e^{i\left[\vec{P}_1\cdot(\pvec{x}_1'-\vec{x}_1)+\vec{P}_2\cdot(\pvec{x}_2'-\vec{x}_2)-\frac{1}{2}\vec{q}_1\cdot(\vec{x}_1+\pvec{x}_1')-\frac{1}{2}\vec{q}_2\cdot(\vec{x}_2+\pvec{x}_2')\right]}\\
&=\frac{1}{(2\pi)^3}\delta(\vec{x}_1-\pvec{x}'_1)\delta(\vec{x}_2-\pvec{x}_2')\int d^3q_1\;d^3q_2\;\rho(\vec{q}_1,\vec{q}_2,\vec{q})\delta(\vec{q}_1+\vec{q}_2-\vec{q})e^{-\frac{i}{2}\left[\vec{q}_1\cdot(\vec{x}_1+\pvec{x}_1')+\vec{q}_2\cdot(\vec{x}_2+\pvec{x}_2')\right]}\\
&=\frac{1}{(2\pi)^3}\delta(\vec{x}_1-\pvec{x}_1')\delta(\vec{x}_2-\pvec{x}_2')\int d^3 q_1\;\rho(\vec{q}_1,\vec{q}-\vec{q}_1,\vec{q})e^{-\frac{i}{2}\left[\vec{q}_1\cdot(\vec{x}_1+\pvec{x}_1')+(\vec{q}-\vec{q}_1)\cdot(\vec{x}_2+\pvec{x}_2')\right]}.
\end{split}
\end{equation}
Next we employ the Feynman parameter representation
\begin{equation}
\frac{1}{AB}=\int_0^1 d\alpha \frac{1}{\left[\alpha A+(1-\alpha)B\right]^2},
\end{equation}
with 
\begin{equation}
A=q_1^2+m_{\pi}^2,\;B=(\vec{q}-\vec{q}_1)^2+m_{\pi}^2.
\end{equation}
Then
\begin{equation}
\alpha A + (1-\alpha)B=\left[\vec{q}_1-(1-\alpha)\vec{q}\right]^2+\alpha(1-\alpha)q^2+m_{\pi}^2.
\end{equation}
Now let us shift the integration variable
\begin{equation}
\vec{q}_1\rightarrow \vec{q}_1+(1-\alpha)\vec{q},
\end{equation}
so that 
\begin{equation}
\begin{split}
\rho(\vec{x}_1,\pvec{x}_1',\vec{x}_2,\pvec{x}_2',\vec{q})&=\frac{1}{(2\pi)^3}\delta(\vec{x}_1-\pvec{x}_1')\delta(\vec{x}_2-\pvec{x}_2')e^{-\frac{i}{2}\vec{q}\cdot(\vec{x}_1+\pvec{x}_1')}\int_0^1 d\alpha \;e^{i\alpha\vec{q}\cdot\vec{r}}\\
&\times \int d^3q_1\frac{\left[\vec{q}_1+(1-\alpha)\vec{q}\;\right]\cdot\vec{\sigma}(1)\left[\alpha\vec{q}-\vec{q}_1\right]\cdot\vec{\sigma}(2)}{\left[q_1^2+\Pi^2(q,\alpha)\right]^2}e^{-i\vec{q}_1\cdot\vec{r}},
\end{split}
\end{equation}
where we have defined
\begin{equation}
\Pi^2(q,\alpha)\equiv \alpha(1-\alpha)q^2+m_{\pi}^2,
\end{equation}
and introduced the relative coordinate
\begin{equation}
\begin{split}
\vec{r}&\equiv\frac{1}{2}\left(\vec{x}_1+\pvec{x}_1'-\vec{x}_2-\pvec{x}_2'\right)\\
&=\vec{x}_1 - \vec{x}_2.
\end{split}
\end{equation}
The quantity $\Pi(q,\alpha)$ can be interpreted as an effective pion mass arising from the finite momentum-transfer. Indeed, $\Pi(q,\alpha)\rightarrow m_{\pi}$ when $q\rightarrow 0$.
Now we can perform the final momentum integral. There are three distinct integrands. First, with a trivial angular integrand:
\begin{equation}
\int d^3 q_1 \frac{e^{-i\vec{q}_1\cdot\vec{r}}}{\left[q_1^2+\Pi^2(q,\alpha)\right]^2}=\pi^2\frac{e^{-r\Pi(q,\alpha)}}{\Pi(q,\alpha)},
\end{equation}
then with one factor of $\hat{q}_1$ in the angular integrand:
\begin{equation}
\int d^3 q_1 \frac{\vec{q}_1\cdot\vec{\sigma}(i)e^{-i\vec{q}_1\cdot\vec{r}}}{\left[q_1^2+\Pi^2(q,\alpha)\right]^2}=-i\pi^2\hat{r}\cdot\vec{\sigma}(i)e^{-r\Pi(q,\alpha)},
\end{equation}
and finally with two factors of $\hat{q}_1$:
\begin{equation}
\begin{split}
\int d^3q_1\frac{\vec{q}_1\cdot\vec{\sigma}(1)\vec{q}_1\cdot\vec{\sigma}(2)e^{-i\vec{q}_1\cdot\vec{r}}}{\left[q_1^2+\Pi^2(q,\alpha)\right]^2}&=\frac{\pi^2}{3}\vec{\sigma}(1)\cdot\vec{\sigma}(2)\frac{e^{-r\Pi(q,\alpha)}}{r}\left[2-r\Pi(q,\alpha)\right]\\
&-\pi^2\sqrt{\frac{8\pi}{15}}\left[\vec{\sigma}(1)\otimes\vec{\sigma}(2)\right]_2\odot Y_2(\hat{r})\frac{e^{-r\Pi(q,\alpha)}}{r}\left[1+\Pi(q,\alpha)r\right].
\end{split}
\end{equation}
For this last integration we have employed the identity
\begin{equation}
\left(\hat{q}\cdot\vec{\sigma}(1)\right)\left(\hat{q}\cdot\vec{\sigma}(2)\right)=\frac{1}{3}\vec{\sigma}(1)\cdot\vec{\sigma}(2)+\sqrt{\frac{8\pi}{15}}Y_2(\hat{q})\odot\left[\vec{\sigma}(1)\otimes\vec{\sigma}(2)\right].
\label{eq:tensor_identity}
\end{equation}
Putting these pieces together yields
\begin{equation}
\begin{split}
&\rho(\vec{x}_1,\pvec{x}_1',\vec{x}_2,\pvec{x}_2',\vec{q})=\frac{1}{8\pi}\delta(\vec{x}_1-\pvec{x}_1')\delta(\vec{x}_2-\pvec{x}'_2)e^{-i\vec{q}\cdot\vec{x}_1}\int_0^1d\alpha\;e^{i\alpha\vec{q}\cdot\vec{r}-\Pi(q,\alpha)r}\Bigg\{-i\alpha\hat{r}\cdot\vec{\sigma}(1)\vec{q}\cdot\vec{\sigma}(2)\\
&+i(1-\alpha)\vec{q}\cdot\vec{\sigma}(1)\hat{r}\cdot\vec{\sigma}(2)-\frac{1}{3}\vec{\sigma}(1)\cdot\vec{\sigma}(2)\frac{1}{r}\left[2-r\Pi(q,\alpha)\right]+\alpha(1-\alpha)\frac{1}{\Pi(q,\alpha)}\vec{q}\cdot\vec{\sigma}(1)\vec{q}\cdot\vec{\sigma}(2)\\
&+\sqrt{\frac{8\pi}{15}}\frac{1}{r}\left[\vec{\sigma}(1)\otimes\vec{\sigma}(2)\right]_2\odot Y_2(\hat{r})\left[1+r\Pi(q,\alpha)\right]\Bigg\}
\label{eq:two_body_master}
\end{split}
\end{equation}
In the $\vec{q}\rightarrow 0$ limit, the Feynman parameter integral becomes trivial, $\Pi(q,\alpha)\rightarrow m_{\pi}$, and the two-body current becomes
\begin{equation}
\begin{split}
\rho(\vec{x}_1,\pvec{x}_1',\vec{x}_2,\pvec{x}_2',0)=\frac{1}{8\pi}\delta(\vec{x}_1-\pvec{x}_1')\delta(\vec{x}_2-\pvec{x}_2')\frac{1}{r}\bigg\{&\frac{1}{3}F_1(r/m_{\pi})\vec{\sigma}(1)\cdot\vec{\sigma}(2)\\
&+\sqrt{\frac{8\pi}{15}}F_2(r/m_{\pi})Y_2(\hat{r})\odot\left[\vec{\sigma}(1)\otimes\vec{\sigma}(2)\right]_2\bigg\},
\label{eq:two_body_q_zero}
\end{split}
\end{equation}
where the form factors are given by
\begin{equation}
F_1(x)\equiv e^{-x}(x-2),\;F_2(x)\equiv e^{-x}(x+1).
\end{equation}
This is a familiar result from studies of $0\nu\beta\beta$-decay (e.g. \cite{Pr_zeau_2003}), where the leading long-range contribution is due to two-pion exchange. In that case, however, the three-momentum transfer is small compared to the pion mass, and so it is justified to work in the limit $\vec{q}\rightarrow 0$. In coherent $\mu\rightarrow e$ conversion, $q\approx m_{\mu}$, and we must work at finite q. 

If we were to Fourier transform the two-body current in Equation \ref{eq:two_body_master} with respect to the momentum transfer $\vec{q}$
\begin{equation}
\rho(\vec{x}_1,\pvec{x}_1',\vec{x}_2,\pvec{x}_2',\vec{x})=\int \frac{d^3q}{(2\pi)^3} \;e^{i\vec{q}\cdot\vec{x}}\rho(\vec{x}_1,\pvec{x}_1',\vec{x}_2,\pvec{x}_2',\vec{q}),
\end{equation}
then the resulting position space current will exhibit a complicated dependence on the Fourier transform coordinate $\vec{x}$. The dominant piece will be the $J=0$ component of this current and so we must project the spatial two-body current onto charge multipoles
\begin{equation}
\begin{split}
\mathcal{M}_{J,M}(\kappa)&=\int d^3x \;j_J(\kappa x)Y_{J,M}(\hat{x})\rho(\vec{x}_1,\pvec{x}_1',\vec{x}_2,\pvec{x}_2',\vec{x})\\
&=\frac{1}{4\pi}i^J\delta(\kappa-q)\int d\Omega_q\; Y_{J,M}(\hat{q})\rho(\vec{x}_1,\pvec{x}_1',\vec{x}_2,\pvec{x}_2',\vec{q}).
\end{split}
\end{equation}
Our focus is then the $J=M=0$ contribution. We begin by considering the first term in the second line of Eq. \ref{eq:two_body_master}. The only angular dependence on $\hat{q}$ in this term is in the exponentials, so we compute
\begin{equation}
\int d\Omega_q\;Y_{0,0}(\hat{q})e^{-i\vec{q}\cdot\vec{R}}e^{i(\alpha-1/2)\vec{q}\cdot\vec{r}}=(4\pi)^{3/2}\sum_{l=0}^{\infty}j_l(q R)j_l\left(q r(\alpha-1/2)\right)Y_l(\hat{R})\odot Y_l(\hat{r}),
\end{equation}
where we have defined the center-of-mass coordinate
\begin{equation}
\vec{R}\equiv \frac{1}{2}\left(\vec{x}_1+\vec{x}_2\right).
\end{equation}

Thus we have the two-body operator
\begin{equation}
\begin{split}
\mathcal{O}_1(\vec{r},\vec{R},q)&=-\frac{1}{12\sqrt{\pi}}\delta(\vec{x}_1-\pvec{x}_1')\delta(\vec{x}_2-\pvec{x}_2')\vec{\sigma}(1)\cdot\vec{\sigma}(2)\sum_{L=0}^{\infty}Y_L(\hat{r})\odot\ Y_L(\hat{R})j_L(q R)\\
&\times\int_0^1d\alpha\;e^{-r\Pi(q,\alpha)}j_L\left(q r(\alpha-1/2)\right)\frac{2-r\Pi(q,\alpha)}{r}.
\end{split}
\end{equation}
The Feynman parameter integral vanishes unless $L$ is even. This can be seen by shifting the Feynman parameter $\beta=\alpha-1/2$ and noting that $\Pi(q,\beta)$ (and thus the entire Feynman parameter integrand) is an even function of $\beta$. Therefore
\begin{equation}
\begin{split}
\mathcal{O}_1(\vec{r},\vec{R},q)&=-\frac{1}{12\sqrt{\pi}}\delta(\vec{x}_1-\pvec{x}_1')\delta(\vec{x}_2-\pvec{x}_2')\vec{\sigma}(1)\cdot\vec{\sigma}(2)\sum_{L=0,2,...}^{\infty}Y_L(\hat{r})\odot\ Y_L(\hat{R})\;j_L(q R)\\
&\times\int_0^1d\alpha\;e^{-r\Pi(q,\alpha)}j_L\left(q r(\alpha-1/2)\right)\frac{2-r\Pi(q,\alpha)}{r}
\end{split}
\end{equation}
Now let's look at the second term in the second line of Eq. \ref{eq:two_body_master} and apply the identity of Eq. \ref{eq:tensor_identity}. We find one term which has the same tensor structure as $\mathcal{O}_1$; that is
\begin{equation}
\begin{split}
\mathcal{O}_2(\vec{r},\vec{R},q)&=\frac{1}{12\sqrt{\pi}}\delta(\vec{x}_1-\pvec{x}_1')\delta(\vec{x}_2-\pvec{x}_2')\vec{\sigma}(1)\cdot\vec{\sigma}(2)\sum_{L=0,2,...}^{\infty}Y_L(\hat{r})\odot Y_L(\hat{R})\;j_L(q R)\\
&\times\int_0^1 d\alpha\;e^{-r\Pi(q,\alpha)}j_L(q r\left(\alpha-1/2)\right)\frac{q^2\alpha(1-\alpha)}{\Pi(q,\alpha)},
\end{split}
\end{equation}
where again the Feynman parameter integral vanishes unless $L$ is even. 

The second term that we get from applying the tensor decomposition of Eq. \ref{eq:tensor_identity} has a quadrupole dependence on $\hat{q}$:
\begin{equation}
\begin{split}
&\int d\Omega_q\;Y_{0,0}(\hat{q})e^{-i\vec{q}\cdot\vec{R}}e^{i(\alpha-1/2)\vec{q}\cdot\vec{r}}Y_2(\hat{q})\odot\left[\vec{\sigma}(1)\otimes\vec{\sigma}(2)\right]_2\\
&=4\pi\sum_{L_1,L_2}(i)^{L_1-L_2}(-1)^{L_1+L_2}C^{20}_{L_10L_20}\sqrt{\frac{(2L_1+1)(2L_2+1)}{5}}j_{L_1}\left(q r(\alpha-1/2)\right)j_{l_2}(q R)\\
&\times\left[Y_{L_1}(\hat{r})\otimes Y_{L_2}(\hat{R})\right]_2\odot\left[\vec{\sigma}(1)\otimes\vec{\sigma}(2)\right]_2.
\end{split}
\end{equation}
Therefore we have the operator
\begin{equation}
\begin{split}
\mathcal{O}_3(\vec{r},\vec{R},q)&=\frac{1}{5\sqrt{24\pi}}\delta(\vec{x}_1-\pvec{x}_1')\delta(\vec{x}_2-\pvec{x}_2')\sum_{L_1,L_2}i^{L_1-L_2}(-1)^{L_1+L_2}C_{L_10L_20}^{20}\sqrt{(2L_1+1)(2L_2+1)}j_{L_2}(q R)\\
&\times\left[Y_{L_1}(\hat{r})\otimes Y_{L_2}(\hat{R})\right]_2\odot\left[\vec{\sigma}(1)\otimes\vec{\sigma}(2)\right]_2\int_0^1d\alpha\;e^{-r\Pi(q,\alpha)}j_{L_1}\left(q r(\alpha-1/2)\right)\frac{q^2\alpha(1-\alpha)}{\Pi(q,\alpha)}
\end{split}
\end{equation}
As before, the Feynman parameter integral vanishes unless $L_1$ is even. The Clebsch-Gordan coefficient then implies that $L_2$ must be even as well. Thus
\begin{equation}
\begin{split}
\mathcal{O}_3(\vec{r},\vec{R},q)&=\frac{1}{5\sqrt{24\pi}}\delta(\vec{x}_1-\pvec{x}_1')\delta(\vec{x}_2-\pvec{x}_2')\sum_{L_1=0,2,...}^{\infty}\sum_{L_2=0,2,...}^{\infty}(-1)^{(L_1-L_2)/2}C_{L_10L_20}^{20}\sqrt{(2L_1+1)(2L_2+1)}\\
&\times j_{L_2}(q R)\left[Y_{L_1}(\hat{r})\otimes Y_{L_2}(\hat{R})\right]_2\odot\left[\vec{\sigma}(1)\otimes\vec{\sigma}(2)\right]_2\int_0^1d\alpha\;e^{-r\Pi(q,\alpha)}j_{L_1}\left(q r(\alpha-1/2)\right)\frac{q^2\alpha(1-\alpha)}{\Pi(q,\alpha)}.
\end{split}
\end{equation}
Now we turn our attention to the term in the third line of Eq. \ref{eq:two_body_master}. The only angular dependence on $\hat{q}$ is in the exponentials but after completing the angular integral we must recouple the resulting two $\hat{r}$ spherical harmonics. For this task we need the identity
\begin{equation}
Y_{l_2}(\hat{r})\odot Y_{l_2}(\hat{R})\;Y_2(\hat{r})\odot\left[\vec{\sigma}(1)\otimes\vec{\sigma}(2)\right]_2=(-1)^{l_2}\sqrt{\frac{2l_2+1}{4\pi}}\sum_{l_1}C_{l_2020}^{l_10}\left[Y_{l_1}(\hat{r})\otimes Y_{l_2}(\hat{R})\right]_2\odot\left[\vec{\sigma}(1)\otimes\vec{\sigma}(2)\right]_2
\end{equation}
As before, we find that the integrand vanishes unless $L_2$ is even and the Clebsch-Gordon coefficient then requires that $L_1$ be even as well. Thus
\begin{equation}
\begin{split}
\mathcal{O}_4(\vec{r},\vec{R},q) &= \frac{1}{5\sqrt{24\pi}}\delta(\vec{x}_1-\pvec{x}_1')\delta(\vec{x}_2-\pvec{x}_2')\sum_{L_1,L_2\;\mathrm{even}}C_{L_10L_20}^{20}\sqrt{(2L_1+1)(2L_2+1)}j_{L_2}(q R)\\
&\times\left[Y_{L_1}(\hat{r})\otimes Y_{L_2}(\hat{R})\right]_2\odot\left[\vec{\sigma}(1)\otimes\vec{\sigma}(2)\right]_2\int_0^1 d\alpha\;e^{-r\Pi(q,\alpha)}j_{L_2}(q r(\alpha-1/2))\frac{1}{r}\left[1+r\Pi(q,\alpha)\right]
\end{split}
\end{equation}
Now we turn our attention to the final set of terms, the first line of Eq. \ref{eq:two_body_master}. First we decompose the tensor structure as
\begin{equation}
-\alpha \hat{r}\cdot\vec{\sigma}(1)\hat{q}\cdot\vec{\sigma}(2)+(1-\alpha)\hat{q}\cdot\vec{\sigma}(1)\hat{r}\cdot\vec{\sigma}(2)=\frac{4\pi}{3}\sum_{K=0}^2\left[Y_1(\hat{q})\otimes Y_1(\hat{r})\right]_K\odot\left[\vec{\sigma}(1)\otimes\vec{\sigma}(2)\right]_K\left[-\alpha + (1-\alpha)(-1)^K\right].
\end{equation}
Next we must perform a more complicated angular integral: the $Y_1(\hat{q})$-dependence is non-trivially coupled to the $Y_1(\hat{r})$. After integrating and re-coupling (using an identity to rewrite a sum of three Clebsch-Gordon coefficients as a single Clebsch-Gordon coefficient and a Wigner 6-$j$ symbol), we find
\begin{equation}
\begin{split}
&\int d\Omega_q\;Y_{00}(\hat{q})e^{-i\vec{q}\cdot{R}}e^{i(\alpha-1/2)\vec{q}\cdot\vec{r}}\left[Y_1(\hat{q})\otimes Y_1(\hat{r})\right]_K\odot\left[\vec{\sigma}(1)\otimes\vec{\sigma}(2)\right]_K\\
&=3\sqrt{4\pi}\sum_{L_1,L_2}(i)^{L_1-L_2}(2L_1+1)C_{L_1010}^{L_20}\sum_J (-1)^{1+J}C_{L_1010}^{J0}\left\{
\begin{array}{ccc}
L_1 & 1 & L_2\\
K & J & 1
\end{array}\right\}
\\ &\times\left[Y_J(\hat{r})\otimes Y_{L_2}(\hat{R})\right]_K\odot\left[\vec{\sigma}(1)\otimes\vec{\sigma}(2)\right]_Kj_{L_1}(q r(\alpha-1/2))j_{L_2}(q R)
\end{split}
\end{equation}
Now let us consider the three cases: $K=0,1,2$. When $K=0$, then $J=L_2$ and many of the angular momentum factors simplify. In particular,
\begin{equation}
\left\{\begin{array}{ccc}
L_1 & 1 & L_2\\
0 & L_2 & 1
\end{array}\right\}=\frac{(-1)^{1+L_1+L_2}}{\sqrt{3(2L_2+1)}},
\end{equation}
\begin{equation}
\left[Y_{L_2}(\hat{r})\otimes Y_{L_2}(\hat{R})\right]_{0,0}=\frac{(-1)^{L_2}}{\sqrt{2L_2+1}}Y_{L_2}(\hat{r})\odot Y_{L_2}(\hat{R}),
\end{equation}
and
\begin{equation}
\left[\vec{\sigma}(1)\otimes\vec{\sigma}(2)\right]_{0,0}=-\frac{1}{\sqrt{3}}\vec{\sigma}(1)\odot\vec{\sigma}(2).
\end{equation}
Therefore we have the operator
\begin{equation}
\begin{split}
\mathcal{O}_5(\vec{r},\vec{R},q)&=\frac{1}{12\sqrt{\pi}}\delta(\vec{x}_1-\pvec{x}_1')\delta(\vec{x}_2-\pvec{x}_2')\sum_{L1,L2}(i)^{1+L_1-L_2}\left(\frac{2L_1+1}{2L_2+1}\right)\left(C_{L_1010}^{L_20}\right)^2\\
&\times j_{L_2}(q R)\;Y_{L_2}(\hat{r})\odot Y_{L_2}(\hat{R})\;\vec{\sigma}(1)\odot\vec{\sigma}(2)\;q\int_0^1 d\alpha\;e^{-r\Pi(q,\alpha)}(1-2\alpha)j_{L_1}(q r(\alpha-1/2)).
\end{split}
\end{equation}
The integral with respect to $\alpha$ vanishes unless $L_1$ is odd, and the Clebsch-Gordon coefficient then implies that $L_2$ must be even. Therefore
\begin{equation}
\begin{split}
\mathcal{O}_5(\vec{r},\vec{R},q)&=\frac{1}{12\sqrt{\pi}}\delta(\vec{x}_1-\pvec{x}_1')\delta(\vec{x}_2-\pvec{x}_2')\sum_{L_1=1,3,...}^{\infty}\sum_{L_2=0,2,...}^{\infty}(-1)^{(1+L_1-L_2)/2}\left(\frac{2L_1+1}{2L_2+1}\right)\left(C_{L_1010}^{L_20}\right)^2\\
&\times j_{L_2}(q R)\;Y_{L_2}(\hat{r})\odot Y_{L_2}(\hat{R})\;\vec{\sigma}(1)\odot\vec{\sigma}(2)\;q\int_0^1 d\alpha\;e^{-r\Pi(q,\alpha)}(1-2\alpha)j_{L_1}(q r(\alpha-1/2))
\end{split}
\end{equation}
Similarly when $K=1$, we have
\begin{equation}
\begin{split}
&\mathcal{O}_6(\vec{r},\vec{R},q)=-\frac{1}{4\sqrt{\pi}}\delta(\vec{x}_1-\pvec{x}_1')\delta(\vec{x}_2-\pvec{x}_2')\sum_{L_1=0,2,...}^{\infty}\sum_{L_2=1,3,...}^{\infty}\sum_{J=1,3,...}^{\infty}(-1)^{(1+L_1-L_2)/2}(2L_1+1)C_{L_1010}^{L_20}C_{L_1010}^{J0}\\
&\times\left\{\begin{array}{ccc}
L_1 & 1 & L_2\\
1 & J & 1
\end{array}\right\}\left[Y_J(\hat{r})\otimes Y_{L_2}(\hat{R})\right]_1\odot\left[\vec{\sigma}(1)\otimes\vec{\sigma}(2)\right]_1 j_{L_2}(q R)\;q\int_0^1d\alpha\;e^{-r\Pi(q,\alpha)}j_{L_1}(qr(\alpha-1/2)).
\end{split}
\end{equation}
Finally when $K=2$,
\begin{equation}
\begin{split}
&\mathcal{O}_7(\vec{r},\vec{R},q)=-\frac{1}{4\sqrt{\pi}}\delta(\vec{x}_1-\pvec{x}_1')\delta(\vec{x}_2-\pvec{x}_2')\sum_{L_1=1,3,...}^{\infty}\sum_{L_2=0,2,...}^{\infty}\sum_{J=0,2,...}^{\infty}(-1)^{(1+L_1-L_2)/2}(2L_1+1)C_{L_1020}^{L_20}C_{L_1020}^{J0}\\
&\times\left\{\begin{array}{ccc}
L_1 & 1 & L_2\\
2 & J & 1
\end{array}\right\}\left[Y_J(\hat{r})\otimes Y_{L_2}(\hat{R})\right]_2\odot\left[\vec{\sigma}(1)\otimes\vec{\sigma}(2)\right]_2 j_{L_2}(q R)\;q \int_0^1d\alpha\;e^{-r\Pi(q,\alpha)}(1-2\alpha)j_{L_1}(qr(\alpha-1/2)).
\end{split}
\end{equation}
The total two-body operator can then be expressed as 
\begin{equation}
\mathcal{O}^{(2)}(q)=-\frac{g_A^2m_{\pi}^2}{4f_{\pi}^2}\frac{1}{2}\sum_{i\neq j}\sum_{k=1}^7\mathcal{O}_k(\vec{r}_{ij},\vec{R}_{ij},q)\;\vec{\tau}(i)\cdot\vec{\tau}(j),
\end{equation}
where the summation over $i,j$ extends over all nucleons in the target nucleus.
\begin{figure}
\centering
\includegraphics[scale=0.63]{relative_op_strength_al27.png}
\caption{The relative strength of various tensor components of the two-nucleon operators evaluated in $^{27}$Al with $q = 104.976$ MeV. Labels in parentheses indicate the angular momenta $(L)$, $(L_1, L_2)$, or $(L_1, L_2, J)$ of the tensor operator. For this chart, we ignore the sign of the resulting matrix element and normalize by the leading operator, the $L=0$ component of $\mathcal{O}_1$.}
\label{fig:rel_op_str}
\end{figure}
Total nuclear matrix elements of this operator can be computed in terms of the reduced two-body density matrix, as discussed in Appendix \ref{app:density}. Each of the seven operator structures $\mathcal{O}_k(\vec{r},\vec{R},q)$ contains an infinite sum of tensor operators indexed by angular momenta $L$, $(L_1,L_2)$, or $(L_1,L_2,J)$. It is worth exploring whether these series can be truncated in practical calculations. Figure \ref{fig:rel_op_str} show the relative strength of various operator components in $^{27}$Al, normalized by the strength of the leading contribution, the $L=0$ component of $\mathcal{O}_1(\vec{r},\vec{R},q)$. We see that there are only five components which contribute above the level of $0.1\%$. Two of these operators, the dominant $L=0$ mode of $\mathcal{O}_1$ and the $(L_1=2,L_2=0)$ mode of $\mathcal{O}_4$, correspond to the two distinct operator structures that survive in the $q\rightarrow 0$ limit (see Eq. \ref{eq:two_body_q_zero}). The three remaining operators that contribute significantly to the total two-body operator are the $L=0$ mode of $\mathcal{O}_2$, the $(L_1=1,L_0=0)$ mode of $\mathcal{O}_5$, and the $(L_1=0,L_2=1,J=1)$ mode of $\mathcal{O}_6$. These operators represent intrinsically finite-$q$ corrections to the two-nucleon response.

We retained the full $q$-dependence of the two-nucleon operator based on the fact that the magnitude of the three-momentum transfer in $\mu\rightarrow e$ conversion is comparable to the pion mass, $q\approx m_{\pi}$. Figure \ref{fig:two_body_finite_q} shows the value of the total two-body nuclear matrix element in $^{27}$Al as a function of three-momentum transfer $q$. The strength of the operator at $q=m_{\mu}$ is reduced by roughly $40$\% from the value at $q=0$. The finite-$q$ corrections are significant and should not be neglected.

\begin{figure}
\centering
\includegraphics[scale=1.2]{two_body_finite_q_v2.png}
\caption{Total nuclear matrix element of the NLO coherent two-body matrix operator contributing to $\mu\rightarrow e$ conversion (see Eq. \ref{eq:eff_equiv}) as a function of three-momentum transfer $q$ for the case of $^{27}$Al. The orange and blue curves, respectively, show the matrix element computed from a NSM wavefunction with and without the additional two-body correlation function of Eq. \ref{eq:f_corr}. The marked points denote the physically relevant value $q\approx m_{\mu}$. The green point and error bars show the value of the 1-body effective operator obtained via the Fermi gas average calculation.}
\label{fig:two_body_finite_q}
\end{figure} 

Figure \ref{fig:two_body_finite_q} also compares the nuclear shell-model result to the Fermi gas average result at the physically relevant momentum-transfer $q\approx m_{\mu}$. As discussed in Section \ref{sec:quarks_2_nucleons}, the FGA result is roughly a factor of two larger than the NSM result. We can compute the value of the effective 1-body coupling implied by the shell-model calculation, $f^{SI}_\mathrm{eff,NSM}$, by equating the matrix elements
\begin{equation}
\braket{J_i||\mathcal{O}^{(2)}(q)||J_i}=-\frac{3g_A^2m_{\pi}^2k_F}{64\pi f_{\pi}^2}f^{SI}_\mathrm{eff,NSM}\braket{J_i||M_0(q)||J_i},
\label{eq:eff_equiv}
\end{equation}
where $M_{00}(q)=\sum_{i=1}^Aj_0(qr_i)Y_{00}(\hat{r}_i)$ is the isoscalar one-body monopole charge operator. The resulting values of $f^{SI}_\mathrm{eff,NSM}$ for the nuclear targets of interest are shown in Table \ref{tab:two_body}. We see that the effective coupling implied by the shell model is essentially constant across the range of nuclei from $^{12}$C to $^{63}$Cu, and the value is insensitive to the interaction employed to obtain the wave function. 

\section{Correlation Function}
\label{sec:cor_fun}
The nuclear shell-model calculation may still overestimate the strength of the two-nucleon contribution. The nuclear shell-model wave functions are obtained by diagonalizing an effective phenomenological interaction which has been tuned to reproduce low-energy nuclear observables such as charge radii and low-lying spectra. The model space for these calculations is severely truncated. In $^{27}$Al, the valence nucleons are restricted to the $2s$-$1d$ harmonic oscillator shells. Although this formalism is capable of accurately reproducing many nuclear observables, it is known to fail in the following regard: at short range, the two-nucleon potential is strongly repulsive, and therefore the two-nucleon correlation function must develop at hole at very small separations.

\begin{figure}
\centering
\includegraphics[scale=1.0]{deuteron_s_comp.png}
\caption{The $S$-wave component of the deuteron wave function projected into various harmonic oscillator spaces indexed by cutoff in oscillator quanta $N$. The exact solution is obtained from the potential Av18. A value for the harmonic oscillator parameter $b=1.7$ was employed in these calculations.}
\label{fig:deuteron_proj}
\end{figure}

Resolving this fine structure, however, requires high momentum modes which are not included in the typically very soft Hilbert spaces of nuclear shell-model calculations. To demonstrate this concretely, we take the exact $S$-channel deuteron wave function obtained by solving the Schrodinger equation with potential given by Av18\cite{PhysRevC.51.38} and project this solution into harmonic oscillator spaces of varying dimension specified by a cutoff in harmonic oscillator quanta of $N=100$ (101 states), $N=50$ (51 states), $N=10$ (6 states) and finally $N=0$ (1 state). In Figure \ref{fig:deuteron_proj}, these projections are compared to exact wave function. In particular, we notice that the ``hole'' in the exact wave function at small separations is removed when the higher momentum modes are integrated out. The closest analog to our nuclear shell-model calculations is the $N=1$ curve. Thus the two-nucleon correlation function obtained from our shell-model calculations is erroneously enhanced at short range.

\begin{figure}
\centering
\includegraphics[scale=1.0]{V1_radial.png}
\caption{Dependence of the leading 2-body multipole on the relative radial coordinate $r$.}
\label{fig:V1_radial}
\end{figure}

We find that the strength of the two-nucleon $\mu\rightarrow e$ conversion operator strongly depends on the short-range behavior of the two-nucleon correlation function. The leading contribution is the $L=0$ multipole of the $\mathcal{O}_1$ operator, which has the following dependence on the relative radial coordinate
\begin{equation}
V_1(r,q)\equiv \int_0^1d\alpha\;e^{-r\Pi(q,\alpha)}j_0(qr(\alpha-1/2))\frac{2-r\Pi(q,\alpha)}{r}
\end{equation}
For small values of $r$, $V_1(r,q)\sim 1/r$ and the potential divergences. This behavior is illustrated in Figure \ref{fig:V1_radial}. In matrix elements, the divergence is regulated by the harmonic oscillator basis yielding a finite result. Nonetheless, the matrix element is very sensitive to the behavior of the two-nucleon correlation function. We can see this explicitly by modifying the two nucleon density with the addition of an ad hoc correlation function
\begin{equation}
\begin{split}
\psi(\vec{r}_i,\vec{r}_j)&\rightarrow\left(1-\beta(r_{ij})\right)\psi(\vec{r}_i,\vec{r}_j)\\
\beta(r)&=e^{-ar^2}\left(1-br^2\right),
\label{eq:f_corr}
\end{split}
\end{equation}
where $a=1.1$[fm]$^{-2}$ and $b=0.68$[fm]$^{-2}$. The radial coordinate here is $r_{ij}=|\vec{r}_i-\vec{r}_j|$. The effect of this modification is to add a hole to the two-nucleon density at separations $r\lesssim 1$ fm while leaving the long-distance behavior unchanged. We see this effect in Figure \ref{fig:V1_radial}, where the additional correlation function removes the divergence. Compared to the bare result, the total two-body matrix element in $^{27}$Al is reduced by $\approx 40\%$ by the additional correlation function. 

The shell-model wave functions are not proper effective wave functions. No attempt is made to properly normalize them in the model space (to account for the proportional of the total wave function which lies outside the model space). As well, to compute matrix elements with the shell-model wave functions, we (erroneously) evaluate only the bare operator. Having truncated the full Hilbert space down to a model space, the bare operator must be replaced by an effective operator, which accounts for the physics that has been integrated out. All of these considerations are beyond the scope of the current work. For now, it suffices to have some estimate of the strength of the two-nucleon operator in $\mu\rightarrow e$ conversion, particularly as our estimates likely represent an upper limit.
\subsection{Comparison to One-Body Average}
We began this section by considering the two-body current in Eq. \ref{eq:two_body_current}. The current associated with the diagram in Figure \ref{fig:two_body_labelled} contains some additional factors. Including the coupling to the lepton current, the two-body Lagrangian is
\begin{equation}
\begin{split}
\mathcal{L}_{2-N}&=-\frac{B_0 \mathring{g}_A^2}{\mathring{f}_{\pi}^2\Lambda^2}\left(J_u+J_d\right)\frac{1}{\left(q_1^2-m_{\pi}^2\right)\left(q_2^2-m_{\pi}^2\right)}\sum_a\left(\bar{N}'_1 S\cdot q_1\tau_a N_1\right)\left(\bar{N}_2' S\cdot q_2 \tau_a N_2\right)\\
&=-\frac{B_0\mathring{g}_A^2}{4\mathring{f}_{\pi}^2\Lambda^2}\left(J_u+J_d\right)\frac{1}{\left(|\vec{q}_1|^2+m_{\pi}^2\right)\left(|\vec{q}_2|^2+m_{\pi}^2\right)}\sum_a \left(\bar{N}_1' \vec{q}_1\cdot\vec{\sigma}(1) N_1 \right)\left(\bar{N}_2' \vec{q}_2\cdot\vec{\sigma}(2) N_2\right),
\end{split}
\end{equation}
where the $\circ$ over $\mathring{g}_A$ and $\mathring{f}_{\pi}$ indicates that these quantities should be evaluated in the chiral limit.

After one-body averaging, the coherent part of the resulting effective one-nucleon operator is written as
\begin{equation}
\mathcal{L}_\mathrm{eff}=-\frac{3B_0K_F \mathring{g}_A^2}{64\pi \mathring{f}_{\pi}^2\Lambda^2}\left(J_u+J_d\right)f_\mathrm{eff}^{SI}\bar{N}N,
\end{equation}
where $K_F$ is the Fermi momentum of the nucleus and $f_\mathrm{eff}^{SI}(\vec{q},\vec{k})$ is a dimensionless function which encodes the momentum-dependence of the effective operator. In their work, Bartolotta and Ramsey-Musolf approximate $f_\mathrm{eff}^{SI}$ as being constant. Comparing to previous studies of one-body averaging of two-nucleon operators, the authors obtain an expected value and error bars which reflect the fact that the effective operator is likely an overestimate of the true two-body operator:
\begin{equation}
f_\mathrm{eff}^{SI}=1.05^{+0.07}_{-0.53}.
\end{equation}

In Walecka's formalism, the leading piece of the one-body charge $\bar{N}N$ is the monopole term
\begin{equation}
M_0(q)=\sum_{j=1}^A Y_{0,0}(\hat{x}_j)j_0(qx_j)
\end{equation}
Disregarding common factors (including the lepton current) we can now make a direct comparison between the true two-body term
\begin{equation}
\braket{J_i||\mathcal{O}^{(2)}||J_i},
\end{equation}
and the effective operator
\begin{equation}
\braket{J_i||\mathcal{O}_\mathrm{eff}||J_i}=\frac{3K_F}{16\pi}f_\mathrm{eff}^{SI}\braket{J_f||M_0(q)||J_i}.
\end{equation}
The one-body monopole operator can be evaluated using either measured proton and neutron densities or the same many-body wavefunction that is used for the two-body operator. Using the measured densities we find
\begin{equation}
\braket{J_i||M_0(q)||J_i}=11.69\pm 0.23,
\end{equation}
whereas the NSM wavefunction yields
\begin{equation}
\braket{J_i||M_0(q)||J_i}_\mathrm{NSM}=11.81.
\end{equation}
Thus the NSM value is consistent with the measured nuclear density result. The uncertainty in the value computed from proton and neutron densities is due entirely to the uncertainty in the neutron density. 

Linearly interpolating between the measured nuclear Fermi momenta for $^{24}$Mg and $^{40}$Ca, the authors in \cite{2018PhRvC..98a5208B} obtain $K_F=238\pm 5$ MeV.  Combining these results for the effective operator we find
\begin{equation}
\braket{J_f||\mathcal{O}_\mathrm{eff}||J_i}=0.88^{+0.06}_{-0.45}\;\left[\mathrm{fm}\right]^{-1}.
\end{equation}
Figure \ref{fig:two_body_finite_q} shows the leading coherent two-body operator evaluated for a range of momentum transfer $q$ for the case of $^{27}$Al. The two-body matrix element is computed from a nuclear shell model wavefunction with and without the additional short-range correlation function. We see that the correlation function reduces the two-body matrix element by roughly 40\%. Even without this addition reduction, the two-body matrix element computed in the nuclear shell model falls outside of the 1$\sigma$ error bar on the 1-body effective operator, though well within the 2$\sigma$ error.

\chapter{Conclusions}
\section{Next Steps: EFT Matching and the Inelastic Case}
\section{Summary}
Inspired by previous work on dark matter direct detection \cite{Fitzpatrick_2013} and spurred by the looming promise of exceptional experimental progress in searches for charged lepton flavor violation, we have developed a nuclear effective theory of $\mu\rightarrow e$ conversion. Whereas previous works in the literature have either focused on the narrow -- though exceptionally interesting -- case of coherent conversion or retained only the relatively simple nuclear charge and spin operators, we have, for the first time, identified the most general set of response functions which can be probed through measurements of elastic $\mu\rightarrow e$ conversion in nuclei. As the nature of possible CLFV operators is yet entirely undetermined, one should consider a complete basis of effective operators (through a given order in some small power-counting parameter). Such a construction can be performed at a variety of energy scales in terms of different degrees of freedom, but the nuclear scale effective theory is the most natural in the sense that it interfaces directly with experiments. Moreover, by constructing the effective theory directly at the nuclear scale, we are able to achieve a factorization between the underlying CLFV physics -- which must be independent of the choice of target -- and the nuclear physics. This separation provides a clean path for constraining the underlying CLFV response functions through an ensemble of $\mu\rightarrow e$ conversion measurements on various nuclear targets. 

Crucial to obtaining the simple yet general form of our effective theory are the approximations that we have employed for the leptonic fields. In particular, the effective momentum approximation allows us to accurately model the outgoing electron as a Dirac plane wave, thereby permitting a straightforward multipole decomposition of the nuclear charges and currents. Our novel application of this technique -- which has heretofore been employed in high-energy electron scattering studies -- yields a powerful compromise between the numerical solution, which is highly-accurate but extremely cumbersome to employ in the general case, and the uncorrected plane wave solution, which allows for a simplified treatment but is a rather poor model of the outgoing electron. Indeed, the effective momentum solution introduces RMS errors $\lesssim 1\%$ in $^{27}$Al while retaining all the advantages of the plane wave.

The muon wave function can be treated exactly in the effective theory while maintaining a reasonably simple expression for the conversion rate. An even more transparent form for the effective theory can be obtained by neglecting entirely the muon's lower component and/or replacing the muon's slowly-varying (compared to the nuclear extent) radial wave function by a constant value. Applying both of these approximations, performing the multipole decomposition of the nuclear charges and currents, and appealing to the approximately good parity and time-reversal symmetries of the nuclear ground state, we find that the most general $\mu\rightarrow e$ conversion response is governed by six response functions and two interference terms. Restoring the muon's lower component will supplement the effective theory with [add this once we know] additional response functions, which are proportional to the muon velocity and therefore subleading.

The form of the effective theory dictates what can and cannot be learned about CLFV operators from measurements of elastic $\mu\rightarrow e$ conversion: in principle, one can constrain the values of the leptonic response functions but one cannot parse out the values of the individual low-energy constants of the nucleon-level effective theory. Only the particular bilinear combinations of LECs specified by the CLFV response functions can be constrained by experiment. The nuclear effective theory provides a blueprint for a program of $\mu\rightarrow e$ conversion measurements on an ensemble of nuclear targets which would allow one to probe the various CLFV response functions.

As the ultimate goal of this experimental program is to constrain and/or determine the nature of beyond-standard-model CLFV physics, one must be able to port any information obtained in the nuclear-level effective theory to candidate UV theories. These apparently disparate descriptions can be connected through a program of effective theory renormalization and matching that relates physics at different energy scales, beginning from the very low nuclear scale of $\mu\rightarrow e$ conversion experiments, past the scale where quarks deconfine, past the scale of electroweak symmetry breaking, and eventually to the scale of new CLFV physics. In fact, this matching program has already been explicitly realized in the special case of scalar-mediated coherent conversion. Work is now underway to extend the matching to the general effective theory. An analogous matching program has already been completed for the previously mentioned case of dark matter direct detection, connecting UV models of dark matter to the corresponding nuclear scale effective theory. 

\bibliographystyle{unsrtnat}
\bibliography{../../bib_erule}

\appendix
\chapter{Notation, Definitions and Basic Proofs}
\subsection{Isospin Matrices}
The isospin matrices $\vec{\tau}$ are exactly the Pauli matrices
\begin{equation}
\tau_1=\left(\begin{array}{cc}
0 & 1\\
1 & 0
\end{array}\right),\;
\tau_2=\left(\begin{array}{cc}
0 & -i\\
i & 0
\end{array}\right),\;
\tau_3=\left(\begin{array}{cc}
1 & 0\\
0 & -1
\end{array}\right).
\end{equation}
We define the isospin raising and lowering operators
\begin{equation}
\tau_{\pm}\equiv \frac{1}{2}\left(\tau_1\pm i \tau_2\right),
\end{equation}
so that
\begin{equation}
\tau_+=\left(\begin{array}{cc}
0 & 1\\
0 & 0
\end{array}\right),\;\tau_- = \left(\begin{array}{cc}
0 & 0\\
1 & 0
\end{array}\right)
\end{equation}
We note the following useful relations
\begin{equation}
\begin{split}
\left[\tau_+,\tau_-\right]&=\tau_3\\
\left[\tau_3,\tau_{\pm}\right]&=\pm 2\tau_{\pm}\\
\left\{\tau_+,\tau_-\right\}&=\mathbf{1}\\
\left\{\tau_3,\tau_{\pm}\right\}&=0
\end{split}
\end{equation}
\subsection{Chiral Projection Operators}
The left and right projections of a Dirac spinor are defined by
\begin{equation}
P_L\equiv \frac{1}{2}\left(1-\gamma_5\right),\;P_R\equiv \frac{1}{2}\left(1+\gamma_5\right)
\end{equation}
Using the fact that 
\begin{equation}
\left(\gamma_5\right)^2=\mathbf{1},
\end{equation}
it follows that
\begin{equation}
\gamma_5 P_L = -P_L,\;\gamma_5P_R=P_R
\end{equation}
Using the fact that the fifth gamma matrix anticommutes with the other four
\begin{equation}
\left\{\gamma^{\mu},\gamma^5\right\}=0,
\end{equation}
we have
\begin{equation}
\gamma^{\mu}P_L=P_R\gamma^{\mu},\;\gamma^{\mu}P_R=P_L\gamma^{\mu}.
\end{equation}
Then we may compute
\begin{equation}
\begin{split}
\bar{q}q&=\bar{q}_Rq_L+\bar{q}_Lq_R\\
\bar{q}\gamma^5q&=\bar{q}_Lq_R-\bar{q}_Rq_L\\
\bar{q}\gamma^{\mu}q&=\bar{q}_L\gamma^{\mu}q_L+\bar{q}_R\gamma^{\mu}q_R\\
\bar{q}\gamma^{\mu}\gamma^5 q&=-\bar{q}_L\gamma^{\mu}q_L +\bar{q}_R\gamma^{\mu}q_R\\
\bar{q}\sigma^{\mu\nu}q&=\bar{q}_R\sigma^{\mu\nu}q_L+\bar{q}_L\sigma^{\mu\nu}q_R
\end{split}
\end{equation}
\chapter{Expansion of Spinor Currents}
Following Bjorken \& Drell, we work in the Dirac representation of the $\gamma$ matrices and employ the following normalization convention for spinors:
\begin{equation}
u(p)=\sqrt{\frac{E+m}{2m}}\left(\begin{array}{c}
\xi\\
\frac{\vec{\sigma}\cdot\vec{p}}{E+m}\xi
\end{array}\right)
\end{equation}
so that
\begin{equation}
\bar{u}u=1.
\end{equation}
The $\gamma$ matrices in the Dirac basis are
\begin{equation}
\gamma^0=\left(\begin{array}{cc}
I_2 & 0 \\
0 & -I_2
\end{array}\right),\;\gamma^k=\left(\begin{array}{cc}
0 & \sigma^k\\
-\sigma^k & 0
\end{array}\right),\;\gamma^5=\left(\begin{array}{cc}
0 & I_2\\
I_2 & 0
\end{array}\right)
\end{equation}
The basis for Dirac spinors is then furnished by the scalar $1$, pseudo-scalar $\gamma^5$, vector $\gamma^{\mu}$, axial-vector $\gamma^{\mu}\gamma^5$ and tensor $\sigma^{\mu\nu}\equiv \frac{i}{2}\left[\gamma^{\mu},\gamma^{\nu}\right]$ matrices. Explicitly, in the Dirac basis, the tensor operator can be written as
\begin{equation}
\begin{split}
\sigma^{0i}&=\frac{i}{2}\left[\gamma^0,\gamma^i\right]=i\left(\begin{array}{cc}
0 & \sigma^i\\
\sigma^i & 0
\end{array}\right)\\
\sigma^{ij}&=\frac{i}{2}\left[\gamma^i,\gamma^j\right]=\epsilon^{ijk}\left(\begin{array}{cc}
\sigma^k & 0\\
0 & \sigma^k
\end{array}\right)
\end{split}
\end{equation}
\section{Non-relativistic Expansion of Nucleon Spinor Currents}
In the non-relativistic limit, the nucleon spinors can be written as
\begin{equation}
N(k)=\left(\begin{array}{c}
\xi\\
\frac{1}{2m_N}\vec{k}\cdot\vec{\sigma}_N\xi
\end{array}\right)
\end{equation}
Then the leading Lorentz-covariant nucleon spinor currents can be reduced in terms of Pauli spinors as
\begin{equation}
\begin{split}
\bar{N}(k')N(k)&\approx \xi^{'\dag}\left[1_N\right]\xi\\
\bar{N}(k')i\gamma^5N(k)&\approx \xi^{'\dag}\left[i\frac{\vec{q}}{2m_N}\cdot\vec{\sigma}_N\right]\xi\\
\bar{N}(k')\gamma^0 N(k)&\approx \xi^{'\dag}\left[1_N\right]\xi\\
\bar{N}(k')\gamma^i N(k)&\approx\xi^{'\dag}\left[\vec{v}_N+i\frac{\vec{q}}{2m_N}\times\vec{\sigma}_N\right]^i\xi\\
\bar{N}(k')\gamma^0\gamma^5N(k)&\approx \xi^{'\dag}\left[\vec{v}_N\cdot\vec{\sigma}_N\right]\xi\\
\bar{N}(k')\gamma^i\gamma^0 N(k)&\approx \xi^{'\dag}\left[\vec{\sigma}_N\right]^i\xi\\
\bar{N}(k')i\sigma^{0i}\frac{q_i}{m_N}N(k)&\approx 0\\
\bar{N}(k')i\sigma^{ij}\frac{q_j}{m_N}N(k)&\approx \xi^{\dag'}\left[-i\frac{\vec{q}}{m_N}\times\vec{\sigma}_N\right]^i\xi\\
\bar{N}(k')\sigma^{0i}\gamma_5\frac{q_i}{m_N}N(k)&\approx \xi^{\dag'}\left[-i\frac{\vec{q}}{m_N}\cdot\vec{\sigma}_N\right]\xi\\
\bar{N}(k')\sigma^{ij}\gamma_5\frac{q_j}{m_N}N(k)&\approx 0
\end{split}
\end{equation}
Thus we see that only four unique Hermitian operators arise in the reduction of the single-nucleon currents: $1_N$, $i\vec{q}$, $\vec{v}_N$, and $\vec{\sigma}_N$. This fact is crucial to the formulation of the single-nucleon effective theory.
%\begin{table}
%\centering
%\begin{tabular}{ccc}
%\hline
%\hline
%Relativistic form & NR & 1st Quantized\\
%\hline
%$\bar{N}(k')N(k)$ & $\xi^{\dag}\xi$ & $J_0(\vec{x})$\\
%$\bar{N}(k')i\gamma_5N(k)$ & $i\xi^{'\dag}\frac{\vec{q}}{2m_N}\cdot\vec{\sigma}\xi$ & $i\frac{\vec{q}}{2m_N}\cdot\vec{J}_A(\vec{x})$\\
%$\bar{N}(k')\gamma^0N(k)$ & $\xi^{\dag}\xi$ & $J_0(\vec{x})$\\
%$\bar{N}(k')\vec{\gamma}N(k)$ & $\vec{v}_N\xi^{'\dag}\xi + i\frac{\vec{q}}{2m_N}\times\xi^{'\dag}\vec{\sigma}\xi$ & $\vec{J}_c(\vec{x})+i\frac{\vec{q}}{2m_N}\times\vec{J}_A(\vec{x})$ \\
%$\bar{N}(k')\gamma^0\gamma_5N(k)$ & $\vec{v}_N\cdot\xi^{\dag}\vec{\sigma}\xi$ & $J_0^A(\vec{x})$ \\
%$\bar{N}(k')\vec{\gamma}\gamma_5N(k)$ & $\xi^{'\dag}\vec{\sigma}\xi$ & $\vec{J}_A(\vec{x})$\\
%$\bar{N}(k')\sigma^{0i}N(k)$ & $i\frac{q}{2m_N}\xi^{'\dag}\xi - \vec{v}_N\times\xi^{'\dag}\vec{\sigma}\xi$ & $i\frac{\vec{q}}{2m_N}J_0(\vec{x})$\\
%$\bar{N}(k')\sigma^{ij}N(k)$ & $\xi^{'\dag}\vec{\sigma}\xi$ & $\vec{J}_A(\vec{x})$\\
%\end{tabular}
%\end{table}
\section{Expansion of Leptonic Spinor Currents}
For the purpose of deriving the Pauli operator form of the leptonic Dirac currents, we define the electron and muon Dirac spinors, respectively, as
\begin{equation}
\chi_e=\left(\begin{array}{c}
\xi\\
\hat{q}\cdot\vec{\sigma}_L\xi
\end{array}\right),\;\;\;\;
\chi_{\mu}=\left(\begin{array}{c}
\xi\\
\frac{1}{2}\vec{v}_{\mu}\cdot\vec{\sigma}_L\xi
\end{array}\right).
\end{equation}
Then it is a straightforward exercise in vector and spinor algebra to derive the following reductions of the Lorentz-covariant lepton currents:
\begin{equation}
\begin{split}
\bar{\chi}_e\chi_{\mu}&\approx \xi^{\dag'}\left[1_L-\frac{1}{2}\hat{q}\cdot\vec{v}_{\mu}-\frac{i}{2}\hat{q}\cdot\left(\vec{v}_{\mu}\times\vec{\sigma}_L\right)\right]\xi\\
\bar{\chi}_ei\gamma_5\chi_{\mu}&\approx \xi^{\dag'}\left[-i\hat{q}\cdot\vec{\sigma}_L+\frac{i}{2}\vec{v}_{\mu}\cdot\vec{\sigma}_L\right]\xi\\
\bar{\chi}_e\gamma^0\chi_{\mu}&\approx \xi^{\dag'}\left[1_L+\frac{1}{2}\hat{q}\cdot\vec{v}_{\mu}+\frac{i}{2}\hat{q}\cdot\left(\vec{v}_{\mu}\times\vec{\sigma}_L\right)\right]\xi\\
\bar{\chi}_e\gamma^i\chi_{\mu}&\approx \xi^{\dag'}\left[\hat{q}-i\hat{q}\times\vec{\sigma}_L+\frac{1}{2}\vec{v}_{\mu}+\frac{i}{2}\vec{v}_{\mu}\times\vec{\sigma}_L\right]^i\xi\\
\bar{\chi}_e\gamma^0\gamma_5\chi_{\mu}&\approx \xi^{\dag'}\left[\hat{q}\cdot\vec{\sigma}_L+\frac{1}{2}\vec{v}_{\mu}\cdot\vec{\sigma}_L\right]\xi\\
\bar{\chi}_e\gamma^i\gamma_5\chi_{\mu}&\approx \xi^{\dag'}\left[\vec{\sigma}_L-\frac{1}{2}i\hat{q}\times\vec{v}_{\mu}+\frac{1}{2}\hat{q}\times\left(\vec{v}_{\mu}\times\vec{\sigma}_L\right)+\frac{1}{2}\left(\vec{v}_{\mu}\cdot\vec{\sigma}_L\right)\hat{q}\right]^i\xi\\
\bar{\chi}_ei\sigma^{0i}\frac{q_i}{m_L}\chi_{\mu}&\approx \frac{q}{m_L}\xi^{\dag'}\left[-1_L+\frac{1}{2}\hat{q}\cdot\vec{v}_{\mu}+\frac{i}{2}\hat{q}\cdot\left(\vec{v}_{\mu}\times\vec{\sigma}_L\right)\right]\xi\\
\bar{\chi}_ei\sigma^{ij}\frac{q_j}{m_L}\chi_{\mu}&\approx -\frac{q}{m_L}\xi^{\dag'}\left[i\hat{q}\times\vec{\sigma}_L+\frac{1}{2}\vec{v}_{\mu}-\frac{1}{2}\left(\hat{q}\cdot\vec{v}_{\mu}\right)\hat{q}-\frac{i}{2}\hat{q}\cdot\left(\vec{v}_{\mu}\times\vec{\sigma}_L\right)\hat{q}+\frac{i}{2}\vec{v}_{\mu}\times\vec{\sigma}_L\right]^i\xi\\
\bar{\chi}_e\sigma^{0i}\gamma_5\frac{q_i}{m_L}\chi_{\mu}&\approx \frac{q}{m_L}\xi^{\dag'}\left[-i\hat{q}\cdot\vec{\sigma}_L+\frac{i}{2}\vec{v}\cdot\vec{\sigma}_L\right]\xi\\
\bar{\chi}_e\sigma^{ij}\gamma_5\frac{q_j}{m_L}\chi_{\mu}&\approx \frac{q}{m_L}\left[i\vec{\sigma}_L-i\left(\hat{q}\cdot\vec{\sigma}_L\right)\hat{q}-\frac{1}{2}\hat{q}\times\vec{v}_{\mu}-\frac{i}{2}\hat{q}\times\left(\vec{v}_{\mu}\times\vec{\sigma}_L\right)\right]^i\xi
\end{split}
\end{equation}
Combining these results with the analogous expressions for the reduction of the nucleon currents, one can arrive at Table \ref{tab:operator_list}. As for the nucleons, there are four unique Hermitian operators which appear in the reduction of the leptonic currents: $1_L$, $i\hat{q}$, $\vec{v}_{\mu}$ and $\vec{\sigma}_L$. 
\chapter{Spherical Tensor and Tensor Operators}
We follow the conventions of Edmonds \cite{edmonds1996angular}. 
\section{Spherical Basis}
Consider a three-dimensional coordinate system where $\hat{e}_x$, $\hat{e}_y$, and $\hat{e}_z$ are the unit vectors along the $x$, $y$, and $z$ axes respectively. The generators of rotations about these three axes are the operators 
\begin{equation}
S_x=i\hat{e}_x\times;\;S_y=i\hat{e}_y\times;\;S_z=i\hat{e}_z\times,
\end{equation}
where $\times$ indicates the vector cross product. The operators $\vec{S}$ satisfy the usual commutation relations of angular momentum operators
\begin{equation}
\left[S_i,S_j\right]=i\epsilon_{ijk}S_k.
\end{equation}
\begin{equation}
\begin{split}
\hat{e}_{+1}&=-\frac{1}{\sqrt{2}}\left(\hat{e}_x+i\hat{e}_y\right)\\
\hat{e}_0&=\hat{e}_z\\
\hat{e}_{-1}&=\frac{1}{\sqrt{2}}\left(\hat{e}_x-i\hat{e}_y\right)
\end{split}
\end{equation}
which satisfy
\begin{equation}
\begin{split}
\vec{S}^2\hat{e}_\lambda&=2\hat{e}_\lambda\\
S_z\hat{e}_\lambda&=\lambda\hat{e}_\lambda,
\end{split}
\end{equation}
where $\lambda=0,\pm 1$. The eigenvalue of the total spin operator $\vec{S}^2$ indicates that the representation is spin 1. The spherical unit vectors $\hat{e}_\lambda$ have the following useful properties:

Under complex conjugation
\begin{equation}
\hat{e}_\lambda^*=(-1)^\lambda\hat{e}_{-\lambda},\;\mathrm{for}\;\lambda=0,\pm 1
\end{equation}
The scalar product of two spherical unit vectors is given by 
\begin{equation}
\hat{e}_\lambda^*\cdot\hat{e}_{\lambda'}=(-1)^\lambda\hat{e}_\lambda\cdot\hat{e}_{-\lambda}=\delta_{\lambda\lambda'}.
\end{equation}
Any vector may be expanded in the spherical basis as
\begin{equation}
\vec{V}=\sum_\lambda V_\lambda \hat{e}_\lambda^*=\sum_{\lambda}(-1)^\lambda V_\lambda \hat{e}_{-\lambda}
\end{equation}
where the vector components are given by
\begin{equation}
\begin{split}
V_{\pm 1}&=\mp \frac{1}{\sqrt{2}}\left(V_x\pm i V_y\right),\\
V_0&=V_z.
\end{split}
\end{equation}
These vector components can be obtained by projecting with the spherical basis vectors
\begin{equation}
V_\lambda=\hat{e}_\lambda\cdot\vec{V}.
\end{equation}
Taking cross products of the spherical basis vectors
\begin{equation}
\begin{split}
\hat{e}_{\lambda}\times\hat{e}_{\lambda}&=0\\
\hat{e}_{\pm}\times\hat{e}_{\mp}&=\pm i\hat{e}_0\\
\hat{e}_{\pm}\times\hat{e}_0&=\pm i\hat{e}_{\pm}
\end{split}
\end{equation}
\section{Vector Spherical Harmonics}
\label{sec:vector_spherical_harmonics}
The ordinary spherical harmonics $Y_{\ell m}(\theta,\phi)=Y_{\ell m}(\hat{r})$ form a basis of scalar functions on the sphere; that is, any (well-behaved) smooth function $f(\theta,\phi)$ can be decomposed as
\begin{equation}
f(\theta,\phi)=\sum_{\ell=0}^{\infty}\sum_{m=-\ell}^\ell f_{\ell m}Y_{\ell m}(\theta,\phi)
\end{equation}
where
\begin{equation}
f_{\ell m}=\int d\Omega\;f(\theta,\phi)Y^*_{\ell m}(\theta,\phi).
\end{equation}
The spherical harmonics are a special basis as they are eigenfunctions of the orbital angular momentum operator
\begin{equation}
\begin{split}
\vec{L}^2Y_{\ell m}&=\ell(\ell+1)Y_{\ell m}\\
L_zY_{\ell m}&=mY_{\ell m}.
\end{split}
\end{equation}
One would like to find an analogous basis for vector-valued functions on the sphere $\vec{V}(\theta,\phi)$. Of course, if we expand in the spherical basis $\vec{V}=\sum_{\lambda}(-1)^\lambda V_{-\lambda}\hat{e}_\lambda$, then each component $V_{-\lambda}(\theta,\phi)$ is a scalar function which can be decomposed in terms of ordinary spherical harmonics
\begin{equation}
\vec{V}(\theta,\phi)=\sum_{\lambda}(-1)^\lambda \hat{e}_{\lambda}\sum_{\ell=0}^{\infty}\sum_{m=-\ell}^\ell\left(V_{-\lambda}\right)_{\ell m}Y_{\ell m}(\theta,\phi),
\end{equation}
where we see that $\hat{e}_{\lambda}$ and $Y_{\ell m}$ carry angular momentum under the operators $\vec{S}$ and $\vec{L}$, respectively. Therefore we combine these two objects into a tensor of definite total angular momentum $\vec{J}=\vec{L}+\vec{S}$
\begin{equation}
\begin{split}
\vec{Y}_{J\;\ell\;M}(\hat{r})&\equiv
\left[Y_\ell(\hat{r})\otimes\hat{e}\right]_{JM}\\
&=\sum_{m\lambda}Y_{\ell\;m}(\hat{r})\hat{e}_\lambda \braket{\ell\;m\;1\;\lambda|J\;M},
\end{split}
\end{equation}
where $J=\ell,\ell\pm 1$. The resulting objects, the vector spherical harmonics, satisfy the orthogonality condition
\begin{equation}
\int d\Omega\;\vec{Y}^*_{J\;\ell\;M}(\hat{r})\cdot\vec{Y}_{J'\;\ell'\;M'}(\hat{r})=\delta_{JJ'}\delta_{\ell\ell'}\delta_{MM'}
\end{equation}
and are eigenfunctions of $\vec{J}$
\begin{equation}
\begin{split}
\vec{J}^2\vec{Y}_{J\;\ell\;M}&=J(J+1)\vec{Y}_{J\;\ell\;M}\\
J_z\vec{Y}_{J\;\ell\;M}&=M\vec{Y}_{J\;\ell\;M}.
\end{split}
\end{equation}
Indeed they furnish a basis of vector-valued functions on the sphere
\begin{equation}
\vec{V}(\theta,\phi)=\sum_{\ell=0}^{\infty}\sum_J\sum_{M=-J}^J V_{J\;\ell\;M}\vec{Y}_{J\;\ell\;M}(\theta,\phi),
\end{equation}
where the coefficients are given by
\begin{equation}
V_{J\;\ell\;M}=\int d\Omega\;\vec{V}(\theta,\phi)\cdot\vec{Y}^*_{J\;\ell\;M}(\theta,\phi).
\end{equation}
The vector spherical harmonics have several other properties which make them an effective tool for, among other uses, performing a multipole expansion of three-current operators. First, they inherit the parity transformation of the ordinary spherical harmonics 
\begin{equation}
Y_{\ell m}(-\hat{r})=(-1)^{\ell}Y_{\ell m}(\hat{r}),
\end{equation}
and therefore $\vec{Y}_{J\;\ell\;M}$ has parity $(-1)^\ell$. 
Next, as one may expect, there are many close relationships between the scalar and vector spherical harmonics. For example, the vector spherical harmonics can be obtained from the ordinary spherical harmonics by application of the unit vector $\hat{r}$
\begin{equation}
\hat{r}Y_{\ell M}(\hat{r})=-\sqrt{\frac{\ell+1}{2\ell+1}}\vec{Y}_{\ell\;\ell+1\;M}(\hat{r})+\sqrt{\frac{\ell}{2\ell+1}}\vec{Y}_{\ell\;\ell-1\;M}(\hat{r}),
\end{equation}
%is the magnetic field of electric multipole  radiation and the electric field of magnetic multipole radiation.is the electric field of electric multipole radiation and the magnetic field of magnetic multipole radiation
or the gradient operator
\begin{equation}
\vec{\nabla}\left(f(r)Y_{\ell\;M}(\hat{r})\right)=-\sqrt{\frac{\ell+1}{2\ell+1}}\left(\frac{d}{dr}-\frac{\ell}{r}\right)f(r)\vec{Y}_{\ell\;\ell+1\;M}(\hat{r})+\sqrt{\frac{\ell}{2\ell+1}}\left(\frac{d}{dr}+\frac{\ell+1}{r}\right)f(r)\vec{Y}_{\ell\;\ell-1\;M}(\hat{r}),
\end{equation}
where $f(r)$ is any (well-behaved) scalar function. Conversely, the ordinary spherical harmonics arise through application of the divergence operator to the vector spherical harmonics
\begin{equation}
\begin{split}
\vec{\nabla}\cdot\left(f(r)\vec{Y}_{\ell\;\ell+1\;M}(\hat{r})\right)&=-\sqrt{\frac{\ell+1}{2\ell+1}}\left(\frac{d}{dr}+\frac{\ell+2}{r}\right)f(r)Y_{\ell\;M}(\hat{r})\\
\vec{\nabla}\cdot\left(f(r)\vec{Y}_{\ell\;\ell\;M}(\hat{r})\right)&=0,\;\mathrm{for\;any}\;f(r)\\
\vec{\nabla}\cdot\left(f(r)\vec{Y}_{\ell\;\ell-1\;M}(\hat{r})\right)&=\sqrt{\frac{\ell}{2\ell+1}}\left(\frac{d}{dr}-\frac{\ell-1}{r}\right)f(r)\;Y_{\ell\;M}(\hat{r}).
\end{split}
\end{equation}
Finally, as the curl operator $\vec{\nabla}\times$ maps vector fields to vector fields, so it relates the vector spherical harmonics
\begin{equation}
\begin{split}
\vec{\nabla}\times\left(f(r)\vec{Y}_{\ell\;\ell+1\;M}(\hat{r})\right)&=i\sqrt{\frac{\ell}{2\ell+1}}\left(\frac{d}{dr}+\frac{\ell+2}{r}\right)f(r)\vec{Y}_{\ell\;\ell\;M}(\hat{r}),\\
\vec{\nabla}\times\left(f(r)\vec{Y}_{\ell\;\ell\;M}(\hat{r})\right)&=i\sqrt{\frac{\ell}{2\ell+1}}\left(\frac{d}{dr}-\frac{\ell}{r}\right)f(r)\vec{Y}_{\ell\;\ell+1\;M}(\hat{r})+i\sqrt{\frac{\ell+1}{2\ell+1}}\left(\frac{d}{dr}+\frac{\ell+1}{r}\right)f(r)\vec{Y}_{\ell\;\ell-1\;M}(\hat{r}),\\
\vec{\nabla}\times\left(f(r)\vec{Y}_{\ell\;\ell-1\;M}(\hat{r})\right)&=i\sqrt{\frac{\ell+1}{2\ell+1}}\left(\frac{d}{dr}-\frac{\ell-1}{r}\right)f(r)\vec{Y}_{\ell\;\ell\;M}(\hat{r}).
\end{split}
\end{equation}
In this work, we often find that the scalar function is the regular spherical Bessel function of order $J$, $f(r)=j_\ell(qr)$. Then it is useful to know the raising and lowering operators for the spherical Bessel functions
\begin{equation}
\begin{split}
\left(\frac{d}{dr}-\frac{\ell}{r}\right)j_\ell(qr)&=-qj_{\ell+1}(qr),\\
\left(\frac{d}{dr}+\frac{\ell+1}{r}\right)j_\ell(qr)&=qj_{\ell-1}(qr).
\end{split}
\end{equation}
Using these relations, we obtain the following expressions for the gradient and curl specialized to the case $f(r)=j_\ell(qr)$:
\begin{equation}
\begin{split}
\vec{\nabla}\left(j_\ell(qy)Y_{\ell\;M}(\hat{r})\right)&=q\left[\sqrt{\frac{\ell+1}{2\ell+1}}j_{\ell+1}(qr)\vec{Y}_{\ell\;\ell+1\;M}(\hat{r})+\sqrt{\frac{\ell}{2\ell+1}}j_{\ell-1}(qr)\vec{Y}_{\ell\;\ell-1\;M}(\hat{r})\right]\\
\vec{\nabla}\times\left(j_\ell(qr)\vec{Y}_{\ell\;\ell\;M}(\hat{r})\right)&=iq\left[-\sqrt{\frac{\ell}{2\ell+1}}j_{\ell+1}(qr)\vec{Y}_{\ell\;\ell+1\;M}(\hat{r})+\sqrt{\frac{\ell+1}{2\ell+1}}j_{\ell-1}(qr)\vec{Y}_{\ell\;\ell-1\;M}(\hat{r})\right].
\end{split}
\end{equation}
We can invert these expressions to obtain
\begin{equation}
\begin{split}
j_{\ell+1}(qr)\vec{Y}_{\ell\;\ell+1\;M}(\hat{r})&=\frac{1}{q}\left[\sqrt{\frac{\ell+1}{2\ell+1}}\vec{\nabla}\left(j_\ell(qr)Y_{\ell\;M}(\hat{r})\right)+i\sqrt{\frac{\ell}{2\ell+1}}\vec{\nabla}\times\left(j_\ell(qr)\vec{Y}_{\ell\;\ell\;M}(\hat{r})\right)\right]\\
j_{\ell-1}(qr)\vec{Y}_{\ell\;\ell-1\;M}(\hat{r})&=\frac{1}{q}\left[\sqrt{\frac{\ell}{2\ell+1}}\vec{\nabla}\left(j_\ell(qr)Y_{\ell\;M}(\hat{r})\right)-i\sqrt{\frac{\ell+1}{2\ell+1}}\vec{\nabla}\times\left(j_\ell(qr)\vec{Y}_{\ell\;\ell\;M}(\hat{r})\right)\right],
\end{split}
\end{equation}
or equivalently
\begin{equation}
\begin{split}
j_{\ell}(qr)\vec{Y}_{\ell-1\;\ell\;M}(\hat{r})&=\frac{1}{q}\left[\sqrt{\frac{\ell}{2\ell-1}}\vec{\nabla}\left(j_{\ell-1}(qr)Y_{\ell-1\;M}(\hat{r})\right)+i\sqrt{\frac{\ell-1}{2\ell-1}}\vec{\nabla}\times\left(j_{\ell-1}(qr)\vec{Y}_{\ell-1\;\ell-1\;M}(\hat{r})\right)\right]\\
j_{\ell}(qr)\vec{Y}_{\ell+1\;\ell\;M}(\hat{r})&=\frac{1}{q}\left[\sqrt{\frac{\ell+1}{2\ell+3}}\vec{\nabla}\left(j_{\ell+1}(qr)Y_{\ell+1\;M}(\hat{r})\right)-i\sqrt{\frac{\ell+2}{2\ell+3}}\vec{\nabla}\times\left(j_{\ell+1}(qr)\vec{Y}_{\ell+1\;\ell+1\;M}(\hat{r})\right)\right],
\end{split}
\end{equation}
The utility of these expressions is that the two vector fields $j_\ell(qr)\vec{Y}_{\ell\pm 1\;\ell\;M}$ have been decomposed into longitudinal (curl-free) and transverse (divergence-free) components. The third vector field $j_{\ell}(qr)\vec{Y}_{\ell\;\ell\;M}$ is automatically transverse.

In general, a plane wave can be expanded in partial waves as
\begin{equation}
e^{i\vec{q}\cdot\vec{r}}=4\pi \sum_{L=0}^{\infty}i^Lj_L(qr)Y_L(\hat{q})\odot Y_L(\hat{r})
\end{equation}
If we define our basis to be along the direction of $\hat{q}$, then the plane wave expansion takes the form
\begin{equation}
e^{i\vec{q}\cdot\vec{r}}=\sum_{L=0}^{\infty}\sqrt{4\pi(2L+1)}i^Lj_L(qr)Y_{L0}(\hat{r})
\end{equation}
Defining the multipole projection functions
\begin{equation}
\begin{split}
M_{JM}(q\vec{x})&\equiv j_J(qr)Y_{JM}(\hat{r})\\
\vec{M}_{J\;L\;M}(q\vec{r})&\equiv j_L(qr)\vec{Y}_{J\;L\;M}(\hat{r})
\end{split}
\end{equation}
\begin{equation}
e^{i\vec{q}\cdot\vec{r}}\hat{e}_{\lambda}=-\sum_{L=1}^{\infty}\sqrt{2\pi(2L+1)}i^L\left[\lambda \vec{M}_{L\;L\;\lambda}(q\vec{r})+\frac{1}{q}\vec{\nabla}\times\vec{M}_{L\;L\;\lambda}(q\vec{r})\right]
\end{equation}
for $\lambda = \pm 1$, whereas when $\lambda = 0$ we have
\begin{equation}
e^{i\vec{q}\cdot\vec{r}}\hat{e}_0=-\frac{i}{q}\sum_{L=0}^{\infty}\sqrt{4\pi(2L+1)}i^L\vec{\nabla}M_{J0}(q\vec{r})
\end{equation}
We identify the basic multipole projections: an arbitrary charge $\rho(\vec{r})$ can be decomposed as
\begin{equation}
\mathcal{M}_{JM}(q)=\int d^3r\;M_{JM}(q\vec{r})\rho(\vec{r})
\end{equation}
whereas an arbitrary three-current has three independent projections 
\begin{equation}
\begin{split}
\mathcal{L}_{JM}(q)&=\frac{i}{q}\int d^3r\;\left[\vec{\nabla}M_{JM}(q\vec{r})\right]\cdot\vec{J}(\vec{r})\\
\mathcal{T}^\mathrm{el}_{JM}(q)&=\frac{1}{q}\int d^3r\;\left[\vec{\nabla}\times\vec{M}_{J\;J\;M}(q\vec{r})\right]\cdot\vec{J}(\vec{r})\\
\mathcal{T}^\mathrm{mag}_{JM}(q)&=\int d^3r\;\vec{M}_{J\;J\;M}(q\vec{r})\cdot\vec{J}(\vec{r})
\end{split}
\end{equation}
\chapter{Nuclear Matrix Elements of One- and Two-body Operators}
\label{app:density}
In order to make contact with $\mu\rightarrow e$ conversion experiments, the underlying nuclear operators must be evaluated between many-body wave functions that accurately capture the detailed structure of the target nucleus. Although many competing methods for these evaluations exist, our preferred paradigm is that of the nuclear shell model. In the shell-model approach, the total nuclear wave function is composed of linear combinations of Slater determinant basis elements. Each Slater determinant is a totally anti-symmetric combination of single-particle harmonic oscillator states labeled by nodal quantum number $n$, orbital angular momentum $\ell$, spin $s=1/2$, and total angular momentum $j=\ell\pm 1/2$. Separate Slater determinants are constructed for neutron and proton degrees of freedom which are distinguished by the isospin quantum number $m_t=+1/2$ $(-1/2)$ for protons (neutrons). 

Having specified the basis, the required wave functions are obtained by diagonalizing a suitable Hamiltonian which describes the interactions among the nucleons. One encounters several difficulties in this construction: first, the exact form of these interactions -- which contain two-body, three-body, and higher contributions -- is not known precisely; the proper theory is that specified by quantum chromodynamics (QCD), which does not lend itself to a simple description in the strongly-coupled regime of nuclear physics. Second, the shell-model space which we have adopted must be severely truncated in practical calculations, requiring one to perform a program of operator and wave function renormalization. These complications can be avoided by abandoning efforts to root the calculation of the nuclear wave function in the first-principles of QCD and instead adopting a phenomenological interaction that has been tuned in the model space to reproduce low-energy nuclear observables such as charge radii and low-lying spectra. This is the strategy which we apply in work; the interactions and shell-mdel spaces which we employ are given in Table \ref{tab:nsm_params}. 

Having thus obtained the nuclear wave functions, there remains a significant computational task in evaluating matrix elements of few-body operators. A well-organized approach is required to avoid unnecessary effort and potential confusion. By formulating the problem in terms of irreducible tensor operators and exploiting the good angular momentum $J$ and (approximate) isospin $T$ quantum numbers of the nuclear ground state, we are able to factorize many-body matrix elements of few-body operators in terms of few-body matrix elements, extracting from the total nuclear wave function only that nuclear structure information which is required to evaluate the few-body operators under consideration. In this appendix, we will demonstrate how this factorization is performed for one- and two-body operators.
\section{One-body Density Matrices}
The total one-body tensor operator for a system of $A$ nucleons can be written in the coordinate representation as
\begin{equation}
\mathcal{O}_{JM}=\sum_{i=1}^A\mathcal{O}_{JM}(\vec{x}_i),
\end{equation}
or in the second-quantized form as
\begin{equation}
\mathcal{O}_{JM}=\sum_{\alpha,\beta}\mathcal{O}_{JM}^{\alpha\beta}c^{\dag}_{\alpha}c_{\beta}
\end{equation}
with 
\begin{equation}
\begin{split}
\mathcal{O}^{\alpha\beta}_{JM}\equiv \braket{\alpha|\mathcal{O}_{JM}|\beta}=\int d^3x\;\phi^{\dag}_{\alpha}(\vec{x})\mathcal{O}_{JM}(\vec{x})\phi_{\beta}(\vec{x}).
\end{split}
\end{equation}
The summation $\alpha$ extends to all single-particle states; that is, $\alpha=(j_{\alpha},m_{\alpha})$. The operators and the states may carry additional isospin quantum numbers which we suppress here for brevity (the generalization to include isospin is straightforward).

Our aim is to evaluate the given operator for a total nuclear wave function. The nuclear response can be factorized into a piece which describes how the operator acts on single-particle states and a piece which encodes how the single-particle states are embedded in the initial and final nuclear wave functions. The latter quantity is known as the one-body density matrix. Its form can be obtained directly from the second-quantized form of the operator:
\begin{equation}
\begin{split}
\braket{J_f||\mathcal{O}_J||J_i}
&=\sum_{a,b}\frac{1}{\sqrt{2J+1}}\braket{J_f||\left[c^{\dag}_a\otimes\tilde{c}_b\right]_J||J_i}\braket{a||\mathcal{O}_J||b}\\
&\equiv \sum_{a,b}\rho^{f,i;J}_{ab}\braket{a||\mathcal{O}_J||b},
\label{eq:one_body_density}
\end{split}
\end{equation}
where we have defined the reduced one-body density matrix
\begin{equation}
\rho^{f,i;J}_{ab}\equiv \frac{1}{\sqrt{2J+1}}\braket{J_f||\left[c^{\dag}_a\otimes\tilde{c}_b\right]_J||J_i},
\end{equation}
and introduced the time-reversed annihilation operator
\begin{equation}
\tilde{c}_b=\tilde{c}_{j_b,m_b}=(-1)^{j_b+m_b}c_{j_b,-m_b}.
\end{equation}
The summation in Eq. \ref{eq:one_body_density} is over single-particle orbits $a,b$, which do not carry magnetic angular momentum quantum numbers, as all of the matrix elements have been reduced. Thus the problem of computing many-body matrix elements of single-nucleon operators has been reduced in terms of single-nucleon matrix elements. Assuming that one has obtained the initial and final nuclear wave functions in a Slater-determinant basis, it is straightforward -- though perhaps computationally intensive -- to obtain the desired one-body density matrix. 

Restoring the isospin quantum numbers, we may define the doubly-reduced one-body density matrix by 
\begin{equation}
\rho^{f,i;J,T}_{ab}\equiv \frac{1}{\sqrt{2J+1}}\frac{1}{\sqrt{2T+1}}\braket{J_f;T_f|||\left[c^{\dag}_a\otimes\tilde{c}_b\right]_{J,T}|||J_i;T_i},
\end{equation}
where the time-reversed annihilation operator now satisfies
\begin{equation}
\tilde{c}_b=\tilde{c}_{j_b,m_b;\frac{1}{2},m_{tb}}=(-1)^{j_b+m_b+1/2+m_{tb}}c_{j_b,-m_b;\frac{1}{2},-m_{tb}}.
\end{equation}
The doubly-reduced density matrix allows for efficient calculation of the doubly-reduced matrix element
\begin{equation}
\braket{J_f;T_f|||\mathcal{O}_{J,T}|||J_i;T_i}=\sum_{a,b}\rho^{f,i}_{J,T}(ab)\braket{a|||\mathcal{O}_{J,T}|||b}.
\end{equation}
\section{Two-Body Density Matrices}
The total two-body operator for a nuclear system can be written in the coordinate representation as
\begin{equation}
\mathcal{O}_{J,M}=\sum_{i<j}\mathcal{O}_{J,M}\left(\vec{x}_i,\vec{x}_j\right)=\frac{1}{2}\sum_{i\neq j}\mathcal{O}_{J,M}(\vec{x}_i,\vec{x}_j),
\end{equation}
or in the occupation number representation as
\begin{equation}
\mathcal{O}_{J,M}=\frac{1}{2}\sum_{\alpha\beta\gamma\delta}\mathcal{O}_{J,M}^{\alpha\beta\gamma\delta}c^{\dag}_{\alpha}c^{\dag}_{\beta}c_{\delta}c_{\gamma},
\end{equation}
where 
\begin{equation}
\mathcal{O}_{J,M}^{\alpha\beta\gamma\delta}\equiv \int d^3\vec{x}_1d^3\vec{x}_2\;\phi^{\dag}_{\alpha}(\vec{x}_1)\phi^{\dag}_{\beta}(\vec{x}_2)\mathcal{O}_{J,M}\left(\vec{x}_1,\vec{x}_2\right)\phi_{\gamma}(\vec{x}_1)\phi_{\delta}(\vec{x}_2).
\end{equation}
In the occupation number representation, the operator can also be written in terms of anti-symmetrized matrix elements
\begin{equation}
\mathcal{O}_{J,M}=\frac{1}{4}\sum_{abcd}\bar{\mathcal{O}}_{J,M}^{\alpha\beta\gamma\delta}c^{\dag}_{\alpha}c^{\dag}_{\beta}c_{\delta}c_{\gamma},
\end{equation}
where
\begin{equation}
\bar{\mathcal{O}}^{\alpha\beta\gamma\delta}_{J,M}\equiv \mathcal{O}^{\alpha\beta\gamma\delta}_{J,M}-\mathcal{O}^{\alpha\beta\delta\gamma}_{J,M}.
\end{equation}
Defining two-particle states by
\begin{equation}
\ket{\alpha\beta}=c^{\dag}_{\alpha}c^{\dag}_{\beta}\ket{0},
\end{equation}
we write
\begin{equation}
\bar{\mathcal{O}}^{\alpha\beta\gamma\delta}_{J,M}=\braket{\alpha\beta|\mathcal{O}_{J,M}|\gamma\delta}.
\end{equation}
We wish to compute many-body expectation values in terms of two-particle states coupled to good total angular momentum; that is, we define the two-body density matrix by
\begin{equation}
\braket{J_f||\mathcal{O}_J||J_i}=\sum_{abcd}\sum_{J_{ab}J_{cd}}\rho_{abcd}^{f,i,J;J_{ab},J_{cd}}\braket{ab;J_{ab}||\mathcal{O}_J||cd;J_{cd}},
\label{eq:2bdy_dm}
\end{equation}
where the normalized two-particle coupled states are defined as
\begin{equation}
\ket{a\;b;J,M}=\mathcal{N}_{ab}(J)\left[c^{\dag}_{a}c^{\dag}_{b}\right]_{J,M}\ket{0},
\end{equation}
where $\ket{0}$ is a suitable vacuum state and the normalization factor is given by
\begin{equation}
\mathcal{N}_{ab}(J)\equiv \frac{\sqrt{1-\delta_{ab}(-1)^J}}{1+\delta_{ab}}.
\end{equation}
The form of the coupled, reduced two-body density matrix follows from inserting the second-quantized form of the operator into Eq. \ref{eq:2bdy_dm}:
\begin{equation}
\rho_{abcd}^{f,i,J;J_{ab}J_{cd}}
=\frac{1}{4}\frac{(-1)^{J_c+J_d-J_{cd}}}{\sqrt{2J+1}}\braket{J_f||\left[[c_a^{\dag}\otimes c_b^{\dag}]_{J_{ab}}\otimes [\tilde{c}_d\otimes\tilde{c}_c]_{J_{cd}}\right]_J||J_i}.
\end{equation}
The preceding section can be generalized in a straightforward manner to include isospin.
\chapter{Single-nucleon Response Functions}
\label{app:single_nucleon_response}
The eleven single-nucleon response functions through order $1/m_N^2$ are given in Eq. \ref{eq:single_nucleon_responses}. One would like to evaluate matrix elements of these operators between nuclear wave functions corresponding to the chosen nuclear target. In Appendix \ref{app:density}, we demonstrated how total nuclear matrix elements of one-body operators can be factorized in terms of single-particle matrix elements multiplied by the relevant one-body density matrix. In this section, we discuss how to evaluate the single-particle matrix elements of the eleven multipole operators when the single-particle states are elements of a harmonic oscillator basis. 

The basic operators from which the multipole responses are constructed are $M_{JM}(q\vec{r})$, $\vec{M}_{J\;L\;M}(q\vec{r})$, $\vec{M}_{J\;L\;M}(q\vec{r})\cdot\frac{1}{q}\vec{\nabla}$, $M_{J,M}(q\vec{r})\;\vec{\sigma}\cdot\frac{1}{q}\vec{\nabla}$ and $\vec{M}_{J\;L\;M}(q\vec{r})\cdot\left(\vec{\sigma}\times\frac{1}{q}\vec{\nabla}\right)$. Matrix elements of the standard charge multipole and the projections of the spin current are relatively straightforward
\begin{equation}
\begin{split}
\braket{n'\left(\ell'\;1/2\right)j'||M_J(q\vec{r})||n\left(\ell\;1/2\right)j}&=\frac{1}{\sqrt{4\pi}}(-1)^{J+j+1/2}[\ell'][\ell][j'][j][J]\\
&\times\left\{\begin{array}{ccc}
\ell' & j' & \frac{1}{2}\\
j & \ell & J
\end{array}\right\}\left(\begin{array}{ccc}
\ell' & J & \ell\\
0 & 0 & 0
\end{array}\right)\braket{n'\ell'|j_J(\rho)|n\ell},
\label{eq:MJ_mat}
\end{split}
\end{equation}
and
\begin{equation}
\begin{split}
\braket{n'\left(\ell'\;1/2\right)j'||\vec{M}_{J\;L}(q\vec{r})\cdot\vec{\sigma}||n\left(\ell\;1/2\right)j}&=\frac{\sqrt{6}}{\sqrt{4\pi}}(-1)^{\ell'}[\ell'][\ell][j'][j][L][J]\\
&\times\left\{\begin{array}{ccc}
\ell' & \ell & L\\
\frac{1}{2} & \frac{1}{2} & 1\\
j' & j & J
\end{array}\right\}\left(\begin{array}{ccc}
\ell' & L & \ell\\
0 & 0 & 0
\end{array}\right)\braket{n'\ell'|j_L(\rho)|n\ell},
\label{eq:MJL_sigma_mat}
\end{split}
\end{equation}
where $[J]=\sqrt{2J+1}$ and $\rho=qr$. The radial matrix elements are computed in terms of radial harmonic oscillator wave functions
\begin{equation}
\braket{n'\ell'|j_L(\rho)|n\ell}=\int_0^{\infty}dr\;r^2H_{n'\ell'}(r)j_L(qr)H_{n\ell}(r).
\end{equation}
We note that in Eq. \ref{eq:MJ_mat}, the 3-$j$ symbol implies that $(-1)^{J+\ell'+\ell}=+1$ or equivalently $J+\ell'+\ell$ is even. Similarly in Eq. \ref{eq:MJL_sigma_mat}, $L+\ell'+\ell$ must be even. This constraint is imposed by the parity transformation of the underlying multipole operator. The fact that $L+\ell'+\ell$ is even for these operators is crucial to demonstrating that the radial matrix element $\braket{n'\ell'|j_L(\rho)|n\ell}$ is a polynomial in $y=(qb/2)^2$. Next we consider the projections of the convection current 
\begin{equation}
\begin{split}
&\braket{n'\left(\ell'\;1/2\right)j'||\vec{M}_{J\;L}(q\vec{r})\cdot\frac{1}{q}\vec{\nabla}||n\left(\ell\;1/2\right)j}=\frac{1}{\sqrt{4\pi}}(-1)^{L+j+1/2}[\ell'][j'][j][L][J]\left\{\begin{array}{ccc}
\ell' & j' & \frac{1}{2}\\
j & \ell & J
\end{array}\right\}\\
&\times\Bigg\{-\sqrt{\ell+1}[\ell+1]\left\{\begin{array}{ccc}
L & 1 & J\\
\ell & \ell' & \ell + 1
\end{array}\right\}\left(\begin{array}{ccc}
\ell' & L & \ell+1\\
0 & 0 & 0
\end{array}\right)\braket{n'\ell'|j_L(\rho)\left(\frac{d}{d\rho}-\frac{\ell}{\rho}\right)|n\ell}\\
&\hspace{15mm}+\sqrt{\ell}[\ell-1]\left\{\begin{array}{ccc}
L & 1 & J\\
\ell & \ell' & \ell-1
\end{array}\right\}\left(\begin{array}{ccc}
\ell' & L & \ell - 1\\
0 & 0 & 0
\end{array}\right)\braket{n'\ell'|j_L(\rho)\left(\frac{d}{d\rho}+\frac{\ell+1}{\rho}\right)|n\ell}\Bigg\}
\end{split}
\end{equation}
Here the 3-$j$ symbols imply that $L+\ell'+\ell$ must be odd. 
\begin{equation}
\begin{split}
&\braket{n'\left(\ell'\;1/2\right)j'||M_J(q\vec{r})\vec{\sigma}\cdot\frac{1}{q}\vec{\nabla}||n\left(\ell\;1/2\right)j}=\frac{1}{\sqrt{4\pi}}(-1)^{\ell'}[\ell'][j'][j][2j-\ell][J]\left\{\begin{array}{ccc}
\ell' & j' & \frac{1}{2}\\
j & 2j-\ell & J
\end{array}\right\}\\
&\times\left(\begin{array}{ccc}
\ell' & J & 2j-\ell\\
0 & 0 & 0
\end{array}\right)\Bigg\{-\delta_{j,\ell+1/2}\braket{n'\ell'|j_J(\rho)\left(\frac{d}{d\rho}-\frac{\ell}{\rho}\right)|n\ell}+\delta_{j,\ell-1/2}\braket{n'\ell'|j_J(\rho)\left(\frac{d}{d\rho}+\frac{\ell+1}{\rho}\right)|n\ell}\Bigg\}
\end{split}
\end{equation}
again we find $J+\ell'+\ell$ is odd.
For the multipoles $\Phi_J$, $\Phi'_J$ and $\Phi_J''$, we reorganize
\begin{equation}
\vec{M}_{J\;L\;M}\cdot\left(\vec{\sigma}\times \vec{\nabla}\right)=-i\sqrt{6}\sum_K (-1)^{K+L}\sqrt{2K+1}\left\{\begin{array}{ccc}
1 & 1 & 1\\
J & K & L
\end{array}\right\}\left[\vec{\sigma}\otimes\left[Y_L\otimes\vec{\nabla}\right]_K\right]_{J,M}
\end{equation}
Then the projections of the spin-velocity current are
\begin{equation}
\begin{split}
&\braket{n'\left(\ell\;1/2\right)j'||\vec{M}_{J\;L}(q\vec{r})\cdot\left(\vec{\sigma}\times\frac{1}{q}\vec{\nabla}\right)||n\left(\ell\;1/2\right)j}=\frac{i}{\sqrt{4\pi}}(-1)^{L+J+\ell}6[l'][j'][j][L][J]\sum_{K=J-1}^{J+1}(-1)^K[K]\\
&\times\left\{\begin{array}{ccc}
\ell' & \ell & K\\
\frac{1}{2} & \frac{1}{2} & 1\\
j' & j & J
\end{array}\right\}\Bigg[\sqrt{\ell+1}[\ell+1]\left\{\begin{array}{ccc}
L & 1 & K\\
\ell & \ell' & \ell + 1
\end{array}\right\}\left(\begin{array}{ccc}
\ell' & L & \ell+1\\
0 & 0 & 0
\end{array}\right)\braket{n'\ell'|j_L(\rho)\left(\frac{d}{d\rho}-\frac{\ell}{\rho}\right)|n\ell}\\
&-\sqrt{\ell}[\ell-1]\left\{\begin{array}{ccc}
L & 1 & K\\
\ell & \ell' & \ell-1
\end{array}\right\}\left(\begin{array}{ccc}
\ell' & L & \ell-1\\
0 & 0 & 0
\end{array}\right)\braket{n'\ell'|j_L(\rho)\left(\frac{d}{d\rho}+\frac{\ell+1}{\rho}\right)|n\ell}\Bigg]\left\{\begin{array}{ccc}
1 & 1 & 1\\
J & K & L
\end{array}\right\}
\end{split}
\end{equation}
and $L+\ell'+\ell$ is odd.
Therefore, each of the eleven single-nucleon multipole operators can be computed in terms of a product of angular momentum factors and one of three radial matrix elements
\begin{equation}
\begin{split}
&\braket{n'\ell'|j_L(\rho)|n\ell},\\
&\braket{n'\ell'|j_L(\rho)\left(\frac{d}{d\rho}-\frac{\ell}{\rho}\right)|n\ell},\\
&\braket{n'\ell'|j_L(\rho)\left(\frac{d}{d\rho}+\frac{\ell+1}{\rho}\right)|n\ell}
\end{split}
\end{equation}
\begin{equation}
H_{n,\ell}(x)=\sqrt{\frac{2\Gamma(n)}{\Gamma\left(n+\ell+1/2\right)}}e^{-x^2/2}x^{\ell}L_{n-1}^{\ell+1/2}(x^2),
\end{equation}
where the coordinate $x=r/b$ is dimensionless.
The harmonic oscillator radial wave functions satisfy recurrence relations which can be used to relate states with $n>1$ in terms of states with $n=1$ and various $\ell$. Explicitly for $n=2$ and $n=3$, we have
\begin{equation}
\begin{split}
H_{2\ell}(x)&=\frac{1}{\sqrt{2}}\left\{[\ell+1]H_{1\ell}(x)-[\ell+2]H_{1\ell+2}(x)\right\}\\
H_{3\ell}(x)&=\frac{1}{\sqrt{8}}\left\{[\ell+1][\ell+2]R_{1\ell}(x)-2[\ell+2]^2R_{1\ell+2}(x)+[\ell+3][\ell+4]R_{1\ell+4}(x)\right\}
\end{split}
\end{equation}
Note that the recurrence relation conserves the parity of $\ell$. After using the recurrence relation to obtain $n=1$ states, we can apply the derivative operators as
\begin{equation}
\begin{split}
\left(\frac{d}{d\rho}-\frac{\ell}{\rho}\right)R_{1\ell}(x)&=-\frac{1}{\sqrt{8y}}[\ell+1]R_{1\ell+1}(x)\\
\left(\frac{d}{d\rho}+\frac{\ell+1}{\rho}\right)R_{1\ell}(x)&=\frac{1}{\sqrt{8y}}\left\{2[\ell]R_{1\ell-1}(x)-[\ell+1]R_{1\ell+1}(x)\right\},
\end{split}
\end{equation}
where the parity of the state is now changed as $\ell\rightarrow \ell\pm 1$. With these relations in hand, the only matrix element that we need to explicitly compute is
\begin{equation}
\braket{1\ell'|j_L(\rho)|1\ell}=\frac{(2y)^{L/2}e^{-y}\left(L+\ell'+\ell+1\right)!!}{(2L+1)!!\{(2\ell'+1)!!(2\ell+1)!!\}^{1/2}}\;_1F_1\left(\frac{L-\ell'-\ell}{2};L+\frac{3}{2};y\right)
\end{equation}
The crucial observation is that the confluent hypergeometric function
\begin{equation}
\;_1F_1(\alpha;\beta;y)=1+\frac{\alpha}{\beta}y+\frac{\alpha(\alpha+1)}{\beta(\beta+1)}\frac{y^2}{2!}+...
\end{equation}
terminates at finite order whenever $\alpha=(L-\ell'-\ell)/2$ is a non-positive integer. Therefore, letting $T_J(q\vec{r})$ represent any of the 11 single-particle multipole operators, 
\begin{equation}
\braket{n'\left(\ell'\;1/2\right)j'||T_J(q\vec{r})||n\left(\ell\;1/2\right)j}=\frac{1}{\sqrt{4\pi}}y^{(J-K)/2}e^{-y}p(y),
\end{equation}
where $K=2$ for the normal parity operators $M$, $\Delta'$, $\Sigma$, $\Phi'$, and $\Phi''$, and where $K=1$ for the abnormal parity operators $\Delta$, $\Sigma'$, $\Sigma''$, $\Omega$, and $\Phi$.

In addition to the recurrence relations, we can derive closed-form expressions for the matrix elements. The Laguerre polynomial can be expanded as
\begin{equation}
L_{n-1}^{\ell+1/2}(x^2)=\sum_{i=0}^{n-1}\left(\begin{array}{c}
n+\ell-1/2\\
n-i-1
\end{array}\right)\frac{(-1)^i}{i!}x^{2i}
\end{equation}
The required radial integrals are of the form
\begin{equation}
I_L(m,y)\equiv
\int_0^{\infty}dx\;x^me^{-x^2}j_L(qbx)=\frac{\sqrt{\pi}}{4}y^{L/2}e^{-y}\frac{\Gamma\left(\frac{1}{2}(L+m+1)\right)}{\Gamma(L+3/2)}\;_1F_1\left(1+\frac{L-m}{2};L+\frac{3}{2};y\right),
\end{equation} 
which converges for $L+m>-1$. Here the confluent hypergeometric function appears with parameter $\alpha=1+(L-m)/2$. As we see below, for the physically relevant values of $m$, $\alpha$ is always a non-positive integer and therefore the summation terminates at order $\alpha$. The radial matrix elements can then be expressed in terms of the basic integral $I_L(m,y)$ as
\begin{equation}
\begin{split}
\braket{n'\ell'|j_L(\rho)|n\ell}&=\sqrt{\frac{2\Gamma(n')}{\Gamma(n'+\ell'+1/2)}\frac{2\Gamma(n)}{\Gamma(n+\ell+1/2)}}\sum_{i=0}^{n'-1}\sum_{j=0}^{n-1}\left(\begin{array}{c}
n'+\ell'-1/2\\
n'-i-1
\end{array}\right)\left(\begin{array}{c}
n+\ell-1/2\\
n-j-1
\end{array}\right)\\
&\times\frac{(-1)^{i+j}}{i!j!}I_L\left(2+2i+2j+\ell'+\ell,y\right),
\end{split}
\end{equation}
\begin{equation}
\begin{split}
&\braket{n'\ell'|j_L(\rho)\left(\frac{d}{d\rho}-\frac{\ell}{\rho}\right)|n\ell}=\sqrt{\frac{2\Gamma(n')}{\Gamma(n'+\ell'+1/2)}\frac{2\Gamma(n)}{\Gamma(n+\ell+1/2)}}\sum_{i=0}^{n'-1}\sum_{j=0}^{n-1}\left(\begin{array}{c}
n'+\ell'-1/2\\
n'-i-1
\end{array}\right)\\
&\times\left(\begin{array}{c}
n+\ell-1/2\\
n-j-1
\end{array}\right)\frac{(-1)^{i+j}}{i!j!}\left[2jI_L\left(1+2i+2j+\ell'+\ell,y\right)-I_L\left(3+2i+2j+\ell'+\ell,y\right)\right],
\end{split}
\end{equation}
\begin{equation}
\begin{split}
&\braket{n'\ell'|j_L(\rho)\left(\frac{d}{d\rho}+\frac{\ell+1}{\rho}\right)|n\ell}=\sqrt{\frac{2\Gamma(n')}{\Gamma(n'+\ell'+1/2)}\frac{2\Gamma(n)}{\Gamma(n+\ell+1/2)}}\sum_{i=0}^{n'-1}\sum_{j=0}^{n-1}\left(\begin{array}{c}
n'+\ell'-1/2\\
n'-i-1
\end{array}\right)\\
&\times\left(\begin{array}{c}
n+\ell-1/2\\
n-j-1
\end{array}\right)\frac{(-1)^{i+j}}{i!j!}\left[\left(2j+2\ell+1\right)I_L\left(1+2i+2j+\ell'+\ell,y\right)-I_L\left(3+2i+2j+\ell'+\ell,y\right)\right],
\end{split}
\end{equation}
\section{Operators Generated by the Muon's Lower Component}
As discussed in Section \ref{sec:muon_lower}, the inclusion of the muon's lower Dirac component leads to the introduction of new nuclear multipole operators. We will now discuss how single-particle matrix elements of these operators can be evaluated in a harmonic oscillator basis. 

\begin{equation}
\begin{split}
\braket{n'\left(\ell'\;1/2\right)j'||M_J^{(1)}(q\vec{r})||n\left(\ell\;1/2\right)j}&=\frac{1}{\sqrt{4\pi}}(-1)^{J+j+1/2}[\ell'][\ell][j'][j][J]\sqrt{J(J+1)}\\
&\times\left\{\begin{array}{ccc}
\ell' & j' & \frac{1}{2}\\
j & \ell & J
\end{array}\right\}\left(\begin{array}{ccc}
\ell' & J & \ell\\
0 & 0 & 0
\end{array}\right)\braket{n'\ell'|\frac{1}{\rho}j_J(\rho)|n\ell}
\end{split}
\end{equation}
\begin{equation}
\begin{split}
\braket{n'\left(\ell'\;1/2\right)j'||M_J^{(1)}(q\vec{r})||n\left(\ell\;1/2\right)j}&=\frac{1}{\sqrt{4\pi}}(-1)^{J+j+1/2}[\ell'][\ell][j'][j][J]\\
&\times\left\{\begin{array}{ccc}
\ell' & j' & \frac{1}{2}\\
j & \ell & J
\end{array}\right\}\left(\begin{array}{ccc}
\ell' & J & \ell\\
0 & 0 & 0
\end{array}\right)\braket{n'\ell'|\frac{dj_J(\rho)}{d\rho}|n\ell}
\end{split}
\end{equation}
In order to compute matrix elements of the modified spin projections $\Sigma'^{(0)}$, $\Sigma''^{(0)}$, we define the operator
\begin{equation}
\vec{M}^{(0)}_{J\;L\;M}(q\vec{r})\equiv j_J(qr)\vec{Y}_{J\;L\;M}(\hat{r})
\end{equation}
with matrix elements
\begin{equation}
\begin{split}
\braket{n'\left(\ell'\;1/2\right)j'||\vec{M}^{(0)}_{J\;L}(q\vec{r})\cdot\vec{\sigma}||n\left(\ell\;1/2\right)j}&=\frac{\sqrt{6}}{\sqrt{4\pi}}(-1)^{\ell'}[\ell'][\ell][j'][j][L][J]\\
&\times\left\{\begin{array}{ccc}
\ell' & \ell & L\\
\frac{1}{2} & \frac{1}{2} & 1\\
j' & j & J
\end{array}\right\}\left(\begin{array}{ccc}
\ell' & L & \ell\\
0 & 0 & 0
\end{array}\right)\braket{n'\ell'|j_J(\rho)|n\ell},
\end{split}
\end{equation}
Similarly, to compute matrix elements of the operators $\Sigma'^{(2)}$, $\Sigma''^{(2)}$, we define
\begin{equation}
\vec{M}^{(2)}_{J\;L\;M}(q\vec{r})\equiv \frac{dj_L(qr)}{dqr}\;\vec{Y}_{J\;L\;M}(\hat{r})
\end{equation}
with corresponding matrix elements
\begin{equation}
\begin{split}
\braket{n'\left(\ell'\;1/2\right)j'||\vec{M}^{(2)}_{J\;L}(q\vec{r})\cdot\vec{\sigma}||n\left(\ell\;1/2\right)j}&=\frac{\sqrt{6}}{\sqrt{4\pi}}(-1)^{\ell'}[\ell'][\ell][j'][j][L][J]\\
&\times\left\{\begin{array}{ccc}
\ell' & \ell & L\\
\frac{1}{2} & \frac{1}{2} & 1\\
j' & j & J
\end{array}\right\}\left(\begin{array}{ccc}
\ell' & L & \ell\\
0 & 0 & 0
\end{array}\right)\braket{n'\ell'|\frac{dj_L(\rho)}{d\rho}|n\ell},
\end{split}
\end{equation}
For the lower component matrix elements we have
\begin{equation}
\begin{split}
\braket{n'\ell'|\frac{1}{\rho}j_L(\rho)|n\ell}&=\sqrt{\frac{2\Gamma(n')}{\Gamma(n'+\ell'+1/2)}\frac{2\Gamma(n)}{\Gamma(n+\ell+1/2)}}\sum_{i=0}^{n'-1}\sum_{j=0}^{n-1}\left(\begin{array}{c}
n'+\ell'-1/2\\
n'-i-1
\end{array}\right)\left(\begin{array}{c}
n+\ell-1/2\\
n-j-1
\end{array}\right)\\
&\times\frac{(-1)^{i+j}}{i!j!}\frac{1}{2\sqrt{y}}I_L\left(1+2i+2j+\ell'+\ell,y\right)
\end{split}
\end{equation}
\begin{equation}
\begin{split}
&\braket{n'\ell'|\frac{dj_L(\rho)}{d\rho}|n\ell}=\sqrt{\frac{2\Gamma(n')}{\Gamma(n'+\ell'+1/2)}\frac{2\Gamma(n)}{\Gamma(n+\ell+1/2)}}\sum_{i=0}^{n'-1}\sum_{j=0}^{n-1}\left(\begin{array}{c}
n'+\ell'-1/2\\
n'-i-1
\end{array}\right)\left(\begin{array}{c}
n+\ell-1/2\\
n-j-1
\end{array}\right)\\
&\times\frac{(-1)^{i+j}}{i!j!}\frac{1}{2L+1}\left[LI_{L-1}\left(2+2i+2j+\ell'+\ell,y\right)-(L+1)I_{L+1}\left(2+2i+2j+\ell'+\ell,y\right)\right]
\end{split}
\end{equation}
As before, these radial matrix elements are finite polynomials in $y$.
\chapter{Fermi Gas Average}
\label{app:fga}
In the Fermi Gas Average (FGA) approach, we choose a target nucleon and sum over its interaction with a spin and isospin symmetric core. The core nucleons occupy momentum states up to the nuclear Fermi momentum, $k_F$. Starting from a two-nucleon operator $\mathcal{O}^{(2)}$, an effective one-body operator is obtained by performing a mean-field-like sum over direct and exchange terms
\begin{equation}
\braket{\alpha|\mathcal{O}^{(1)}|\beta}\equiv \sum_{\gamma}\braket{\alpha\gamma|\mathcal{O}^{(2)}|\beta\gamma}-\braket{\alpha\gamma|\mathcal{O}^{(2)}|\gamma\beta},
\end{equation}
where $\gamma$ sums over occupied core states. Each core state is a direct product of space, spin, and isospin components
\begin{equation}
\ket{\alpha}=\ket{\vec{p}(\alpha)}\otimes \ket{\frac{1}{2}m_s(\alpha)}\otimes \ket{\frac{1}{2}m_t(\alpha)}.
\end{equation}
Therefore, the summations over space, spin, and isospin components can be performed independently.
The two-nucleon operator which arises in scalar-mediated coherent $\mu\rightarrow e$ conversion has the form
\begin{equation}
\mathcal{O}^{(2)}=\frac{\vec{q}_1\cdot\vec{\sigma}_1}{|\vec{q}_1|^2+m_{\pi}^2}\frac{\vec{q}_2\cdot\vec{\sigma}_2}{|\vec{q}_2|^2+m_{\pi}^2}\vec{\tau}_1\cdot\vec{\tau}_2
\end{equation}
We begin by decomposing $\mathcal{O}^{(2)}$ into irreducible tensor operators
\begin{equation}
\begin{split}
\vec{q}_1\cdot\vec{\sigma}_1\vec{q}_2\cdot\vec{\sigma}_2&=\sum_{J=0}^2(-1)^J\left[\vec{q}_1\otimes\vec{q}_2\right]_J\odot\left[\vec{\sigma}_1\otimes\vec{\sigma}_2\right]_J\\
&=\frac{1}{3}\vec{q}_1\cdot\vec{q}_2\vec{\sigma}_1\cdot\vec{\sigma}_2+\frac{1}{2}\left(\vec{q}_1\times\vec{q}_2\right)\cdot\left(\vec{\sigma}_1\times\vec{\sigma}_2\right)+\left[\vec{q}_1\otimes\vec{q}_2\right]_2\odot\left[\vec{\sigma}_1\otimes\vec{\sigma}_2\right]_2
\end{split}
\end{equation}
Computing the average over the spin operators (see Table \ref{tab:fga_spin_avg}) we find that all direct contributions vanish. We have only to compute the exchange terms, where we find that the scalar spin operator $\vec{\sigma}_1\cdot\vec{\sigma}_2$ averages to the spin-independent operator $I_2$ whereas the vector operator $\vec{\sigma}_1\times\vec{\sigma}_2$ averages to the spin-dependent operator $\vec{\sigma}$. 

In the exchange term, the momentum transfers are $\vec{q}_1=\vec{p}_\alpha-\vec{p}_\gamma$, $\vec{q}_2=\vec{p}_\gamma-\vec{p}_\beta$, $\vec{q}=\vec{p}_\alpha-\vec{p}_\beta$. Let us introduce the average momentum of the single nucleon $\vec{k}=\frac{1}{2}\left(\vec{p}_{\alpha}+\vec{p}_\beta\right)$. We will write the effective one-body operator as
\begin{equation}
\mathcal{O}^{(1)}(\vec{p}_\alpha,\vec{p}_\beta)=\frac{3}{16\pi}\left[f^\mathrm{SI}(\vec{q},\vec{k})I_2-f^\mathrm{SD}(\vec{q},\vec{k})i\vec{\sigma}\cdot\left(\vec{q}\times\vec{k}\right)\right],
\end{equation}
where $f^\mathrm{SI}(\vec{q},\vec{k})$ and $f^\mathrm{SD}(\vec{q},\vec{k})$ are, respectively, the spin-independent and spin-dependent effective one-body form factors to be computed. 
\begin{table}
\centering
\begin{tabular}{ccc}
\hline
\hline
2-body & 1-body direct & 1-body exchange\\
\hline
$\vec{\sigma}_1\cdot\vec{\sigma}_2$ & 0 & $3I_2$\\
$\vec{\sigma}_1\times\vec{\sigma}_2$ & 0 & $2i\vec{\sigma}$\\
$3\sigma_{1z}\sigma_{2z}-\vec{\sigma}_1\cdot\vec{\sigma}_2$ & 0 & 0\\
$\vec{\tau}_1\cdot\vec{\tau}_2$ & 0 & 3$I_2$\\
\hline
\hline
\end{tabular}
\caption{One-body average of two-body spin and isospin operators for direct and exchange contributions. We assume that the Fermi distributions for the protons and neutrons are identical. This assumption may be violated in very heavy nuclei.}
\label{tab:fga_spin_avg}
\end{table}
Let us begin by considering the spin-independent form factor, which we express as an integral over the Fermi sphere
\begin{equation}
\begin{split}
f^\mathrm{SI}(\vec{q},\vec{k})&=-16\pi\int \frac{d^3\vec{p}_{\gamma}}{(2\pi)^3}\frac{\left(\vec{p}_\alpha-\vec{p}_{\gamma}\right)\cdot\left(\vec{p}_{\gamma}-\vec{p}_\beta\right)}{\left[\left(\vec{p}_\alpha-\vec{p}_{\gamma}\right)^2+m_{\pi}^2\right]\left[\left(\vec{p}_{\gamma}-\vec{p}_\beta\right)^2+m_{\pi}^2\right]}\\
&=-\frac{2}{\pi^2}\int_0^{K_F}d|\vec{p}_{\gamma}|\;|\vec{p}_{\gamma}|^2\int d\Omega_{p_{\gamma}}\frac{\left(\vec{k}+\frac{1}{2}\vec{q}-\vec{p}_{\gamma}\right)\cdot\left(\vec{p}_\gamma-\vec{k}+\frac{1}{2}\vec{q}\right)}{\left[\left(\vec{k}+\frac{1}{2}\vec{q}-\vec{p}_{\gamma}\right)^2+m_{\pi}^2\right]\left[\left(\vec{p}_{\gamma}-\vec{k}+\frac{1}{2}\vec{q}\right)^2+m_{\pi}^2\right]}
\end{split}
\end{equation}


We introduce the Feynman parameter representation
\begin{equation}
\frac{1}{AB}=\int_{-1/2}^{1/2}d\beta \frac{1}{\left[(1/2-\beta)A+(1/2+\beta)B\right]^2},
\end{equation}
with 
\begin{equation}
A=\left(\vec{k}+\frac{1}{2}\vec{q}-\vec{p}_{\gamma}\right)^2+m_{\pi}^2,\;\;\;B=\left(\vec{p}_{\gamma}-\vec{k}+\frac{1}{2}\vec{q}\right)^2+m_{\pi}^2
\end{equation}
which yields
\begin{equation}
f^\mathrm{SI}(\vec{q},\vec{k})=-\frac{2}{\pi^2}\int_{-1/2}^{1/2}d\beta \int_0^{K_F}d|\vec{p}_{\gamma}|\;|\vec{p}_{\gamma}|^2\int d\Omega_{p_{\gamma}}\frac{2\vec{p}_{\gamma}\cdot\vec{k}+\frac{1}{4}|\vec{q}|^2-|\vec{k}|^2-|\vec{p}_{\gamma}|^2}{\left[|\vec{p}_{\gamma}|^2-2\vec{p}_{\gamma}\cdot\left(\vec{k}-\beta\vec{q}\right)+\Delta\right]^2},
\end{equation}
where we have defined
\begin{equation}
\Delta\equiv -2\beta\vec{k}\cdot\vec{q}+\frac{1}{4}|\vec{q}|^2+|\vec{k}|^2+m_{\pi}^2.
\end{equation}
Now we need to compute the angular $\Omega_{p_{\gamma}}$ integral. Let us orient our coordinate system so that $\hat{z}$ is along $\vec{k}-\beta\vec{q}$. Then
\begin{equation}
\vec{p}_{\gamma}\cdot \left(\vec{k}-\beta\vec{q}\right)=|\vec{p}_{\gamma}||\vec{k}-\beta\vec{q}|\cos\theta.
\end{equation}
We also have to consider the angular dependence of $\vec{p}_{\gamma}\cdot\vec{k}$. The azimuthal $\phi$-dependence of this dot product must integrate to zero so we are left with only the $\hat{z}$ component
\begin{equation}
\vec{p}_{\gamma}\cdot \vec{k}\rightarrow |\vec{p}_{\gamma}|\cos\theta \frac{\vec{k}\cdot(\vec{k}-\beta\vec{q})}{|\vec{k}-\beta\vec{q}|}
\end{equation}
Therefore
\begin{equation}
f^\mathrm{SI}(\vec{q},\vec{k})=-\frac{4}{\pi}\int_{-1/2}^{1/2}d\beta \int_0^{K_F}dp\;p^2 \int_{-1}^1d\mu \left(2p\frac{\vec{k}\cdot\left(\vec{k}-\beta\vec{q}\right)}{Q}\mu+\frac{1}{4}|\vec{q}|^2-|\vec{k}|^2-p^2\right)\frac{1}{\left[p^2-2pQ\mu+\Delta\right]^2},
\end{equation}
where we have defined $Q=|\vec{k}-\beta\vec{q}|$ and $p=|\vec{p}_\gamma|$.

Before proceeding with the angular integration, let us pause to consider the spin-dependent term $f^\mathrm{SD}(\vec{q},\vec{k})$. We begin with the fact that
\begin{equation}
\vec{q}_1\times\vec{q}_2=-\vec{q}\times\vec{k}+\vec{q}\times\vec{p}_{\gamma}
\end{equation}
and introduce the same Feynman parameter representation as in the spin-independent case. In the angular integration, the only non-vanishing part of $\vec{q}\times\vec{p}_{\gamma}$ is the $\hat{z}$ component of $p_{\gamma}$, so we may write
\begin{equation}
-\vec{q}\times\vec{k}+\vec{q}\times\vec{p}\rightarrow \vec{q}\times\vec{k}\left(\frac{p}{Q}\cos\theta-1\right)
\end{equation}
leading to
\begin{equation}
f^\mathrm{SD}(\vec{q},\vec{k})=\frac{4}{\pi}\int_{-1/2}^{1/2}d\beta \int_0^{K_F}dp\;p^2 \int_{-1}^1d\mu \left(1-\frac{p}{Q}\mu\right)\frac{1}{\left[p^2-2pQ\mu+\Delta\right]^2},
\end{equation}
Both of the required angular integrals can be computed analytically
\begin{equation}
\begin{split}
\int_{-1}^1 d\mu\;\frac{1}{\left[p^2-2pQ\mu+\Delta\right]^2}&=\frac{2}{\left(p^2+\Delta\right)^2-4p^2Q^2}\\
\int_{-1}^1 d\mu\;\frac{\mu}{\left[p^2-2pQ\mu+\Delta\right]^2}&=\frac{p^2+\Delta}{pQ}\frac{1}{(p^2+\Delta)^2-4p^2Q^2}-\frac{1}{2Q^2}\mathrm{arctanh}\left(\frac{2pQ}{p^2+\Delta}\right).
\end{split}
\end{equation}
Introducing dimensionless quantities $\bar{p}=p/k_F$, $\bar{k}=\vec{k}/k_F$, $\bar{q}=\vec{k}/k_F$, $\bar{m}=m_{\pi}/k_F$, $\bar{Q}=Q/k_F$, and $\bar{\Delta}=\Delta/k_F^2$, allows us to write the form factors as
\begin{equation}
\begin{split}
f^\mathrm{SI}(\vec{q},\vec{k})&=-\frac{4}{\pi}k_F\int_{-1/2}^{1/2}d\beta \int_0^1d\bar{p}\;\Bigg\{2\left(\frac{\bar{k}\cdot(\bar{k}-\beta\bar{q})}{\bar{Q}^2}\bar{\Delta}+\frac{1}{4}\bar{q}^2-\bar{k}^2\right)\frac{\bar{p}^2}{(\bar{p}^2+\bar{\Delta})^2-4\bar{p}^2\bar{Q}^2}\\
&+2\left(\frac{\bar{k}\cdot\left(\bar{k}-\beta\bar{q}\right)}{\bar{Q}^2}-1\right)\frac{\bar{p}^4}{(\bar{p}^2+\Delta)^2-4\bar{p}^2Q^2}-\frac{\bar{k}\cdot(\bar{k}-\beta\bar{q})}{\bar{Q}^3}\bar{p}\;\mathrm{arctanh}\left(\frac{2\bar{p}\bar{Q}}{\bar{p}^2+\bar{\Delta}}\right)\Bigg\}\\
f^\mathrm{SD}(\vec{q},\vec{k})&=\frac{4}{\pi}k_F\int_{-1/2}^{1/2}d\beta\int_0^1d\bar{p}\;\Bigg\{\left(2-\frac{\bar{\Delta}}{\bar{Q}^2}\right)\frac{\bar{p}^2}{(\bar{p}^2+\bar{\Delta})^2-4\bar{p}^2\bar{Q}^2}\\
&-\frac{1}{\bar{Q}^2}\frac{\bar{p}^4}{(\bar{p}^2+\bar{\Delta})^2-4\bar{p}^2\bar{Q}^2}+\frac{1}{2\bar{Q}^3}\bar{p}\;\mathrm{arctanh}\left(\frac{2\bar{p}\bar{Q}}{\bar{p}^2+\bar{\Delta}}\right)\Bigg\}.
\label{eq:fsi_fsd_pre_int}
\end{split}
\end{equation}
The following three integrals necessary to compute the form factors can be performed analytically:
\begin{equation}
\begin{split}
\int_0^1 d\bar{p}\;\frac{\bar{p}^2}{(\bar{p}^2+\bar{\Delta})^2-4\bar{p}^2\bar{Q}^2}&=\frac{1}{4}\frac{1}{\sqrt{\bar{\Delta}-\bar{Q}^2}}\left[\arctan\left(\frac{1-\bar{Q}}{\sqrt{\bar{\Delta}-\bar{Q}^2}}\right)+\arctan\left(\frac{1+\bar{Q}}{\sqrt{\bar{\Delta}-\bar{Q}^2}}\right)\right]\\
&-\frac{1}{8\bar{Q}}\log\left(\frac{1+\bar{\Delta}+2\bar{Q}}{1+\bar{\Delta}-2\bar{Q}}\right)\\
\int_0^1 d\bar{p}\;\frac{\bar{p}^4}{(\bar{p}^2+\bar{\Delta})^2-4\bar{p}^2\bar{Q}^2}&=-\frac{1}{4}\frac{3\bar{\Delta}-4\bar{Q}^2}{\sqrt{\bar{\Delta}-\bar{Q}^2}}\left[\arctan\left(\frac{1-\bar{Q}}{\sqrt{\bar{\Delta}-\bar{Q}^2}}\right)+\arctan\left(\frac{1+\bar{Q}}{\sqrt{\bar{\Delta}-\bar{Q}^2}}\right)\right]\\
&+1+\frac{\bar{\Delta}-4\bar{Q}^2}{8\bar{Q}}\log\left(\frac{1+\bar{\Delta}+2\bar{Q}}{1+\bar{\Delta}-2\bar{Q}}\right)\\
\int_0^1d\bar{p}\;\bar{p}\;\mathrm{arctanh}\left(\frac{2\bar{p}\bar{Q}}{\bar{p}^2+\bar{\Delta}}\right)&=\bar{Q}+\bar{Q}\sqrt{\bar{\Delta}-\bar{Q}^2}\left[\mathrm{arctan}\left(\frac{1-\bar{Q}}{\sqrt{\bar{\Delta}-\bar{Q}^2}}\right)+\mathrm{arctan}\left(\frac{1+\bar{Q}}{\sqrt{\bar{\Delta}-\bar{Q}^2}}\right)\right]\\
&-\frac{1}{4}\left(1-2\bar{Q}^2+\Delta\right)\log\left(\frac{1+2\bar{Q}+\bar{\Delta}}{1-2\bar{Q}+\bar{\Delta}}\right)
\end{split}
\end{equation}
Combining these results with the proper prefactors in Eq. \ref{eq:fsi_fsd_pre_int}, the resulting spin-independent and spin-dependent form factors are 
\begin{equation}
\begin{split}
f^{SI}(\bar{q},\bar{k})&=\frac{2}{\pi}\int_{-1/2}^{1/2}d\beta\Bigg\{2\left(1+\frac{-\beta\bar{k}\cdot\bar{q}+\beta^2\bar{q}^2}{\bar{k}^2-2\beta\bar{k}\cdot\bar{q}+\beta^2\bar{q}^2}\right)-\left(\frac{4(\frac{1}{4}-\beta^2)\bar{q}^2+3\bar{m}^2}{\sqrt{(\frac{1}{4}-\beta^2)\bar{q}^2+\bar{m}^2}}\right)\\
&\times\left[\arctan\left(\frac{1+\sqrt{\bar{k}^2-2\beta\bar{k}\cdot\bar{q}+\beta^2\bar{q}^2}}{\sqrt{(\frac{1}{4}-\beta^2)\bar{q}^2+\bar{m}^2}}\right)+\arctan\left(\frac{1-\sqrt{\bar{k}^2-2\beta\bar{k}\cdot\bar{q}+\beta^2\bar{q}^2}}{\sqrt{(\frac{1}{4}-\beta^2)\bar{q}^2+\bar{m}^2}}\right)\right]\\
&+\frac{1}{2\sqrt{\bar{k}^2-2\beta\bar{k}\cdot\bar{q}+\beta^2\bar{q}^2}}\log\left(\frac{1+2\sqrt{\bar{k}^2-2\beta\bar{k}\cdot\bar{q}+\beta^2\bar{q}^2}+\bar{k}^2-2\beta\bar{k}\cdot\bar{q}+\frac{1}{4}\bar{q}^2+\bar{m}^2}{1-2\sqrt{\bar{k}^2-2\beta\bar{k}\cdot\bar{q}+\beta^2\bar{q}^2}+\bar{k}^2-2\beta\bar{k}\cdot\bar{q}+\frac{1}{4}\bar{q}^2+\bar{m}^2}\right)\\
&\times\Bigg[1+2\bar{m}^2+\left(\frac{3}{4}-4\beta^2\right)\bar{q}^2-\bar{k}^2+2\beta\bar{k}\cdot\bar{q}+\beta\frac{\left(1+\frac{1}{4}\bar{q}^2+\bar{m}^2+\bar{k}^2-2\beta\bar{k}\cdot\bar{q}\right)\left(\bar{k}\cdot\bar{q}-\beta\bar{q}^2\right)}{\bar{k}^2-2\beta\bar{k}\cdot\bar{q}+\beta^2\bar{q}^2}\Bigg]\Bigg\}\\
f^{SD}(\bar{q},\bar{k})&=-\frac{2}{\pi}\int_{-1/2}^{1/2}d\beta\frac{1}{\sqrt{\bar{k}^2-2\beta\bar{k}\cdot\bar{q}+\beta^2\bar{q}^2}}\left[\frac{1}{\sqrt{\bar{k}^2-2\beta\bar{k}\cdot\bar{q}+\beta^2\bar{q}^2}}\right.\\
&\left.-\frac{1+\bar{m}^2+\bar{k}^2-2\beta\bar{k}\cdot\bar{q}+\frac{1}{4}\bar{q}^2}{4\left(\bar{k}^2-2\beta\bar{k}\cdot\bar{q}+\beta^2\bar{q}^2\right)}\log\left(\frac{1+2\sqrt{\bar{k}^2-2\beta\bar{k}\cdot\bar{q}+\beta^2\bar{q}^2}+\bar{k}^2-2\beta\bar{k}\cdot\bar{q}+\frac{1}{4}\bar{q}^2+\bar{m}^2}{1-2\sqrt{\bar{k}^2-2\beta\bar{k}\cdot\bar{q}+\beta^2\bar{q}^2}+\bar{k}^2-2\beta\bar{k}\cdot\bar{q}+\frac{1}{4}\bar{q}^2+\bar{m}^2}\right)\right]
\end{split}
\end{equation}
Note that these functions depend on not only the magnitude of the dimensionless momentum transfer $\bar{q}$ and average momentum $\bar{k}$, but on their relative angle. Fortunately, for the physically relevant values of these momenta, $f^{SI}$ and $f^{SD}$ do not vary significantly over the range of possible angular values. Therefore we may replace each function by its angular average. The angle-averaged functions then depend only on the magnitude of the momentum transfer and the average momentum. For $\mu^-\rightarrow e^-$ conversion in $^{27}$Al, $|\vec{q}|\approx m_{\mu}$. Fixing the magnitude of the momentum transfer, $f^{SI}$ and $f^{SD}$ are now functions of the dimensionless average nucleon momentum $\bar{k}$, as shown in figure \ref{fig:1body_f_functions}. In order to recover a local one-body effective operator, we now wish to replace these slowly-varying functions of $\bar{k}$ by a constant. We can weight our average by the nucleon momentum probability distribution obtained from the measured nucleon density.
\begin{figure}
\centering
\subfloat[]{
\includegraphics[scale=0.5]{fsi_plot.png}
}
\hfill
\subfloat[]{
\includegraphics[scale=0.5]{fsd_plot.png}
}
\caption{(a) Angle averaged value of $f^{SI}$ at $|\vec{q}|=m_{\mu}$ and its constant approximation $f_\mathrm{eff}^{SI}$ as a function of the dimensionless average momentum $\bar{k}$. 
\\
(b) Angle averaged value of $f^{SD}$ at $|\vec{q}|=m_{\mu}$ and its constant approximation $f_\mathrm{eff}^{SD}$ as a function of the dimensionless average momentum $\bar{k}$.}
\label{fig:1body_f_functions}
\end{figure}


This calculation was first performed in \cite{2018PhRvC..98a5208B} but several errors were committed which resulted in incorrect expressions for $f^\mathrm{SI}(\bar{q},\bar{k}$, $f^\mathrm{SD}(\bar{q},\bar{k})$ and all quantities derived from them.

\end{document}
